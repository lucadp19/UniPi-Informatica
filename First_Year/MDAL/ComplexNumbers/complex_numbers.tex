\documentclass[italian,oneside,headinclude,10pt]{scrartcl}
    \usepackage[utf8]{inputenc}
    \usepackage[italian]{babel}
    \usepackage[T1]{fontenc}
    \usepackage{textcomp, microtype}
    \usepackage{amsmath, amsthm, amssymb, cases, mathtools, bm, nicefrac}
    % \usepackage{breqn}
    \usepackage{stmaryrd} % per \trianglelefteqslant
    \usepackage{array, multicol}
    \usepackage{centernot}
    \usepackage{faktor}



    \usepackage{xparse}

    \usepackage{enumitem}
    \usepackage{float}
    \usepackage{letltxmacro}

    \LetLtxMacro\amsproof\proof
    \LetLtxMacro\amsendproof\endproof

    
    % \usepackage[pdfspacing]{classicthesis}
    \usepackage[style=arsclassica, pdfspacing, eulermath]{classicthesis}
    % \usepackage{lmodern}

    \usepackage{tikz}
    \usetikzlibrary{angles, arrows.meta, quotes}
    
    \usepackage{thmtools}
    \usepackage[framemethod=TikZ]{mdframed}
    \usepackage{tikz-cd}
    \usepackage{hyperref} % penultimo package da caricare!
    \usepackage{cleveref} % ultimo package da caricare!


\restylefloat{table}

% \AtBeginDocument{
%     \LetLtxMacro\proof\amsproof
%     \LetLtxMacro\endproof\amsendproof
% }
\makeatletter %only needed in preamble
\renewcommand\large{\@setfontsize\large{11.5pt}{18}}
\makeatother

% \titleformat*{\chapter}{\LARGE\scshape}
% \titleformat*{\section}{\Large\scshape}

% % \renewenvironment{proof}[1][\proofname]{{\scshape #1. }}{\qed\medskip}
% \renewcommand*{\chapterformat}{%
% \mbox{\chapappifchapterprefix{\nobreakspace}%
% \scalebox{2}{\color{gray}\thechapter\autodot}\enskip}}

\makeatletter
\renewenvironment{proof}[1][\proofname]
  {\par\pushQED{\qed}%
   \normalfont \topsep3\p@\@plus6\p@\relax
   \list{}{\leftmargin=2em
          \rightmargin=\leftmargin
          \settowidth{\itemindent}{\itshape#1}%
          \labelwidth=\itemindent
          % the following line is not needed with amsart, but might be with other classes
          \parsep=0pt \listparindent=\parindent 
  }
   \item[\hskip\labelsep
        %  \scshape
         \bfseries
         #1\@addpunct{.\hspace{1em}}]\ignorespaces}
  {\popQED\endlist\@endpefalse}
\makeatother

\declaretheoremstyle[
    spaceabove=2\topsep, spacebelow=2\topsep,
    headindent=0pt,
    headfont=\bfseries,
    notefont=\normalfont\normalsize\bfseries, notebraces={}{.},
    bodyfont=\itshape\normalsize,
    headformat={\llap{\smash{\parbox[t]{1.1in}{\centering \NAME\\ \NUMBER}}} \NOTE},
    headpunct={},
    postheadspace=10pt
]{thmstyle}
\declaretheorem[numberwithin=section, style=thmstyle]{principle}
\declaretheorem[name=Teorema, numberwithin=section, style=thmstyle]{theorem}
\declaretheorem[name=Assioma, sibling=theorem, style=thmstyle]{axiom}
\declaretheorem[name=Corollario, sibling=theorem, style=thmstyle]{corollary}
\declaretheorem[name=Proposizione, sibling=theorem, style=thmstyle]{proposition}
\declaretheorem[name=Lemma, sibling=theorem, style=thmstyle]{lemma}

\declaretheoremstyle[
    spaceabove=2\topsep, spacebelow=2\topsep,
    headindent=0pt,
    % headfont=\bfseries,
    % notefont=\bfseries, notebraces={ (}{)},
    headfont=\bfseries,
    notefont=\bfseries, notebraces={}{},
    bodyfont=\itshape\normalsize,
    headformat={\llap{\smash{\parbox[t]{1.1in}{\centering \NUMBER\\ \NAME}}} \NOTE},
    headpunct={},
    % qed={$\triangleright$},
    postheadspace=10pt
]{unnamedstyle}
\declaretheorem[name={\ignorespaces}, sibling=theorem, style=unnamedstyle]{unnamed}

\declaretheoremstyle[
    spaceabove=2\topsep, spacebelow=2\topsep,
    headindent=0pt,
    headfont=\bfseries,
    notefont=\normalfont\normalsize\bfseries, notebraces={}{.},
    bodyfont=\normalfont\normalsize,
    headformat={\llap{\smash{\parbox[t]{1.1in}{\centering \NAME\\ \NUMBER}}} \NOTE},
    headpunct={},
    % qed={$\triangleright$},
    postheadspace=10pt
]{defstyle}
\declaretheorem[name=Definizione, sibling=theorem, style=defstyle]{definition}

\declaretheoremstyle[
    headfont=\scshape,
    notefont=\normalfont, notebraces={ - }{.},
    bodyfont=\normalfont,
    postheadspace=1em
]{exmplstyle}
\declaretheorem[name=Esempio, sibling=theorem, style=exmplstyle]{example}
\declaretheorem[name=Esercizio, sibling=theorem, style=exmplstyle]{exercise}


\declaretheoremstyle[
    headfont=\scshape,
    notefont=\normalfont, notebraces={(}{)},
    bodyfont=\normalfont,
    numbered=no,
    postheadspace=1em
]{remarkstyle}
\declaretheorem[name=Osservazione, style=remarkstyle]{remark}
\declaretheorem[name=Soluzione, style=remarkstyle]{solution}
\declaretheorem[name=Intuizione, style=remarkstyle]{intuition}

\newcolumntype{z}{r<{{}}}
\newcolumntype{o}{@{}>{{}}c<{{}}@{}}

% \newenvironment{FuncDef}[1]
%     {  \begin{split} \begin{align} } 
%     {  \end{align} \end{split} }
% \newenvironment{FuncDefN}[1]{
%     \begin{split*}
%         \begin{align*}
%             #1
%         \end{align*}
%     \end{split*}
% }

% Set related symbols
\newcommand{\set}[1]{\left\{\;#1\;\right\}}
\newcommand{\union}{\cup}
\newcommand{\inters}{\cap}
\newcommand{\bigunion}{\bigcup}
\newcommand{\biginters}{\bigcap}
\newcommand{\disjunion}{\sqcup}
\newcommand{\bigdisjunion}{\bigsqcup}
\newcommand{\suchthat}{\,:\,} % oppure con {:}
\DeclareMathOperator{\tc}{\text{ tale che }}

\renewcommand{\epsilon}{\varepsilon}
\renewcommand{\theta}{\vartheta}
\renewcommand{\rho}{\varrho}
\renewcommand{\phi}{\varphi}

\DeclarePairedDelimiter{\braces}{[}{]}
\DeclarePairedDelimiter{\abs}{\lvert}{\rvert}
\DeclarePairedDelimiter{\norm}{\lVert}{\rVert}
% \DeclarePairedDelimiter{\ang}{\langle}{\rangle}
\DeclarePairedDelimiter{\cycl}{\langle}{\rangle}

\let\oldre\Re
\let\oldim\Im
\renewcommand{\Re}[1]{\operatorname{Re}(#1)}
\renewcommand{\Im}[1]{\operatorname{Im}(#1)}
\newcommand{\conj}[1]{\overline{#1}}

\newcommand{\deq}{:=}
\newcommand{\iseq}{\overset{?}{=}}
\newcommand{\seteq}{\overset{!}{=}}
\newcommand{\divides}{\mid} % divide esattamente
\newcommand{\ndivides}{\not\mid} % non divide esattamente
\newcommand{\congr}{\equiv} % congruo 
\newcommand{\ncongr}{\not\congr} % non congruo
% \newcommand{\isomorph}{\cong}
\newcommand{\isomorph}{\simeq}
\newcommand{\normal}{\trianglelefteqslant}
\newcommand{\grindex}[2]{\braces*{#1\;:\;#2}}

\newcommand{\Mod}[1]{\ \left(#1\right)}
\newcommand{\mcm}[2]{\left[#1, #2\right]}
\newcommand{\mcd}[2]{\left(#1, #2\right)}
\newcommand{\ord}[2][]{\operatorname{ord}_{#1}\!\left( #2 \right)}


% \renewcommand{\prime}{^\prime}
\newcommand{\inv}{^{-1}}
\newcommand{\Imm}[1]{\operatorname{Im} #1}
% \newcommand{\ang}[1]{\left\langle #1 \right\rangle}
\newcommand{\invertible}[1]{#1^{\times}}
\newcommand{\compl}[1]{#1^C}

\newcommand{\Mat}[2]{\operatornamewithlimits{Mat}_{#1 \times #1}\!(#2)}
\newcommand{\Hom}[2]{\operatorname{Hom}\left(#1, #2\right)}
\newcommand{\Aut}[1]{\operatorname{Aut}\left(#1\right)}
\DeclareMathOperator{\id}{id}

\NewDocumentCommand{\eqclass}{sm}{
    \IfBooleanTF{#1}{
        \left[C_{#2}\right]
    }{
        \overline{#2}
    }
}
\newcommand{\quot}[2]{{#1}/_{#2}}
% \newcommand{\quot}[2]{\faktor{#1}{#2}}
\newcommand{\N}{\mathbb{N}}
\newcommand{\Z}{\mathbb{Z}}
\newcommand{\Zmod}[1]{\Z/_{#1\Z}}
% \newcommand{\Zmod}[1]{\quot{\Z}{#1\Z}}
\newcommand{\Q}{\mathbb{Q}}
\newcommand{\R}{\mathbb{R}}
\newcommand{\C}{\mathbb{C}}
\newcommand{\K}{\mathbb{K}}


\newcommand{\HH}{\mathcal{H}}
\newcommand{\KK}{\mathcal{K}}
\renewcommand{\SS}{\mathcal{S}}
\newcommand{\PP}{\mathcal{P}}

\begin{document}

\author{Luca De Paulis}
\title{Appunti sui numeri complessi}
\maketitle

% \tableofcontents

\section{Numeri complessi}

Se consideriamo l'insieme dei numeri reali $\R$ e i polinomi a coefficienti reali $\R[x]$ notiamo che non tutti i polinomi sono fattorizzabili \emph{completamente}: alcuni polinomi di grado $2$, in particolare quelli con discriminante negativo, non ammettono fattorizzazione in polinomi di grado $1$. Lo scopo dei numeri complessi è quindi quello di permettere la risoluzione di equazioni di secondo grado con delta negativo.

La più semplice equazione di secondo grado senza soluzioni in $\R$ è \[
    x^2 + 1 = 0.  
\] Infatti essa è equivalente a $x^2 = -1$, e siccome il quadrato di ogni numero reale è non negativo, nessun $x \in \R$ può soddisfarla. Si introduce per questo l'\emph{unità immaginaria} $i$.

\begin{definition}
    [Unità immaginaria] \label{def:immaginary_unit}
    Si dice \emph{unità immaginaria} il numero $i$ tale che \begin{equation}
        i^2 \deq -1.
    \end{equation}
\end{definition}

\begin{definition}
    [Insieme dei numeri complessi]
    Si dice \emph{insieme dei numeri complessi} l'insieme $C$ tale che \begin{equation}
        \C \deq \set{a + ib \suchthat a, b \in \R, i^2 = -1}.
    \end{equation}
\end{definition}

Un numero complesso $z$ può dunque essere pensato come una coppia di numeri reali: il primo viene detto \emph{parte reale di $z$}, e lo si indica con $\Re z$; il secondo viene detto \emph{parte immaginaria di $z$}, e lo si indica con $\Im z$.

Questa rappresentazione ci consente di rappresentare i numeri complessi come se fossero vettori nel piano (che viene quindi detto \emph{piano complesso}): la parte reale di un numero complesso è l'ascissa, la parte immaginaria è l'ordinata.

\begin{center}\begin{tikzpicture}[>=latex]
    \draw[step=1cm,gray!25!,very thin] (-2,-2) grid (4,4);
    \draw[thick,->] (-1,0) -- (3,0) node[anchor=north west] {Re};
    \draw[thick,->] (0,-1) -- (0,3) node[anchor=south east] {Im};
    \draw[red,thick,->] (0,0) coordinate (O) -- (1, 2)
    node[midway,below] {$z$};
    \filldraw [black] (1,0) circle (1pt)
    node[below] {$a$};
    \filldraw [black] (0,2) circle (1pt)
    node[left] {$b$};
    \filldraw [black] (1,2) circle (1pt)
    node[right] {$(a, b)$};
\end{tikzpicture}\end{center}

Avendo rappresentato i numeri complessi come vettori, viene spontaneo definire una quantità che rappresenti la \emph{lunghezza} del vettore.
\begin{definition}
    [Modulo di un numero complesso] Sia $z = a + ib \in \C$. Si dice \emph{modulo} di $z$ il numero reale \begin{equation}
        \abs*{z} \deq \sqrt{a^2 + b^2}. 
    \end{equation}
\end{definition}

Il modulo di $z$ è la lunghezza del vettore associato a $z$ per il teorema di Pitagora: il vettore è l'ipotenusa di un triangolo rettangolo che ha un cateto lungo $a$ e un cateto lungo $b$. Per questo l'unico numero complesso che ha modulo $0$ è il numero $0 +i0$.

Ovviamente due numeri complessi $z, w \in \C$ sono uguali se e solo se hanno la stessa parte reale e la stessa parte immaginaria, ovvero se e solo se rappresentano lo stesso vettore nel piano complesso.

Possiamo inoltre definire due operazioni sui numeri complessi: una somma \begin{equation}
    (a + ib) + (c + id) = (a + c) + i(b + d) \in \C
\end{equation} e un prodotto \begin{equation}
    (a + ib) \cdot (c + id) = ac + iad + ibc + i^2bd = (ac - bd) + i(ad + bc).
\end{equation} Notiamo che $i^2bd = -bd$ per definizione dell'unità immaginaria.

La somma è definita come una qualunque somma tra vettori: nel piano complesso la somma si effettua con il metodo del parallelogramma, oppure sommando tra di loro le componenti (ovvero la parte reale e la parte immaginaria). Il prodotto non sembra avere un significato concreto; tuttavia nel seguito riusciremo a far vedere come i prodotti tra numeri complessi corrispondono a \emph{rotazioni} dei vettori nel piano.

Le due operazioni di somma e prodotto soddisfano la proprietà commutativa, la proprietà associativa e la proprietà distributiva della somma rispetto al prodotto. Notiamo inoltre che il numero complesso $0 + i0$ è elemento neutro rispetto alla somma: \[
    (a + ib) + (0 + i0) = (a + 0) + i(b + 0) = a + ib,    
\] e il numero complesso $1 + i0$ è l'elemento neutro del prodotto: \[
    (a + ib) \cdot (1 + i0) = (a\cdot 1 - b\cdot 0) + i(a\cdot 0 + b\cdot 1) = a + ib.    
\] Inoltre, ogni numero complesso ha un opposto: infatti dato $z = a+ib \in \C$ il suo opposto è dato da $-z = -a-ib$. Infatti \[
    z + (-z) = (a+ib) + (-a-ib) = 0 + i0,
\] cioè l'elemento neutro della somma.

Prima di vedere se ogni numero ammette un \emph{inverso moltiplicativo}, ovvero un reciproco, introduciamo una nuova operazione sui numeri complessi.

\begin{definition}
    [Coniugato complesso]
    Sia $z \in \C$ un numero complesso tale che $z = a+ib$ (con $a, b \in \R$). Allora si dice \emph{coniugato complesso} di $z$ il numero \begin{equation}
        \conj{z} = a - ib.
    \end{equation}
\end{definition}

Nel piano complesso il coniugato di un numero è il vettore ribaltato rispetto all'asse delle ascisse: \begin{center}
    \begin{tikzpicture}
        \draw[step=1cm,gray!25!,very thin] (-2,-3) grid (5,3);
        \draw[thick,->] (-1,0) -- (4,0) node[anchor=north west] {Re};
        \draw[thick,->] (0,-2) -- (0,2) node[anchor=south east] {Im};
        \draw[red,thick,->] (0,0) coordinate (O) -- (2, 1)
        node[midway,above] {$z$};
        \draw[blue,thick,->] (0,0) coordinate (O) -- (2, -1)
        node[midway,below] {$\conj z$};
        \filldraw [black] (2,0) circle (1pt)
        node[below] {$a$};
        \filldraw [black] (0,1) circle (1pt)
        node[left] {$b$};
        \filldraw [black] (0,-1) circle (1pt)
        node[left] {$-b$};
        \filldraw [black] (2,1) circle (1pt)
        node[right] {$(a, b)$};
        \filldraw [black] (2,-1) circle (1pt)
        node[right] {$(a, -b)$};
    \end{tikzpicture}
\end{center}

L'operazione di coniugio si comporta bene rispetto alla somma e al prodotto. Vale infatti la seguente proposizione.

\begin{proposition}\label{somma_prodotto_tra_coniugati}
    Siano $z, w \in \C$ tali che $z = a+ib$, $w = c + id$ (con $a, b, c, d \in \R$). Valgono le seguenti affermazioni.
     \begin{enumerate}[label={(\roman*)}]
        \item La somma dei coniugati è il coniugato della somma: \[
            \conj{z} + \conj{w} = \conj{z + w}.
        \]
        \item Il prodoto dei coniugati è il coniugato del prodoto: \[
            \conj{z}\cdot\conj{w} = \conj{zw}.
        \]
        \item $(\conj{z})^n = \conj{z^n}$.
    \end{enumerate}
\end{proposition}
\begin{proof}
    Dimostriamo i tre fatti separatamente.
    \begin{enumerate}[label={(\roman*)}]
        \item Per definizione di somma \begin{alignat*}
            {1}
            \conj{z} + \conj{w} &= (a-ib) + (c - id)\\
            &= (a+c) - i(b+d)\\
            &= \conj{z + w}.
        \end{alignat*}
        \item Per definizione di prodotto \begin{alignat*}
            {1}
            \conj{z}\cdot\conj{w} &= (a-ib)(c - id)\\
            &= (ac - bd) + i(-ad-bc)\\
            &= (ac - bd) - i(ad+bc)\\
            &= \conj{zw}.
        \end{alignat*}
        \item Dimostriamolo per induzione su $n$.
        \begin{description}
            \item[Caso base.] Se $n = 1$ allora banalmente $(\conj{z})^1 = \conj{z} = \conj{z^1}$.
            \item[Passo induttivo.] Supponiamo che la tesi valga per $n$ e dimostriamola per $n+1$. Allora \[
                (\conj{z})^{n+1} = (\conj{z})^{n} \cdot \conj{z} = \conj{z^n} \cdot \conj{z} = \conj{z^{n+1}}
            \] dove l'ultimo passaggio è giustificato dal punto precedente della dimostrazione. \qedhere
        \end{description}
    \end{enumerate}
\end{proof}

Possiamo studiare inoltre le relazioni che un numero complesso ha con il suo coniugato.

\begin{proposition}\label{somma_prodotto_col_coniugato}
    Sia $z = a+ib \in \C$ (con $a, b \in \R$). Allora valgono i seguenti fatti:
    \begin{enumerate}[label={(\roman*)}]
        \item La somma di un $z$ con il proprio coniugato è un numero reale, ed in particolare è il doppio della parte reale di $z$: \[
            z + \conj{z} = 2\Re{z}.
        \]
        \item Il prodotto di $z$ con il suo coniugato è un numero reale, ed in particolare è il modulo di $z$ al quadrato: \[
            z\conj{z} = \abs{z}^2.
        \]
    \end{enumerate}
\end{proposition}
\begin{proof}
    Dimostriamo i due fatti.
    \begin{enumerate}[label={(\roman*)}]
        \item Per definizione di somma vale che \[
           z + \conj{z} = (a + ib) + (a - ib) = 2a = 2\Re z.   
        \]
        \item Per definizione di prodotto vale che \begin{align*}
            z\conj{z} &= (a + ib)(a - ib)\\ 
            &= (a^2 - (-b^2)) + i(ab - ab) \\
            &= a^2 + b^2 \\
            &= \abs{z}^2. \qedhere
        \end{align*} 
    \end{enumerate}
\end{proof}

Dalla seconda relazione della proposizione precendente possiamo ricavare il reciproco del numero complesso $z = a+ib$:
\begin{equation}
    z \cdot \conj z = \abs*{z}^2 
    \iff \frac{1}{z} = \frac{\conj z}{\abs*{z}^2} = \frac{a}{a^2 + b^2} - i\frac{b}{a^2 + b^2} \in \C.
\end{equation}

Dunque ogni numero complesso diverso da $0 + i0$ (in quanto $\abs*{0}^2 = 0$) ha un inverso, calcolabile con la formula di sopra. L'insieme dei numeri complessi con le operazioni di somma e prodotto è quindi un \emph{campo}: \begin{itemize}
    \item valgono la proprietà commutativa e associativa per entrambe le operazioni;
    \item vale la proprietà distributiva;
    \item esiste un elemento neutro per la somma e ogni numero ammette un opposto;
    \item esiste un elemento neutro per il prodotto e ogni numero non nullo ammette un reciproco.
\end{itemize}

\section{I numeri reali come sottoinsieme dei complessi}

I numeri complessi con parte immaginaria nulla, ovvero della forma \[
    a + i0    
\] possono essere interpretati molto semplicemente come numeri reali veri e propri.
Infatti \begin{itemize}
    \item la somma di $z, w \in \C$ con $\Im z = \Im w = 0$ ha ancora parte immaginaria nulla, e corrisponde alla somma delle parti reali: \[
        z + w = (a + i0) + (b + i0) = (a + b) + i0.    
    \]
    \item il prodotto di $z, w \in \C$ con $\Im z = \Im w = 0$ ha ancora parte immaginaria nulla e corrisponde al prodotto delle parti reali: \[
        zw = (a + i0)(b + i0) = (ab + 0) + i0 = ab + i0.    
    \]
    \item il modulo di $z \in \C$ con $\Im z = 0$ corrisponde al valore assoluto della parte reale: \[
        \abs*{z} = \sqrt{a^2 + 0^2} = \sqrt{a^2} = \abs*{a}.    
    \]
    \item il coniugato di $z \in \C$ con $\Im z = 0$ è $z$ stesso: \[
        \conj z = a - i0 = a + i0 = z.    
    \]
    \item il reciproco di $z \in \C$ con $\Im z = 0$ corrisponde al reciproco della sua parte reale: \[
        \frac1z = \frac{a}{\abs*{z}^2} + i\frac{0}{\abs*{z}^2} = \frac{a}{a^2} + i0 = \frac1a + i0. 
    \]
\end{itemize}

Possiamo quindi identificare i numeri reali con il sottoinsieme dei numeri complessi con parte immaginaria nulla: graficamente, essi corrispondono all'asse delle ascisse.
\section{Forma polare}

Nella sezione precedente abbiamo visto come ad ogni numero complesso può essere associato un vettore nel piano complesso le cui coordinate corrispondono alla parte reale e alla parte immaginaria del numero complesso in esame.

I vettori nel piano possono però essere rappresentati anche da un altro punto di vista: ad ogni vettore può essere associata la sua lunghezza e l'angolo che il vettore forma con ĺ'asse delle ascisse: 
\begin{center}
    \begin{tikzpicture}
        \coordinate (a) at (1, 0);
        \draw[step=1cm,gray!25!,very thin] (-4,-3) grid (4,3);
        \draw[thick,->] (-3,0) -- (3,0) node[anchor=north west] {Re};
        \draw[thick,->] (0,-2) -- (0,2) node[anchor=south east] {Im};
        \draw[red,thick,->] (0,0) coordinate (O) -- (2, 1) coordinate (P)
        node[midway,above] {$\rho$};

        \draw pic[draw,angle radius=1cm,"$\theta$" shift={(6mm,1mm)}] {angle=a--O--P};
    \end{tikzpicture}
\end{center}

Per formalizzare questa associazione, consideriamo innanzitutto l'insieme dei numeri complessi con modulo uguale ad $1$. Per definizione di modulo, un numero complesso $z = a+ib$ ha modulo $1$ se e solo se \[
    \sqrt{a^2 + b^2} = 1 \iff a^2 + b^2 = 1.
\] I numeri di modulo unitario formano quindi una circonferenza di raggio $1$ con centro nell'origine degli assi:
\begin{center}
    \begin{tikzpicture}
        \coordinate (a) at (1, 0);
        \draw[step=1.5cm,gray!25!,very thin] (-4,-3) grid (4,3);
        \draw[thick,->] (-3,0) -- (3,0) node[anchor=north west] {Re};
        \draw[thick,->] (0,-3) -- (0,3) node[anchor=south east] {Im};
        % \draw[red,thick,->] (0,0) coordinate (O) -- (60:1) coordinate (P)
        % node[midway,above] {$1$};

        \draw[blue,thick,->] (0,0) coordinate (O) -- (135:1.5) coordinate (Q)
        node[midway,above] {$1$};

        \draw[color=black, very thick](0,0) circle (1.5);

        % \draw pic[draw,color=red,angle radius=0.5cm,"$\nicefrac{\pi}{3}$" shift={(6mm,1mm)}] {angle=a--O--P};
        \draw pic[draw,color=blue,angle radius=0.75cm,"$\frac{3\pi}{4}$" shift={(6mm,1mm)}] {angle=a--O--Q};
    \end{tikzpicture}
\end{center}

Ogni vettore di questa circonferenza è univocamente determinato dall'angolo che forma con l'asse delle ascisse: dato un angolo $\theta$, il vettore che corrisponde a $\theta$ avrà come coordinate $(\cos \theta, \sin \theta)$, dunque il corrispondente numero complesso sarà \[
    z = \cos \theta + i\sin\theta.    
\]

La prossima proposizione ci mostra come moltiplicare tra di loro numeri complessi di modulo unitario.
\begin{proposition}\label{prop:product_unitary}
    Siano $z, w \in \C$ tali che \[
        z = \cos \theta + i\sin\theta, \quad w = \cos \phi + i\sin\phi.
    \]
    Allora vale che \begin{equation}
        zw = \cos (\theta + \phi) + i\sin (\theta + \phi).
    \end{equation}
\end{proposition}
\begin{proof}
    \begin{align*}
        zw &= (\cos \theta + i\sin\theta)(\cos \phi + i\sin\phi)\\
        &= (\cos \theta\cos \phi - \sin\theta\sin\phi) + i(\cos\theta\sin\phi + \sin\theta\cos\phi)\\
        &= \cos(\theta + \phi) + i\sin(\theta+\phi). \qedhere
    \end{align*}
\end{proof}

Dunque molitplicare tra di loro due numeri complessi di angoli $\theta$ e $\phi$ e di modulo unitario ci restituisce un numero complesso di modulo unitario e di angolo $\theta + \phi$: equivale quindi a ruotare uno dei due vettori per l'angolo associato all'altro.

Consideriamo ora un vettore con modulo $\rho \geq 0$ qualunque. Tramite la trigonometria possiamo ricavare le sue coordinate:
\begin{center}
    \begin{tikzpicture}
        % \coordinate (a) at (1, 0);
        \draw[step=1cm,gray!25!,very thin] (-2,-2) grid (4,3);
        \draw[thick,->] (-1,0) -- (3,0) node[anchor=north west] {Re};
        \draw[thick,->] (0,-1) -- (0,2) node[anchor=south east] {Im};
        \draw[red,thick,->] (0,0) coordinate (O) -- (2, 1) coordinate (P)
        node[midway,above] {$\rho$};

        \draw[black,<->] (2,0) coordinate (A) -- (P)
        node[midway,right] {$\rho\sin\theta$};
        \draw[black,<->] (O) -- (A)
        node[midway,below] {$\rho\cos\theta$};

        \draw pic[draw,angle radius=1cm,"$\theta$" shift={(6mm,1mm)}] {angle=A--O--P};
    \end{tikzpicture}
\end{center}

Dunque un vettore di modulo $\rho$ e angolo $\theta$ ha come coordinate \[
    (\rho\cos\theta, \rho\sin\theta),
\] da cui segue che il corrispondente numero complesso è della forma \[
    z = \rho\cos\theta + \rho\sin\theta = \rho(\cos\theta + \sin\theta).
\]

Anche in questo caso moltiplicare due numeri complessi è particolarmente facile:
\begin{proposition} \label{prop:product_polar}
    Siano $z, w \in \C$ tali che \[
        z = r_1(\cos \theta + i\sin\theta), \quad w = r_2(\cos \phi + i\sin\phi).
    \]
    Allora vale che \begin{equation}
        zw = r_1r_2(\cos (\theta + \phi) + i\sin (\theta + \phi)).
    \end{equation}
\end{proposition}
\begin{proof}
    \begin{align*}
        zw &= r_1(\cos \theta + i\sin\theta) \cdot r_2(\cos \phi + i\sin\phi)\\
        &= r_1r_2 \cdot ((\cos \theta + i\sin\theta)(\cos \phi + i\sin\phi)) \tag{per la \autoref{prop:product_unitary}}\\
        &= r_1r_2(\cos (\theta + \phi) + i\sin (\theta + \phi)). \qedhere
    \end{align*}
\end{proof}

In questo caso il prodotto tra due numeri complessi corrisponde al vettore con \begin{itemize}
    \item modulo uguale al prodotto dei moduli,
    \item angolo dato dalla rotazione di uno dei due vettori per l'angolo definito dal secondo.
\end{itemize}

Possiamo quindi introdurre la \emph{forma polare} di un numero complesso.
\begin{definition}
    [Forma polare]
    Sia $z \in \C$ un numero complesso con modulo $\rho$ e angolo associato $\theta$. Si dice forma polare di $z$ la forma \begin{equation}
        z = \rho(\cos\theta +i\sin\theta) = \rho e^{i\theta}.
    \end{equation}
    L'angolo $\theta$ viene detto \emph{argomento} del numero complesso $z$ e lo si indica con $\arg z$.
\end{definition}

Per trasformare un numero complesso da una forma all'altra basta sfruttare un po' di trigonometria:
\paragraph{Dalla forma cartesiana alla polare} Consideriamo un numero $z = a+ib \in \C$ espresso in forma cartesiana. Per portarlo in forma polare dobbiamo trovare $\rho = \abs*{z}$ e $\arg z$.

Per definizione di modulo, $\abs*{z} = \sqrt{a^2 + b^2}$. Per trovare l'argomento basta fare l'arcotangente del rapporto tra i cateti, facendo attenzione al quadrante in cui ci troviamo:
\begin{equation}\label{eq:argument}
    \arg z = \begin{cases}
        \arctan \frac{b}{a}, &\text{se } a > 0 \\
        \arctan \frac{b}{a} + \pi, &\text{se } a < 0 \\
        \nicefrac{\pi}{2} &\text{se } a = 0, b > 0\\
        \nicefrac{3\pi}{2} &\text{se } a = 0, b < 0.
    \end{cases}    
\end{equation}
\paragraph{Dalla forma polare alla cartesiana} Se $z = \rho(\cos\theta +i\sin\theta)$ è un numero complesso in forma polare, per portarlo in forma cartesiana basta calcolare le due funzioni trigonometriche: \begin{align*}
    z = \rho\cos\theta + i\rho\sin\theta\\
    \implies a = \rho\cos\theta, b = \rho\sin\theta.
\end{align*}

\begin{example}
    Il numero $i = 0 + 1i$ ha come forma polare $e^{i\frac{\pi}{2}}$. Infatti: \begin{itemize}
        \item $\abs*{i} = \abs*{0 + 1i} = \sqrt{0^2 + 1^2} = 1$.
        \item $\arg i = \nicefrac{\pi}{2}$ poiché ci troviamo nel terzo caso della \eqref{eq:argument}.
    \end{itemize}

    Ciò è evidente anche disegnando il numero $i$ nel piano complesso:
    \begin{center}
        \begin{tikzpicture}
            \coordinate (a) at (1, 0);
            \draw[step=1cm,gray!25!,very thin] (-3,-3) grid (3,3);
            \draw[thick,->] (-2,0) -- (2,0) node[anchor=north west] {Re};
            \draw[thick,->] (0,-2) -- (0,2) node[anchor=south east] {Im};
            \draw[red,thick,->] (0,0) coordinate (O) -- (0, 1) coordinate (i)
            node[midway,left] {$\abs*{i} = 1$};
    
            \draw pic[draw,angle radius=0.5cm,"$\frac{\pi}{2}$" shift={(3mm,3mm)}] {angle=a--O--i};

            \filldraw [black] (i) circle (1pt)
            node[above right] {$i = 0 + 1i$};
        \end{tikzpicture}
    \end{center}
\end{example}

La forma "esponenziale", data da $z = \rho e^{i\theta}$ è comoda poiché più sintetica della forma con le funzioni trigonometriche. Inoltre, essa continua a rispettare la \autoref{prop:product_polar}: \[
    r_1e^{i\theta} \cdot r_2e^{i\phi} = r_1r_2 e^{i(\theta + \phi)}.    
\]

Prima di studiare le potenze e le radici $n$-esime nei complessi,facciamo alcune osservazioni finali.
\begin{remark}
    I numeri complessi di modulo unitario sono tutti e soli della forma $z = e^{i\theta}$, in quanto il loro modulo è uguale ad $1$.
\end{remark}
\begin{remark}
    Il coniugato in forma polare di $\rho e^{i\theta}$ è il numero $\rho e^{-i\theta}$, dunque la forma polare è comoda anche per calcolare i coniugati di numeri complessi.
\end{remark} 
\begin{remark}
    I numeri reali, essendo tutti sull'asse delle ascisse, hanno argomento $0$ (se sono positivi) oppure $\pi$ (se sono negativi): dunque i numeri reali sono tutti e solo delle forme $\rho e^{i0} = rho$ oppure $\rho e^{i\pi} = -\rho$.
\end{remark}
\begin{remark}
    Due numeri complessi in forma polare sono uguali se e solo se \begin{itemize}
        \item i loro moduli sono uguali,
        \item i loro argomenti sono uguali, a meno di un multiplo intero di $2\pi$.
    \end{itemize}
    Infatti gli angoli $\theta$ e $\theta + 2k\pi$ sono uguali per ogni $k \in \Z$, dunque è necessario considerare che gli argomenti non sono necessariamente in $[0, 2\pi)$.
\end{remark}
\section{Potenze e radici complesse}

Per risolvere equazioni nel campo dei complessi (o equivalentemente per fattorizzare polinomi in $\C[x]$) è necessario saper calcolare potenze di numeri complessi e radici $n$-esime.

La forma cartesiana non è particolarmente di aiuto in questo caso: calcolare le potenze è difficile in quanto dovremmo ricorrere costantemente a prodotti tra binomi della forma $a+ib$, mentre calcolare le radici è impossibile a causa della somma tra parte reale e immaginaria.

La forma polare risulta invece molto più comoda, come ci garantisce la seguente proposizione.
\begin{proposition}\label{prop:power_complex}
    Sia $z = \rho e^{i\theta}$ un numero complesso. Allora la sua potenza $n$-esima è \begin{equation}
        z^n = \rho^n e^{in\theta}.
    \end{equation}
\end{proposition}
\begin{proof}
    Lo mostriamo per induzione su $n$.
    \begin{description}
        \item[Caso base] Se $n = 1$ allora \[
            z^1 = (\rho e^{i\theta})^1 = \rho^1 e^{1 \cdot 1\theta}.    
        \] 
        \item[Passo induttivo] Supponiamo la formula valga per $k$ e dimostriamola per $k + 1$.
        \begin{align*}
            z^{k+1} &= z^k \cdot z \tag{per hp. induttiva}\\
            &= \rho^k e^{ik\theta} \cdot \rho e^{i\theta} \tag{per la \autoref{prop:product_polar}}\\
            &= (\rho^k \rho) e^{i(k\theta+\theta)} \\
            &= \rho^{k+1} e^{i(k+1)\theta}.
        \end{align*} 
    \end{description}
    Dunque la formula è vera per ogni valore di $n$, come volevasi dimostrare.
\end{proof}

La potenza $n$-esima di un numero complesso di modulo unitario (diciamo $z = e^{i\theta}$) corrisponde alla rotazione del vettore corrispondente fino ad arrivare al vettore di angolo $n\theta$: equivale infatti a moltiplicare il vettore per se stesso $n$ volte, e ognuna di queste moltiplicazioni ruota il vettore di un angolo di $\theta$ radianti (come abbiamo osservato nella sezione precedente).

Il problema di trovare la radice $n$-esima di un numero è completamente riconducibile al problema di calcolare potenze di numeri complessi. Supponiamo di voler calcolare la radice $n$-esima di un numero complesso $w  \in \C$ dato, ovvero vogliamo trovare $z \in \C$ tale che \begin{equation}
    z = \sqrt[n]{w}.
\end{equation} Riformulando il problema, vogliamo trovare $z \in \C$ tale che \begin{equation}
    z^n = w.
\end{equation}

\paragraph{Caso $w = 1$} Iniziamo studiando un caso più semplice: consideriamo il numero complesso $w = 1$. Nel campo dei numeri reali il numero $1$ ha sempre una e una sola radice, ovvero se stesso. Nei numeri complessi, come vedremo, il numero $1$ ha più di una radice $n$-esima; in particolare, ne ha esattamente $n$.

Sia $z \in \C$ una radice $n$-esima di $1$ ($z = \sqrt[n]{1}$), ovvero equivalentemente sia $z \in \C$ tale che $z^n = 1$. Siccome stiamo calcolando potenze di numeri complessi scriviamo ogni numero in forma polare: il numero $1$ è esprimibile come $1\cdot e^{i0}$; siano inoltre $\rho$ e $\theta$ rispettivamente il modulo e l'argomento di $z$, cosicché $z = \rho e^{i\theta}$. Per la \autoref{prop:power_complex} segue quindi che $z^n = \rho^n e^{in\theta}$.

Come abbiamo osservato alla fine della sezione precedente, due numeri complessi in forma polare sono uguali se e solo se \begin{itemize}
    \item hanno lo stesso modulo;
    \item i loro argomenti differiscono per un multiplo di $2\pi$.
\end{itemize} Applicando questo ragionamento all'equazione $z^n = 1$ (che è l'equazione che definisce $z$) otteniamo due condizioni che possiamo sfruttare per calcolare $z$:
\begin{itemize}
    \item I moduli devono essere uguali, dunque segue che $\rho^n = 1$. Dato che $\rho$ è un numero reale, questa equazione ha una e una sola soluzione: $\rho = 1$.
    \item Gli argomenti devono differire per un multiplo di $2\pi$, ovvero $\arg z - \arg 1 = n\theta - 0 = 2k\pi$ per qualche $k \in \Z$.
\end{itemize} Dal primo vincolo otteniamo $\rho = 1$, mentre dal secondo ricaviamo che $\theta$ deve essere della forma $\dfrac{2k\pi}{n}$, al variare di $k \in \Z$.

Anche se l'equazione sembra avere infinite soluzioni (una per ogni valore di $k$ intero), in realtà le soluzioni distinte sono solo $n$, e si ottengono scegliendo $k = 0, \dots, n-1$. Infatti scegliendo $k = n$ l'argomento di $z$ diventa $\nicefrac{(2n\pi)}{n} = 2\pi = 0$ (poiché gli argomenti rappresentano angoli in radianti), che è lo stesso argomento ottenuto scegliendo $k = 0$.

Le possibili soluzioni dell'equazione $z^n = 1$, e quindi dell'equazione $z = \sqrt[n]{1}$, sono dunque della forma \[
    1\cdot e^{i\frac{2k\pi}{n}}
\] con $k = 0, \dots, n-1$. Questi numeri vengono detti \emph{radici $n$-esime dell'unità} e vengono spesso chiamati $\mu_0, \dots, \mu_{n-1}$.

Rappresentando ad esempio le radici seste dell'unità possiamo notare immediatamente alcune caratteristiche interessanti:
\begin{center}
    \begin{tikzpicture}[>=latex]
        \coordinate (mu0) at (1.5, 0);
        \draw[step=1.5cm,gray!25!,very thin] (-4,-3) grid (4,3);
        \draw[thin,->] (-3,0) -- (3,0) node[anchor=north west] {Re};
        \draw[thin,->] (0,-3) -- (0,3) node[anchor=south east] {Im};
        \draw[black,thick,->] (0,0) coordinate (O) -- (mu0);
        \draw[black,thick,->] (O) -- (60:1.5) coordinate (mu1);
        \draw[black,thick,->] (O) -- (120:1.5) coordinate (mu2);
        \draw[black,thick,->] (O) -- (180:1.5) coordinate (mu3);
        \draw[black,thick,->] (O) -- (240:1.5) coordinate (mu4);
        \draw[black,thick,->] (O) -- (300:1.5) coordinate (mu5);
        \filldraw [red,very thick] (mu0) circle (1pt)
        node[black,above right,fill=white] {$\mu_0$};
        \filldraw [red,very thick] (mu1) circle (1pt)
        node[black,above right,fill=white] {$\mu_1$};
        \filldraw [red,very thick] (mu2) circle (1pt)
        node[black,above left,fill=white] {$\mu_2$};
        \filldraw [red,very thick] (mu3) circle (1pt)
        node[black,above left,fill=white] {$\mu_3$};
        \filldraw [red,very thick] (mu4) circle (1pt)
        node[black,below left,fill=white] {$\mu_4$};
        \filldraw [red,very thick] (mu5) circle (1pt)
        node[black,below right,fill=white] {$\mu_5$};

        \draw[black,thick] (mu5) -- (mu0);
        \draw[black,thick] (mu0) -- (mu1);
        \draw[black,thick] (mu1) -- (mu2);
        \draw[black,thick] (mu2) -- (mu3);
        \draw[black,thick] (mu3) -- (mu4);
        \draw[black,thick] (mu4) -- (mu5);

        \draw[color=black, very thin,dashed](0,0) circle (1.5);
    \end{tikzpicture}
\end{center}

Siccome l'angolo tra due radici consecutive è sempre di $\frac{\pi}3$ i vertici dei vettori corrispondenti alle sei radici delle unità formano un esagono regolare, inscritto nella circonferenza unitaria: in generale le radici $n$-esimi dell'unità formano un $n$-agono regolare inscritto nella circonferenza unitaria, e $1$ è sempre un vertice di questo $n$-agono.

Inoltre le radici "non-reali" (come ad esempio $\mu_1$, $\mu_2$, $\mu_4$ e $\mu_5$) sono complesse coniugate a coppie, come si vede evidentemente dal disegno nel caso $n = 6$.

\end{document}
\documentclass[italian,oneside,headinclude,10pt]{scrbook}
    \usepackage[utf8]{inputenc}
    \usepackage[italian]{babel}
    \usepackage[T1]{fontenc}
    \usepackage{textcomp, microtype}
    \usepackage{amsmath, amsthm, amssymb, cases, mathtools, bm}
    \usepackage{array, multicol}
    \usepackage{centernot}
    \usepackage{faktor}

    \usepackage{enumerate}
    \usepackage{float}
    \usepackage{letltxmacro}

    \LetLtxMacro\amsproof\proof
    \LetLtxMacro\amsendproof\endproof

    
    \usepackage[pdfspacing, eulermath]{classicthesis}
    % \usepackage{lmodern}
    
    \usepackage{thmtools}
    \usepackage[framemethod=TikZ]{mdframed}
    \usepackage{hyperref} % ultimo package da caricare!

\restylefloat{table}

\AtBeginDocument{
    \LetLtxMacro\proof\amsproof
    \LetLtxMacro\endproof\amsendproof
}

\titleformat*{\chapter}{\LARGE\scshape}
\titleformat*{\section}{\Large\scshape}

\renewcommand*{\proofname}{\textsc{Dimostrazione}}
\addto\italian{\renewcommand*{\proofname}{\textsc{Dimostrazione}}}

\declaretheoremstyle[
    spaceabove=6pt, spacebelow=6pt,
    % headindent=6pt,
    headfont=\bfseries,
    notefont=\bfseries, notebraces={ (}{)},
    % headfont=\scshape,
    % notefont=\scshape, notebraces={ (}{)},
    bodyfont=\itshape\normalsize,
    shaded={rulecolor={rgb}{255,255,255}},
    % mdframed={backgroundcolor=halfgray},
    headpunct={\vspace{0.5\topsep}\newline}
]{thmstyle}
\declaretheorem[name=, numberwithin=section, style=thmstyle]{principle}
\declaretheorem[name=Teorema, numberwithin=section, style=thmstyle]{theorem}
\declaretheorem[name=Corollario, sibling=theorem, style=thmstyle]{corollary}
\declaretheorem[name=Proposizione, sibling=theorem, style=thmstyle]{proposition}
\declaretheorem[name=Lemma, sibling=theorem, style=thmstyle]{lemma}

\declaretheoremstyle[
    spaceabove=0pt, spacebelow=0pt,
    headfont=\bfseries,
    notefont=\normalfont, notebraces={ (}{)},
    % headfont=\scshape,
    % notefont=\scshape, notebraces={(}{)},
    bodyfont=\normalfont\normalsize,
    headpunct={\vspace{0.5\topsep}\newline},
    mdframed={
        linecolor=halfgray,
        linewidth=1pt,
        backgroundcolor=white,
        topline=false,
        bottomline=false,
        rightline=false
    },
]{defstyle}
\declaretheorem[name=Definizione, sibling=theorem, style=defstyle]{definition}

\declaretheoremstyle[
    headfont=\bfseries,
    notefont=\normalfont, notebraces={ (}{)},
    bodyfont=\normalfont,
    postheadspace=1em
]{exmplstyle}
\declaretheorem[name=Esempio, sibling=theorem, style=exmplstyle]{example}
\declaretheorem[name=Esercizio, sibling=theorem, style=exmplstyle]{exercise}


\declaretheoremstyle[
    headfont=\scshape,
    notefont=\normalfont, notebraces={(}{)},
    bodyfont=\normalfont,
    numbered=no,
    postheadspace=1em
]{remarkstyle}
\declaretheorem[name=Osservazione, style=remarkstyle]{remark}
\declaretheorem[name=Soluzione, style=remarkstyle]{solution}
\declaretheorem[name=Intuizione, style=remarkstyle]{intuition}

\newcolumntype{z}{r<{{}}}
\newcolumntype{o}{@{}>{{}}c<{{}}@{}}

% Set related symbols
\newcommand{\set}[1]{\left\{\;#1\;\right\}}
\newcommand{\union}{\cup}
\newcommand{\inters}{\cap}
\newcommand{\suchthat}{\,\mid\,} % oppure con {:}
\DeclareMathOperator{\tc}{\text{ tale che }}

\renewcommand{\epsilon}{\varepsilon}
\renewcommand{\theta}{\vartheta}
\renewcommand{\rho}{\varrho}
\renewcommand{\phi}{\varphi}

\let\oldre\Re
\let\oldim\Im
\renewcommand{\Re}[1]{\operatorname{Re}(#1)}
\renewcommand{\Im}[1]{\operatorname{Im}(#1)}
\newcommand{\conj}[1]{\overline{#1}}

\newcommand{\deq}{:=}
\newcommand{\iseq}{\overset{?}{=}}
\newcommand{\seteq}{\overset{!}{=}}
\newcommand{\divides}{\mid} % divide esattamente
\newcommand{\ndivides}{\centernot\mid} % non divide esattamente
\newcommand{\congr}{\equiv} % congruo 
\newcommand{\ncongr}{\not\congr} % non congruo

\newcommand{\Mod}[1]{\ \left(#1\right)}
\newcommand{\mcm}[2]{\operatorname{mcm}\!\left(#1, #2\right)}
\newcommand{\mcd}[2]{\operatorname{mcd}\!\left(#1, #2\right)}
\newcommand{\ord}[2]{\operatorname{ord}_{#2}\!\left( #1 \right)}

\DeclarePairedDelimiter{\abs}{\lvert}{\rvert}
\DeclarePairedDelimiter{\norm}{\lVert}{\rVert}
\DeclarePairedDelimiter{\ang}{\langle}{\rangle}

\newcommand{^{-1}}{^{-1}}
\newcommand{\Imm}[1]{\operatorname{Im} #1}
% \newcommand{\ang}[1]{\left\langle #1 \right\rangle}
\newcommand{\units}[1]{(#1)^{\times}}
\newcommand{\compl}[1]{#1^C}

\newcommand{\N}{\mathbb{N}}
\newcommand{\Z}{\mathbb{Z}}
\newcommand{\Zmod}[1]{\faktor{\Z}{/#1\Z}}
\newcommand{\Q}{\mathbb{Q}}
\newcommand{\R}{\mathbb{R}}
\newcommand{\C}{\mathbb{C}}
\newcommand{\K}{\mathbb{K}}


\begin{document}

\author{Luca De Paulis}
\title{Esercizi di Matematica Discreta}
\maketitle

\tableofcontents

\chapter{Congruenze lineari}

\section{Teoremi e definizioni utili}

\begin{definition}
    [Congruenza lineare]

    Siano $a, b, n \in \Z$, $n > 0$. Allora la congruenza \[
        ax \congr b \Mod n    
    \] si dice congruenza lineare.
\end{definition}

\begin{proposition}
    [Condizione necessaria e sufficiente per la risolubilita']
    \label{cond_risolubilita'}
    Siano $a, b, n \in \Z$, $n > 0$. Allora la congruenza \[
        ax \congr b \Mod n    
    \] ha soluzione se e solo se $\mcd{a}{n} \divides b$.
\end{proposition}

\begin{definition}[Invertibilità e inverso]
    Siano $a \in \Z$.
    
    Allora si dice che $a$ è invertibile modulo $n$ se esiste  $y \in \Z$ tale che \[
        ay \congr 1 \Mod{n}
    .\]

    In particolare tra tutti gli $y$ che soddisfano la relazione precedente, il numero $y$ tale che $0 \leq y < n$ si dice inverso di $a$ modulo $n$.
\end{definition}

\begin{proposition}
    [Condizione necessaria e sufficiente per l'invertibilità]
    Siano $a, n \in \Z$, $n > 0$. 
    
    Allora $a$ è invertibile modulo $n$ se e solo se $\mcd{a}{n} \divides 1$.
\end{proposition}

\begin{proposition}[Risoluzione di una congruenza lineare] \label{risoluzione_congr_lineare}
    Siano $a, b, n \in \Z$, $n > 0$. 
    
    Allora per risolvere l'equazione $ax \congr b \Mod{n}$ possiamo ricondurci ad uno dei seguenti tre casi:
    \begin{description}
        \item[$(1):\ \mcd{a}{n} = 1$.] L'equazione ha soluzione \[
            x \congr by \Mod{n},   
        \] dove $y$ è l'inverso di $a$ modulo $n$;
        \item[$(2):\ \mcd{a}{n} \divides b$.]L'equazione è equivalente all'equazione \[
            \frac{a}{d} x \congr \frac{b}{d} \Mod{\frac{n}{d}};
        \]
        \item[$(3):\ \mcd{a}{n} \ndivides b$.] L'equazione non ha soluzione.
    \end{description}
\end{proposition}

\section{Esercizi}

\begin{example}
    Proviamo a risolvere la congruenza \[
        57x \congr 81 \Mod{ 21}.
    \]

    Innanzitutto notiamo che il coefficiente della $x$ e il termine noto sono maggiori del modulo, dunque possiamo semplificarli: \begin{align*}
        &57 = 2\cdot 21 + 15 \implies 57 \congr 15 \Mod{ 21}\\
        &81 = 3\cdot 21 + 18 \implies 57 \congr 18 \Mod{ 21}
    \end{align*}

    La congruenza diventa quindi \[
        15x \congr 18 \Mod{ 21}.    
    \]
    Notiamo che $\mcd{15}{21} = 3 \divides 18$, quindi la congruenza ha soluzione (per la proposizione \ref{cond_risolubilita'}) e possiamo applicare la regola $(ii)$ della proposizione \label{risoluzione_congr_lineare}, ottenendo \[
        5x \congr 6 \Mod 7.    
    \]

    A questo punto $\mcd{5}{7} = 1$, dunque $5$ e' invertibile modulo $7$ e possiamo trovare l'inverso a tentativi (dato che $7$ e' piccolo).
    \begin{align*}
        &5 \cdot 1 \congr 5 \ncongr 1 \Mod 7\\
        &5 \cdot 2 \congr 10 \congr 3 \ncongr 1 \Mod 7\\
        &5 \cdot 3 \congr 15 \congr 1 \Mod 7
    \end{align*}
    dunque $3$ e' l'inverso di $5$ modulo $7$. 
    
    Moltiplicando entrambi i membri dell'equazione per $3$ otteniamo quindi \begin{align*}
        &3 \cdot 5x \congr 6 \cdot 3 \Mod 7\\
        \iff & x \congr 18 \congr 4 \Mod 7.
    \end{align*}

    La soluzione della congruenza e' quindi $x \congr 4 \Mod 7$.
\end{example}


\end{document}
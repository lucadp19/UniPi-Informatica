\chapter{Congruenze lineari}

\section{Teoremi e definizioni utili}

\begin{definition}
    [Congruenza lineare]

    Siano $a, b, n \in \Z$, $n > 0$. Allora la congruenza \[
        ax \congr b \Mod n    
    \] si dice congruenza lineare.
\end{definition}

\begin{proposition}
    [Condizione necessaria e sufficiente per la risolubilita']
    \label{cond_risolubilita'}
    Siano $a, b, n \in \Z$, $n > 0$. Allora la congruenza \[
        ax \congr b \Mod n    
    \] ha soluzione se e solo se $\mcd{a}{n} \divides b$.
\end{proposition}

\begin{definition}[Invertibilità e inverso]
    Siano $a \in \Z$.
    
    Allora si dice che $a$ è invertibile modulo $n$ se esiste  $y \in \Z$ tale che \[
        ay \congr 1 \Mod{n}
    .\]

    In particolare tra tutti gli $y$ che soddisfano la relazione precedente, il numero $y$ tale che $0 \leq y < n$ si dice inverso di $a$ modulo $n$.
\end{definition}

\begin{proposition}
    [Condizione necessaria e sufficiente per l'invertibilità]
    Siano $a, n \in \Z$, $n > 0$. 
    
    Allora $a$ è invertibile modulo $n$ se e solo se $\mcd{a}{n} \divides 1$.
\end{proposition}

\begin{proposition}[Risoluzione di una congruenza lineare] \label{risoluzione_congr_lineare}
    Siano $a, b, n \in \Z$, $n > 0$. 
    
    Allora per risolvere l'equazione $ax \congr b \Mod{n}$ possiamo ricondurci ad uno dei seguenti tre casi:
    \begin{description}
        \item[$(1):\ \mcd{a}{n} = 1$.] L'equazione ha soluzione \[
            x \congr by \Mod{n},   
        \] dove $y$ è l'inverso di $a$ modulo $n$;
        \item[$(2):\ \mcd{a}{n} \divides b$.]L'equazione è equivalente all'equazione \[
            \frac{a}{d} x \congr \frac{b}{d} \Mod{\frac{n}{d}};
        \]
        \item[$(3):\ \mcd{a}{n} \ndivides b$.] L'equazione non ha soluzione.
    \end{description}
\end{proposition}

\section{Esercizi}

\begin{example}
    Proviamo a risolvere la congruenza \[
        57x \congr 81 \Mod{ 21}.
    \]

    Innanzitutto notiamo che il coefficiente della $x$ e il termine noto sono maggiori del modulo, dunque possiamo semplificarli: \begin{align*}
        &57 = 2\cdot 21 + 15 \implies 57 \congr 15 \Mod{ 21}\\
        &81 = 3\cdot 21 + 18 \implies 57 \congr 18 \Mod{ 21}
    \end{align*}

    La congruenza diventa quindi \[
        15x \congr 18 \Mod{ 21}.    
    \]
    Notiamo che $\mcd{15}{21} = 3 \divides 18$, quindi la congruenza ha soluzione (per la proposizione \ref{cond_risolubilita'}) e possiamo applicare la regola $(ii)$ della proposizione \label{risoluzione_congr_lineare}, ottenendo \[
        5x \congr 6 \Mod 7.    
    \]

    A questo punto $\mcd{5}{7} = 1$, dunque $5$ e' invertibile modulo $7$ e possiamo trovare l'inverso a tentativi (dato che $7$ e' piccolo).
    \begin{align*}
        &5 \cdot 1 \congr 5 \ncongr 1 \Mod 7\\
        &5 \cdot 2 \congr 10 \congr 3 \ncongr 1 \Mod 7\\
        &5 \cdot 3 \congr 15 \congr 1 \Mod 7
    \end{align*}
    dunque $3$ e' l'inverso di $5$ modulo $7$. 
    
    Moltiplicando entrambi i membri dell'equazione per $3$ otteniamo quindi \begin{align*}
        &3 \cdot 5x \congr 6 \cdot 3 \Mod 7\\
        \iff & x \congr 18 \congr 4 \Mod 7.
    \end{align*}

    La soluzione della congruenza e' quindi $x \congr 4 \Mod 7$.
\end{example}
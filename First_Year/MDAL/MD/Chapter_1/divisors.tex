\chapter{Divisori e gcd}

\section{Divisori di un numero}

Introduciamo ora formalmente i concetti intuitivi di multiplo e divisore.

\begin{definition}[Divisore]
    Siano $a, b \in \Z$. Si dice che $a$ divide $b$ se esiste $k \in \Z$ tale che $ak = b$. In tal caso si scrive $a \divides b$.
\end{definition}

\begin{definition}[Multiplo]
    Siano $a, b \in \Z$. Allora si dice che $b$ è multiplo di $a$ se esiste $k \in \Z$ tale che $b = ak$.
\end{definition}

\begin{remark}
    Ovviamente definizione di multiplo è speculare a quella di divisore: se $a$ è divisore di $b$ allora $b$ è multiplo di $a$.
\end{remark}

La relazione di divisibilità si comporta bene rispetto alla somma e al prodotto. 

\begin{proposition} \label{divides_sum_subtr_mult}
    Siano $a, b, n \in \Z$ tali che $n \divides a$ e $n \divides b$. Allora valgono le seguenti affermazioni:
    \begin{enumerate}
        \item $n \divides a + b$ \label{divides_sum}
        \item $n \divides a - b$ \label{divides_subtr}
        \item per ogni $x \in \Z$ vale che $n \divides ax$ \label{divides_mult}
    \end{enumerate} 
\end{proposition}
\begin{proof}  
    Per ipotesi, dato che $n \divides a$ e $n \divides b$, allora $\exists h, k \in \Z$ tali che
    $nh = a$ e $nk = b$. Dunque:
    \begin{alignat*}{2}
        a + b = nh + nk = n(h + k) &\iff n \divides a + b \\
        a - b = nh - nk = n(h - k) &\iff n \divides a - b \\
        ax = nhx = n(hx) &\iff n \divides ax
    \end{alignat*}
    che è la tesi.
\end{proof}

\begin{definition}[Massimo comun divisore]
    Siano $a, b \in \Z$; allora si dice $\gcd{a, b}$ il più grande intero positivo
    tale che \[
        \gcd{a, b} \divides a \qquad \text{e} \qquad \gcd{a, b} \divides b.
    \]
\end{definition}

\begin{definition}[Minimo comune multiplo]
    Siano $a, b \in \Z$. Allora si dice minimo comune multiplo di $a$ e $b$ il numero $d = \lcm{a, b}$ tale che $d$ è il più piccolo multiplo positivo sia di $a$ che di $b$.
\end{definition}

\begin{definition}[Coprimo]
    Siano $a, b \in \Z$. Se $\gcd{a, b} = 1$ allora $a$ e $b$ si dicono coprimi.
\end{definition}

\begin{remark}
    Siano $a, b \in \Z$. Allora valgono le seguenti proprietà per $\gcd{a, b}$:
    \begin{enumerate}
        \item $\gcd{a, b} = \gcd{\pm a, \pm b}$
        \item $\gcd{a, 1} = \gcd{1, a} = 1$
        \item $\gcd{a, 0} = \gcd{0, a} = 0$
        \item $\gcd{0, 0}$ non esiste.
    \end{enumerate}
\end{remark}

Il prossimo teorema ci fornisce la strategia della \emph{divisione euclidea tra interi}.

\begin{theorem}[Esistenza e unicità del resto] \label{esistenza_resto}
    Siano $a, b \in \Z$, con $b \neq 0$. Allora esistono e sono unici $q, r \in \Z$ tali che
    \begin{align}
        a = bq + r, \qquad0 \leq r < \abs{b}
    \end{align}
    Tale $r$ si dice resto della divisione di $a$ per $b$, e si indica anche con $r = a\bmod b$.
\end{theorem}
\begin{proof}
    Notiamo che i numeri della forma $a - bq$ formano una progressione aritmetica di passo $b$ al variare di $q \in \Z$.
    Il resto $r$ definito in questo modo è l'unico elemento di questa progressione compreso tra $0$ e $b - 1$.
\end{proof}

\begin{proposition}\label{mcm|c_iff_a,b|c}
    Siano $a, b, c \in \Z$. Allora \begin{equation}
        \lcm{a, b} \divides c \iff a \divides c \;\land\; b \divides c
    \end{equation}
\end{proposition}
\begin{proof}
    Dimostriamo separatamente i due versi dell'implicazione.
    \begin{itemize}
        \item[($\implies$)] Dato che $\lcm{a, b}$ è un multiplo di $a$ e di $b$ e per ipotesi $c$ è un multiplo di $\lcm{a, b}$, allora per transitività segue che $c$ è un multiplo di $a$ e di $b$.
        \item[($\impliedby$)] Supponiamo che $c$ sia un multiplo di $a$ e di $b$. Allora per il \autoref{esistenza_resto} esistono $q, r \in \Z$ tali che \[
            c = \lcm{a, b}q + r
        \]
        con $0 \leq r < \lcm{a, b}$.

        Dato che $a$, $b$ dividono sia $c$ (per ipotesi) che $\lcm{a, b}$ (per definizione di mcm), allora segue che essi dividono anche $r$. 

        Ma $0 \leq r < \lcm{a, b}$, dunque necessariamente $r = 0$, cioè $c = \lcm{a, b}q$ e quindi $\lcm{a, b} \divides c$. \qedhere
    \end{itemize}
\end{proof}
\chapter{Sottospazi vettoriali}

In questo capitolo vogliamo scoprire come verificare se un dato sottoinsieme di uno spazio vettoriale e' un sottospazio vettoriale.

\section{Teoremi e definizioni utili}

\begin{definition}[Sottospazio vettoriale]
    Sia $V$ uno spazio vettoriale, $A \subseteq V$. Allora si dice che $A$ e' un sottospazio vettoriale di $V$ (o semplicemente sottospazio) se
    \begin{align}
        &\vec{0_V} \in A \label{0_in_A}\\
        &(\vec{v} + \vec{w}) \in A    &&\forall \vec{v}, \vec{w} \in A \label{somma_in_A}\\
        &(k\vec{v}) \in A            &&\forall k \in \R, \vec{v} \in A \label{prodotto_in_A}
    \end{align}
\end{definition}

\section{Verifica}

\begin{example}
    Sia $S \subseteq \R^3$ tale che \[
        S = \set{ \begin{pmatrix}
            x \\ y \\ z
        \end{pmatrix} \in \R^3 \suchthat x -2y + 3z = 0}.
    \]

    Per verificare se $S$ e' un sottospazio di $\R^3$ e' sufficiente verificare che $S$ rispetti le tre condizioni di sopra:
    \begin{description}
        \item[($0 \in S$)] Verifichiamo che il vettore $\vec{0_{\R^3}} = (0, 0, 0)$ appartenga ad $S$, ovvero soddisfi la condizione $x -2y + 3z = 0$: \[
            0 -2\cdot 0 + 3 \cdot 0 = 0 + 0 + 0 = 0
        \] La condizione quindi e' verificata e $0_V \in S$.
        \item[($\vec v + \vec w \in S$)] Verifichiamo che se \[
            \vec v = \begin{pmatrix}
                v_1 \\ v_2 \\ v_3
            \end{pmatrix}, \quad \vec w = \begin{pmatrix}
                w_1 \\ w_2 \\ w_3
            \end{pmatrix} 
        \] appartengono ad $S$, cioe' \[
            v_1 - 2v_2 + 3v_3 = 0, \quad w_1 - 2w_2 + 3w_3 = 0
        \] allora il vettore $\vec{v} + \vec{w} = (v_1+w_1, v_2+w_2, v_3+w_3)$ appartiene ad $S$, ovvero soddisfa la condizione $x -2y + 3z = 0$. \begin{align*}
            (v_1+w_1) -2(v_2+w_2) + 3(v_3+w_3) \\
            =\ &v_1+w_1 -2v_2-2w_2 + 3v_3+3w_3 \\
            =\ &(v_1 -2v_2 +3v_3) + (w_1 -2w_2 +3w_3)\\
            \intertext{dunque per l'ipotesi che $\vec v \in S$ e $\vec w \in S$}
            =\ &0 + 0\\
            =\ &0.
        \end{align*}
        Dunque $\vec v + \vec w \in S$.
        \item[($k\vec v  \in S$)] Verifichiamo che se \[
            \vec v = \begin{pmatrix}
                v_1 \\ v_2 \\ v_3
            \end{pmatrix}
        \] appartiene ad $S$, cioe' \[
            v_1 - 2v_2 + 3v_3 = 0
        \] allora per ogni $k \in \R$ il vettore $k\vec{v} = (kv_1, kv_2, kv_3)$ appartiene ad $S$, ovvero soddisfa la condizione $x -2y + 3z = 0$. \begin{align*}
            (kv_1) -2(kv_2) + 3(kv_3) &= kv_1 -2kv_2 + 3kv_3 \\
            &= k(v_1 -2v_2 +3v_3)\\
            \intertext{dunque per l'ipotesi che $\vec v \in S$}
            &= k \cdot 0\\
            &= 0.
        \end{align*}
        Dunque $k\vec v \in S$.
    \end{description}
    Concludiamo che $S$ e' un sottospazio di $\R^3$.
\end{example}

\begin{example}
    Sia $S \subseteq \R^3$ tale che \[
        S = \set{ \begin{pmatrix}
            x \\ y \\ z
        \end{pmatrix} \in \R^3 \suchthat x -2y + 3z = 4}.
    \]

    Per verificare se $S$ e' un sottospazio di $\R^3$ e' sufficiente verificare che $S$ rispetti le tre condizioni di sopra:
    \begin{description}
        \item[($0 \in S$)] Verifichiamo che il vettore $\vec{0_\R^3} = (0, 0, 0)$ appartenga ad $S$, ovvero soddisfi la condizione $x -2y + 3z = 4$: \[
            0 -2\cdot 0 + 3 \cdot 0 = 0 + 0 + 0 = 0 \neq 4
        \] La condizione quindi non e' verificata.
    \end{description}
    Possiamo subito concludere che $S$ non e' un sottospazio di $\R^3$.
\end{example}

\begin{example}
    Sia $S \subseteq \R^3$ tale che \[
        S = \set{ \begin{pmatrix}
            x \\ y \\ z
        \end{pmatrix} \in \R^3 \suchthat x^2 - y^2 = 0}.
    \]

    Per verificare se $S$ e' un sottospazio di $\R^3$ e' sufficiente verificare che $S$ rispetti le tre condizioni di sopra:
    \begin{description}
        \item[($0 \in S$)] Verifichiamo che il vettore $\vec{0_\R^3} = (0, 0, 0)$ appartenga ad $S$, ovvero soddisfi la condizione $x^2 - y^2 = 0$: \[
            0^2 - 0y^2 = 0 + 0 = 0
        \] La condizione quindi e' verificata e $0_V \in S$.
        \item[($\vec v + \vec w \in S$)] Verifichiamo che se \[
            \vec v = \begin{pmatrix}
                v_1 \\ v_2 \\ v_3
            \end{pmatrix}, \quad \vec w = \begin{pmatrix}
                w_1 \\ w_2 \\ w_3
            \end{pmatrix} 
        \] appartengono ad $S$, cioe' \[
            v_1^2 - v_2^2 = 0, \quad w_1^2 - w_2^2 = 0
        \] allora il vettore $\vec{v} + \vec{w} = (v_1+w_1, v_2+w_2, v_3+w_3)$ appartiene ad $S$, ovvero soddisfa la condizione $x^2 - y^2 = 0$. \begin{align*}
            &(v_1+w_1)^2 - (v_2+w_2)^2 \\
            =\ &v_1^2+w_1^2 + 2v_1w_1 -v_2^2-w_2^2-2v_2w_2 \\
            =\ &(v_1^2 - v_2^2) + (w_1^2 - w_2^2) + 2v_1w_1 -2v_2w_2\\
            \intertext{dunque per l'ipotesi che $\vec v \in S$ e $\vec w \in S$}
            =\ &0 + 0  + 2v_1w_1 -2v_2w_2\\
            =\ &2v_1w_1 -2v_2w_2.
        \end{align*}
        Ma nessuno ci assicura che questa somma sia uguale a $0$ (ad esempio basta scegliere $\vec v = (1, -1, 0)$ e $\vec w = (1, 1, 0)$), dunque la condizione non e' sempre rispettata.
    \end{description}
    Concludiamo che $S$ non e' un sottospazio di $\R^3$.
\end{example}

\begin{example}
    Sia $V = \M_{2 \times 2}(\R)$ e sia $A \subseteq V$ tale che \[
        A = \set{M \in \M_{2 \times 2}(\R) \suchthat M = M^T}.
    \]

    Vedere se questo e' un sottospazio sembra piu' difficile degli esercizi precedenti. Tuttavia, possiamo cercare di rendere la definizione di $A$ piu' esplicita in modo da capire meglio quale sia la condizione di appartenenza al sottospazio.

    Notiamo che tutta la definizione di $A$ si basa su una matrice generica $M \in \M_{2 \times 2}(\R)$. Rendiamo piu' esplicita questa definizione, scrivendo \[
        M = \begin{pmatrix}
            a & b \\ c & d
        \end{pmatrix}    
    \] con $a, b, c, d \in \R$ generici.

    A questo punto ricordando la definizione di matrice trasposta (ovvero una matrice ottenuta trasformando le righe in colonne) possiamo scrivere la condizione di appartenenza al sottospazio $A$ come \[
        M = M^T \iff \begin{pmatrix}
            a & b \\ c & d
        \end{pmatrix} = \begin{pmatrix}
            a & c \\ b & d
        \end{pmatrix}
    \].
    
    Dunque la matrice $M$ appartiene ad $A$ se e solo se $b = c$ (ovvero la seconda e la terza coordinata sono uguali), cioe' \[
        A = \set{\begin{pmatrix}
            a & b \\ c & d
        \end{pmatrix} \in \M_{2 \times 2}(\R) \suchthat b = c}.
    \]

    A questo punto possiamo verificare se $A$ e' effettivamente un sottospazio di $\M_{2 \times 2}(\R)$.

    \begin{description}
        \item[($0 \in A$)] Verifichiamo che il vettore $\vec{0_V} = \begin{psmallmatrix}0 & 0 \\ 0 & 0 \end{psmallmatrix}$ appartenga ad $A$, ovvero soddisfi la condizione $b = c$. La condizione e' ovviamente verificata e dunque $0_V \in A$.
        \item[($\vec v + \vec w \in A$)] Verifichiamo che se \[
            M_1 = \begin{pmatrix}
                p & q \\ r & s
            \end{pmatrix}, \quad M_2 = \begin{pmatrix}
                x & y \\ z & t
            \end{pmatrix} 
        \] appartengono ad $A$, cioe' \[
            q = r, \quad y = z
        \] allora la matrice \[
            M_1 + M_2 = \begin{pmatrix}
                p+x & q+y \\ r+z & s+t
            \end{pmatrix}
        \] appartiene ad $A$, ovvero soddisfa la condizione $b = c$. \begin{align*}
            &(q+y) \iseq (r+z) \\
            \intertext{Per l'ipotesi che $M_1 \in A$ e $M_2 \in A$ sappiamo che $q = r$ e $y = z$:}
            \iff &r + z = r+z
        \end{align*}
        che e' ovvia. Dunque $M_1 + M_2 \in A$.
        \item[($k\vec v  \in A$)] Verifichiamo che se \[
            M = \begin{pmatrix}
                x & y \\ z & t
            \end{pmatrix} 
        \] appartiene ad $A$, cioe' \[
            y = z
        \] allora per ogni $k \in \R$ la matrice \[
            kM = \begin{pmatrix}
                kx & ky \\ kz & kt
            \end{pmatrix} 
        \]
        appartiene ad $A$, ovvero soddisfi la condizione $ky = kz$. 
        
        Ma per ipotesi $y = z$, dunque moltiplicando entrambi i membri per $k$ otteniamo $ky = kz$, che e' quello che stavamo cercando di dimostrare.

        Dunque $k\M \in A$.
    \end{description}

    Segue quindi che $A$ e' un sottospazio di $\M_{2 \times 2}(\R)$. Tale sottospazio si chiama \emph{spazio delle matrici simmetriche}.
\end{example}

\begin{exercise}
    Dire se i seguenti sottoinsiemi sono sottospazi oppure no.

    \begin{enumerate}
        \item $V \subseteq \R^4$ tale che \[
            V = \set{ \begin{pmatrix}
                x \\ y \\ z
            \end{pmatrix} \in R^4 \suchthat 
                3x - 2y + z + t = 0 }
        \]
        \item $V \subseteq \R^3$ tale che \[
            V = \set{ \begin{pmatrix}
                x \\ y \\ z
            \end{pmatrix} \in R^3 \suchthat \left\{
                \begin{array}{@{}rororor }
                3x & - & 2y & + & z & = & 0 \\
                -x & + & y & + & 4z & = & 2
                \end{array}
            \right.}
        \]
        \item $V \subseteq \R^3$ tale che \[
            V = \Span{\begin{pmatrix}
                1 \\ 2 \\ 0
            \end{pmatrix}, \begin{pmatrix}
                -1 \\ 0 \\ 1
            \end{pmatrix}}   
        \]
        \item $V \subseteq \R[x]^{\leq 3}$ tale che \[
            V = \set{p(x) \in \R[x]^{\leq 3} \suchthat p(2) = 0}
        \]
        \item $V \subseteq \R[x]^{\leq 3}$ tale che \[
            V = \set{p(x) \in \R[x]^{\leq 3} \suchthat p(2) = -1}
        \]
        \item $V \subseteq \M_{2\times 2}(\R)$ tale che \[
            V = \set{M \in \M_{2\times 2}(\R) \suchthat AM - MA = O_2}
        \] dove $A$ e $O_2$ sono due matrici $2 \times 2$ tali che \begin{align*}
            A = \begin{pmatrix}
                0 & -1 \\ -1 & 0
            \end{pmatrix}, &&O_2 = \begin{pmatrix}
                0 & 0 \\ 0 & 0
            \end{pmatrix}.
        \end{align*}
    \end{enumerate}
\end{exercise}
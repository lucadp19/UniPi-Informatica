\documentclass{report}
\usepackage{amsmath}
\usepackage{amsthm}
\usepackage{amsfonts}
\usepackage[italian]{babel}

\author{Luca De Paulis}
\title{Logica per la programmazione}

\newcommand{\AND}{\wedge}
\newcommand{\OR}{\vee}

\begin{document}
    \maketitle
    
    \newtheorem*{defin}{}
    \newtheorem{obs}{Osservazione}
    \chapter{Calcolo proposizionale}    
    

    \section{Definizione di proposizione}
    \begin{defin}[Proposizione]
        Una proposizione e' un enunciato dichiarativo per il quale valgono
    il principio del terzo escluso e il principio di non contraddittorieta'.
    \end{defin}

    \begin{defin}
        [Principio del terzo escluso]
        Una proposizione e' vera oppure e' falsa.
    \end{defin}

    \begin{defin}
        [Principio di non contraddittorieta']
        Una proposizione non e' contemporaneamente vera e falsa.
    \end{defin}

    \section{Connettivi logici}
    \begin{table}[]
        \centering
        \begin{tabular}{|c|c|c|}
        \hline
        CONNETTIVI       & SIMBOLO                         & OPERAZIONE   \\ \hline
        not              & $\neg$                            & negazione    \\ \hline
        and              & $\And$                          & congiunzione \\ \hline
        or               & $\OR$                            & disgiunzione \\ \hline
        se p allora q    & $\implies$                        & implicazione \\ \hline
        p se e solo se q & $\equiv$                          & equivalenza  \\ \hline
        p se q           & $\impliedby$ & conseguenza  \\ \hline
        \end{tabular}
    \end{table}

    \section{Interpretazioni}
    \begin{defin}
        [Interpretazione]
        Una funzione che associa ogni variabile proposizionale che compare in una formula
        ad un valore compreso in ${0; 1}$ si chiama interpretazione.
    \end{defin}

    \begin{defin}
        [Tavola di verita']
        Si chiama tavola di verita' una tavola che raccoglie tutte le interpretazioni
        di una formula.
    \end{defin}

    \section{Tautologie}
    \begin{defin}
        [Tautologia]
        Si dice tautologia una formula che vale $T$ per ogni possibile interpretazione.
    \end{defin}

    \begin{defin}
        [Contraddizione]
        Si dice contraddizione una formula che vale $F$ per ogni possibile interpretazione.
    \end{defin}

    \begin{obs}
        Se $p$ e' una tautologia allora $\neg p$ e' una contraddizione.
    \end{obs}

    \begin{defin}
        [Implicazione tautologica]
        Si dice che $p$ implica tautologicamente $q$ se e solo se $(p \implies q)$ e' una tautologia.
    \end{defin}

    \begin{defin}
        [Equivalenza tautologica]
        Si dice che $p$ equivale tautologicamente a $q$ se e solo se $(p \equiv q)$ e' una tautologia.
    \end{defin}

    \section{Leggi}
    \subsection{Leggi per l'equivalenza}
    \begin{alignat*}{2}
        & p \equiv p                                                \qquad && \tag{riflessivita'} \\
        & (p \equiv q) \equiv (q \equiv p)                          \qquad && \tag{simmetria} \\
        & (p \equiv q) \AND (q \equiv r) \implies (p \equiv r)      \qquad && \tag{transitivita'} \\
        & ((p \equiv q) \equiv r) \equiv (p \equiv (q \equiv r))    \qquad && \tag{associativita'} \\
        & (p \equiv T) \equiv p                                     \qquad && \tag{unita'}
    \end{alignat*}

    \subsection{Leggi per congiunzione e disgiunzione}
    \begin{alignat*}{2}
        & p \OR q \equiv q \OR p                            \qquad && \tag{riflessivita'} \\
        & p \AND q \equiv q \AND p    \\
        & ((p \OR q) \OR r) \equiv (p \OR (q \OR r))        \qquad && \tag{associativita'} \\
        & ((p \AND q) \AND r) \equiv (p \AND (q \AND r)) \\
        & p \OR p \equiv p                                  \qquad && \tag{idempotenza} \\
        & p \AND p \equiv p  \\
        & p \OR F \equiv p                                  \qquad && \tag{unita'} \\
        & p \AND T \equiv p  \\
        & p \OR T \equiv T                                  \qquad && \tag{zero/dominanza} \\
        & p \AND F \equiv F  \\
        & p \AND (q \OR r) \equiv (p \AND q) \OR (p \AND r) \qquad && \tag{distributivita'} \\
        & p \OR (q \AND r) \equiv (p \OR q) \AND (p \OR r) \\
    \end{alignat*}

    \subsection{Leggi per la negazione}
    \begin{alignat*}{2}
        & \neg(\neg p) \equiv p                     \qquad &&\tag{doppia negazione} \\
        & p \OR \neg p \equiv T                     \qquad &&\tag{terzo escluso} \\
        & p \AND \neg p \equiv F                    \qquad &&\tag{contraddizione} \\
        & \neg (p \AND q) \equiv \neg p \OR \neg q  \qquad &&\tag{De Morgan} \\
        & \neg (p \OR q) \equiv \neg p \AND \neg q \\
    \end{alignat*}
    
    \subsection{Leggi di eliminazione}
    \begin{alignat*}{2}   
        & (p \implies q) \equiv (\neg p \OR q)                      \qquad &&\tag{elim-$\implies$} \\
        & \neg(p \implies q) \equiv (p \AND \neg q)                 \qquad &&\tag{elim-$\neg\implies$} \\
        & (p \equiv q) \equiv (p \implies q) \AND (q \implies p)    \qquad &&\tag{elim-$\equiv$} \\
        & (p \equiv q) \equiv (p \AND q) \AND (\neg p \AND \neg q)  \qquad &&\tag{elim-$\equiv$-bis} \\
        & (p \impliedby q) \equiv (q \implies p)                    \qquad &&\tag{elim-$\impliedby$}
    \end{alignat*} 

    \subsection{Transitivita' dell'implicazione}
    \begin{alignat*}{2}
        & ((p \implies q) \AND (q \implies r)) \implies (p \implies b)      \qquad &&\tag{transitivita'}        
    \end{alignat*}

    \subsection{Leggi di complemento e assorbimento}
    \begin{alignat}
        {2}
        & p \OR (\neg p \AND q) \equiv p \OR q      \qquad &&\tag{complemento}\\
        & p \AND (\neg p \OR q) \equiv p \AND q     \nonumber\\
        & p \OR (p \AND q) \equiv p                 \qquad &&\tag{assorbimento}\\
        & p \AND (p \OR q) \equiv p                 \nonumber
    \end{alignat}

    \subsection{Altre leggi utili}
    \begin{alignat*}
        {2}
        & p \AND q \implies p                                           \qquad &&\tag{sempl-$\AND$}\\
        & p \implies p \OR q                                            \qquad &&\tag{introd-$\OR$}\\
        & p \implies q \equiv \neg q \implies p                         \qquad &&\tag{contropositiva}\\
        & p \AND q \implies r \equiv p \AND \neg r \implies \neg q      \qquad &&\tag{scambio}\\
        & (p \implies q) \AND p \implies q                              \qquad &&\tag{Modus Ponens}
    \end{alignat*}

    \section{Dimostrazione delle leggi}
    \subsection{Leggi di complemento}
    \begin{alignat*}
        {2}
                & p \OR (\neg p \AND q)\\
        \equiv  &{\ref{distributivita'}}\\
                & (p \OR \neg p) \AND (p \OR q)\\
        \equiv  &{\ref{terzo escluso}}\\
                & T \AND (p \OR q)\\
        \equiv  &{\ref{unita'}}\\
                & p \OR q
    \end{alignat*}

    \section{Inferenza corretta}
    Il calcolo proposizionale permette di dimostrare semplici inferenze, dimostrazioni o deduzioni del linguaggio
    naturale. Un'inferenza si dice corretta se e solo se la formula e' una tautologia.

    \section{Precedenza tra operatori}
    Per eliminare alcune delle parentesi e alleggerire la scrittura di formule si usano delle leggi
    sulla precedenza degli operatori.

    \begin{table}[]
        \centering
        \begin{tabular}{|c|c|}
        \hline
        OPERATORE                   & LIVELLO DI PRECEDENZA (crescente)     \\ \hline
        $\neg$                      & 0                                     \\ \hline
        $\AND$, $\OR$               & 1                                     \\ \hline
        $\implies$, $\impliedby$    & 2                                     \\ \hline
        $\equiv$                    & 3                                     \\ \hline
        \end{tabular}
    \end{table}

    La tabella non descrive l'associativita' degli operatori, che quindi va descritta tramite
    l'uso delle parentesi per non commettere errori sintattici.

    \subsection{Esempi di formule sintatticamente errate}
    \begin{alignat*}
        {1}
        p \AND & q \OR r \\
        \intertext{Errata perche' nessuna legge descrive l'associativita' di $\AND$ e $\OR$,
        dunque la formula e' ambigua.}
        p \implies & q \implies r\\
        \intertext{Errata perche' l'implicazione non e' associativa, dunque la formula e'
        ambigua.}
    \end{alignat*}


\end{document}
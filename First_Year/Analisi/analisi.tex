\documentclass{report}
\usepackage{amsmath}
\usepackage{amsthm}
\usepackage{amsfonts}
\usepackage[italian]{babel}

\author{Luca De Paulis}
\title{Analisi matematica}

\newcommand{\AND}{\wedge}
\newcommand{\OR}{\vee}
\newcommand{\NN}{\mathbb{N}}
\newcommand{\ZZ}{\mathbb{Z}}
\newcommand{\QQ}{\mathbb{Q}}
\newcommand{\RR}{\mathbb{R}}
\newcommand{\extR}{\overline{\RR}}

\begin{document}
    \maketitle
    
    \newtheorem{defin}{Definizione}[section]
    \newtheorem*{obs}{Osservazione}
    \newtheorem{theorem}{Teorema}[section]
    \newtheorem{corol}{Corollario}[theorem]

    \chapter{Fondamentali}

    \theoremstyle{definition}
    \begin{defin}
        [Intervallo di $\RR$]
        Un sottoinsieme $I$ contenuto in $RR$ ($I \subset \RR$)
        si dice intervallo se e solo se $\forall x, y \in I$ con $x < y$
        e per ogni $z$ t.c. $x < y < z$, allora $z \in \RR$.
    \end{defin}

    \section{Insiemi}
    \theoremstyle{definition}
    \begin{defin}
        [Massimo]
        Sia $A \subset \RR$, $A \neq \emptyset$, allora $m \in \RR$ si dice massimo 
        di $A$ se ${m \geq a \;\forall a \in A \AND m \in A}$. \newline
        Il massimo di un insieme $A$ indica con $\max{A}$.
    \end{defin}

    \begin{defin}
        [Minimo]
        Sia $A \subset \RR$, $A \neq \emptyset$, allora $m \in \RR$ si dice minimo 
        di $A$ se ${m \leq a \;\forall a \in A \AND m \in A}$. \newline
        Il minimo di un insieme $A$ indica con $\min{A}$.
    \end{defin}

    \begin{defin}
        [Maggiorante]
        Sia $A \subset \RR$, $A \neq \emptyset$, allora $m \in \RR$ si dice maggiorante 
        di $A$ se ${m \geq a \;\forall a \in A}$. \newline
    \end{defin}

    \begin{defin}
        [Minorante]
        Sia $A \subset \RR$, $A \neq \emptyset$, allora $m \in \RR$ si dice minorante 
        di $A$ se ${m \leq a \;\forall a \in A}$. \newline
    \end{defin}

    \theoremstyle{remark}
    \begin{obs}
        Se esiste un maggiorante/minorante per $A$, allora ne esistono infiniti.
        Al contrario, esistono insiemi che non ammettono maggioranti o minoranti o entrambi.
    \end{obs}

    \theoremstyle{definition}
    \begin{defin}
        [Insieme limitato superiormente]
        Un insieme si dice limitato superiormente se l'insieme dei suoi maggioranti
        non e' vuoto.
    \end{defin}
    
    \begin{defin}
        [Insieme limitato inferiormente]
        Un insieme si dice limitato inferiormente se l'insieme dei suoi minoranti
        non e' vuoto.
    \end{defin}

    \begin{defin}
        [Insieme limitato]
        Un insieme si dice limitato se e' limitato sia superiormente che inferiormente.
    \end{defin}

    \theoremstyle{remark}
    \begin{obs}
        $A \subset \RR$, $A \neq \emptyset$ si dice limitato se e solo se 
        $\exists \;n, k \in \RR \text{ t.c. } n \leq a \leq k \;\forall a \in A$.
    \end{obs}

    \theoremstyle{plain}
    \begin{theorem}[Esistenza dell'estremo]
        Se $A \subset \RR$, $A \neq \emptyset$ e' superiormente limitato, allora
        l'insieme dei maggioranti di $A$ ha minimo. Tale minimo si chiama estremo
        superiore di $A$ e si indica con $\sup{A}$. \newline
        Allo stesso modo, se $A \subset \RR$, $A \neq \emptyset$ e' inferiormente 
        limitato, allora 
        l'insieme dei minoranti di $A$ ha massimo. Tale massimo si chiama estremo
        inferiore di $A$ e si indica con $\inf{A}$.
    \end{theorem}  

    \theoremstyle{remark}
    \begin{obs}
        Se $\sup{A} \in A$, allora $\sup{A} = \max{A}$. \newline
        Se $\inf{A} \in A$, allora $\inf{A} = \min{A}$. 
    \end{obs}

    \theoremstyle{definition}
    \begin{defin}
        Se $A$ non e' superiormente limitato, allora per definizione $\sup{A} = +\infty$. \newline
        Se $A$ non e' inferiormente limitato, allora per definizione $\inf{A} = -\infty$.
    \end{defin}

    \theoremstyle{remark}
    \begin{obs}
        Sia $A \subset \RR$, $A \neq \emptyset$ e $A$ superiormente limitato. Allora $\sup{A} = m$
        se e solo se
        \begin{alignat}
            {1}
            &m \geq a \qquad \forall a \in A\\
            &\forall \varepsilon > 0\;\exists a_0 \text{ t.c. } a_0 > m - \varepsilon
        \end{alignat}
    \end{obs}

    \section{Retta reale estesa}
    \theoremstyle{definition}
    \begin{defin}
        [Retta reale estesa]
        Si definisce retta reale estesa l'insieme $\extR = \RR \cup \{+\infty\} \cup \{-\infty\}$
        in modo che valgano le seguenti condizioni:
        \begin{alignat}
            {2}
            &-\infty \leq x \leq +\infty    \qquad &&\forall x \in \extR\\
            &-\infty < x < +\infty          \qquad &&\forall x \in \RR
        \end{alignat} 
    \end{defin}

    \theoremstyle{remark}
    \begin{obs}
        Per definizione, $\max{\extR} = +\infty$ e $\min{\extR} = -\infty$.
    \end{obs}

    \begin{obs}
        Sia $A \subset \RR$, $A \neq \emptyset$. Se $\sup{A} < +\infty$, allora $A$ e'
        superiormente limitato. Se $\inf{A} > -\infty$, allora $A$ e'
        inferiormente limitato.
    \end{obs}

    \theoremstyle{definition}
    \begin{defin}
        [Operazioni in $\extR$]
        In $\extR$ valgono tutte le operazioni che valgono in $\RR$, piu' alcune:
        \begin{alignat}
            {2}
            &x + (-\infty) = -\infty                                            \qquad &&\forall x \neq +\infty\\
            &x + (+\infty) = +\infty                                            \qquad &&\forall x \neq -\infty\\
            &x \cdot (+\infty) = +\infty \text{, } x \cdot (-\infty) = -\infty  \qquad &&\forall x > 0\\
            &x \cdot (+\infty) = -\infty \text{, } x \cdot (+\infty) = +\infty  \qquad &&\forall x < 0 \\
        \end{alignat}
    \end{defin}

    \section{Funzioni particolari}
    \theoremstyle{remark}
    \begin{obs}
        Se $A \subset \ZZ$, $A \neq \emptyset$, allora 
        \begin{itemize}
            \item se $A$ e' superiormente limitato $A$ ammette massimo;
            \item se $A$ e' inferiormente limitato $A$ ammette minimo.
        \end{itemize}
    \end{obs}

    \theoremstyle{definition}
    \begin{defin}
        [Parte intera]
        Dato $x \in \RR$ si dice parte intera di $x$ il numero
        \[[x] = \max{\{m \in \ZZ \text{, } m \leq x\}}\]
    \end{defin}

\end{document}
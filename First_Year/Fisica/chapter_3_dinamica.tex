\chapter{Dinamica}

La dinamica e' la branca della fisica che si occupa di studiare le cause delle variazioni del moto. Ci occuperemo inizialmente della dinamica del punto materiale, cioe' il moto del corpo e le forze che agiscono su di esso sono riferiti ad un punto $(x, y, z)$ dello spazio; passeremo poi a studiare la dinamica di sistemi con piu' gradi di liberta'.

\begin{definition}
    Si dice massa inerziale di un corpo la resistenza che il corpo oppone al cambiamento di velocita' risultante dall'azione di una forza.    
\end{definition}
La massa e' una grandezza scalare e additiva, la cui unita' di misura e' il chilogrammo.

\begin{definition}
    La forza e' una quantita' vettoriale che descrive le interazioni fra corpi.
\end{definition}
La forza e' rappresentata da un vettore applicato, cioe' viene descritta da intensita', direzione, verso e punto di applicazione. Da un punto di vista pratico, per misurare una forza si sfrutta la proprieta' che esa ha di deformare gli oggetti.

Vi sono due tipi di forze, distinte dal loro raggio di azione.
\begin{enumerate}
    \item Le forze a distanza (o forze a lungo raggio) non richiedono che i corpi su cui agiscono siano a contatto tra loro. Gli esempi piu' comuni sono le 4 iterazioni fondamentali:
        \begin{itemize}
            \item iterazione gravitazionale (mediata dal gravitone (forse));
            \item iterazione elettromagnetica (mediata dai fotoni);
            \item iterazione nucleare debole (mediata dai bosoni W e Z);
            \item iterazione nucleare forte (mediata dai gluoni);
        \end{itemize}
    \item Le forze a contatto (o forze a corto raggio) sono forze che agiscono a contatto tra i corpi macroscopici, e derivano dalle interazioni elettromagnetiche fra atomi e molecole che costituiscono la materia. Alcuni esempi sono:
        \begin{itemize}
            \item forze esplicate dai vincoli (tensione di fili; forza normale associata ad una superficie che si oppone alla deformazione);
            \item forze di attrito (dinamico e statico);
            \item forze elastiche (interazioni elettromagnetiche che si oppongono alle deformazioni dei corpi).
        \end{itemize}
\end{enumerate}

\section{Principi della dinamica}

\begin{principle}[Primo principio della dinamica]
    Sia $\bvv{R}$ la risultante delle forze che agiscono su un punto materiale. Allora se $\bvv{R} = \bvv{0}$ allora il corpo permane nel suo stato di quiete o moto rettilineo uniforme.
\end{principle}

\begin{definition}
    Si dice che un sistema di riferimento e' inerziale se dato un corpo e appurato che la risultante delle forze che agiscono su quel corpo e' nulla, allora il corpo e' in quiete o si muove di moto rettilineo uniforme rispetto al sistema.
\end{definition}

Segue dalla definizione che un sistema in moto rettilineo uniforme rispetto a un sistema inerziale e' anch'esso inerziale.

\begin{principle}[Secondo principio della dinamica]
    Sia $\bvv{R}$ la risultante delle forze che agiscono su un punto materiale. Allora se $\bvv{R} \neq \bvv{0}$ allora il corpo subisce un'accelerazione che e' direttamente proporzionale alla risultante delle forze. La costante di proporzionalita' e' la massa inerziale, secondo la formula:
    \begin{equation} \label{second_principle}
        \bvv{R} = m\bvv{a}
    \end{equation}
\end{principle}

Facciamo ora delle considerazioni sui primi due principi.
I primi due principi della dinamica sono validi soltanto in sistemi di riferimento inerziali. Essi ci danno un modo per studiare il moto di un corpo a partire dalle forze che agiscono su di esso. Studiando il diagramma di corpo libero, cioe' il diagramma delle forze che agiscono su un corpo possiamo calcolarne la risultante e stabilire, a seconda del risultato, se il corpo ha un'accelerazione nulla o meno.

\begin{principle}[Terzo principio della dinamica]
    Siano $A$, $B$ due corpi puntiformi di massa $m_A, m_B$. Allora se il corpo $A$ esercita sul corpo $B$ una forza $\bvv{F}_{B,A}$, allora il corpo $B$ esercitera' necessariamente una forza $\bvv{F}_{A, B}$ sul corpo $A$, tale che
    \begin{equation} \label{third_principle}
        \bvv{F}_{B, A} = -\bvv{F}_{A, B}   
    \end{equation}
\end{principle}

Non bisogna confondere il terzo principio della dinamica con le forze vincolari: le forze vincolari sono applicate allo stesso corpo che esercita la forza, mentre il secondo principio coinvolge due corpi.

\section{Iterazione gravitazionale}
L'iterazione gravitazionale e' l'iterazione fondamentale che spiega la caduta dei gravi e il moto dei pianeti.

\begin{definition}
    Siano $A$, $B$, due corpi di massa $m_A$, $m_B$. Allora i due corpi si attraggono con forze che sono:
    \begin{itemize}
        \item dirette lungo la congiungente dei centri di massa;
        \item attrattive;
        \item di intensita' uguali, proporzionali al prodotto delle masse e inversamente proporzionali al quadrato della distanza dei centri di massa, secondo la formula
        \begin{equation} \label{forza_grav}
            \abs{\bvv{F}_{A, B}} = \abs{\bvv{F}_{B, A}} = G\frac{m_Am_b}{r^2}
        \end{equation}
        dove $G = 6,64 \times 10^{-11} \text{Nm}^2\text{/kg}$ e' la costante di gravitazione universale.
    \end{itemize}
\end{definition}

Dato che il valore di $G$ e' molto piccolo la forza gravitazionale ha un effetto trascurabile a meno che la massa dei corpi in esame sia grande (almeno $10^{10}$ kg) ed essi sono relativamente vicini.

In teoria la massa inerziale e la massa gravitazionale, cioe' la grandezza scalare proporzionale alla forza di gravita', rappresentano due concetti distinti di massa.

Se consideriamo un corpo sulla superficie terrestre oppure ad un'altezza trascurabile, la forza peso e' costante in modulo e in direzione (radiale, diretta verso il centro di massa della Terra). Dunque possiamo approssimarla con:
\begin{equation}
    \vmag{P} = G\frac{m_Tm_A}{(R_T + h)^2} \approx m_AG\frac{m_T}{R^2_T} = m_Ag
\end{equation}
dove $g = 9,81$ N/kg e' l'accelerazione di gravita' terrestre.

Dagli esperimenti e' stato poi dimostrato che la massa gravitazionale e' equivalente alla massa inerziale: dunque tutti i corpi in caduta libera hanno la stessa accelerazione quando si trovano sullo stesso pianeta e ad altitudini comparabili, sotto l'azione della sola forza gravitazionale.
L'accelerazione di un corpo in caduta libera e' dunque
\begin{equation}
    \bvv{a} = \frac{\bvv{P}}{m_A} = \bvv{g} = -g\bh{j}
\end{equation}

\section{Forze di contatto}
Quando due corpi macroscopici sono a contatto tra di loro agiscono delle forze che derivano dalle interazioni elettromagnetiche della materia. Il primo tipo che studieremo sono le forze legate ai vincoli
Possono essere legate a vincoli di due tipi:
\begin{itemize}
    \item vincoli di superfici (come forze di attrito, forze normali);
    \item vincoli unidimensionali (tensioni di corde, funi, fili).
\end{itemize}

\subsection{Forze legate a vincoli di superfici}
Le forze legate ai vincoli di superfici possono essere
\begin{enumerate}
    \item forze di attrito statico o dinamico: esse sono parallele alla superficie di contatto e opposte al moto relativo dei due corpi;
    \item forze normali: esse sono perpendicolari alla superficie di contatto e impediscono ai corpi di compenetrarsi.
\end{enumerate}
Se la superficie di contatto e' piana allora le forze normali bilanciano il peso del corpo sulla superficie, dunque 
\begin{equation}
    \left(\sum_i \bvv{F}_i \right)_{\perp} = \bvv{0} \implies \bvv{a}_{\perp} = \bvv{0}.
\end{equation}
Se la superficie di contatto non e' piana allora le forze normali non bilanciano il peso del corpo sulla superficie, dunque 
\begin{equation}
    \left(\sum_i \bvv{F}_i \right)_{\perp} \neq \bvv{0} \implies \bvv{a}_{\perp} \neq \bvv{0}.
\end{equation}

\begin{example}[Piano inclinato liscio]
    Supponiamo di avere un piano inclinato con angolo alla base $\theta$. Per descrivere il moto del corpo sul piano inclinato, scegliamo come sistema di riferimento un sistema $XY$ dove l'asse $X$ e' parallelo al piano inclinato, l'asse $Y$ perpendicolare ad esso, e l'origine degli assi sia nel punto dove si trova il corpo al tempo $t_0 = 0$s.

    Se disegnamo il diagramma del corpo libero notiamo che le forze in gioco sono il peso $\bvv{P}$ e la reazione vincolare del piano $\bvv{N}$.
    Sia $\bvv{R}$ la forza risultante; allora
    \begin{equation}
        \bvv{R} = \begin{cases}
            R_x = mg\sin\theta \\
            R_y = N - mg\cos\theta
        \end{cases}
    \end{equation}
    Per il secondo principio della dinamica vale allora
    \begin{align}
        \begin{cases}
            R_x = mg\sin\theta = ma_x \\
            R_y = N - mg\cos\theta = ma_y = 0 
        \end{cases}
        &\implies
        \begin{cases}
            a_x = g\sin\theta \\
            N = mg\cos\theta
        \end{cases}
    \end{align}

    Supponendo che la rampa sia lunga $L$ e che il corpo si muova inizialmente con una velocita' $v_0 = 0$ possiamo calcolare la velocita' con cui il corpo giunge alla fine e il tempo che impiega per percorrerla $t_f$. Dalla legge oraria del moto uniformemente accelerato otteniamo
    \begin{alignat*}{1}
        L &= \frac{1}{2}g\sin\theta t_f^2 \\
        \intertext{da cui possiamo ricavare $t_f$}
        \implies t_f &= \sqrt{\frac{2L}{g\sin\theta}} \\
        \intertext{Sostituendo $h = L\sin\theta$}
              &= \sqrt{\frac{2h}{g\sin^2\theta}} \\
              &= \frac{1}{\sin\theta}\sqrt{\frac{2h}{g}}
        \intertext{Sostituendo $t_f$ nella formula per la velocita' otteniamo infine}
        \implies v_f &= a_xt_f\\
              &= g\sin\theta t_f\\
              &= \sqrt{2gh}
    \end{alignat*}
\end{example}

\begin{example}
    [Piano inclinato liscio e forza orizzontale]
    Supponiamo di avere un piano inclinato con angolo alla base $\theta$ e un corpo di massa $m$ che viene spinto su per il piano inclinato tramite una forza orizzontale $\bvv{F_e}$. Sappiamo inoltre che la velocita' del corpo e' costante. Calcoliamo quanto vale la forza orizzontale e la reazione vincolare $\bvv{N}$.

    Dato che per ipotesi il corpo si muove a velocita' costante, allora $\bvv{R} = \bvv{0}$, cioe' $R_x = 0$ e $R_y = 0$. Dunque disegnando il diagramma del corpo libero:
    \begin{alignat*}{1}
        &\begin{cases}
            R_x = mg\sin\theta - F_e\cos\theta = 0\\
            R_y = N - mg\cos\theta - F_e\sin\theta = 0
        \end{cases} \\
        \implies &\begin{cases}
            F_e\cos\theta = mg\sin\theta\\
            N = mg\cos\theta + F_e\sin\theta
        \end{cases} \\
        \implies &\begin{cases}
            F_e = mg\tan\theta\\
            N = mg\cos\theta + mg\tan\theta \sin\theta = mg\cos\theta (1 + \tan^2 \theta)
        \end{cases}
    \end{alignat*}
\end{example}

Le forze di attrito sono esercitate parallelamente alla superficie di contatto. Esse si dividono in due categorie: \begin{itemize}
    \item attrito statico: si oppone all'inizio del moto del corpo;
    \item attrito dinamico: si oppone al moto di un corpo che non e' in quiete.
\end{itemize}
Le forze di attrito dinamico sono inferiori a quelle di attrito statico.

\subsubsection{Attrito statico}

L'attrito statico si oppone al moto, e ha un'intensita' tale che l'accelerazione e' nulla. La forza di attrito ha direzione tangente alla superficie e ha verso opposto alla forza applicata. Finche' il corpo rimane in uno stato di quiete si ha che $\vmag{f_s} = \vmag{F_{\text{ext}}}$ (dove $\bvv{f_s}$ e' la forza di attrito e $\bvv{F_{\text{ext}}}$ e' la forza esterna). 
In generale si ha che $\vmag{f_s} \leq \mu_s N$, dove $\mu_s$ e' il coefficiente di attrito statico, dunque al massimo $\vmag{f_s} = \mu_s N$.

\subsubsection{Attrito dinamico}

Se la forza esterna supera il valore $\mu_s N$ il corpo comincia a muoversi e l'attrito diventa attrito dinamico, che ha la stessa direzione del moto, verso opposto e modulo \[
    \vmag{f_d} = \mu_d N    
\]
dove $\mu_d$ e' il coefficiente di attrito dinamico, e risulta sempre $\mu_d \leq \mu_s$.

\begin{example}
    Un corpo e' lanciato con velocita' $\bvv{v_0}$ lungo un piano scabro con attrito dinamico $\mu_d$. Dopo quanto tempo si ferma? Che tratto percorre prima di fermarsi?

    Consideriamo il modo nelle due dimensioni. Sull'asse $X$ il corpo si muove di moto accelerato (con accelerazione negativa data dall'attrito dinamico), mentre sull'asse $Y$ il corpo e' fermo dunque la somma delle forze sara' $0$. Dunque:
    \begin{equation*}
        \begin{cases}
            R_x = -f_d \\
            R_y = N - mg = 0            
        \end{cases}
        \implies \begin{cases}
            -f_d = -\mu_d N = -\mu_d mg = ma_x\\
            N = mg
        \end{cases} 
    \end{equation*}
    Quindi $a_x = -\mu_d g$. Sostituendolo nelle equazioni del moto otteniamo\[
        t_f = \frac{v_0}{-a_x} = \frac{v_0}{\mu_d g}
    \] e inoltre \[
        x_f = v_0t_f + \frac{1}{2}a_xt_f^2 = \frac{v_0^2}{\mu_d g} - \frac{1}{2}\mu_d g\frac{v_0^2}{\mu_d^2g^2} = \frac{v_0^2}{2\mu_d g}.
    \]
\end{example}


\begin{example}
    Un corpo e' lanciato con velocita' $\bvv{v_0}$ lungo un piano scabro con attrito dinamico $\mu_d$. Se volessi mantenere il corpo a velocita' costante, che forza esterna dovrei fornire? Ci vuole meno forza a tirare verso l'alto ($\theta > 0$) oppure a spingere verso il basso ($\theta < 0$)?

    Consideriamo il modo nelle due dimensioni includendo una forza $F$ inclinata con un angolo $\theta$: per il primo principio della dinamica la risultante delle forze sul corpo dovra' essere nulla. Dunque:
    \begin{equation*}
        \begin{cases}
            R_x = -f_d + F\cos\theta = -\mu_d N + F\cos\theta = 0\\
            R_y = N - mg + F\sin\theta = 0       
        \end{cases}
        \implies \begin{cases}
            -\mu_d N + F\cos\theta = -\mu_d (mg - F\sin\theta) + F\cos\theta = -\mu_d mg + F(\cos\theta + \mu_d\sin\theta) = 0\\
            N = mg - F\sin\theta
        \end{cases} 
    \end{equation*}
    dunque \[
        F = \frac{\mu_d mg}{\cos\theta + \mu_d \sin\theta}.    
    \]    
\end{example}

\begin{example}[Piano inclinato con attrito]
    Supponiamo di avere un piano inclinato con attrito e che il corpo con massa $m$ sia in uno stato di quiete.

    Questo significa che la forza di attrito $\bvv{f_a}$ riesce a bilanciare le altre forze, in modo che la risultante lungo l'asse $X$ sia nulla. Al massimo vale quindi $f_a = \mu_s N$.
    \begin{equation*}
        \begin{cases}
            R_x = N - mg\cos\theta = 0\\
            R_y = -f_a + mg\sin\theta = 0
        \end{cases} \implies
        \begin{cases}
            N = mg\cos\theta\\
            f_a = mg\sin\theta
        \end{cases} 
    \end{equation*}
    
    Definiamo pendenza critica l'angolo $\theta_c$ oltre il quale la forza di attrito statica non riesce a controbilanciare la forza di gravita'. Quindi quando $\theta > \theta_c$ segue che NON HO CAPITO E NON MI VA DI CAPIRE
\end{example}

\subsection{Tensioni}

Quando non ci sono altre forze applicate in una qualsiasi sua parte una fune ideale (cioe' inestensibile e di massa trascurabile) esercita sui corpi fissati ai suoi estremi due forze che hanno la direzione della fune, stessa intensita' e versi opposti.

\section{Moto circolare}

Si dice moto circolare uniforme un moto su una traiettoria circolare di raggio $R$ con velocita' costante in modulo. Dato che la velocita' cambia direzione avremo che $\bvv a \neq 0$.

Chiamiamo $s$ lo spazio percorso sulla circonferenza, $\theta = \frac{s}{R}$ l'angolo spazzato. Allora valgono le seguenti: \[
    \begin{cases}
        \Delta \bvv r = \bvv{r_f} - \bvv{r_i}\\
        \Delta \theta= \theta_f - \theta_i\\
        \Delta s = R\Delta \theta
    \end{cases}    
\]

\begin{definition}
    Diciamo velocita' angolare media di un moto circolare la grandezza $\ang{\omega} = \frac{\Delta \theta}{\Delta t}$. Dunque la velocita' angolare istantanea sara' $\omega = \lim_{\Delta t \to 0} \ang{\omega} = \frac{d\theta}{dt}$.
\end{definition}

\begin{definition}
    Diciamo accelerazione angolare media di un moto circolare la grandezza $\ang{\alpha} = \frac{\Delta \omega}{\Delta t}$. Dunque la velocita' angolare istantanea sara' $\alpha = \lim_{\Delta t \to 0} \ang{\alpha} = \frac{d\omega}{dt}$.
\end{definition}
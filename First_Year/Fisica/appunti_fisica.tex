\documentclass[a4paper]{report}
    \usepackage[utf8]{inputenc}
    \usepackage[italian]{babel}
    \usepackage[T1]{fontenc}
    \usepackage{textcomp, microtype}
    \usepackage{amsmath, amsthm, amssymb, longtable, physics, mathtools, esvect, cases}

    \usepackage{hyperref} % ultimo package da caricare!


\theoremstyle{plain}
\newtheorem{principle}{Principio}[section]

\theoremstyle{definition}
\newtheorem{definition}{Definizione}[section]

\theoremstyle{remark}
\newtheorem*{remark}{Osservazione}

\theoremstyle{definition}
\newtheorem{example}{Example}[section]

\newcommand{\vmag}[1]{\abs{\bvv{#1}}}
\newcommand{\ang}[1]{\left\langle#1\right\rangle}
\newcommand{\bvv}[1]{\vv{\mathbf{#1}}}
\newcommand{\bh}[1]{\hat{\mathbf{#1}}}

\begin{document}

% \author{Luca De Paulis}
\title{Fisica}
\maketitle

\tableofcontents

\chapter{Insiemi numerici}

\section{Strutture algebriche fondamentali}

\begin{definition}[Gruppo]
    Si dice \textbf{gruppo} una tripla ($G$, $\cdot$, $e$) formata da \begin{itemize}
        \item un insieme di elementi $G$;
        \item un operazione $\cdot : A \times A \to A$ detta prodotto;
        \item un elemento $e \in G$
    \end{itemize} per cui valgono i seguenti assiomi: 
    \begin{description}
        \item[(Assiomi di gruppo)] Per ogni $a, b, c \in G$ vale che
        \begin{align*}
            &\text{(P1)}      &&(ab) \in G            &\text{(chiusura rispetto a $\cdot$)}\\
            &\text{(P2)}      &&(ab)c = a(bc)         &\text{(associatività di $\cdot$)}\\
            &\text{(P3)}      &&a \cdot e=e \cdot a=a &\text{($e$ el. neutro di $\cdot$)}\\
            &\text{(P4)}     &&\exists a^{-1} \in G. \quad aa^{-1} = e &\text{(inverso per $\cdot$)}
            \intertext{Si dice \textbf{gruppo commutativo} un gruppo per cui vale inoltre il seguente assioma:}
            &\text{(P5)}     &&ab = ba               &\text{(commutatività di $\cdot$)}
        \end{align*}
    \end{description}
\end{definition}

\begin{definition}[Anello]
    Si dice \textbf{anello} una quintupla ($A$, $+$, $\cdot$, $0$, $1$) formata da
    \begin{itemize}
        \item un insieme di elementi $A$;
        \item un operazione $+ : A \times A \to A$ detta somma;
        \item un operazione $\cdot : A \times A \to A$ detta prodotto;
        \item un elemento $0 \in A$;
        \item un elemento $1 \in A$
    \end{itemize} per cui valgono i seguenti assiomi: 
    \begin{description}
        \item[(Assiomi di anello)] Per ogni $a, b, c \in A$ vale che
        \begin{align*}
            &\text{(S1)}      &&(a+b) \in A           &\text{(chiusura rispetto a $+$)}\\
            &\text{(S2)}      &&a+b = b+a             &\text{(commutatività di $+$)}\\
            &\text{(S3)}      &&(a+b)+c = a+(b+c)     &\text{(associatività di $+$)}\\
            &\text{(S4)}      &&a+0=0+a=a             &\text{(0 el. neutro di $+$)}\\
            &\text{(S5)}      &&\exists (-a) \in A. \quad a+(-a) = 0 &\text{(opposto per $+$)}\\
            &\text{(P1)}      &&(ab) \in A            &\text{(chiusura rispetto a $\cdot$)}\\
            &\text{(P2)}      &&(ab)c = a(bc)         &\text{(associatività di $\cdot$)}\\
            &\text{(P3)}      &&a \cdot 1=1 \cdot a=a &\text{(1 el. neutro di $\cdot$)}\\
            &\text{(P4)}      &&(a+b)c = ac + bc      &\text{(distributività 1)} \\
            &\text{(P5)}     &&a(b+c) = ab + ac      &\text{(distributività 2)}
            \intertext{Si dice \textbf{anello commutativo} un anello per cui vale inoltre il seguente assioma:}
            &\text{(P6)}     &&ab = ba               &\text{(commutatività di $\cdot$)}
        \end{align*}
    \end{description} 
\end{definition}

Un tipico esempio di anello commutativo è $\Z$: infatti gli anelli generalizzano le operazioni che possiamo fare sui numeri interi e le loro proprietà fondamentali per estenderle ad altri insiemi con la stessa struttura algebrica.

\begin{definition}[Campo]
    Si dice \textbf{campo} una quintupla ($F$, $+$, $\cdot$, $0$, $1$) formata da
    \begin{itemize}
        \item un insieme di elementi $F$;
        \item un operazione $+ : F \times F \to F$ detta somma;
        \item un operazione $\cdot : F \times F \to F$ detta prodotto;
        \item un elemento $0 \in F$;
        \item un elemento $1 \in F$
    \end{itemize}  per cui valgono i seguenti assiomi: 
    \begin{description}
        \item[(Assiomi di campo)] Per ogni $a, b, c \in F$ vale che
        \begin{align*}
            &\text{(S1)}      &&(a+b) \in F           &\text{(chiusura rispetto a $+$)}\\
            &\text{(S2)}      &&a+b = b+a             &\text{(commutatività di $+$)}\\
            &\text{(S3)}      &&(a+b)+c = a+(b+c)     &\text{(associatività di $+$)}\\
            &\text{(S4)}      &&a+0=0+a=a             &\text{(0 el. neutro di $+$)}\\
            &\text{(S5)}      &&\exists (-a) \in F. \quad a+(-a) = 0 &\text{(opposto per $+$)}\\
            &\text{(P1)}      &&(ab) \in F            &\text{(chiusura rispetto a $\cdot$)}\\        
            &\text{(P2)}      &&ab = ba               &\text{(commutatività di $\cdot$)}\\
            &\text{(P3)}      &&(ab)c = a(bc)         &\text{(associatività di $\cdot$)}\\
            &\text{(P4)}      &&a \cdot 1=1 \cdot a=a &\text{(1 el. neutro di $\cdot$)}\\
            &\text{(P5)}     &&(a+b)c = ac + bc      &\text{(distributività)} \\
            &\text{(P6)}     &&a \neq 0 \implies \exists a^{-1} \in F. \quad aa^{-1} = 1 &\text{(inverso per $\cdot$)}
        \end{align*}
    \end{description} 

    La definizione sopra è equivalente a dire che $F$ è un anello commutativo per cui ogni elemento non nullo ha un inverso moltiplicativo.
\end{definition}

Tra gli insiemi numerici classici, gli insiemi $\Q, \R$ e $\C$ sono tutti esempi di campi: infatti le operazioni di addizione e moltiplicazione sono chiuse rispetto all'insieme, rispettano le proprietà commutativa, associativa e distributiva ed esistono gli inversi per la somma e per il prodotto (per ogni numero diverso da $0$). Il concetto di campo serve quindi a generalizzare la struttura algebrica dei numeri razionali/reali/complessi per altri insiemi numerici.

Nei campi vale la seguente proposizione.
\begin{proposition}
    [Regola di annullamento del prodotto] \label{annullamento_prodotto}
    Sia $\K$ un campo e siano $a, b \in \K$. Allora \[
        ab = 0 \implies a = 0 \lor b = 0.    
    \]
\end{proposition}
\begin{proof}
    Sappiamo che $a = 0 \lor b = 0$ è equivalente a $a \neq 0 \implies b = 0$, dunque supponiamo che $a$ sia diverso da $0$ e dimostriamo che $b$ è zero.

    Dato che $a \neq 0$ allora ammette un inverso. Chiamiamolo $a^{-1}$ e moltiplichiamo entrambi i membri per esso:
    \begin{alignat*}
        {1}
        &a^{-1}(ab) = a^{-1} \cdot 0\\
        \iff &(a^{-1}a)b = 0 \\
        \iff &b = 0
    \end{alignat*}
    che è la tesi.
\end{proof}

\section{Numeri complessi}

\begin{definition}[Unità immaginaria]
    Si dice unità immaginaria il numero $i$ tale che \[
        i^2 = -1.    
    \]
\end{definition}

\begin{definition}[Numeri complessi]
    L'insieme dei numeri complessi $\C$ è l'insieme dei numeri della forma $a+ib$ per qualche $a, b \in \R$, ovvero \[
        \C = \{ a + ib \mid a, b \in \R, i^2 = -1\}.  
    \]
\end{definition}

\begin{definition}[Parte reale e immaginaria]
    Sia $z \in \C$ tale che $z = a + ib$. Allora si dicono rispettivamente \begin{itemize}
        \item parte reale di $z$ il numero $\Re z = a$;
        \item parte immaginaria di $z$ il numero $\Im z = b$.
    \end{itemize}
\end{definition}

\begin{definition}[Somma e prodotto sui complessi]
    Definiamo le seguenti due operazioni su $\C$:
    \begin{itemize}
        \item $+ : \C \times \C \to \C$ tale che $(a + ib) + (c + id) = (a + c) + i(b + d)$;
        \item $\cdot : \C \times \C \to \C$ tale che $(a + ib) \cdot (c + id) = (ac - bd) + i(ad + bc)$.
    \end{itemize}
\end{definition}

\begin{remark}
    Le due operazioni vengono naturalmente dalla somma e dal prodotto tra monomi. Infatti \begin{gather*}
        (a + ib) + (c + id) = a + c + ib + id = (a + c) + i(b + d);\\
        \begin{alignedat}{1}
            (a + ib) \cdot (c + id) &= ac + iad + ibc + i^2bd \\
            &= ac + i(ad + bc) -bd \\
            &= (ac - bd) + i(ad + bc).
        \end{alignedat}
    \end{gather*}
\end{remark}

Notiamo che i numeri complessi della forma $a + i0$ sono numeri reali, dunque $\R \subset \C$. Inoltre possiamo rappresentare i numeri complessi come punti in uno spazio bidimensionale dove la parte reale rappresenta l'ascissa e la parte immaginaria rappresenta l'ordinata: la retta corrispondente all'asse x è il sottoinsieme dei numeri reali.

\begin{definition}[Coniugato complesso]
    Sia $z = a+ib \in \C$. Allora si dice coniugato complesso (o semplicemente coniugato) di $z$ il numero \[
        \conj{z} = a - ib.    
    \]
\end{definition}

\begin{definition}[Norma di un numero complesso]
    Sia $z = a + ib \in \C$. Allora si dice norma di $z$ il numero reale \[
        \abs{z} = \sqrt{a^2 + b^2}.    
    \]
\end{definition}

Notiamo che $\abs{z} = 0$ se e solo se $a = b = 0$, ovvero se $z = 0$.

\begin{proposition}\label{somma_prodotto_tra_coniugati}
    Siano $z, w \in \C$ tali che $z = a+ib$, $w = c + id$. Allora \begin{enumerate}[(i)]
        \item $\conj{z} + \conj{w} = \conj{z + w}$;
        \item $\conj{z}\cdot\conj{w} = \conj{zw}$;
        \item $(\conj{z})^n = \conj{z^n}$.
    \end{enumerate}
\end{proposition}
\begin{proof}
    Dimostriamo i tre fatti.
    \begin{enumerate}[(i)]
        \item Per definizione di somma \begin{alignat*}
            {1}
            \conj{z} + \conj{w} &= (a-ib) + (c - id)\\
            &= (a+c) - i(b+d)\\
            &= \conj{z + w}.
        \end{alignat*}
        \item Per definizione di prodotto \begin{alignat*}
            {1}
            \conj{z}\cdot\conj{w} &= (a-ib)(c - id)\\
            &= (ac - bd) + i(-ad-bc)\\
            &= (ac - bd) - i(ad+bc)\\
            &= \conj{zw}.
        \end{alignat*}
        \item Dimostriamolo per induzione su $n$.
        \begin{description}
            \item[Caso base.] Se $n = 1$ allora banalmente $(\conj{z})^1 = \conj{z} = \conj{z^1}$.
            \item[Passo induttivo.] Supponiamo che la tesi valga per $n$ e dimostriamola per $n+1$. Allora \[
                (\conj{z})^{n+1} = (\conj{z})^{n} \cdot \conj{z} = \conj{z^n} \cdot \conj{z} = \conj{z^{n+1}}
            \] dove l'ultimo passaggio è giustificato dal punto precedente della dimostrazione. \qedhere
        \end{description}
    \end{enumerate}
\end{proof}

\begin{proposition}\label{somma_prodotto_col_coniugato}
    Sia $z = a+ib \in \C$. Allora valgono i seguenti fatti:
    \begin{enumerate}[(i)]
        \item $z + \conj{z} = 2\Re{z}$;
        \item $z\conj{z} = \abs{z}^2$.
    \end{enumerate}
\end{proposition}
\begin{proof}
    Dimostriamo i due fatti.
    \begin{enumerate}[(i)]
        \item Per definizione di somma $z + \conj{z} = (a + ib) + (a - ib) = 2a = 2\Re z$.
        \item Per definizione di prodotto \[
            z\conj{z} = (a + ib)(a - ib) = a^2 - iab + iab - i^2b^2 = a^2 + b^2 = \abs{z}^2.\qedhere
        \] 
    \end{enumerate}
\end{proof}

La proposizione precedente ci consente di trovare l'inverso di qualunque numero non nullo in $\C$.

\begin{proposition}[Inverso tra i complessi]
    Sia $z \in \C, z \neq 0$. Allora \[\frac{1}{z} = \frac{\conj{z}}{\abs{z}^2}.\]
\end{proposition}
\begin{proof}
    Per la proposizione \ref{somma_prodotto_col_coniugato} segue che \[
        z\conj{z} = \abs{z}^2 \iff \frac{1}{z} = \frac{\conj{z}}{\abs{z}^2}. \qedhere   
    \]
\end{proof}

\begin{proposition}[I numeri complessi formano un campo]
    L'insieme $\C$ insieme alle operazioni di somma e prodotto con i rispettivi elementi neutri $0, 1 \in \C$ forma un campo.
\end{proposition}

\subsection{Rappresentazione polare dei numeri complessi}

Dato che possiamo considerare i numeri complessi come punti di un piano bidimensionale possiamo rappresentarli in forma polare, cioè considerando il vettore che congiunge l'origine degli assi con il punto $(a, b)$ che rappresenta il numero complesso $a + ib$. La forma polare di un numero complesso è data dalla coppia $(r, \theta)$, dove $r$ è il raggio del vettore e $\theta$ è l'angolo tra l'asse x e il vettore.

Dunque se $z = a+ib$ è un numero complesso in forma cartesiana, possiamo esprimerlo come $r(\cos\theta + i\sin\theta)$, dove $r = \sqrt{a^2 + b^2} = \abs{z}$ e $\theta = \arctan \frac{a}{b}$.

\begin{definition}[Esponenziale complesso]
    $e^{i\theta} = \cos\theta + i\sin\theta.$
\end{definition}

Sfruttando la definizione precedente possiamo scrivere ogni numero complesso nella forma $re^{i\theta}$ che è la forma polare del numero.

\begin{proposition}
    Siano $e^{i\alpha}, e^{i\beta} \in \C$. Allora vale \[
        e^{i\alpha} e^{i\beta} = e^{i(\alpha + \beta)}.
    \]
\end{proposition}
\begin{proof}
    Per definizione di esponenziale complesso:
    \begin{alignat*}{1}
        e^{i\alpha} e^{i\beta} &= (\cos\alpha + i\sin\alpha)(\cos\beta + i\sin\beta)\\
        &= (\cos\alpha \cos\beta - \sin\alpha \sin\beta) + i(\sin\alpha \cos\beta + \cos\alpha \sin\beta)\\
        &= \cos(\alpha + \beta) + i\sin(\alpha + \beta)\\
        &= e^{i(\alpha + \beta)}. \tag*{\qedhere}
    \end{alignat*}
\end{proof}

\section{Successioni per ricorrenza}

\begin{definition}[Successione]
    Si dice successione a valori in un insieme $A$ una funzione $(a_n) : \N \to A$.
\end{definition}

Solitamente analizzeremo successioni a valori reali, ovvero $(a_n) : \N \to \R$. Inoltre usiamo equivalentemente le notazioni $(a_n)_k$ o $a_k$ per riferirci alla funzione valutata nel punto $k \in \N$.

\begin{definition}[Somma di successioni e prodotto per una costante]
    Sia $S_{\R}$ l'insieme delle successioni a valori reali. Allora definisco una somma tra successioni $+ : S_{\R} \times S_{\R} \to S_{\R}$ tale che \[
        (a_n) + (b_n) = (a_n + b_n)  
    \] e un prodotto per una costante $\cdot : \R \times S_{\R} \to S_{\R}$ tale che \[
        k(a_n) = (ka_n).    
    \]
\end{definition}

\begin{example}
    Sia $a_n = 3^n$ e $b_n = 2n + 1$. Allora $(c_n) = (a_n) + (b_n)$ è la successione definita dalla legge $c_n = 3^n + 2n + 1$, mentre $(d_n) = 3(b_n)$ è la successione definita da $d_n = 6n + 3$.
\end{example}

Queste operazioni rispettano le solite proprietà (associativa, commutativa, distributiva). In particolare vale quindi la seguente proposizione.

\begin{proposition}[L'insieme delle successioni è uno spazio vettoriale]
    L'insieme delle successioni a valori reali $S_{\R}$ insieme alle operazioni di somma e prodotto per costanti e alla successione identicamente nulla $(0_n)$ è uno spazio vettoriale su $\R$.
\end{proposition}

\begin{definition}[Ricorrenza lineare omogenea]
    Si dice ricorrenza lineare omogenea di ordine $k$ un'equazione della forma \begin{equation} \label{ricorrenza}
        a_{n+k} = r_{k-1}a_{n+k-1} + r_{k-2}a_{n+k-2} + \dots + r_{1}a_{n+1} + r_0a_n. 
    \end{equation}
    Una soluzione della ricorrenza lineare \ref{ricorrenza} è una successione $(s_n)$ tale che per ogni $n \in \N$ vale che $s_n, s_{n+1}, \dots, s_{n+k}$ soddisfano la ricorrenza.
\end{definition}

\begin{proposition}
    Sia $A$ l'insieme delle successioni che soddisfano la ricorrenza lineare omogenea \[
        s_{n+k} = r_{k-1}s_{n+k-1} + r_{k-2}s_{n+k-2} + \dots + r_{1}s_{n+1} + r_0s_n.
    \] Allora $A$ è un sottospazio vettoriale di $S_{\R}$.
\end{proposition}
\begin{proof}
    Dobbiamo dimostrare tre fatti:
    \begin{enumerate}[(i)]
        \item $(0_n) \in A$;
        \item se $(a_n), (b_n) \in A$ allora $(c_n) = (a_n) + (b_n) \in A$;
        \item se $h \in \R$, $(a_n) \in A$ allora $(d_n) = h(a_n) \in A$.
    \end{enumerate}

    Sia $n \in \N$ qualsiasi.
    \begin{enumerate}[(i)]
        \item Verifichiamo che $(0_n)$ sia soluzione. La ricorrenza da verificare è \[
            0_{n+k} = r_{k-1}0_{n+k-1} + \dots + r_{1}0_{n+1} + r_00_n.
        \] Ma dato che $(0_n)$ è la successione identicamente nulla, allora questo equivale a dire $0 = 0r_{k-1} + \dots +  + 0r_0 = 0$, che è verificata e quindi $(0_n) \in A$.
        \item Verifichiamo che $(c_n)$ sia soluzione. \begin{align*}
            c_{n+k} &= a_{n+k} + b_{n+k}\\
            &= (r_{k-1}a_{n+k-1} + \dots + r_0a_n) + (r_{k-1}b_{n+k-1} + \dots + r_0b_n) \\
            &= r_{k-1}(a_{n+k-1} + b_{n+k-1}) + \dots + r_0(a_n + b_n) \\
            &= r_{k-1}c_{n+k-1} + \dots + r_0c_0
        \end{align*}
        dunque $(c_n) \in A$.
        \item Verifichiamo che $(d_n)$ sia soluzione. \begin{align*}
            d_{n+k} &= ha_{n+k}\\
            &= h(r_{k-1}a_{n+k-1} + \dots + r_0a_n)\\
            &= r_{k-1}(ha_{n+k-1}) + \dots + r_0(ha_n) \\
            &= r_{k-1}d_{n+k-1} + \dots + r_0d_0
        \end{align*}
        dunque $(d_n) \in A$. \qedhere
    \end{enumerate}
\end{proof}

La proposizione precedente ci permette di trovare una soluzione generale ad una ricorrenza lineare omogenea.

\begin{example}
    Siano $a_n = 3^n$ e $b_n = (-1)^n$ due soluzioni di una ricorrenza lineare omogenea. Allora per la proposizione precedente anche $k_1a_n = k_13^n$ e $k_2b_n = k_2(-1)^n$ saranno soluzioni (per ogni $k_1, k_2 \in \R$), e di conseguenza anche $k_1a_n + k_2b_n = k_13^n + k_2(-1)^n$.
\end{example}

Cerchiamo di risolvere una ricorrenza lineare omogenea.
\begin{example}
    Sia $a_{n+2} = 2a_{n+1} + 3a_n$ una ricorrenza lineare omogenea di ordine $2$. Trovare la soluzione generale. Inoltre trovare una soluzione particolare che soddisfi le condizioni iniziali $a_0 = 0$ e $a_1 = 1$.
\end{example}
\begin{solution}
    Proviamo a risolvere la ricorrenza con una soluzione esponenziale della forma $(\lambda^n)$ al variare di $n \in \N$. Sostituendo otteniamo \begin{alignat*}{1}
        &\lambda^{n+2} = 2\lambda^{n+1} + 3\lambda^{n} \\
        \iff &\lambda^2 = 2\lambda + 3 \\
        \iff &\lambda^2 - 2\lambda - 3.
    \end{alignat*}
    Dunque se $(\lambda^n)$ è una soluzione allora $\lambda$ deve essere radice di quel polinomio di secondo grado, detto polinomio caratteristico della ricorrenza.
    Risolvendolo segue che $\lambda_1 = 3$ e $\lambda_2 = -1$ sono soluzioni, dunque le successioni $(3^n)$ e $((-1)^n)$ sono soluzioni della ricorrenza.

    La soluzione generale della ricorrenza è dunque una successione della forma $(a_n) = k_1(3^n) + k_2((-1)^n)$ al variare di $k_1, k_2 \in \R$.

    Imponiamo ora che $a_0 = 0$ e $a_1 = 1$.
    \begin{equation*}
        \left\{
        \begin{array}{@{}roror }
        3^0k_1 & + & (-1)^0k_2 & = & 0 \\
        3^1k_1 & + & (-1)^1k_2 & = & 1 \\
        \end{array}
        \right. \iff \left\{
        \begin{array}{@{}ror }
        k_1 + k_2 & = & 0 \\
        3k_1 -k_2 & = & 1 \\
        \end{array}
        \right. 
    \end{equation*}
    da cui segue $k_1 = \frac14$, $k_2 = -\frac14$. La successione che soddisfa le condizioni iniziali è dunque $a_n = \frac14(3)^n - \frac14(-1)^n$.
\end{solution}

\begin{definition}
    [Polinomio caratteristico di una ricorrenza]
    Sia $a_{n+k} = r_{k-1}a_{n+k-1} +  \dots +  r_0a_n$ una ricorrenza lineare omogenea di ordine $k$. Allora si dice polinomio caratteristico associato alla ricorrenza il polinomio \[
        p(\lambda) = \lambda^k - r_{k-1}\lambda^{k-1} - \dots - r_0.    
    \]
\end{definition}

Il polinomio caratteristico si ottiene sostituendo alla ricorrenza lineare la successione $(\lambda^n)$, esattamente come abbiamo fatto nell'esempio precedente.

\begin{example}
    Consideriamo la successione di Fibonacci $f_{n+2} = f_{n+1} + f_n$ con $f_0 = 0$, $f_1 = 1$. Trovare una successione che risolva la ricorrenza e soddisfi i casi base.
\end{example}
\begin{solution}
    Il polinomio caratteristico di questa ricorrenza è \[
        p(\lambda) = \lambda^2 - \lambda - 1    
    \] che ha come radici i numeri $\varphi = \frac12(1 + \sqrt5)$ e $\bar{\varphi} = \frac12(1 - \sqrt5)$.

    La soluzione generale della ricorrenza è dunque una successione della forma $(f_n) = k_1(\varphi^n) + k_2(\bar{\varphi}^n)$ al variare di $k_1, k_2 \in \R$.

    Imponiamo ora che $f_0 = 0$ e $f_1 = 1$.
    \begin{equation*}
        \arraycolsep=1.2pt\def\arraystretch{1.3}
        \left\{
        \begin{array}{@{}roror }
        \varphi^0k_1 & + & \bar{\varphi}^0k_2 & = & 0\\
        \varphi^1k_1 & + & \bar{\varphi}^1k_2 & = & 1 \\
        \end{array}
        \right. \iff \left\{
        \begin{array}{@{}roror }
        k_1 & + & k_2 & = & 0\\
        \varphi k_1 & + & \bar{\varphi}k_2 & = & 1 \\
        \end{array}
        \right. 
    \end{equation*}
    da cui segue $k_1 = \frac{1}{\sqrt5}$, $k_2 = -\frac{1}{\sqrt5}$. La successione che soddisfa le condizioni iniziali è dunque \[
        f_n = \frac{1}{\sqrt5}\left(\frac{1 + \sqrt5}{2}\right)^n - \frac{1}{\sqrt5}\left(\frac{1 - \sqrt5}{2}\right)^n.
    \]
\end{solution}

Nel caso che una radice del polinomio caratteristico abbia una molteplicità maggiore di $1$ essa darà luogo a più di una soluzione della ricorrenza, come ci dice la seguente proposizione.
\begin{proposition}
    Sia $p(\lambda)$ il polinomio caratteristico di una ricorrenza lineare omogenea e sia $\lambda_0$ una radice di molteplicità $h$ (ovvero $h$ è il massimo intero per cui $(x - \lambda_0)^h$ compare nella fattorizzazione di $p(\lambda)$) con $h \leq 2$. 
    
    Allora $(\lambda_0^n), (n\lambda_0^n), \dots, (n^{h-1}\lambda_0^n)$ sono tutte soluzioni della ricorrenza lineare omogenea.
\end{proposition}

\begin{example}
    Sia $p(\lambda) = (\lambda - 3)^3(\lambda + 1)^2(\lambda - \sqrt2)^4$. Allora le seguenti sono tutte soluzioni indipendenti della ricorrenza lineare omogenea associata a $p(\lambda)$:
    \begin{multicols}{3}
        \begin{enumerate}[(i)]
        \item $(3^n)$;
        \item $(n3^n)$;
        \item $(n^23^n)$;
        \item $((-1)^n)$;
        \item $(n(-1)^n)$;
        \item $(\sqrt{2}^n)$;
        \item $(n\sqrt{2}^n)$;
        \item $(n^2\sqrt{2}^n)$;
        \item $(n^3\sqrt{2}^n)$.
    \end{enumerate}
    \end{multicols}
    
    La soluzione generale sarà dunque della forma \begin{align*}
        (a_n) = 
            &\ k_1(3^n) + k_2(n3^n) + k_3(n^23^n) + k_4(n(-1)^n) + k_5(n(-1)^n) + \\
            + &\ k_6(\sqrt{2}^n) + k_7(n\sqrt{2}^n) + k_8(n^2\sqrt{2}^n) + k_9(n^3\sqrt{2}^n)
    \end{align*}
    al variare di $k_1, \dots, k_9 \in \R$.
\end{example}
\chapter{Spazi vettoriali}

\section{Spazi vettoriali}
\begin{definition}
    Si dice \textbf{spazio vettoriale su un campo $\K$} un insieme $V$ di elementi, detti \textbf{vettori}, insieme con due operazioni $+ : V \times V \to V$ e $\cdot : \K \times V \to V$ e un elemento $\bm{0_V} \in V$ che soddisfano i seguenti assiomi:
    \[\forall \bm{v}, \bm{w}, \bm{u} \in V, \quad\forall h, k \in \K \]
    \begin{align}
        &\text{1.} &&(\bm{v} + \bm{w}) \in V                                        &\text{(chiusura di V rispetto a $+$)} \\      
        &\text{2.} &&\bm{v} + \bm{w} = \bm{w} + \bm{v}                              &\text{(commutativita' di $+$)} \\
        &\text{3.} &&(\bm{v} + \bm{w}) + \bm{u} = \bm{w} + (\bm{v} + \bm{u})        &\text{(associativita' di $+$)} \\
        &\text{4.} &&\bm{0_V} + \bm{v} = \bm{v} + \bm{0_V} = \bm{v}                 &\text{($\bm{0_V}$ el. neutro di $+$)} \\
        &\text{5.} &&\exists (-\bm{v}) \in V. \quad\bm{v} + (\bm{-v}) = \bm{0_V}    &\text{(opposto per $+$)} \\
        &\text{6.} &&k\bm{v} \in V                                                  &\text{(chiusura di V rispetto a $\cdot$)} \\
        &\text{7.} &&k(\bm{v} + \bm{w}) = k\bm{v} + k\bm{w}                         &\text{(distributivita' 1)} \\
        &\text{8.} &&(k + h)\bm{v}= k\bm{v} + h\bm{v}                               &\text{(distributivita' 2)} \\
        &\text{9.} &&(kh)\bm{v}= k(h\bm{v})                                         &\text{(associativita' di $\cdot$)} \\
        &\text{10.}&&1\bm{v}= \bm{v}                                                &\text{(1 el. neutro di $\cdot$)}
    \end{align}
\end{definition}
 
Spesso il campo $\K$ su cui e' definito uno spazio vettoriale $V$ e' il campo dei numeri reali $\R$ o il campo dei numeri complessi $\C$. Supporremo che gli spazi vettoriali siano definiti su $\R$ a meno di diverse indicazioni. Le definizioni valgono comunque in generale anche su campi $\K$ diversi da $\R$ o $\C$.

\begin{example}
    Possiamo fare diversi esempi di spazi vettoriali. Ad esempio sono spazi vettoriali:
    \begin{enumerate}
        \item i vettori geometrici dove:
        \begin{itemize}
            \item l'elemento neutro e' il vettore nullo;
            \item la somma e' definita tramite la regola del parallelogramma;
            \item il prodotto per scalare e' definito nel modo usuale;
        \end{itemize}
        \item i vettori colonna $n \times 1$ o i vettori riga $1 \times n$ dove:
        \begin{itemize}
            \item l'elemento neutro e' il vettore composto da $n$ elementi $0$;
            \item la somma e' definita come somma tra componenti;
            \item il prodotto per scalare e' definito come prodotto tra lo scalare e ciascuna componente;
        \end{itemize}
        \item le matrici $n \times m$, indicate con $\M_{n \times m}(\K)$;
        \item i polinomi di grado minore o uguale a $n$, indicati con $\K[x]^{\leq n}$;
        \item tutti i polinomi, indicati con $\K[x]$.
    \end{enumerate}
\end{example}

\begin{definition}
    Sia $V$ uno spazio vettoriale, $A \subseteq V$. Allora si dice che $A$ e' un sottospazio vettoriale di $V$ (o semplicemente sottospazio) se
    \begin{align}
        &\bm{0_V} \in A \\
        &(\bm{v} + \bm{w}) \in A    &&\forall \bm{v}, \bm{w} \in A \\
        &(k\bm{v}) \in A            &&\forall k \in \R, \bm{v} \in A
    \end{align}
\end{definition}

\begin{proposition}
    Le soluzioni di un sistema omogeneo $A\bm{x} = \bm{0}$ con $n$ variabili formano un sottospazio di $\R^n$.
\end{proposition}
\begin{proof}
    Chiamiamo $S$ l'insieme delle soluzioni. Dato che le soluzioni sono vettori colonna di $n$ elementi, $S \subseteq \R^n$. Verifichiamo ora le condizioni per cui $S$ e' un sottospazio di $\R^n$:
    \begin{enumerate}
        \item $\bm{0}$ appartiene a $S$, poiche' $A\bm{0} = \bm{0}$;
        \item Se $\bm{x}, \bm{y}$ appartengono ad $S$, allora $A(\bm{x} + \bm{y}) = A\bm{x} + A\bm{y} = \bm{0} + \bm{0} = \bm{0}$, dunque $\bm{x} + \bm{y} \in S$;
        \item Se $\bm{x}$ appartiene ad $S$, allora $A(k\bm{x}) = kA\bm{x} = k\bm{0} = \bm{0}$, dunque $k\bm{x} \in S$.
    \end{enumerate}
    Dunque $S$ e' un sottospazio di $\R^n$.
\end{proof}

\section{Combinazioni lineari e span}
\begin{definition}
    Sia $V$ uno spazio vettoriale e $\bm{v_1}, \bm{v_2}, \dots, \bm{v_n} \in V$. Allora il vettore $\bm{v} \in V$ si dice combinazione lineare di $\bm{v_1}, \bm{v_2}, \dots, \bm{v_n}$ se 
    \begin{equation}
        \bm{v}= a_1\bm{v_1} + a_2\bm{v_2} + \dots + a_n\bm{v_n} 
    \end{equation}
    per qualche $a_1, a_2, \dots, a_n \in \R$.
\end{definition}

\begin{definition}
    Sia $V$ uno spazio vettoriale e $\bm{v_1}, \dots, \bm{v_n} \in V$. Si indica con $\Span{\bm{v_1}, \dots, \bm{v_n}}$ l'insieme dei vettori che si possono ottenere come combinazione lineare di $\bm{v_1}, \dots, \bm{v_n}$:
    \begin{equation}
        \Span{\bm{v_1}, \dots, \bm{v_n}} = \left\{a_1\bm{v_1} + \dots + a_n\bm{v_n} \mid a_1, \dots, a_n \in \R\right\}
    \end{equation}
\end{definition}

\begin{proposition}
    Sia $A \in \M_{n \times m}(\R)$ e siano $\bm{a_1}, \bm{a_2}, \dots, \bm{a_m} \in \R^n$ le sue colonne. Allora l'immagine della matrice e' uguale allo span delle sue colonne.
\end{proposition}
\begin{proof}
    L'immagine della matrice e' l'insieme di tutti i vettori del tipo $A \cdot \begin{pmatrix}
        x_1 & \dots & x_m
    \end{pmatrix}^T$ al variare di $x_1, \dots, x_m \in \R$. 
    \begin{alignat*}
        {1}
        A\begin{pmatrix} x_1 \\ \vdots \\ x_m \end{pmatrix}
            &= A\left(\begin{pmatrix} x_1 \\ \vdots \\ 0 \end{pmatrix} + \dots + \begin{pmatrix} 0 \\ \vdots \\ x_m \end{pmatrix}\right)\\
            &= A\left(x_1\begin{pmatrix} 1 \\ \vdots \\ 0 \end{pmatrix} + \dots + x_m\begin{pmatrix} 0 \\ \vdots \\ 1 \end{pmatrix}\right)\\
            &= x_1A\begin{pmatrix} 1 \\ \vdots \\ 0 \end{pmatrix} + \dots + x_mA\begin{pmatrix} 0 \\ \vdots \\ 1 \end{pmatrix}\\
        \intertext{Ma sappiamo per la proposizione \ref{j-esima_colonna} che moltiplicare una matrice per un vettore che contiene tutti $0$ tranne un $1$ in posizione $j$ ci da' come risultato la $j$-esima colonna della matrice, dunque:}
            &= x_1\bm{a_1} + \dots + x_m\bm{a_m} 
    \end{alignat*}
    Ma i vettori che appartengono allo span delle colonne di $A$ sono tutti e solo del tipo $x_1\bm{a_1} + \dots + x_m\bm{a_m}$, dunque $\Imm{A} = \Span{\bm{a_1}, \bm{a_2}, \dots, \bm{a_m}}$, come volevasi dimostrare.
\end{proof}

\begin{proposition}
    Sia $V$ uno spazio vettoriale, $\bm{v_1}, \dots, \bm{v_n} \in V$. Allora $A = \Span{v_1, \dots, v_n} \subseteq V$ e' un sottospazio di $V$.
\end{proposition}
\begin{proof}
    Dimostriamo che valgono le tre condizioni per cui $A$ e' un sottospazio di $V$:
    \begin{enumerate}
        \item $\bm{0_V}$ appartiene ad $A$, in quanto basta scegliere $a_1 = \dots = a_n = 0$;
        \item Siano $\bm{v}, \bm{w} \in A$. Allora per qualche $a_1, \dots, a_n, b_1, \dots, b_n \in \R$ vale che \begin{alignat*}{1}
            \bm{v} + \bm{w} &= (a_1\bm{v_1} + \dots + a_n\bm{v_n}) + (b_1\bm{v_1} + \dots + b_n\bm{v_n}) \\
            &= (a_1 + b_1)\bm{v_1} + \dots + (a_n + b_n)\bm{v_n} \in A
        \end{alignat*}
        \item Siano $\bm{v} \in A, k \in \R$. Allora per qualche $a_1, \dots, a_n \in \R$ vale che \begin{alignat*}{1}
            k\bm{v} &= k(a_1\bm{v_1} + \dots + a_n\bm{v_n})  \\
            &= (ka_1)\bm{v_1} + \dots + (ka_n)\bm{v_n} \in A
        \end{alignat*}
    \end{enumerate}
    cioe' $A$ e' un sottospazio di $V$.
\end{proof}

Vale anche l'implicazione inversa: ogni sottospazio di $V$ puo' essere descritto come span di alcuni suoi vettori.

\subsubsection{Forma parametrica e cartesiana}

\begin{proposition}
    Ogni sottospazio vettoriale di $R^n$ puo' essere descritto in due forme:
    \begin{itemize}
        \item forma parametrica: come span di alcuni vettori, cioe' come immagine di una matrice;
        \item forma cartesiana: come insieme delle soluzioni di un sistema lineare omogeneo, cioe' come kernel di una matrice.
    \end{itemize}
\end{proposition}
\begin{remark}
    Per essere piu' precisi dovremmo parlare di immagine e di kernel dell'applicazione lineare associata alla matrice. 
\end{remark}
\begin{example}
    Consideriamo il sottospazio di $R^3$ generato dall'insieme delle soluzioni dell'equazione $3x + 4y + 5z = 0$ (forma cartesiana) e chiamiamolo $W$.
    Cerchiamo di esprimere $W$ in forma parametrica: \begin{alignat*}{1}
        W &= \left\{ \begin{pmatrix} x \\ y \\ z \end{pmatrix} \in \R^3 \mid 3x + 4y + 5z = 0\right\} \\
        \intertext{Scegliamo $y, z$ libere, da cui segue $x = -\frac{4}{3}y -\frac{5}{3}z$. Sostituendolo otteniamo: }
        &= \left\{ \begin{pmatrix} -\frac{4}{3}y -\frac{5}{3}z \\ y \\ z \end{pmatrix} \mid y, z \in \R \right\}\\
        &= \left\{ y\begin{pmatrix} -\frac{4}{3} \\ 1 \\ 0 \end{pmatrix} + z\begin{pmatrix} -\frac{5}{3} \\ 0 \\ 1 \end{pmatrix} \mid y, z \in \R \right\}\\
        &= \left\{ y\begin{pmatrix} -\frac{4}{3} \\ 1 \\ 0 \end{pmatrix} + z\begin{pmatrix} -\frac{5}{3} \\ 0 \\ 1 \end{pmatrix} \mid y, z \in \R \right\}\\
        &= \Span{\begin{pmatrix} -\frac{4}{3} \\ 1 \\ 0 \end{pmatrix}; \begin{pmatrix} -\frac{5}{3} \\ 0 \\ 1 \end{pmatrix}}
    \end{alignat*}

    Se torniamo indietro notiamo che \begin{alignat*}{1}
        W &= \left\{ \begin{pmatrix} -\frac{4}{3}y -\frac{5}{3}z \\ y \\ z \end{pmatrix} \mid y, z \in \R \right\}\\
        &= \left\{ \begin{pmatrix} -\frac{4}{3}y -\frac{5}{3}z \\ y + 0z \\ 0y+z \end{pmatrix} \mid y, z \in \R \right\}\\
        &= \left\{ \begin{pmatrix} -\frac{4}{3} & -\frac{5}{3} \\ 1 & 0 \\ 0 & 1 \end{pmatrix}\begin{pmatrix} y \\ z \end{pmatrix} \mid y, z \in \R \right\}
    \end{alignat*}
    che e' la definizione di immagine della matrice $\begin{psmallmatrix}
        -\frac{4}{3} & -\frac{5}{3} \\ 1 & 0 \\ 0 & 1 
    \end{psmallmatrix}$.

    Dunque $W = \Imm{\begin{psmallmatrix}
        -\frac{4}{3} & -\frac{5}{3} \\ 1 & 0 \\ 0 & 1 
    \end{psmallmatrix}}$.
\end{example}
\begin{example}
    Consideriamo il sottospazio di $R^3$ generato dallo span dei vettori $\begin{psmallmatrix} 1 \\ 2 \\ 3 \end{psmallmatrix}$, $\begin{psmallmatrix} 4 \\ 5 \\ 6 \end{psmallmatrix}$ (forma parametrica) e chiamiamolo $W$.
    Cerchiamo di esprimere $W$ in forma cartesiana: \begin{alignat*}{1}
        W &= \Span{\begin{pmatrix} 1 \\ 2 \\ 3 \end{pmatrix}, \begin{pmatrix} 4 \\ 5 \\ 6 \end{pmatrix}} \\
            &= \left\{ \begin{pmatrix} x \\ y \\ z \end{pmatrix} \in \R^3 \mid \exists a, b \in \R. \begin{pmatrix} x \\ y \\ z \end{pmatrix} = a\begin{pmatrix} 1 \\ 2 \\ 3 \end{pmatrix} + b\begin{pmatrix} 4 \\ 5 \\ 6 \end{pmatrix} \right\}\\
            &= \left\{ \begin{pmatrix} x \\ y \\ z \end{pmatrix} \in \R^3 \mid \exists a, b \in \R. \begin{pmatrix} x \\ y \\ z \end{pmatrix} = \begin{pmatrix} a + 4b \\ 2a+5b \\ 3a+6b \end{pmatrix}\right\}\\
            &= \left\{ \begin{pmatrix} x \\ y \\ z \end{pmatrix} \in \R^3 \mid \exists a, b \in \R. \begin{pmatrix} x \\ y \\ z \end{pmatrix} = \begin{pmatrix} 1 & 4 \\ 2&5 \\ 3&6 \end{pmatrix} \begin{pmatrix}a \\ b\end{pmatrix}\right\}
    \end{alignat*}
    dunque e' sufficiente capire in che casi il sistema ha soluzione.
    Risolviamo il sistema e imponiamo che non ci siano equazioni impossibili:
    \begin{gather*}
        \begin{pmatrix}[cc|c]
            1&4&x \\ 2&5&y \\ 3&6&z 
        \end{pmatrix} \xrightarrow[R_2 - 2R_1]{R_3 - 3R_1}
        \begin{pmatrix}[cc|c]
            1&4&x \\ 0&-3&y-2x \\ 0&-6&z-3x 
        \end{pmatrix} \xrightarrow[R_3 - 2R_2]{}
        \begin{pmatrix}[cc|c]
            1&4&x \\ 0&-3&y-2x \\ 0&0&x-2y+z 
        \end{pmatrix}
    \end{gather*}
    Dato che non devono esserci equazioni impossibili, segue che tutti i vettori di $W$ sono della forma $\begin{psmallmatrix}x\\y\\z\end{psmallmatrix}$ con $x - 2y + z = 0$. Dunque \[
        W = \left\{ \begin{pmatrix}
            x\\y\\z
        \end{pmatrix}\in \R^3 \mid x-2y+z = 0\right\}    
    \] e' la forma cartesiana di $W$.

    Notiamo che dire che $x, y, z \in \R$ sono tali che $x-2y+z = 0$ e' equivalente a dire che \[
        \begin{pmatrix}
            1 &-2 &1
        \end{pmatrix} \cdot \begin{pmatrix}
            x \\ y \\ z
        \end{pmatrix} = 0
    \]
    cioe' $W$ e' formato da tutti e solo i vettori che fanno parte del kernel della matrice $A = \begin{pmatrix} 1 &-2 &1 \end{pmatrix}$, cioe' $W = \ker \begin{pmatrix} 1 &-2 &1 \end{pmatrix}$.
\end{example}

\subsubsection{Indipendenza e dipendenza lineare}

\begin{definition}
    Sia $V$ uno spazio vettoriale, $\bm{v_1}, \dots, \bm{v_n} \in V$. Allora l'insieme $\left\{ \bm{v_1}, \dots, \bm{v_n} \right\}$ si dice insieme di vettori linearmente indipendenti se
    \begin{equation}
        a_1\bm{v_1} + \dots + a_n\bm{v_n} = \bm{0_V} \iff a_1 = \dots = a_n = 0
    \end{equation}
    cioe' se l'unica combinazione lineare di $\bm{v_1}, \dots, \bm{v_n}$ che da' come risultato il vettore nullo e' quella con $a_1 = \dots = a_n = 0$.
\end{definition}

Possiamo usare una definizione alternativa di dipendenza lineare, equivalente alla precedente, tramite questa proposizione:
\begin{proposition}\label{dip_se_e'_comb_lin}
    Sia $V$ uno spazio vettoriale, $\bm{v_1}, \dots, \bm{v_n} \in V$. Allora l'insieme dei vettori $\left\{ \bm{v_1}, \dots, \bm{v_n} \right\}$ e' linearmente dipendente se e solo se almeno uno di essi e' esprimibile come combinazione lineare degli altri. 
\end{proposition}
\begin{proof}
    Dimostriamo entrambi i versi dell'implicazione.
    \begin{itemize}
        \item Supponiamo che $\left\{ \bm{v_1}, \dots, \bm{v_n} \right\}$ sia linearemente dipendente, cioe' che esistano $a_1, \dots, a_n$ non tutti nulli tali che \[
            a_1\bm{v_1} + a_2\bm{v_2} + \dots + a_n\bm{v_n} = \bm{0_V}   
        .\]
        Supponiamo senza perdita di generalita' $a_1 \neq 0$, allora segue che \[
            \bm{v_1} = -\frac{a_2}{a_1}\bm{v_1} - \dots - \frac{a_n}{a_1}\bm{v_n}
        \]
        dunque $\bm{v_1}$ puo' essere espresso come combinazione lineare degli altri vettori.
        \item Supponiamo che il vettore $\bm{v_1}$ sia esprimibile come combinazione lineare degli altri (senza perdita di generalita'), cioe' che esistano $k_2, \dots, k_n \in \R$ tali che \[
            \bm{v_1} = k_2\bm{v_2} + \dots + k_n\bm{v_n}
        .\]
        Consideriamo una generica combinazione lineare di $v_1, v_2, \dots, v_n$:
        \begin{alignat*}
            {1}
            & a_1\bm{v_1} + a_2\bm{v_2} + \dots + a_n\bm{v_n} \\
            = & a_1(k_2\bm{v_2} + \dots + k_n\bm{v_n}) + a_2\bm{v_2} + \dots + a_n\bm{v_n} \\
            = & (a_1k_2 + a_2)\bm{v_2} + \dots + (a_1k_n + a_n)\bm{v_n}
        \end{alignat*}
        Se scegliamo $a_1 \in \R$ libero, $a_i = -a_1k_i$ per ogni $2 \leq i \leq n$, otterremo
        \begin{alignat*}{1}
            & (a_1k_2 + a_2)\bm{v_2} + \dots + (a_1k_n + a_n)\bm{v_n} \\
            = & (a_1k_2 - a_1k_2)\bm{v_2} + \dots + (a_1k_n - a_1k_n)\bm{v_n} \\
            = & 0\bm{v_2} + \dots + 0\bm{v_n} \\
            = & \bm{0_V}
        \end{alignat*}
        dunque esiste una scelta dei coefficienti $a_1, a_2, \dots, a_n$ diversa da $a_1 = \dots = a_n = 0$ per cui la combinazione lineare da' come risultato il vettore nullo, cioe' l'insieme dei vettori non e' linearmente indipendente. \qedhere
    \end{itemize}
\end{proof}

Inoltre per comodita' spesso si dice che i vettori $\bm{v_1}, \dots, \bm{v_n}$ sono indipendenti, invece di dire che l'insieme formato da quei vettori e' un insieme linearmente indipendente.

\begin{proposition}\label{aggiunto_vettore_indipendente}
    Sia $V$ uno spazio vettoriale, $\bm v \in V$ e $\bm{v_1}, \dots, \bm{v_n} \in V$ indipendenti. Allora i due fatti seguenti sono equivalenti:
    \begin{enumerate}
        \item $\bm v \notin \Span{\bm{v_1}, \dots, \bm{v_n}}$;
        \item $\bm{v_1}, \dots, \bm{v_n}, \bm v$ e' ancora un insieme di vettori linearmente indipendenti.
    \end{enumerate}
\end{proposition}
\begin{proof}
    Dimostriamo entrambi i versi dell'implicazione.
    \begin{itemize}
        \item[($\implies$)] Supponiamo che $\bm v \notin \Span{\bm{v_1}, \dots, \bm{v_n}}$.
        
        Se $\bm{v_1}, \dots, \bm{v_n}, \bm v$ e' un insieme di vettori linearmente indipendenti per definizione l'unica combinazione lineare $a_1\bm{v_1} + \dots + a_n\bm{v_n} + b\bm{v}$ che da' come risultato il vettore nullo $\bm{0}$ deve essere quella con coefficienti tutti nulli.

        Supponiamo per assurdo $b \neq 0$,. Allora \begin{alignat*}{1}
            &\bm 0 = a_1\bm{v_1} + \dots + a_n\bm{v_n} + b\bm{v}\\
            \iff &-b\bm{v} = a_1\bm{v_1} + \dots + a_n\bm{v_n}\\
            \iff &\bm{v} = -\frac{a_1}{b}\bm{v_1} - \dots - \frac{a_n}{b}\bm{v_n}
        \end{alignat*}
        cioe' $v \in \Span{\bm{v_1}, \dots, \bm{v_n}}$, che pero' e' assurdo perche' per ipotesi $\bm v \notin \Span{\bm{v_1}, \dots, \bm{v_n}}$.

        Dunque $b = 0$, cioe' \begin{alignat*}{1}
            &\bm 0 = a_1\bm{v_1} + \dots + a_n\bm{v_n} + b\bm{v}\\
            \iff &\bm 0 = a_1\bm{v_1} + \dots + a_n\bm{v_n}
            \intertext{Tuttavia $\bm{v_1}, \dots, \bm{v_n}$ sono linearmente indipendenti, dunque l'unica scelta dei coefficienti che annulla la combinazione lineare e' quella con tutti i coefficienti nulli:}
            \iff &a_1 = \dots = a_n = b = 0
        \end{alignat*}
        cioe' $\bm{v_1}, \dots, \bm{v_n}, \bm v$ e' ancora un insieme di vettori linearmente indipendenti.
        \item[($\impliedby$)] Supponiamo che $\bm{v_1}, \dots, \bm{v_n}, \bm v$ sia un insieme di vettori linearmente indipendenti. 
        
        Per la proposizione \ref{dip_se_e'_comb_lin} sappiamo che un insieme di vettori e' linearmente dipendente se e solo se almeno uno di essi puo' essere scritto come combinazione lineare degli altri, cioe' se e solo se almeno uno di essi e' nello span degli altri.
        Ma questo e' equivalente a dire che un insieme di vettori e' linearmente indipendente se e solo se nessuno di essi e' nello span degli altri, dunque dato che $\bm{v_1}, \dots, \bm{v_n}, \bm v$ e' un insieme di vettori linearmente indipendenti segue che $\bm{v}$ non puo' appartenere a $\Span{\bm{v_1}, \dots, \bm{v_n}}$. \qedhere
    \end{itemize}
\end{proof}

\begin{proposition} \label{span_Gauss}
    Sia $V$ uno spazio vettoriale e $\bm{v_1}, \dots, \bm{v_n} \in V$. Allora per ogni $k \in \R$ e per ogni $i, j \leq n$.
    \begin{equation}
        \Span{\bm{v_1}, \dots, \bm{v_i}, \bm{v_j}, \dots, \bm{v_n}} = \Span{\bm{v_1}, \dots, \bm{v_i} + k\bm{v_j}, \bm{v_j}, \dots, \bm{v_n}}.
    \end{equation}
\end{proposition}
\begin{proof}
    Supponiamo che $v \in \Span{\bm{v_1}, \dots, \bm{v_i}, \bm{v_j}, \dots, \bm{v_n}}$. Allora per definizione esisteranno $a_1, \dots, a_n \in \R$ tali che
    \begin{alignat*}{1}
        v &= a_1\bm{v_1} + \dots + a_i\bm{v_i} + a_j\bm{v_j} + \dots + a_n\bm{v_n} \\
        \intertext{Aggiungiamo e sottraiamo $a_ik\bm{v_j}$ al secondo membro.}
        &= a_1\bm{v_1} + \dots + a_i\bm{v_i} + a_j\bm{v_j} + \dots + a_n\bm{v_n} + a_ik\bm{v_j} - a_ik\bm{v_j}\\
        &= a_1\bm{v_1} + \dots + a_i\bm{v_i} + a_ik\bm{v_j} + a_j\bm{v_j} - a_ik\bm{v_j} + \dots + a_n\bm{v_n}\\
        &= a_1\bm{v_1} + \dots + a_i(\bm{v_i} + k\bm{v_j}) + (a_j - a_ik)\bm{v_j} + \dots + a_n\bm{v_n}\\
        \implies v &\in \Span{\bm{v_1}, \dots, \bm{v_i} + k\bm{v_j}, \bm{v_j}, \dots, \bm{v_n}}. 
    \end{alignat*}

    Si dimostra l'altro verso nello stesso modo.

    Dunque in entrambi gli insiemi ci sono gli stessi elementi, cioe' i due span sono uguali.
\end{proof}

Notiamo inoltre che se scambiamo due vettori o se moltiplichiamo un vettore per uno scalare otteniamo uno span equivalente a quello di partenza. Quindi possiamo "semplificare" uno span di vettori tramite mosse di Gauss per colonna, come suggerisce la prossima proposizione.

\begin{proposition} \label{span_colonne_indipendenti}
    Siano $\bm{v_1}, \dots, \bm{v_n} \in \R^m$ dei vettori colonna. Allora per stabilire quali di questi vettori sono indipendenti consideriamo la matrice $A$ che contiene come colonna $i$-esima il vettore colonna $v_i$ e riduciamola a scalini per colonna. Lo span delle colonne non nulle della matrice ridotta a scalini e' uguale allo span di $\bm{v_1}, \dots, \bm{v_n}$.
\end{proposition}
\begin{proof} 
    Consideriamo la matrice $\bar{A}$ ridotta a scalini. Allora per la proposizione \ref{span_Gauss} lo span delle sue colonne e' uguale allo span dei vettori iniziali. 

    Tutte le colonne nulle possono essere eliminate da questo insieme, in quanto il vettore nullo e' sempre linearmente dipendente.

    Le colonne rimanenti sono sicuramente linearmente indipendenti: infatti dato che la matrice e' a scalini per colonna per annullare il primo pivot dobbiamo annullare il primo vettore, per annullare il secondo dobbiamo annullare il secondo e cosi' via. Dunque lo span dei vettori colonna non nulli rimanenti e' uguale allo span dei vettori iniziali.
\end{proof}

Notiamo che alla fine di questo procedimento otteniamo vettori colonna che sono diversi dai vettori iniziali, ma questi vettori hanno pivot ad "altezze diverse".

\begin{example}
    Siano $\bm{v_1}, \bm{v_2}, \bm{v_3}, \bm{v_4} \in \R^3$ tali che \[
        \bm{v_1} = \begin{pmatrix}
            1 \\ 2 \\ 3
        \end{pmatrix}, \bm{v_2} = \begin{pmatrix}
            3 \\ 7 \\ 4
        \end{pmatrix}, \bm{v_3} = \begin{pmatrix}
            2 \\ 4 \\ 6
        \end{pmatrix}, \bm{v_4} = \begin{pmatrix}
            -1 \\ 7 \\ 2
        \end{pmatrix}.
    \] Si trovi un insieme di vettori di $\R^3$ indipendenti con lo stesso span di $\bm{v_1}, \bm{v_2}, \bm{v_3}, \bm{v_4}$.
\end{example}
\begin{solution}
    Per la proposizione precedente mettiamo i vettori come colonne di una matrice e semplifichiamola tramite mosse di colonna:
    \begin{gather*}
        \begin{pmatrix}[c|c|c|c]
            1 & 3 & 2 & -1 \\ 2 & 7 & 4 & 7 \\ 3 & 4 & 6 & 2
        \end{pmatrix} \xrightarrow[C_4 + C_1]{C_2 - 3C_1, C_3 - 2C_1} \begin{pmatrix}
            [c|c|c|c]
            1 & 0 & 0 & 0 \\ 2 & 1 & -2 & 1 \\ 3 & -5 & -3 & -7
        \end{pmatrix} \\ 
        \xrightarrow[C_4 - C_2]{C_3 + 2C_2} \begin{pmatrix}
            [c|c|c|c]
            1 & 0 & 0 & 0 \\ 2 & 1 & 0 & 0 \\ 3 & -5 & -13 & -2
        \end{pmatrix} \xrightarrow[]{C_4 - \frac{2}{13}C_3} \begin{pmatrix}
            [c|c|c|c]
            1 & 0 & 0 & 0 \\ 2 & 1 & 0 & 0 \\ 3 & -5 & -13 & 0
        \end{pmatrix}
    \end{gather*}
    Dunque i vettori $\bm{w_1} = \begin{psmallmatrix} 1 \\ 2 \\ 3 \end{psmallmatrix}, \bm{w_2} = \begin{psmallmatrix} 0 \\ 1 \\ -5 \end{psmallmatrix}, \bm{w_3} = \begin{psmallmatrix} 0 \\ 0 \\ -13 \end{psmallmatrix}$ sono indipendenti e per la proposizione precedente vale che \[
        \Span{\bm{v_1}, \bm{v_2}, \bm{v_3}, \bm{v_4}} = \Span{\bm{w_1}, \bm{w_2}, \bm{w_3}}.
    \]
\end{solution}

\section{Generatori e basi}
\begin{definition}
    Sia $V$ uno spazio vettoriale, $\bm{v_1}, \dots, \bm{v_n} \in V$. Allora si dice che ${\bm{v_1}, \dots, \bm{v_n}}$ e' un insieme di generatori di $V$, oppure che l'insieme ${\bm{v_1}, \dots, \bm{v_n}}$ genera $V$, se
    \begin{equation}
        \Span{\bm{v_1}, \dots, \bm{v_n}} = V.
    \end{equation}
\end{definition}

Per comodita' spesso si dice che i vettori $\bm{v_1}, \dots, \bm{v_n}$ sono generatori di $V$, invece di dire che l'insieme formato da quei vettori e' un insieme di generatori.

\begin{definition}
    Sia $V$ uno spazio vettoriale, $\bm{v_1}, \dots, \bm{v_n} \in V$. Allora si dice che $\mathcal{B} = \ang{\bm{v_1}, \dots, \bm{v_n}}$ e' una base di $V$ se
    \begin{itemize}
        \item i vettori $\bm{v_1}, \dots, \bm{v_n}$ generano $V$;
        \item i vettori $\bm{v_1}, \dots, \bm{v_n}$ sono linearmente indipendenti.
    \end{itemize}
\end{definition}

\begin{definition}
    Sia $V$ uno spazio vettoriale. Allora il numero di vettori in una sua base si dice dimensione dello spazio vettoriale $V$, e si indica con $\dim V$.
\end{definition}

Sapendo che un insieme di vettori genera un sottospazio di $\R^n$ (o $\R^n$ stesso) si puo' trovare una base del sottospazio (o di $\R^n$) disponendo i vettori come colonne di una matrice e semplificandoli, come abbiamo visto in precedenza. Tuttavia se vogliamo \textbf{estrarre una base} dal nostro insieme di vettori allora possiamo procedere in un modo leggermente diverso, che utilizza le mosse di Gauss per riga.

\begin{proposition}\label{estrarre_una_base}
    Siano $\bm{v_1}, \dots, \bm{v_m} \in \R^n$ dei vettori che generano $V \subseteq \R^n$ sottospazio di $\R^n$. Allora possiamo porre i vettori come colonne di una matrice e ridurla a scalini per riga. Alla fine del procedimento i vettori che originariamente erano nelle colonne con i pivot sono indipendenti e generano $V$, dunque formano una base di $V$.
\end{proposition}
\begin{proof}
    Consideriamo i $k$ vettori indipendenti che sono nell'insieme $\bm{v_1}, \dots, \bm{v_m}$ e chiamiamoli $\bm{w_1}, \dots, \bm{w_k}$.
    Consideriamo una loro combinazione lineare qualunque $x_1\bm{w_1} + \dots + x_k\bm{w_k}$ e la poniamo uguale a $\bm{0}$; questo e' equivalente a dire \[
        A\begin{pmatrix}
            x_1 \\ \vdots \\ x_k
        \end{pmatrix} = \bm{0}
    \] dove $A$ e' la matrice le cui colonne sono i vettori $\bm{w_1}, \dots, \bm{w_k}$. 
    
    Dato che i $k$ vettori sono indipendenti l'unica soluzione di questo sistema e' il vettore nullo, dunque il sistema ha una sola soluzione e quindi deve avere $0$ variabili libere, cioe' il numero di pivot della matrice ridotta a scalini deve essere uguale al numero di colonne.

    Se aggiungessimo vettori non indipendenti a questo insieme per definizione di dipendenza lineare allora non avremmo piu' una singola soluzione, dunque le colonne che abbiamo aggiunto non possono contenere pivot.
\end{proof}

Notiamo che alla fine del procedimento non otteniamo dei vettori colonna che generano il nostro sottospazio, ma dobbiamo andarli a scegliere dall'insieme iniziale: in questo senso possiamo estrarre una base da un insieme di generatori.

\begin{example}
    Sia $V \subseteq \R^4$ tale che $V = \Span{\bm{c_1}, \bm{c_2}, \bm{c_3}, \bm{c_4}}$ dove \[
        \bm{c_1} = \begin{pmatrix}
            2 \\ 0 \\ 1 \\ 1
        \end{pmatrix}, \bm{c_2} = \begin{pmatrix}
            3 \\ -2 \\ -2 \\ 0
        \end{pmatrix}, \bm{c_3} = \begin{pmatrix}
            1 \\ 0 \\ -1 \\ 1
        \end{pmatrix}, \bm{c_4} = \begin{pmatrix}
            0 \\ 1 \\ -2 \\ \frac{1}{3}
        \end{pmatrix}.
    \] Si estragga una base di $V$ da questi quattro vettori.
\end{example}
\begin{solution}
    Utilizziamo il metodo proposto dalla proposizione precedente. 
    \begin{gather*}
        \begin{pmatrix}
            2&3&1&0\\0&-2&0&1\\1&-2&-1&-2\\1&0&\frac{1}{3}&\frac13
        \end{pmatrix} \xrightarrow[R_4 - \frac{1}{2}R_2]{R_3 - \frac{1}{2}R_2}
        \begin{pmatrix}
            2&3&1&0\\0&-2&0&1\\0&-\frac72&-\frac32&-2\\0&-\frac{3}{2}&\frac{1}{6}&\frac13
        \end{pmatrix} \xrightarrow[R_4 \times 6]{R_2\times \frac12, R_3\times 2} \\
        \begin{pmatrix}
            2&3&1&0\\0&-1&0&\frac12\\0&-7&-3&-2\\0&-9&1&2
        \end{pmatrix} \xrightarrow[R_4-9R_2]{R_3 -7R_2} 
        \begin{pmatrix}
            2&3&1&0\\0&-1&0&\frac12\\0&0&-3&-\frac{15}{2}\\0&0&-1&-\frac{5}{2}
        \end{pmatrix}\\ \xrightarrow[]{R_4 -\frac13R_3} 
        \begin{pmatrix}
            2&3&1&0\\0&-1&0&\frac12\\0&0&-3&-\frac{15}{2}\\0&0&0&0
        \end{pmatrix}.
    \end{gather*}
    Notiamo dunque che i pivot sono nelle colonne $1$, $2$ e $3$, che corrispondono ai vettori $\bm{c_1}, \bm{c_2}, \bm{c_3}$ che per la proposizione precedente sono indipendenti e generano $V$, dunque $\ang{\bm{c_1}, \bm{c_2}, \bm{c_3}}$ e' una base di $V$.
\end{solution}

\begin{proposition}\label{base=dim_gener_indip}
    Sia $V$ uno spazio vettoriale e sia $\left\{\bm{v_1}, \dots, \bm{v_n} \right\}$ un insieme di $n$ vettori di $V$. Se valgono due dei seguenti fatti
    \begin{itemize}
        \item $n = \dim V$;
        \item $\left\{\bm{v_1}, \dots, \bm{v_n} \right\}$ e' un insieme di generatori di $V$;
        \item $\left\{\bm{v_1}, \dots, \bm{v_n} \right\}$ sono linearmente indipendenti;
    \end{itemize}
    allora vale anche il terzo e $\ang{\bm{v_1}, \dots, \bm{v_n}}$ e' una base di $V$.
\end{proposition}

\begin{example}
    Consideriamo lo spazio dei polinomi di grado minore o uguale a due $\R[x]^{\leq 2}$. Mostrare che $\alpha = \ang{1, (x-1), (x-1)^2}$ e' una base di $\R[x]^{\leq 2}$.
\end{example}
\begin{solution}
    Sappiamo che la base standard di $\R[x]^{\leq 2}$ e' la base $\ang{1, x, x^2}$, dunque $\dim \left( \R[x]^{\leq 2} \right) = 3$. Dato che la base $\alpha$ ha esattamente $3$ vettori, per la proposizione \ref{base=dim_gener_indip} ci basta dimostrare una tra:
    \begin{itemize}
        \item i tre vettori sono indipendenti;
        \item i tre vettori generano $\R[x]^{\leq 2}$.
    \end{itemize}
    Per esercizio, verifichiamole entrambe.
    \begin{itemize}
        \item Verifichiamo che sono linearmente indipendenti: consideriamo una generica combinazione lineare dei tre vettori e poniamola uguale al vettore $\bm{0} = 0 + 0x + 0x^2$. \begin{alignat*}{1}
            a \cdot 1 + b \cdot (x - 1) + c \cdot (x - 1)^2 &= 0+0x+0x^2 \\
            \iff a + bx - b + cx^2 -2cx + c &= 0+0x+0x^2 \\
            \iff (a-b+c) +(b-2c)x + cx^2 &= 0+0x+0x^2
        \end{alignat*} Dunque $a, b, c$ devono soddisfare il seguente sistema:
        \begin{equation*}
            \left\{
            \begin{array}{@{}rororor }
            a & - & b & + & c  & = & 0 \\
              &   & b & - & 2c & = & 0 \\
              &   &   &   & c  & = & 0 \\
            \end{array}  
            \right.
        \end{equation*}
        che ha soluzione solo per $a = b = c = 0$. Dunque i tre vettori sono indipendenti e, sapendo che $\dim \left( \R[x]^{\leq 2} \right) = 3$, sono una base di $\R[x]^{\leq 2}$.
        \item Verifichiamo che i tre vettori generano $\R[x]^{\leq 2}$. Un modo per farlo e' verificare che i vettori che compongono la base canonica di $\R[x]^{\leq 2}$ sono nello span di $\{1, (x-1), (x-1)^2\}$: infatti, dato che la base canonica genera tutto lo spazio, se essa e' nello span anche tutto il resto dello spazio sara' nello span dei nostri tre vettori.
        \begin{itemize}
            \item $1 = 1\cdot 1 + 0\cdot (x-1) + 0 \cdot (x-1)^2$, dunque $1 \in \Span{1, (x-1), (x-1)^2}$
            \item $x = 1\cdot 1 + 1\cdot (x-1) + 0 \cdot (x-1)^2$, dunque $x \in \Span{1, (x-1), (x-1)^2}$
            \item Dato che non e' immediato vedere come scrivere $x^2$ in termini di $1, (x-1), (x-1)^2$ cerchiamo di trovare i coefficienti algebricamente:
            \begin{alignat*}{1}
                x^2 &= a \cdot 1 + b\cdot (x-1) + c \cdot (x-1)^2 \\
                    &= (a-b+c) +(b-2c)x + cx^2
            \end{alignat*}
            dunque uguagliando i coefficienti dei termini dello stesso grado otteniamo
            \begin{equation*}
                \left\{\begin{array}{@{}rororor }
                    a & - & b & + & c  & = & 0 \\
                      &   & b & - & 2c & = & 0 \\
                      &   &   &   & c  & = & 1 \\
                \end{array} \right. \implies 
                \left\{\begin{array}{@{}rororor }
                    a & = & 1 \\
                    b & = & 2 \\
                    c & = & 1 \\
                \end{array} \right.
            \end{equation*}
            Quindi $x^2$ e' esprimibile come combinazione lineare di $1, (x-1), (x-1)^2$ (in particolare $x^2 = 1 + 2(x-1) + (x-1)^2$), dunque \[x^2 \in \Span{1, (x-1), (x-1)^2}.\]
        \end{itemize}
        Abbiamo quindi verificato che i vettori che formano la base canonica di $\R[x]^{\leq 2}$ fanno parte dello span dei nostri tre vettori, dunque se la base canonica genera tutto lo spazio anche $\{1, (x-1), (x-1)^2\}$ sono generatori. Inoltre, dato che $\dim (\R[x]^{\leq 2}) = 3$ segue che $\ang{1, (x-1), (x-1)^2}$ e' una base di $\R[x]^{\leq 2}$.
    \end{itemize}
\end{solution}

\begin{definition}
    Sia $V$ uno spazio vettoriale, $\bm{v} \in V$ e $\mathcal{B} = \ang{\bm{v_1}, \dots, \bm{v_n}}$ una base di $V$. Allora si dice vettore delle coordinate di $\bm{v}$ rispetto a $\mathcal{B}$ il vettore colonna
    \begin{equation}
        [\bm{v}]_{\mathcal{B}} = \begin{bmatrix}
                                    a_1 \\
                                    a_2 \\
                                    \vdots \\
                                    a_n
                                 \end{bmatrix} \in \R^n
    \end{equation}
    tale che \begin{equation}
        \bm{v} = a_1\bm{v_1} + \dots + a_n\bm{v_n}
    \end{equation}
\end{definition}

\begin{proposition}
    Sia $V$ uno spazio vettoriale, $\bm{v} \in V$ e $\mathcal{B} = \ang{\bm{v_1}, \dots, \bm{v_n}}$ una base di $V$. Allora le coordinate di $\bm{v}$ rispetto a $\mathcal{B}$ sono uniche.
\end{proposition}
\begin{proof}
    Supponiamo per assurdo che esistano due vettori colonna distinti $\bm{a}$, $\bm{b}$ che rappresentino le coordinate di $\bm{v}$ rispetto a $\mathcal{B}$. Allora
    \begin{alignat*}
        {1}
        \bm{0_V}  &= \bm{v} - \bm{v} \\
                &= (a_1\bm{v_1} + \dots + a_n\bm{v_n}) - (b_1\bm{v_1} + \dots + b_n\bm{v_n}) \\
                &= (a_1 - b_1)\bm{v_1} + \dots + (a_n - b_n)\bm{v_n}
    \end{alignat*}
    Ma per definizione di base $\bm{v_1}, \dots, \bm{v_n}$ sono linearmente indipendenti, dunque l'unica combinazione lineare che da' come risultato il vettore $\bm{0_V}$ e' quella in cui tutti i coefficienti sono $0$. Da cio' segue che
    \begin{gather*}
        a_1 - b_1 = a_2 - b_2 = \dots = a_n - b_n = 0 \\
        \implies \bm{a} = \begin{bmatrix}
            a_1 \\
            a_2 \\
            \vdots \\
            a_n
        \end{bmatrix}
        = 
        \begin{bmatrix}
            b_1 \\
            b_2 \\
            \vdots \\
            b_n
        \end{bmatrix} = \bm{b}
    \end{gather*}
    cioe' i due vettori sono uguali. Ma cio' e' assurdo poiche' abbiamo supposto $\bm{a} \neq \bm{b}$, dunque le coordinate di $\bm{v}$ rispetto a $\mathcal{B}$ devono essere uniche.
\end{proof}

\begin{example}
    Sia $V \subseteq \M_{2 \times 2}(\R)$ tale che $V$ e' il sottospazio delle matrici simmetriche. Trovare una base di $V$ e trovare le coordinate di $\bm{u} = \begin{psmallmatrix}
        3 & 4 \\ 4 & 6
    \end{psmallmatrix} \in V$ rispetto alla base trovata.
\end{example}
\begin{solution}
    Cerco di esprimere un generico vettore $\bm{v} \in V$ in termini della condizione che definisce il sottospazio.
    \begin{alignat*}
        {1}
        V &= \left\{ \begin{pmatrix} a&b\\b&c \end{pmatrix}\mid a, b, c \in \R\right\} \\
        \intertext{Isolando i contributi di $a$, $b$ e $c$ ottengo}
        &= \left\{ \bm{v} \in \M_{2 \times 2}(\R) \mid \exists a, b, c \in \R. \bm{v} = \begin{pmatrix} a&0\\0&0 \end{pmatrix} + \begin{pmatrix} 0&b\\b&0 \end{pmatrix} + \begin{pmatrix} 0&0\\0&c \end{pmatrix}\right\} \\
        &= \left\{ \bm{v} \in \M_{2 \times 2}(\R) \mid \exists a, b, c \in \R. \bm{v} = a\begin{pmatrix} 1&0\\0&0 \end{pmatrix} + b\begin{pmatrix} 0&1\\1&0 \end{pmatrix} + c\begin{pmatrix} 0&0\\0&1 \end{pmatrix}\right\} \\
        &= \Span{\begin{pmatrix} 1&0\\0&0 \end{pmatrix}, \begin{pmatrix} 0&1\\1&0 \end{pmatrix}, \begin{pmatrix} 0&0\\0&1 \end{pmatrix}} \\
    \end{alignat*}

    Ora dobbiamo mostrare che $\begin{psmallmatrix} 1&0\\0&0 \end{psmallmatrix}, \begin{psmallmatrix} 0&1\\1&0 \end{psmallmatrix}, \begin{psmallmatrix} 0&0\\0&1 \end{psmallmatrix}$ sono linearmente indipendenti. Consideriamo una loro generica combiazione lineare e imponiamola uguale a $0$:
    \begin{gather*}
        x\begin{pmatrix} 1&0\\0&0 \end{pmatrix} + y\begin{pmatrix} 0&1\\1&0 \end{pmatrix} + z\begin{pmatrix} 0&0\\0&1 \end{pmatrix} = \begin{pmatrix} 0&0\\0&0 \end{pmatrix} \\
        \iff \begin{pmatrix} x&y\\y&z \end{pmatrix} = \begin{pmatrix} 0&0\\0&0 \end{pmatrix} \\
        \iff x = y = z = 0.
    \end{gather*}
    Dunque $\mathcal{B} = \ang{\begin{psmallmatrix} 1&0\\0&0 \end{psmallmatrix}, \begin{psmallmatrix} 0&1\\1&0 \end{psmallmatrix}, \begin{psmallmatrix} 0&0\\0&1 \end{psmallmatrix}}$ e' una base di $V$.

    Per trovare le coordinate di $\bm{u}$ esprimiamo in termini della base:
    \begin{equation*}
        \begin{pmatrix}
            3 & 4 \\ 4 & 6
        \end{pmatrix} = 3\begin{pmatrix} 1&0\\0&0 \end{pmatrix} + 4\begin{pmatrix} 0&1\\1&0 \end{pmatrix} + 6\begin{pmatrix} 0&0\\0&1 \end{pmatrix}
    \end{equation*}
    dunque $[\bm{u}]_{\mathcal{B}} = \begin{pmatrix}
        3 \\ 4 \\ 6
    \end{pmatrix}$.
\end{solution}

Notiamo che sembra esserci una relazione biunivoca tra un vettore di $V$ e le sue coordinate in $\R^n$ rispetto ad una base. Infatti (come vedremo nella prossima parte) la relazione tra vettore di $V$ e vettore colonna delle sue coordinate e' un isomorfismo: essi rappresentano lo stesso oggetto sotto forme diverse. Quindi spesso per fare calcoli (ad esempio semplificare un insieme di vettori per trovare una base) possiamo passare allo spazio isomorfo $\R^n$, sfruttare i vettori colonna e le matrici (ad esempio facendo mosse di Gauss per riga o per colonna) e infine passare di nuovo allo spazio originale.

Abbiamo mostrato come estrarre una base di un sottospazio a partire da un insieme di generatori. Ora vogliamo \textbf{completare una base} di un sottospazio ad una base dello spazio vettoriale che lo contiene.

\begin{theorem}
    [del completamento ad una base] \label{th_completamento}
    Sia $V$ uno spazio vettoriale di dimensione finita $n = \dim V$ e sia $\mathcal{B} = \ang{\bm{v_1}, \dots, \bm{v_k}}$ un insieme di $k$ vettori linearmente indipendenti. Allora vale che $k \leq n$ ed esistono $n - k$ vettori di $V$, diciamo ${\bm{w_1}, \dots, \bm{w_{n-k}}}$ tali che $\mathcal{B}' = \ang{\bm{v_1}, \dots, \bm{v_k}, \bm{w_1}, \dots, \bm{w_{n-k}}}$ e' una base di $V$.
\end{theorem}
\begin{proof}
    Non possono esserci piu' di $n$ vettori indipendenti in uno spazio di dimensione $n$, dunque $k \leq n$. Ora dimostriamo che possiamo completare $\mathcal{B}$ ad una base di $V$.

    Se $k = n$ allora per la proposizione \ref{base=dim_gener_indip} gli $n$ vettori indipendenti sono gia' una base, dunque abbiamo finito.

    Se $k < n$ allora esistera' sicuramente $\bm{w_1} \notin \Span{\bm{v_1}, \dots, \bm{v_k}}$ (altrimenti i vettori genererebbero l'intero spazio vettoriale e sarebbero quindi una base), dunque $\ang{\bm{v_1}, \dots, \bm{v_k}, \bm{w_1}}$ sono ancora indipendenti.
    
    Continuiamo a ripetere questo processo fino a quando l'insieme di vettori non genera l'intero spazio vettoriale $V$. Sia $\mathcal{B}'$ l'insieme creato tramite questo processo. Allora $\mathcal{B}'$ e' un insieme di vettori indipendenti che generano $V$, dunque e' una base di $V$, dunque dovra' contenere $n$ vettori. Ma dato che inizialmente avevamo $k$ vettori, per completare ad una base di $V$ abbiamo dovuto aggiungere $n-k$ vettori di $V$.
\end{proof}

\subsubsection{Procedimento per completare ad una base di $\R^n$}

Sia $V = \R^n$ e siano $\left\{ \bm{v_1}, \dots, \bm{v_k} \right\}$ indipendenti.

Allora formo la matrice $M$ che ha come colonne i vettori $\bm{v_1}, \dots, \bm{v_k}$ e la riduco a scalini per colonna tramite mosse di Gauss di colonna, ottenendo una matrice $M'$ che ha come colonne i vettori $\bm{v'_1}, \dots, \bm{v'_k}$.

Questi vettori sono indipendenti (poiche' le mosse di colonna non modificano lo span) e sono a scalini, dunque dovranno avere pivot su righe diverse, e dovranno averne esattamente $k \leq n$. Allora aggiungo $n - k$, ognuno con un pivot su una riga diversa da quelle gia' occupate: la matrice finale sara' una matrice quadrata con $n$ pivot, dunque sara' formata da colonne indipendenti che formano una base di $\R^n$.

\begin{example}
    Sia $V = \R^4$, $A \subseteq V$ tale che \[
        A = \Span{\begin{pmatrix}2\\0\\1\\1\end{pmatrix}, \begin{pmatrix}3\\-2\\-2\\0\end{pmatrix}, \begin{pmatrix}1\\0\\-1\\\frac13\end{pmatrix}, \begin{pmatrix}0\\1\\-2\\\frac13\end{pmatrix}}.    
    \] Trovare una base di $A$ e completarla ad una base di $\R^4$.
\end{example}
\begin{solution}
    Troviamo una base di $A$ tramite mosse di colonna:
    \begin{gather*}
        \begin{pmatrix}[c|c|c|c]
            2&3&1&0\\0&-2&0&1\\1&-2&-1&-2\\1&0&\frac13&\frac13
        \end{pmatrix} \xrightarrow[]{\text{scambio}}
        \begin{pmatrix}[c|c|c|c]
            1&0&3&2\\0&1&-2&0\\-1&-2&-2&1\\\frac13&\frac13&0&1
        \end{pmatrix} \xrightarrow[C_4 - 2C_1]{C_3 - 3C_1} \\
        \xrightarrow[C_4 - 2C_1]{C_3 - 3C_1} \begin{pmatrix}[c|c|c|c]
            1&0&0&0\\0&1&-2&0\\-1&-2&1&3\\\frac13&\frac13&-1&\frac13
        \end{pmatrix} \xrightarrow[]{C_3 + 2C_1}
        \begin{pmatrix}[c|c|c|c]
            1&0&0&0\\0&1&0&0\\-1&-2&-3&3\\\frac13&\frac13&-\frac13&\frac13
        \end{pmatrix} \xrightarrow[]{C_4 + C_3} \\ \xrightarrow[]{C_4 + C_3}
        \begin{pmatrix}[c|c|c|c]
            1&0&0&0\\0&1&0&0\\-1&-2&-3&0\\\frac13&\frac13&-\frac13&0
        \end{pmatrix}
    \end{gather*}
    Dunque una base di $A$ e' formata dai vettori $\ang{\begin{psmallmatrix}1\\0\\-1\\\frac13 \end{psmallmatrix}, \begin{psmallmatrix}0\\1\\-2\\\frac13 \end{psmallmatrix}, \begin{psmallmatrix}0\\0\\-3\\-\frac13 \end{psmallmatrix} }$.

    Notiamo che i pivot di questi vettori sono ad altezza $1$, $2$ e $3$, dunque per completare ad una base di $\R^4$ basta aggiungere un vettore che ha un pivot ad altezza $4$, come $\begin{psmallmatrix} 0\\0\\0\\1 \end{psmallmatrix}$.

    Dunque abbiamo completato la base di $A$ alla seguente base di $\R^4$: \[
        \ang{
            \begin{pmatrix}1\\0\\-1\\\frac13 \end{pmatrix}, 
            \begin{pmatrix}0\\1\\-2\\\frac13 \end{pmatrix}, 
            \begin{pmatrix}0\\0\\-3\\-\frac13 \end{pmatrix},
            \begin{pmatrix} 0\\0\\0\\1 \end{pmatrix}
        }.      
    \]
\end{solution}

\section{Sottospazi somma e intersezione}

\begin{definition}
    Sia $V$ uno spazio vettoriale e siano $A, B \subseteq V$ due sottospazi di $V$. Allora sono sottospazi vettoriali di $V$:
    \begin{subequations}
        \begin{equation}
            A \cap B = \left\{ \bm{v} \in V \mid \bm v \in A, \bm v \in B\right\}
        \end{equation}
        \begin{equation}
            A + B = \left\{ (\bm{v} + \bm{w}) \in V \mid \bm v \in A, \bm w \in B\right\}
        \end{equation}
    \end{subequations}
\end{definition}

\begin{remark}
    Possiamo verificare molto semplicemente che i due spazi sopra sono effettivamente sottospazi di $V$. Inoltre $A \cup B$ non e' un sottospazio vettoriale, ma possiamo notare che $(A \cup B) \subset (A + B)$ in quanto $A \subset A + B$ e $B \subset A + B$.
\end{remark}

\begin{proposition}
    Sia $V$ uno spazio vettoriale e siano $A, B \subseteq V$ due sottospazi di $V$ tali che $A = \Span{\bm{v_1}, \dots, \bm{v_n}}$ e $B = \Span{\bm{w_1}, \dots, \bm{w_m}}$. Allora \begin{equation}
        A + B = \Span{\bm{v_1}, \dots, \bm{v_n}, \bm{w_1}, \dots, \bm{w_m}}.
    \end{equation}
\end{proposition}
\begin{proof}
    Consideriamo un generico $\bm u \in A + B$. Allora per definizione di $A + B$ segue che $\bm u = \bm v + \bm w$ per qualche $\bm v \in A, \bm w \in B$.
    Dato che $A = \Span{\bm{v_1}, \dots, \bm{v_n}}$ e $B = \Span{\bm{w_1}, \dots, \bm{w_m}}$, allora possiamo scrivere 
    \begin{alignat*}{1}
        &\bm v = a_1\bm{v_1} + \dots + a_n\bm{v_n},  \quad \bm w = b_1\bm{w_1} + \dots + b_n\bm{w_m} \\
        \intertext{per qualche $a_1, \dots, a_n, b_1, \dots, b_m \in \R$. Quindi $\bm u = \bm v + \bm w$ diventa}
        \implies &\bm v + \bm w = a_1\bm{v_1} + \dots + a_n\bm{v_n} + b_1\bm{w_1} + \dots + b_n\bm{w_m}
    \end{alignat*}
    dunque ogni vettore in $A + B$ puo' essere scritto come combinazione lineare di $\bm{v_1}, \dots, \bm{v_n}, \bm{w_1}, \dots, \bm{w_m}$, dunque questi vettori generano $A + B$. 
\end{proof}

\begin{remark}
    I vettori $\bm{v_1}, \dots, \bm{v_n}, \bm{w_1}, \dots, \bm{w_m}$ \textbf{generano} $A+B$ ma non sono una base: dobbiamo prima assicurarci che siano linearmente indipendenti.
\end{remark}

\begin{definition}
    Sia $V$ uno spazio vettoriale e siano $A, B \subseteq V$ due sottospazi di $V$. Allora il sottospazio somma $A + B$ si dice in somma diretta se per ogni $\bm v \in A$, $\bm w \in B$ allora $\bm v, \bm w$ sono indipendenti. Se la somma e' diretta scrivo $A \oplus B$.
\end{definition}

\begin{proposition}
    Sia $V$ uno spazio vettoriale e siano $A, B \subseteq V$ due sottospazi di $V$. Allora il sottospazio somma $A + B$ e' in somma diretta se e solo se $A \cap B = \{\bm 0\}$.
\end{proposition}
\begin{proof}
    Innanzitutto notiamo che $\bm{0} \in A$ e $\bm 0 \in B$, dunque $\{\bm 0\} \subseteq A \cap B$.
    \begin{itemize}
        \item[($\implies$)] Supponiamo $A \oplus B$. 
        
        Allora supponiamo per assurdo che esista $\bm u \in A \cap B$ non nullo. Per definizione di intersezione segue che $\bm u \in A$ e $\bm u \in B$, ma questo significa che in $A$ e in $B$ ci sono almeno due vettori $\bm v \in A$ e $\bm w \in B$ non indipendenti tra loro: basta scegliere $\bm v = \bm u$ e $\bm w = \bm u$. 
        
        Tuttavia questo e' assurdo poiche' abbiamo assunto che $A$ e $B$ siano in somma diretta, dunque non puo' esserci un $\bm u \in A \cap B$ non nullo, dunque $A \cap B = \{\bm 0\}$.
        \item[($\impliedby$)] Supponiamo che $A \cap B = \{\bm 0\}$. 
        
        Siano $\bm v \in A$, $\bm{w} \in B$ entrambi non nulli. Per dimostrare che $A$ e $B$ sono in somma diretta e' sufficiente dimostrare che sono necessariamente indipendenti, cioe' che l'unica combinazione lineare $a\bm v + b\bm w$ che e' uguale a $\bm 0$ e' quella con $a = b = 0$. 
        
        Notiamo che $a\bm v + b\bm w = \bm 0$ se e solo se $a\bm v = -b\bm w$; ma dato che i due vettori (che fanno parte di sottospazi diversi) sono uguali segue che devono entrambi far parte del sottospazio intersezione, cioe' $a\bm v, -b\bm w \in A \cap B$.
        
        Per ipotesi $A \cap B = \{\bm 0\}$, dunque $a\bm v = -b\bm w = \bm 0$. Inoltre abbiamo assunto che i vettori $\bm v$ e $\bm{w}$ siano non nulli, dunque segue che $a = b = 0$, come volevasi dimostrare. \qedhere
    \end{itemize}
\end{proof}

\begin{theorem}
    [di Grassman] \label{th_grassman}
    Sia $V$ uno spazio vettoriale e $A, B \subseteq V$ due sottospazi. Allora \begin{equation}
        \dim(A + B) = \dim A + \dim B - \dim(A \cap B).
    \end{equation}
\end{theorem}
\begin{proof}
    Consideriamo una base $\gamma = \ang{\bm{u_1}, \dots, \bm{u_k}}$ di $A \cap B$.

    Dato che $A \cap B$ e' un sottospazio sia di $A$ che di $B$, allora per il teorema del completamento ad una base (\ref{th_completamento}) possiamo completarla ad una base $\alpha = \ang{\bm{u_1}, \dots, \bm{u_k}, \bm{v_1}, \dots, \bm{v_n}}$ di $A$ e ad una base $\beta = \ang{\bm{u_1}, \dots, \bm{u_k}, \bm{w_1}, \dots, \bm{w_m}}$ di $B$.

    Dimostriamo che $\ang{\bm{u_1}, \dots, \bm{u_k}, \bm{v_1}, \dots, \bm{v_n}, \bm{w_1}, \dots, \bm{w_m}}$ e' una base di $A + B$.

    \begin{itemize}
        \item Mostriamo che $\{\bm{u_1}, \dots, \bm{u_k}, \bm{v_1}, \dots, \bm{v_n}, \bm{w_1}, \dots, \bm{w_m}\}$ generano $A + B$.
        
        Sia $\bm u \in A + B$ generico. Allora esistono $\bm v \in A, \bm w \in B$ tali che $\bm{u} = \bm{v} + \bm{w}$. Dato che $\alpha$ e' una base di $A$ e $\beta$ e' una base di $B$ allora possiamo scrivere $\bm v$ e $\bm w$ come \begin{alignat*}{2}
            \bm{v} &=\ && a_1\bm{u_1} + \dots + a_k\bm{u_k} + a_{k+1}\bm{v_1} + \dots + a_{k+n}\bm{v_n} \\
            \bm{w} &=\ && b_1\bm{u_1} + \dots + b_k\bm{u_k} + b_{k+1}\bm{w_1} + \dots + b_{k+m}\bm{w_m} \\
            \intertext{dunque}
            \bm{u} &=\ && \bm{v} + \bm{w}\\
                &=\ && a_1\bm{u_1} + \dots + a_k\bm{u_k} + a_{k+1}\bm{v_1} + \dots + a_{k+n}\bm{v_n} + \\
                & && + b_1\bm{u_1} + \dots + b_k\bm{u_k} + b_{k+1}\bm{w_1} + \dots + b_{k+m}\bm{w_m} \\
                &=\ && (a_1 + b_1)\bm{u_1} + \dots + (a_k + b_k)\bm{u_k} + \\
                & && + a_{k+1}\bm{v_1} + \dots + a_{k+n}\bm{v_n} + \\
                & && + b_{k+1}\bm{w_1} + \dots + b_{k+m}\bm{w_m}.
        \end{alignat*}
        Dunque ogni elemento di $A + B$ puo' essere scritto come combinazione lineare di $\bm{u_1}, \dots, \bm{u_k}, \bm{v_1}, \dots, \bm{v_n}, \bm{w_1}, \dots, \bm{w_m}$, cioe' essi sono generatori di $A + B$.

        \item Mostriamo che $\{\bm{u_1}, \dots, \bm{u_k}, \bm{v_1}, \dots, \bm{v_n}, \bm{w_1}, \dots, \bm{w_m}\}$ e' un insieme di vettori linearmente indipendenti.
        
        Consideriamo una combinazione lineare dei vettori $\bm{u_1}, \dots, \bm{u_k}$, $\bm{v_1}, \dots, \bm{v_n}$, $\bm{w_1}, \dots, \bm{w_m}$ e verifichiamo che essa e' uguale al vettore $\bm{0}$ se e solo se tutti i coefficienti sono uguali a $0$.

        \begin{alignat*}{1}
            &\bm{0} = x_1\bm{v_1} + \dots + x_n\bm{v_n} + y_1\bm{u_1} + \dots + y_k\bm{u_k} + z_1\bm{w_1} + \dots +z_m\bm{w_m} \\
            \iff &x_1\bm{v_1} + \dots + x_n\bm{v_n} = -(y_1\bm{u_1} + \dots + y_k\bm{u_k} + z_1\bm{w_1} + \dots +z_m\bm{w_m}).
        \end{alignat*}
        Notiamo che il primo membro e' un vettore del sottospazio $A$, mentre il secondo membro e' un vettore del sottospazio $B$: dato che i due vettori sono uguali allora devono trovarsi in entrambi i sottospazi e dunque anche nel sottospazio $A \cap B$. Dato che $\gamma$ e' una base di $A \cap B$ possiamo scrivere \begin{alignat*}{1}
            &x_1\bm{v_1} + \dots + x_n\bm{v_n} = a_1\bm{u_1} + \dots + a_k\bm{u_k} \\
            \iff &x_1\bm{v_1} + \dots + x_n\bm{v_n} - a_1\bm{u_1} - \dots - a_k\bm{u_k} = \bm{0}
            \intertext{ma $\bm{u_1}, \dots, \bm{u_k}, \bm{v_1}, \dots, \bm{v_n}$ formano una base di $A$, dunque devono essere indipendenti, quindi per definizione segue che}
            \iff &x_1 = \dots = x_n = 0
        \end{alignat*}
        Dunque nella combinazione lineare i termini con $\bm{v_i}$ scompaiono, e rimangono solo
        \begin{alignat*}{1}
            &\bm{0} = y_1\bm{u_1} + \dots + y_k\bm{u_k} + z_1\bm{w_1} + \dots +z_m\bm{w_m}
            \intertext{ma questi vettori formano la base $\beta$ di $B$, dunque devono essere indipendenti, cioe' per definizione}
            \iff &y_1 = \dots = y_k = z_1 = \dots = z_m = 0.
        \end{alignat*}

        Segue quindi che $\{\bm{u_1}, \dots, \bm{u_k}, \bm{v_1}, \dots, \bm{v_n}, \bm{w_1}, \dots, \bm{w_m}\}$ e' un insieme di vettori linearmente indipendenti.
    \end{itemize}
    Dato che l'insieme $\{\bm{u_1}, \dots, \bm{u_k}, \bm{v_1}, \dots, \bm{v_n}, \bm{w_1}, \dots, \bm{w_m}\}$ e' un insieme di vettori linearmente indipendenti e genera $A+B$, allora esso e' una base di $A + B$.

    Dunque \begin{alignat*}
        {1}
        \dim(A + B) &= n + k + m\\ 
                &= (n + k) + (m + k) - k \\
                &= \dim A + \dim B - dim(A \cap B)
    \end{alignat*}
    come volevasi dimostrare.
\end{proof}



\chapter{Dinamica}

La dinamica e' la branca della fisica che si occupa di studiare le cause delle variazioni del moto. Ci occuperemo inizialmente della dinamica del punto materiale, cioe' il moto del corpo e le forze che agiscono su di esso sono riferiti ad un punto $(x, y, z)$ dello spazio; passeremo poi a studiare la dinamica di sistemi con piu' gradi di liberta'.

\begin{definition}
    Si dice massa inerziale di un corpo la resistenza che il corpo oppone al cambiamento di velocita' risultante dall'azione di una forza.    
\end{definition}
La massa e' una grandezza scalare e additiva, la cui unita' di misura e' il chilogrammo.

\begin{definition}
    La forza e' una quantita' vettoriale che descrive le interazioni fra corpi.
\end{definition}
La forza e' rappresentata da un vettore applicato, cioe' viene descritta da intensita', direzione, verso e punto di applicazione. Da un punto di vista pratico, per misurare una forza si sfrutta la proprieta' che esa ha di deformare gli oggetti.

Vi sono due tipi di forze, distinte dal loro raggio di azione.
\begin{enumerate}
    \item Le forze a distanza (o forze a lungo raggio) non richiedono che i corpi su cui agiscono siano a contatto tra loro. Gli esempi piu' comuni sono le 4 iterazioni fondamentali:
        \begin{itemize}
            \item iterazione gravitazionale (mediata dal gravitone (forse));
            \item iterazione elettromagnetica (mediata dai fotoni);
            \item iterazione nucleare debole (mediata dai bosoni W e Z);
            \item iterazione nucleare forte (mediata dai gluoni);
        \end{itemize}
    \item Le forze a contatto (o forze a corto raggio) sono forze che agiscono a contatto tra i corpi macroscopici, e derivano dalle interazioni elettromagnetiche fra atomi e molecole che costituiscono la materia. Alcuni esempi sono:
        \begin{itemize}
            \item forze esplicate dai vincoli (tensione di fili; forza normale associata ad una superficie che si oppone alla deformazione);
            \item forze di attrito (dinamico e statico);
            \item forze elastiche (interazioni elettromagnetiche che si oppongono alle deformazioni dei corpi).
        \end{itemize}
\end{enumerate}

\section{Principi della dinamica}

\begin{principle}[Primo principio della dinamica]
    Sia $\bvv{R}$ la risultante delle forze che agiscono su un punto materiale. Allora se $\bvv{R} = \bvv{0}$ allora il corpo permane nel suo stato di quiete o moto rettilineo uniforme.
\end{principle}

\begin{definition}
    Si dice che un sistema di riferimento e' inerziale se dato un corpo e appurato che la risultante delle forze che agiscono su quel corpo e' nulla, allora il corpo e' in quiete o si muove di moto rettilineo uniforme rispetto al sistema.
\end{definition}

Segue dalla definizione che un sistema in moto rettilineo uniforme rispetto a un sistema inerziale e' anch'esso inerziale.

\begin{principle}[Secondo principio della dinamica]
    Sia $\bvv{R}$ la risultante delle forze che agiscono su un punto materiale. Allora se $\bvv{R} \neq \bvv{0}$ allora il corpo subisce un'accelerazione che e' direttamente proporzionale alla risultante delle forze. La costante di proporzionalita' e' la massa inerziale, secondo la formula:
    \begin{equation} \label{second_principle}
        \bvv{R} = m\bvv{a}
    \end{equation}
\end{principle}

Facciamo ora delle considerazioni sui primi due principi.
I primi due principi della dinamica sono validi soltanto in sistemi di riferimento inerziali. Essi ci danno un modo per studiare il moto di un corpo a partire dalle forze che agiscono su di esso. Studiando il diagramma di corpo libero, cioe' il diagramma delle forze che agiscono su un corpo possiamo calcolarne la risultante e stabilire, a seconda del risultato, se il corpo ha un'accelerazione nulla o meno.

\begin{principle}[Terzo principio della dinamica]
    Siano $A$, $B$ due corpi puntiformi di massa $m_A, m_B$. Allora se il corpo $A$ esercita sul corpo $B$ una forza $\bvv{F}_{B,A}$, allora il corpo $B$ esercitera' necessariamente una forza $\bvv{F}_{A, B}$ sul corpo $A$, tale che
    \begin{equation} \label{third_principle}
        \bvv{F}_{B, A} = -\bvv{F}_{A, B}   
    \end{equation}
\end{principle}

Non bisogna confondere il terzo principio della dinamica con le forze vincolari: le forze vincolari sono applicate allo stesso corpo che esercita la forza, mentre il secondo principio coinvolge due corpi.

\section{Iterazione gravitazionale}
L'iterazione gravitazionale e' l'iterazione fondamentale che spiega la caduta dei gravi e il moto dei pianeti.

\begin{definition}
    Siano $A$, $B$, due corpi di massa $m_A$, $m_B$. Allora i due corpi si attraggono con forze che sono:
    \begin{itemize}
        \item dirette lungo la congiungente dei centri di massa;
        \item attrattive;
        \item di intensita' uguali, proporzionali al prodotto delle masse e inversamente proporzionali al quadrato della distanza dei centri di massa, secondo la formula
        \begin{equation} \label{forza_grav}
            \abs{\bvv{F}_{A, B}} = \abs{\bvv{F}_{B, A}} = G\frac{m_Am_b}{r^2}
        \end{equation}
        dove $G = 6,64 \times 10^{-11} \text{Nm}^2\text{/kg}$ e' la costante di gravitazione universale.
    \end{itemize}
\end{definition}

Dato che il valore di $G$ e' molto piccolo la forza gravitazionale ha un effetto trascurabile a meno che la massa dei corpi in esame sia grande (almeno $10^{10}$ kg) ed essi sono relativamente vicini.

In teoria la massa inerziale e la massa gravitazionale, cioe' la grandezza scalare proporzionale alla forza di gravita', rappresentano due concetti distinti di massa.

Se consideriamo un corpo sulla superficie terrestre oppure ad un'altezza trascurabile, la forza peso e' costante in modulo e in direzione (radiale, diretta verso il centro di massa della Terra). Dunque possiamo approssimarla con:
\begin{equation}
    \vmag{P} = G\frac{m_Tm_A}{(R_T + h)^2} \approx m_AG\frac{m_T}{R^2_T} = m_Ag
\end{equation}
dove $g = 9,81$ N/kg e' l'accelerazione di gravita' terrestre.

Dagli esperimenti e' stato poi dimostrato che la massa gravitazionale e' equivalente alla massa inerziale: dunque tutti i corpi in caduta libera hanno la stessa accelerazione quando si trovano sullo stesso pianeta e ad altitudini comparabili, sotto l'azione della sola forza gravitazionale.
L'accelerazione di un corpo in caduta libera e' dunque
\begin{equation}
    \bvv{a} = \frac{\bvv{P}}{m_A} = \bvv{g} = -g\bh{j}
\end{equation}

\section{Forze di contatto}
Quando due corpi macroscopici sono a contatto tra di loro agiscono delle forze che derivano dalle interazioni elettromagnetiche della materia. Il primo tipo che studieremo sono le forze legate ai vincoli
Possono essere legate a vincoli di due tipi:
\begin{itemize}
    \item vincoli di superfici (come forze di attrito, forze normali);
    \item vincoli unidimensionali (tensioni di corde, funi, fili).
\end{itemize}

\subsection{Forze legate a vincoli di superfici}
Le forze legate ai vincoli di superfici possono essere
\begin{enumerate}
    \item forze di attrito statico o dinamico: esse sono parallele alla superficie di contatto e opposte al moto relativo dei due corpi;
    \item forze normali: esse sono perpendicolari alla superficie di contatto e impediscono ai corpi di compenetrarsi.
\end{enumerate}
Se la superficie di contatto e' piana allora le forze normali bilanciano il peso del corpo sulla superficie, dunque 
\begin{equation}
    \left(\sum_i \bvv{F}_i \right)_{\perp} = \bvv{0} \implies \bvv{a}_{\perp} = \bvv{0}.
\end{equation}
Se la superficie di contatto non e' piana allora le forze normali non bilanciano il peso del corpo sulla superficie, dunque 
\begin{equation}
    \left(\sum_i \bvv{F}_i \right)_{\perp} \neq \bvv{0} \implies \bvv{a}_{\perp} \neq \bvv{0}.
\end{equation}

\begin{example}[Piano inclinato liscio]
    Supponiamo di avere un piano inclinato con angolo alla base $\theta$. Per descrivere il moto del corpo sul piano inclinato, scegliamo come sistema di riferimento un sistema $XY$ dove l'asse $X$ e' parallelo al piano inclinato, l'asse $Y$ perpendicolare ad esso, e l'origine degli assi sia nel punto dove si trova il corpo al tempo $t_0 = 0$s.

    Se disegnamo il diagramma del corpo libero notiamo che le forze in gioco sono il peso $\bvv{P}$ e la reazione vincolare del piano $\bvv{N}$.
    Sia $\bvv{R}$ la forza risultante; allora
    \begin{equation}
        \bvv{R} = \begin{cases}
            R_x = mg\sin\theta \\
            R_y = N - mg\cos\theta
        \end{cases}
    \end{equation}
    Per il secondo principio della dinamica vale allora
    \begin{align}
        \begin{cases}
            R_x = mg\sin\theta = ma_x \\
            R_y = N - mg\cos\theta = ma_y = 0 
        \end{cases}
        &\implies
        \begin{cases}
            a_x = g\sin\theta \\
            N = mg\cos\theta
        \end{cases}
    \end{align}

    Supponendo che la rampa sia lunga $L$ e che il corpo si muova inizialmente con una velocita' $v_0 = 0$ possiamo calcolare la velocita' con cui il corpo giunge alla fine e il tempo che impiega per percorrerla $t_f$. Dalla legge oraria del moto uniformemente accelerato otteniamo
    \begin{alignat*}{1}
        L &= \frac{1}{2}g\sin\theta t_f^2 \\
        \intertext{da cui possiamo ricavare $t_f$}
        \implies t_f &= \sqrt{\frac{2L}{g\sin\theta}} \\
        \intertext{Sostituendo $h = L\sin\theta$}
              &= \sqrt{\frac{2h}{g\sin^2\theta}} \\
              &= \frac{1}{\sin\theta}\sqrt{\frac{2h}{g}}
        \intertext{Sostituendo $t_f$ nella formula per la velocita' otteniamo infine}
        \implies v_f &= a_xt_f\\
              &= g\sin\theta t_f\\
              &= \sqrt{2gh}
    \end{alignat*}
\end{example}

\begin{example}
    [Piano inclinato liscio e forza orizzontale]
    Supponiamo di avere un piano inclinato con angolo alla base $\theta$ e un corpo di massa $m$ che viene spinto su per il piano inclinato tramite una forza orizzontale $\bvv{F_e}$. Sappiamo inoltre che la velocita' del corpo e' costante. Calcoliamo quanto vale la forza orizzontale e la reazione vincolare $\bvv{N}$.

    Dato che per ipotesi il corpo si muove a velocita' costante, allora $\bvv{R} = \bvv{0}$, cioe' $R_x = 0$ e $R_y = 0$. Dunque disegnando il diagramma del corpo libero:
    \begin{alignat*}{1}
        &\begin{cases}
            R_x = mg\sin\theta - F_e\cos\theta = 0\\
            R_y = N - mg\cos\theta - F_e\sin\theta = 0
        \end{cases} \\
        \implies &\begin{cases}
            F_e\cos\theta = mg\sin\theta\\
            N = mg\cos\theta + F_e\sin\theta
        \end{cases} \\
        \implies &\begin{cases}
            F_e = mg\tan\theta\\
            N = mg\cos\theta + mg\tan\theta \sin\theta = mg\cos\theta (1 + \tan^2 \theta)
        \end{cases}
    \end{alignat*}
\end{example}

Le forze di attrito sono esercitate parallelamente alla superficie di contatto. Esse si dividono in due categorie: \begin{itemize}
    \item attrito statico: si oppone all'inizio del moto del corpo;
    \item attrito dinamico: si oppone al moto di un corpo che non e' in quiete.
\end{itemize}
Le forze di attrito dinamico sono inferiori a quelle di attrito statico.

\subsubsection{Attrito statico}

L'attrito statico si oppone al moto, e ha un'intensita' tale che l'accelerazione e' nulla. La forza di attrito ha direzione tangente alla superficie e ha verso opposto alla forza applicata. Finche' il corpo rimane in uno stato di quiete si ha che $\vmag{f_s} = \vmag{F_{\text{ext}}}$ (dove $\bvv{f_s}$ e' la forza di attrito e $\bvv{F_{\text{ext}}}$ e' la forza esterna). 
In generale si ha che $\vmag{f_s} \leq \mu_s N$, dove $\mu_s$ e' il coefficiente di attrito statico, dunque al massimo $\vmag{f_s} = \mu_s N$.

\subsubsection{Attrito dinamico}

Se la forza esterna supera il valore $\mu_s N$ il corpo comincia a muoversi e l'attrito diventa attrito dinamico, che ha la stessa direzione del moto, verso opposto e modulo \[
    \vmag{f_d} = \mu_d N    
\]
dove $\mu_d$ e' il coefficiente di attrito dinamico, e risulta sempre $\mu_d \leq \mu_s$.

\begin{example}
    Un corpo e' lanciato con velocita' $\bvv{v_0}$ lungo un piano scabro con attrito dinamico $\mu_d$. Dopo quanto tempo si ferma? Che tratto percorre prima di fermarsi?

    Consideriamo il modo nelle due dimensioni. Sull'asse $X$ il corpo si muove di moto accelerato (con accelerazione negativa data dall'attrito dinamico), mentre sull'asse $Y$ il corpo e' fermo dunque la somma delle forze sara' $0$. Dunque:
    \begin{equation*}
        \begin{cases}
            R_x = -f_d \\
            R_y = N - mg = 0            
        \end{cases}
        \implies \begin{cases}
            -f_d = -\mu_d N = -\mu_d mg = ma_x\\
            N = mg
        \end{cases} 
    \end{equation*}
    Quindi $a_x = -\mu_d g$. Sostituendolo nelle equazioni del moto otteniamo\[
        t_f = \frac{v_0}{-a_x} = \frac{v_0}{\mu_d g}
    \] e inoltre \[
        x_f = v_0t_f + \frac{1}{2}a_xt_f^2 = \frac{v_0^2}{\mu_d g} - \frac{1}{2}\mu_d g\frac{v_0^2}{\mu_d^2g^2} = \frac{v_0^2}{2\mu_d g}.
    \]
\end{example}


\begin{example}
    Un corpo e' lanciato con velocita' $\bvv{v_0}$ lungo un piano scabro con attrito dinamico $\mu_d$. Se volessi mantenere il corpo a velocita' costante, che forza esterna dovrei fornire? Ci vuole meno forza a tirare verso l'alto ($\theta > 0$) oppure a spingere verso il basso ($\theta < 0$)?

    Consideriamo il modo nelle due dimensioni includendo una forza $F$ inclinata con un angolo $\theta$: per il primo principio della dinamica la risultante delle forze sul corpo dovra' essere nulla. Dunque:
    \begin{equation*}
        \begin{cases}
            R_x = -f_d + F\cos\theta = -\mu_d N + F\cos\theta = 0\\
            R_y = N - mg + F\sin\theta = 0       
        \end{cases}
        \implies \begin{cases}
            -\mu_d N + F\cos\theta = -\mu_d (mg - F\sin\theta) + F\cos\theta = -\mu_d mg + F(\cos\theta + \mu_d\sin\theta) = 0\\
            N = mg - F\sin\theta
        \end{cases} 
    \end{equation*}
    dunque \[
        F = \frac{\mu_d mg}{\cos\theta + \mu_d \sin\theta}.    
    \]    
\end{example}

\begin{example}[Piano inclinato con attrito]
    Supponiamo di avere un piano inclinato con attrito e che il corpo con massa $m$ sia in uno stato di quiete.

    Questo significa che la forza di attrito $\bvv{f_a}$ riesce a bilanciare le altre forze, in modo che la risultante lungo l'asse $X$ sia nulla. Al massimo vale quindi $f_a = \mu_s N$.
    \begin{equation*}
        \begin{cases}
            R_x = N - mg\cos\theta = 0\\
            R_y = -f_a + mg\sin\theta = 0
        \end{cases} \implies
        \begin{cases}
            N = mg\cos\theta\\
            f_a = mg\sin\theta
        \end{cases} 
    \end{equation*}
    
    Definiamo pendenza critica l'angolo $\theta_c$ oltre il quale la forza di attrito statica non riesce a controbilanciare la forza di gravita'. Quindi quando $\theta > \theta_c$ segue che NON HO CAPITO E NON MI VA DI CAPIRE
\end{example}

\subsection{Tensioni}

Quando non ci sono altre forze applicate in una qualsiasi sua parte una fune ideale (cioe' inestensibile e di massa trascurabile) esercita sui corpi fissati ai suoi estremi due forze che hanno la direzione della fune, stessa intensita' e versi opposti.

\section{Moto circolare}

Si dice moto circolare uniforme un moto su una traiettoria circolare di raggio $R$ con velocita' costante in modulo. Dato che la velocita' cambia direzione avremo che $\bvv a \neq 0$.

Chiamiamo $s$ lo spazio percorso sulla circonferenza, $\theta = \frac{s}{R}$ l'angolo spazzato. Allora valgono le seguenti: \[
    \begin{cases}
        \Delta \bvv r = \bvv{r_f} - \bvv{r_i}\\
        \Delta \theta= \theta_f - \theta_i\\
        \Delta s = R\Delta \theta
    \end{cases}    
\]

\begin{definition}
    Diciamo velocita' angolare media di un moto circolare la grandezza $\ang{\omega} = \frac{\Delta \theta}{\Delta t}$. Dunque la velocita' angolare istantanea sara' $\omega = \lim_{\Delta t \to 0} \ang{\omega} = \frac{d\theta}{dt}$.
\end{definition}

\begin{definition}
    Diciamo accelerazione angolare media di un moto circolare la grandezza $\ang{\alpha} = \frac{\Delta \omega}{\Delta t}$. Dunque la velocita' angolare istantanea sara' $\alpha = \lim_{\Delta t \to 0} \ang{\alpha} = \frac{d\omega}{dt}$.
\end{definition}

\end{document}
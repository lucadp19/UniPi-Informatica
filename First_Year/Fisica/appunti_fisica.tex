\documentclass[a4paper]{report}
    \usepackage[utf8]{inputenc}
    \usepackage[italian]{babel}
    \usepackage[T1]{fontenc}
    \usepackage{textcomp, microtype}
    \usepackage{amsmath, amsthm, amssymb, longtable, physics, mathtools, esvect, cases}

    \usepackage{hyperref} % ultimo package da caricare!


\theoremstyle{plain}
\newtheorem{principle}{Principio}[section]

\theoremstyle{definition}
\newtheorem{definition}{Definizione}[section]

\theoremstyle{remark}
\newtheorem*{remark}{Osservazione}

\theoremstyle{definition}
\newtheorem{example}{Example}[section]

\newcommand{\vmag}[1]{\abs{\bvv{#1}}}
\newcommand{\ang}[1]{\left\langle#1\right\rangle}
\newcommand{\bvv}[1]{\vv{\mathbf{#1}}}
\newcommand{\bh}[1]{\hat{\mathbf{#1}}}

\begin{document}

% \author{Luca De Paulis}
\title{Fisica}
\maketitle

\tableofcontents

\chapter{Insiemi numerici}

\section{Strutture algebriche fondamentali}

\begin{definition}[Gruppo]
    Si dice \textbf{gruppo} una tripla ($G$, $\cdot$, $e$) formata da \begin{itemize}
        \item un insieme di elementi $G$;
        \item un operazione $\cdot : A \times A \to A$ detta prodotto;
        \item un elemento $e \in G$
    \end{itemize} per cui valgono i seguenti assiomi: 
    \begin{description}
        \item[(Assiomi di gruppo)] Per ogni $a, b, c \in G$ vale che
        \begin{align*}
            &\text{(P1)}      &&(ab) \in G            &\text{(chiusura rispetto a $\cdot$)}\\
            &\text{(P2)}      &&(ab)c = a(bc)         &\text{(associatività di $\cdot$)}\\
            &\text{(P3)}      &&a \cdot e=e \cdot a=a &\text{($e$ el. neutro di $\cdot$)}\\
            &\text{(P4)}     &&\exists a^{-1} \in G. \quad aa^{-1} = e &\text{(inverso per $\cdot$)}
            \intertext{Si dice \textbf{gruppo commutativo} un gruppo per cui vale inoltre il seguente assioma:}
            &\text{(P5)}     &&ab = ba               &\text{(commutatività di $\cdot$)}
        \end{align*}
    \end{description}
\end{definition}

\begin{definition}[Anello]
    Si dice \textbf{anello} una quintupla ($A$, $+$, $\cdot$, $0$, $1$) formata da
    \begin{itemize}
        \item un insieme di elementi $A$;
        \item un operazione $+ : A \times A \to A$ detta somma;
        \item un operazione $\cdot : A \times A \to A$ detta prodotto;
        \item un elemento $0 \in A$;
        \item un elemento $1 \in A$
    \end{itemize} per cui valgono i seguenti assiomi: 
    \begin{description}
        \item[(Assiomi di anello)] Per ogni $a, b, c \in A$ vale che
        \begin{align*}
            &\text{(S1)}      &&(a+b) \in A           &\text{(chiusura rispetto a $+$)}\\
            &\text{(S2)}      &&a+b = b+a             &\text{(commutatività di $+$)}\\
            &\text{(S3)}      &&(a+b)+c = a+(b+c)     &\text{(associatività di $+$)}\\
            &\text{(S4)}      &&a+0=0+a=a             &\text{(0 el. neutro di $+$)}\\
            &\text{(S5)}      &&\exists (-a) \in A. \quad a+(-a) = 0 &\text{(opposto per $+$)}\\
            &\text{(P1)}      &&(ab) \in A            &\text{(chiusura rispetto a $\cdot$)}\\
            &\text{(P2)}      &&(ab)c = a(bc)         &\text{(associatività di $\cdot$)}\\
            &\text{(P3)}      &&a \cdot 1=1 \cdot a=a &\text{(1 el. neutro di $\cdot$)}\\
            &\text{(P4)}      &&(a+b)c = ac + bc      &\text{(distributività 1)} \\
            &\text{(P5)}     &&a(b+c) = ab + ac      &\text{(distributività 2)}
            \intertext{Si dice \textbf{anello commutativo} un anello per cui vale inoltre il seguente assioma:}
            &\text{(P6)}     &&ab = ba               &\text{(commutatività di $\cdot$)}
        \end{align*}
    \end{description} 
\end{definition}

Un tipico esempio di anello commutativo è $\Z$: infatti gli anelli generalizzano le operazioni che possiamo fare sui numeri interi e le loro proprietà fondamentali per estenderle ad altri insiemi con la stessa struttura algebrica.

\begin{definition}[Campo]
    Si dice \textbf{campo} una quintupla ($F$, $+$, $\cdot$, $0$, $1$) formata da
    \begin{itemize}
        \item un insieme di elementi $F$;
        \item un operazione $+ : F \times F \to F$ detta somma;
        \item un operazione $\cdot : F \times F \to F$ detta prodotto;
        \item un elemento $0 \in F$;
        \item un elemento $1 \in F$
    \end{itemize}  per cui valgono i seguenti assiomi: 
    \begin{description}
        \item[(Assiomi di campo)] Per ogni $a, b, c \in F$ vale che
        \begin{align*}
            &\text{(S1)}      &&(a+b) \in F           &\text{(chiusura rispetto a $+$)}\\
            &\text{(S2)}      &&a+b = b+a             &\text{(commutatività di $+$)}\\
            &\text{(S3)}      &&(a+b)+c = a+(b+c)     &\text{(associatività di $+$)}\\
            &\text{(S4)}      &&a+0=0+a=a             &\text{(0 el. neutro di $+$)}\\
            &\text{(S5)}      &&\exists (-a) \in F. \quad a+(-a) = 0 &\text{(opposto per $+$)}\\
            &\text{(P1)}      &&(ab) \in F            &\text{(chiusura rispetto a $\cdot$)}\\        
            &\text{(P2)}      &&ab = ba               &\text{(commutatività di $\cdot$)}\\
            &\text{(P3)}      &&(ab)c = a(bc)         &\text{(associatività di $\cdot$)}\\
            &\text{(P4)}      &&a \cdot 1=1 \cdot a=a &\text{(1 el. neutro di $\cdot$)}\\
            &\text{(P5)}     &&(a+b)c = ac + bc      &\text{(distributività)} \\
            &\text{(P6)}     &&a \neq 0 \implies \exists a^{-1} \in F. \quad aa^{-1} = 1 &\text{(inverso per $\cdot$)}
        \end{align*}
    \end{description} 

    La definizione sopra è equivalente a dire che $F$ è un anello commutativo per cui ogni elemento non nullo ha un inverso moltiplicativo.
\end{definition}

Tra gli insiemi numerici classici, gli insiemi $\Q, \R$ e $\C$ sono tutti esempi di campi: infatti le operazioni di addizione e moltiplicazione sono chiuse rispetto all'insieme, rispettano le proprietà commutativa, associativa e distributiva ed esistono gli inversi per la somma e per il prodotto (per ogni numero diverso da $0$). Il concetto di campo serve quindi a generalizzare la struttura algebrica dei numeri razionali/reali/complessi per altri insiemi numerici.

Nei campi vale la seguente proposizione.
\begin{proposition}
    [Regola di annullamento del prodotto] \label{annullamento_prodotto}
    Sia $\K$ un campo e siano $a, b \in \K$. Allora \[
        ab = 0 \implies a = 0 \lor b = 0.    
    \]
\end{proposition}
\begin{proof}
    Sappiamo che $a = 0 \lor b = 0$ è equivalente a $a \neq 0 \implies b = 0$, dunque supponiamo che $a$ sia diverso da $0$ e dimostriamo che $b$ è zero.

    Dato che $a \neq 0$ allora ammette un inverso. Chiamiamolo $a^{-1}$ e moltiplichiamo entrambi i membri per esso:
    \begin{alignat*}
        {1}
        &a^{-1}(ab) = a^{-1} \cdot 0\\
        \iff &(a^{-1}a)b = 0 \\
        \iff &b = 0
    \end{alignat*}
    che è la tesi.
\end{proof}

\section{Numeri complessi}

\begin{definition}[Unità immaginaria]
    Si dice unità immaginaria il numero $i$ tale che \[
        i^2 = -1.    
    \]
\end{definition}

\begin{definition}[Numeri complessi]
    L'insieme dei numeri complessi $\C$ è l'insieme dei numeri della forma $a+ib$ per qualche $a, b \in \R$, ovvero \[
        \C = \{ a + ib \mid a, b \in \R, i^2 = -1\}.  
    \]
\end{definition}

\begin{definition}[Parte reale e immaginaria]
    Sia $z \in \C$ tale che $z = a + ib$. Allora si dicono rispettivamente \begin{itemize}
        \item parte reale di $z$ il numero $\Re z = a$;
        \item parte immaginaria di $z$ il numero $\Im z = b$.
    \end{itemize}
\end{definition}

\begin{definition}[Somma e prodotto sui complessi]
    Definiamo le seguenti due operazioni su $\C$:
    \begin{itemize}
        \item $+ : \C \times \C \to \C$ tale che $(a + ib) + (c + id) = (a + c) + i(b + d)$;
        \item $\cdot : \C \times \C \to \C$ tale che $(a + ib) \cdot (c + id) = (ac - bd) + i(ad + bc)$.
    \end{itemize}
\end{definition}

\begin{remark}
    Le due operazioni vengono naturalmente dalla somma e dal prodotto tra monomi. Infatti \begin{gather*}
        (a + ib) + (c + id) = a + c + ib + id = (a + c) + i(b + d);\\
        \begin{alignedat}{1}
            (a + ib) \cdot (c + id) &= ac + iad + ibc + i^2bd \\
            &= ac + i(ad + bc) -bd \\
            &= (ac - bd) + i(ad + bc).
        \end{alignedat}
    \end{gather*}
\end{remark}

Notiamo che i numeri complessi della forma $a + i0$ sono numeri reali, dunque $\R \subset \C$. Inoltre possiamo rappresentare i numeri complessi come punti in uno spazio bidimensionale dove la parte reale rappresenta l'ascissa e la parte immaginaria rappresenta l'ordinata: la retta corrispondente all'asse x è il sottoinsieme dei numeri reali.

\begin{definition}[Coniugato complesso]
    Sia $z = a+ib \in \C$. Allora si dice coniugato complesso (o semplicemente coniugato) di $z$ il numero \[
        \conj{z} = a - ib.    
    \]
\end{definition}

\begin{definition}[Norma di un numero complesso]
    Sia $z = a + ib \in \C$. Allora si dice norma di $z$ il numero reale \[
        \abs{z} = \sqrt{a^2 + b^2}.    
    \]
\end{definition}

Notiamo che $\abs{z} = 0$ se e solo se $a = b = 0$, ovvero se $z = 0$.

\begin{proposition}\label{somma_prodotto_tra_coniugati}
    Siano $z, w \in \C$ tali che $z = a+ib$, $w = c + id$. Allora \begin{enumerate}[(i)]
        \item $\conj{z} + \conj{w} = \conj{z + w}$;
        \item $\conj{z}\cdot\conj{w} = \conj{zw}$;
        \item $(\conj{z})^n = \conj{z^n}$.
    \end{enumerate}
\end{proposition}
\begin{proof}
    Dimostriamo i tre fatti.
    \begin{enumerate}[(i)]
        \item Per definizione di somma \begin{alignat*}
            {1}
            \conj{z} + \conj{w} &= (a-ib) + (c - id)\\
            &= (a+c) - i(b+d)\\
            &= \conj{z + w}.
        \end{alignat*}
        \item Per definizione di prodotto \begin{alignat*}
            {1}
            \conj{z}\cdot\conj{w} &= (a-ib)(c - id)\\
            &= (ac - bd) + i(-ad-bc)\\
            &= (ac - bd) - i(ad+bc)\\
            &= \conj{zw}.
        \end{alignat*}
        \item Dimostriamolo per induzione su $n$.
        \begin{description}
            \item[Caso base.] Se $n = 1$ allora banalmente $(\conj{z})^1 = \conj{z} = \conj{z^1}$.
            \item[Passo induttivo.] Supponiamo che la tesi valga per $n$ e dimostriamola per $n+1$. Allora \[
                (\conj{z})^{n+1} = (\conj{z})^{n} \cdot \conj{z} = \conj{z^n} \cdot \conj{z} = \conj{z^{n+1}}
            \] dove l'ultimo passaggio è giustificato dal punto precedente della dimostrazione. \qedhere
        \end{description}
    \end{enumerate}
\end{proof}

\begin{proposition}\label{somma_prodotto_col_coniugato}
    Sia $z = a+ib \in \C$. Allora valgono i seguenti fatti:
    \begin{enumerate}[(i)]
        \item $z + \conj{z} = 2\Re{z}$;
        \item $z\conj{z} = \abs{z}^2$.
    \end{enumerate}
\end{proposition}
\begin{proof}
    Dimostriamo i due fatti.
    \begin{enumerate}[(i)]
        \item Per definizione di somma $z + \conj{z} = (a + ib) + (a - ib) = 2a = 2\Re z$.
        \item Per definizione di prodotto \[
            z\conj{z} = (a + ib)(a - ib) = a^2 - iab + iab - i^2b^2 = a^2 + b^2 = \abs{z}^2.\qedhere
        \] 
    \end{enumerate}
\end{proof}

La proposizione precedente ci consente di trovare l'inverso di qualunque numero non nullo in $\C$.

\begin{proposition}[Inverso tra i complessi]
    Sia $z \in \C, z \neq 0$. Allora \[\frac{1}{z} = \frac{\conj{z}}{\abs{z}^2}.\]
\end{proposition}
\begin{proof}
    Per la proposizione \ref{somma_prodotto_col_coniugato} segue che \[
        z\conj{z} = \abs{z}^2 \iff \frac{1}{z} = \frac{\conj{z}}{\abs{z}^2}. \qedhere   
    \]
\end{proof}

\begin{proposition}[I numeri complessi formano un campo]
    L'insieme $\C$ insieme alle operazioni di somma e prodotto con i rispettivi elementi neutri $0, 1 \in \C$ forma un campo.
\end{proposition}

\subsection{Rappresentazione polare dei numeri complessi}

Dato che possiamo considerare i numeri complessi come punti di un piano bidimensionale possiamo rappresentarli in forma polare, cioè considerando il vettore che congiunge l'origine degli assi con il punto $(a, b)$ che rappresenta il numero complesso $a + ib$. La forma polare di un numero complesso è data dalla coppia $(r, \theta)$, dove $r$ è il raggio del vettore e $\theta$ è l'angolo tra l'asse x e il vettore.

Dunque se $z = a+ib$ è un numero complesso in forma cartesiana, possiamo esprimerlo come $r(\cos\theta + i\sin\theta)$, dove $r = \sqrt{a^2 + b^2} = \abs{z}$ e $\theta = \arctan \frac{a}{b}$.

\begin{definition}[Esponenziale complesso]
    $e^{i\theta} = \cos\theta + i\sin\theta.$
\end{definition}

Sfruttando la definizione precedente possiamo scrivere ogni numero complesso nella forma $re^{i\theta}$ che è la forma polare del numero.

\begin{proposition}
    Siano $e^{i\alpha}, e^{i\beta} \in \C$. Allora vale \[
        e^{i\alpha} e^{i\beta} = e^{i(\alpha + \beta)}.
    \]
\end{proposition}
\begin{proof}
    Per definizione di esponenziale complesso:
    \begin{alignat*}{1}
        e^{i\alpha} e^{i\beta} &= (\cos\alpha + i\sin\alpha)(\cos\beta + i\sin\beta)\\
        &= (\cos\alpha \cos\beta - \sin\alpha \sin\beta) + i(\sin\alpha \cos\beta + \cos\alpha \sin\beta)\\
        &= \cos(\alpha + \beta) + i\sin(\alpha + \beta)\\
        &= e^{i(\alpha + \beta)}. \tag*{\qedhere}
    \end{alignat*}
\end{proof}

\section{Successioni per ricorrenza}

\begin{definition}[Successione]
    Si dice successione a valori in un insieme $A$ una funzione $(a_n) : \N \to A$.
\end{definition}

Solitamente analizzeremo successioni a valori reali, ovvero $(a_n) : \N \to \R$. Inoltre usiamo equivalentemente le notazioni $(a_n)_k$ o $a_k$ per riferirci alla funzione valutata nel punto $k \in \N$.

\begin{definition}[Somma di successioni e prodotto per una costante]
    Sia $S_{\R}$ l'insieme delle successioni a valori reali. Allora definisco una somma tra successioni $+ : S_{\R} \times S_{\R} \to S_{\R}$ tale che \[
        (a_n) + (b_n) = (a_n + b_n)  
    \] e un prodotto per una costante $\cdot : \R \times S_{\R} \to S_{\R}$ tale che \[
        k(a_n) = (ka_n).    
    \]
\end{definition}

\begin{example}
    Sia $a_n = 3^n$ e $b_n = 2n + 1$. Allora $(c_n) = (a_n) + (b_n)$ è la successione definita dalla legge $c_n = 3^n + 2n + 1$, mentre $(d_n) = 3(b_n)$ è la successione definita da $d_n = 6n + 3$.
\end{example}

Queste operazioni rispettano le solite proprietà (associativa, commutativa, distributiva). In particolare vale quindi la seguente proposizione.

\begin{proposition}[L'insieme delle successioni è uno spazio vettoriale]
    L'insieme delle successioni a valori reali $S_{\R}$ insieme alle operazioni di somma e prodotto per costanti e alla successione identicamente nulla $(0_n)$ è uno spazio vettoriale su $\R$.
\end{proposition}

\begin{definition}[Ricorrenza lineare omogenea]
    Si dice ricorrenza lineare omogenea di ordine $k$ un'equazione della forma \begin{equation} \label{ricorrenza}
        a_{n+k} = r_{k-1}a_{n+k-1} + r_{k-2}a_{n+k-2} + \dots + r_{1}a_{n+1} + r_0a_n. 
    \end{equation}
    Una soluzione della ricorrenza lineare \ref{ricorrenza} è una successione $(s_n)$ tale che per ogni $n \in \N$ vale che $s_n, s_{n+1}, \dots, s_{n+k}$ soddisfano la ricorrenza.
\end{definition}

\begin{proposition}
    Sia $A$ l'insieme delle successioni che soddisfano la ricorrenza lineare omogenea \[
        s_{n+k} = r_{k-1}s_{n+k-1} + r_{k-2}s_{n+k-2} + \dots + r_{1}s_{n+1} + r_0s_n.
    \] Allora $A$ è un sottospazio vettoriale di $S_{\R}$.
\end{proposition}
\begin{proof}
    Dobbiamo dimostrare tre fatti:
    \begin{enumerate}[(i)]
        \item $(0_n) \in A$;
        \item se $(a_n), (b_n) \in A$ allora $(c_n) = (a_n) + (b_n) \in A$;
        \item se $h \in \R$, $(a_n) \in A$ allora $(d_n) = h(a_n) \in A$.
    \end{enumerate}

    Sia $n \in \N$ qualsiasi.
    \begin{enumerate}[(i)]
        \item Verifichiamo che $(0_n)$ sia soluzione. La ricorrenza da verificare è \[
            0_{n+k} = r_{k-1}0_{n+k-1} + \dots + r_{1}0_{n+1} + r_00_n.
        \] Ma dato che $(0_n)$ è la successione identicamente nulla, allora questo equivale a dire $0 = 0r_{k-1} + \dots +  + 0r_0 = 0$, che è verificata e quindi $(0_n) \in A$.
        \item Verifichiamo che $(c_n)$ sia soluzione. \begin{align*}
            c_{n+k} &= a_{n+k} + b_{n+k}\\
            &= (r_{k-1}a_{n+k-1} + \dots + r_0a_n) + (r_{k-1}b_{n+k-1} + \dots + r_0b_n) \\
            &= r_{k-1}(a_{n+k-1} + b_{n+k-1}) + \dots + r_0(a_n + b_n) \\
            &= r_{k-1}c_{n+k-1} + \dots + r_0c_0
        \end{align*}
        dunque $(c_n) \in A$.
        \item Verifichiamo che $(d_n)$ sia soluzione. \begin{align*}
            d_{n+k} &= ha_{n+k}\\
            &= h(r_{k-1}a_{n+k-1} + \dots + r_0a_n)\\
            &= r_{k-1}(ha_{n+k-1}) + \dots + r_0(ha_n) \\
            &= r_{k-1}d_{n+k-1} + \dots + r_0d_0
        \end{align*}
        dunque $(d_n) \in A$. \qedhere
    \end{enumerate}
\end{proof}

La proposizione precedente ci permette di trovare una soluzione generale ad una ricorrenza lineare omogenea.

\begin{example}
    Siano $a_n = 3^n$ e $b_n = (-1)^n$ due soluzioni di una ricorrenza lineare omogenea. Allora per la proposizione precedente anche $k_1a_n = k_13^n$ e $k_2b_n = k_2(-1)^n$ saranno soluzioni (per ogni $k_1, k_2 \in \R$), e di conseguenza anche $k_1a_n + k_2b_n = k_13^n + k_2(-1)^n$.
\end{example}

Cerchiamo di risolvere una ricorrenza lineare omogenea.
\begin{example}
    Sia $a_{n+2} = 2a_{n+1} + 3a_n$ una ricorrenza lineare omogenea di ordine $2$. Trovare la soluzione generale. Inoltre trovare una soluzione particolare che soddisfi le condizioni iniziali $a_0 = 0$ e $a_1 = 1$.
\end{example}
\begin{solution}
    Proviamo a risolvere la ricorrenza con una soluzione esponenziale della forma $(\lambda^n)$ al variare di $n \in \N$. Sostituendo otteniamo \begin{alignat*}{1}
        &\lambda^{n+2} = 2\lambda^{n+1} + 3\lambda^{n} \\
        \iff &\lambda^2 = 2\lambda + 3 \\
        \iff &\lambda^2 - 2\lambda - 3.
    \end{alignat*}
    Dunque se $(\lambda^n)$ è una soluzione allora $\lambda$ deve essere radice di quel polinomio di secondo grado, detto polinomio caratteristico della ricorrenza.
    Risolvendolo segue che $\lambda_1 = 3$ e $\lambda_2 = -1$ sono soluzioni, dunque le successioni $(3^n)$ e $((-1)^n)$ sono soluzioni della ricorrenza.

    La soluzione generale della ricorrenza è dunque una successione della forma $(a_n) = k_1(3^n) + k_2((-1)^n)$ al variare di $k_1, k_2 \in \R$.

    Imponiamo ora che $a_0 = 0$ e $a_1 = 1$.
    \begin{equation*}
        \left\{
        \begin{array}{@{}roror }
        3^0k_1 & + & (-1)^0k_2 & = & 0 \\
        3^1k_1 & + & (-1)^1k_2 & = & 1 \\
        \end{array}
        \right. \iff \left\{
        \begin{array}{@{}ror }
        k_1 + k_2 & = & 0 \\
        3k_1 -k_2 & = & 1 \\
        \end{array}
        \right. 
    \end{equation*}
    da cui segue $k_1 = \frac14$, $k_2 = -\frac14$. La successione che soddisfa le condizioni iniziali è dunque $a_n = \frac14(3)^n - \frac14(-1)^n$.
\end{solution}

\begin{definition}
    [Polinomio caratteristico di una ricorrenza]
    Sia $a_{n+k} = r_{k-1}a_{n+k-1} +  \dots +  r_0a_n$ una ricorrenza lineare omogenea di ordine $k$. Allora si dice polinomio caratteristico associato alla ricorrenza il polinomio \[
        p(\lambda) = \lambda^k - r_{k-1}\lambda^{k-1} - \dots - r_0.    
    \]
\end{definition}

Il polinomio caratteristico si ottiene sostituendo alla ricorrenza lineare la successione $(\lambda^n)$, esattamente come abbiamo fatto nell'esempio precedente.

\begin{example}
    Consideriamo la successione di Fibonacci $f_{n+2} = f_{n+1} + f_n$ con $f_0 = 0$, $f_1 = 1$. Trovare una successione che risolva la ricorrenza e soddisfi i casi base.
\end{example}
\begin{solution}
    Il polinomio caratteristico di questa ricorrenza è \[
        p(\lambda) = \lambda^2 - \lambda - 1    
    \] che ha come radici i numeri $\varphi = \frac12(1 + \sqrt5)$ e $\bar{\varphi} = \frac12(1 - \sqrt5)$.

    La soluzione generale della ricorrenza è dunque una successione della forma $(f_n) = k_1(\varphi^n) + k_2(\bar{\varphi}^n)$ al variare di $k_1, k_2 \in \R$.

    Imponiamo ora che $f_0 = 0$ e $f_1 = 1$.
    \begin{equation*}
        \arraycolsep=1.2pt\def\arraystretch{1.3}
        \left\{
        \begin{array}{@{}roror }
        \varphi^0k_1 & + & \bar{\varphi}^0k_2 & = & 0\\
        \varphi^1k_1 & + & \bar{\varphi}^1k_2 & = & 1 \\
        \end{array}
        \right. \iff \left\{
        \begin{array}{@{}roror }
        k_1 & + & k_2 & = & 0\\
        \varphi k_1 & + & \bar{\varphi}k_2 & = & 1 \\
        \end{array}
        \right. 
    \end{equation*}
    da cui segue $k_1 = \frac{1}{\sqrt5}$, $k_2 = -\frac{1}{\sqrt5}$. La successione che soddisfa le condizioni iniziali è dunque \[
        f_n = \frac{1}{\sqrt5}\left(\frac{1 + \sqrt5}{2}\right)^n - \frac{1}{\sqrt5}\left(\frac{1 - \sqrt5}{2}\right)^n.
    \]
\end{solution}

Nel caso che una radice del polinomio caratteristico abbia una molteplicità maggiore di $1$ essa darà luogo a più di una soluzione della ricorrenza, come ci dice la seguente proposizione.
\begin{proposition}
    Sia $p(\lambda)$ il polinomio caratteristico di una ricorrenza lineare omogenea e sia $\lambda_0$ una radice di molteplicità $h$ (ovvero $h$ è il massimo intero per cui $(x - \lambda_0)^h$ compare nella fattorizzazione di $p(\lambda)$) con $h \leq 2$. 
    
    Allora $(\lambda_0^n), (n\lambda_0^n), \dots, (n^{h-1}\lambda_0^n)$ sono tutte soluzioni della ricorrenza lineare omogenea.
\end{proposition}

\begin{example}
    Sia $p(\lambda) = (\lambda - 3)^3(\lambda + 1)^2(\lambda - \sqrt2)^4$. Allora le seguenti sono tutte soluzioni indipendenti della ricorrenza lineare omogenea associata a $p(\lambda)$:
    \begin{multicols}{3}
        \begin{enumerate}[(i)]
        \item $(3^n)$;
        \item $(n3^n)$;
        \item $(n^23^n)$;
        \item $((-1)^n)$;
        \item $(n(-1)^n)$;
        \item $(\sqrt{2}^n)$;
        \item $(n\sqrt{2}^n)$;
        \item $(n^2\sqrt{2}^n)$;
        \item $(n^3\sqrt{2}^n)$.
    \end{enumerate}
    \end{multicols}
    
    La soluzione generale sarà dunque della forma \begin{align*}
        (a_n) = 
            &\ k_1(3^n) + k_2(n3^n) + k_3(n^23^n) + k_4(n(-1)^n) + k_5(n(-1)^n) + \\
            + &\ k_6(\sqrt{2}^n) + k_7(n\sqrt{2}^n) + k_8(n^2\sqrt{2}^n) + k_9(n^3\sqrt{2}^n)
    \end{align*}
    al variare di $k_1, \dots, k_9 \in \R$.
\end{example}
\chapter{Cinematica del punto materiale}

\section{Definizioni fondamentali}
\begin{definition}
    Si dice raggio vettore o \textbf{vettore posizione} il vettore $\bvv{r}$ che descrive la posizione del punto materiale rispetto ai tre assi al variare del tempo.
    \[\bvv{r}(t) = x(t)\bh{i} + y(t)\bh{j} + z(t)\bh{k}\]
\end{definition}

\begin{definition}
    Si dice \textbf{vettore spostamento} il vettore $\bvv{s}$ che descrive lo spostamento del punto materiale tra due istanti di tempo.
    \[\bvv{s} = \Delta\bvv{r} = \bvv{r_f} - \bvv{r_i} = \bvv{r}(t_f) - \bvv{r}(t_i)\]
\end{definition}

\begin{definition}
    Si dice \textbf{velocita' media} il vettore $\ang{\bvv{v}}$ che descrive la velocita' media del punto materiale tra due istanti di tempo.
        \begin{equation} \label{def_vel_media}
            \ang{\bvv{v}} = \bvv{v_m}(t_1, t_2) = \frac{\Delta\bvv{r}}{\Delta t} = \frac{\bvv{r}(t_2) - \bvv{r}(t_1)}{t_2-t_1}
        \end{equation}
    Si dice invece \textbf{velocita' istantanea} il vettore $\bvv{v}$ che descrive la velocita' del punto materiale in ogni istante di tempo.
    La velocita' istantanea e' definita come il limite della velocita' media quando $t_2 \to t_1$, o equivalentemente se supponiamo
    $t_2 = t_1 + \Delta t$ la velocita' istantanea e' il limite della velocita' media quando $\Delta t \to 0$.
    \[\bvv{v} = \lim_{t_2 \to t_1} \frac{\bvv{r}(t_2) - \bvv{r}(t_1)}{t_2-t_1} = \lim_{\Delta t \to 0} \frac{\bvv{r}(t_1 + \Delta t) - \bvv{r}(t_1)}{\Delta t} = \dot{\bvv{r}}(t)\]
\end{definition}

Notiamo che la velocita' istantanea e' un vettore parallelo allo spostamento infinitesimo, e quindi e' in particolare tangente alla traiettoria
del punto materiale. Scrivendola come derivata delle componenti otteniamo:
\[\bvv{v} = \dot{x}(t)\bh{i} + \dot{y}(t)\bh{j} + \dot{z}(t)\bh{k} = v_x\bh{i} + v_y\bh{j} + v_z\bh{k}\]

\begin{definition}
    Si dice \textbf{accelerazione media} il vettore $\ang{\bvv{a}}$ che descrive il cambiamento medio della velocita' del punto materiale tra due istanti di tempo.
    \[\ang{\bvv{a}} = \bvv{a_m}(t_1, t_2) = \frac{\Delta\bvv{v}}{\Delta t} = \frac{\bvv{v}(t_2) - \bvv{v}(t_1)}{t_2-t_1}\] 
    Si dice invece \textbf{accelerazione istantanea} il vettore $\bvv{a}$ che descrive l'accelerazione del punto materiale in ogni istante di tempo.
    \[\bvv{a} = \lim_{t_2 \to t_1} \frac{\bvv{v}(t_2) - \bvv{v}(t_1)}{t_2-t_1} = \lim_{\Delta t \to 0} \frac{\bvv{v}(t_1 + \Delta t) - \bvv{v}(t_1)}{\Delta t} = \dot{\bvv{v}}(t) = \ddot{\bvv{r}}(t)\]
\end{definition}

\theoremstyle{plain}
\begin{remark}
    L'accelerazione e' diversa da 0 se la velocita' cambia in modulo, ma anche se cambia in direzione!
\end{remark}

\section{Moto ad una dimensione}
\theoremstyle{definition}
\begin{definition}
    Si dice \textbf{legge oraria del moto} la legge che associa ad ogni istante di tempo la posizione del corpo
    sull'asse di riferimento:
    \[x(t) = f(t)\]
\end{definition}

Dalla legge oraria possiamo ricavare la velocita' e l'accelerazione del corpo tramite la derivata:
\[x(t) = f(t) \implies v(t) = \dot{x}(t) \implies a(t) = \dot{v}(t) = \ddot{x}(t)\]

\subsection{Stato di quiete}
Si dice che un punto materiale e' in stato di quiete se vale che $x(t) = x_0$ costante $\forall t > 0$.
Da questa relazione si ricavano la velocita' e l'accelerazione:
\begin{subequations}
\begin{align}
    &x(t) = x_0 \\
    &v(t) = \dot{x}(t) = 0 \\
    &a(t) = \dot{v}(t) = 0 
\end{align}
\end{subequations}

\subsection{Moto a velocita' costante}
Si dice che un punto materiale si muove di moto rettilineo uniforme se vale che $v(t) = v_0$ costante $\forall t > 0$.
Da questa relazione si ricavano la posizione e l'accelerazione:
\begin{subequations}
\begin{align}
    &x(t) = \int_{0}^t v(t) dt = x_0 + v_0t \\
    &v(t) = v_0 \\
    &a(t) = \dot{v}(t) = 0 
\end{align}    
\end{subequations}
dove $x_0 = x(0)$.
Se consideriamo il vettore posizione e il vettore velocita', otteniamo che
\[\bvv{v}(t) = \ang{\bvv{v}} = \frac{\Delta \bvv{r}}{\Delta t} = \frac{\bvv{r}(t) - \bvv{r_0}}{\Delta t}\]
da cui segue
\[\bvv{r}(t) = \bvv{r_0} + \ang{\bvv{v}}(t - t_0)\]
dove $\ang{\bvv{v}}$ e' il vettore velocita' media (che e' sempre uguale alla velocita' 
istantanea nel caso di moto a velocita' costante). 
Da questo segue il fatto che il moto sia lungo una traiettoria rettilinea.

\subsection{Moto ad accelerazione costante}
Si dice che un punto materiale si muove di moto rettilineo uniformemente accelerato
se vale che $a(t) = a_0$ costante $\forall t > 0$.
Da questa relazione si ricavano la posizione e la velocita':
\begin{subequations}
\begin{align}
&a(t) = a_0 \\
&v(t) = \int_{0}^t a(t) dt = v_0 + a_0t \\
&x(t) = \int_{0}^t v(t) dt = x_0 + v_0t + \frac{a_0}{2}t^2
\end{align}
\end{subequations}

dove $x_0 = x(0)$ e $v_0 = v(0)$.
Dalla seconda possiamo ricavare
\begin{numcases}{v(t) = v_0 + a_0t \implies}
    t = \frac{v(t) - v_0}{a_0} \label{time} \\
    a_0 = \frac{v(t) - v_0}{t} \label{accel}
\end{numcases}
Sostituendo la \ref{time} nell'espressione per $x(t)$ e riordinando otteniamo
\begin{equation}
    v^2(t) = v_0^2 + 2a_0(x-x_0) \label{MUA_senza_t}
\end{equation}
Sostituendo invece la \ref{accel} nell'espressione per $x(t)$ e riordinando otteniamo
\begin{equation}
    x(t) = x_0 + \frac{1}{2}(v(t)+v_0)t \label{MUA_senza_a}
\end{equation}

Se consideriamo il vettore velocita' e il vettore accelerazione, otteniamo che
\[\bvv{a}(t) = \ang{\bvv{a}} = \frac{\Delta \bvv{v}}{\Delta t} = \frac{\bvv{v}(t) - \bvv{v_0}}{\Delta t}\]
da cui segue
\[\bvv{v}(t) = \bvv{v_0} + \ang{\bvv{a}}(t - t_0)\]
dove $\ang{\bvv{a}}$ e' il vettore accelerazione media (che e' sempre uguale all'accelerazione 
istantanea nel caso di moto ad accelerazione costante).
Da questo segue il fatto che il moto sia lungo una traiettoria rettilinea.


\subsection{Moto a caduta libera}
E' un caso particolare di un moto uniformemente accelerato. Consideriamo un sistema ortogonale $XY$ e un corpo
che si trova inizialmente nel punto $(x_0, y_0) = (0, y_0)$ e che si muove verso il basso con una velocita' di modulo iniziale $v_0$. 
I vettori che rappresentano lo stato del corpo saranno quindi:
\begin{subequations}
\begin{align}
    &\bvv{r}(0) = y_0\bh{j}\\
    &\bvv{v}(0) = v_0\bh{j}\\
    &\bvv{a}(0) = -g\bh{j}
\end{align}    
\end{subequations}
dove $g$ e' l'accelerazione di gravita' terrestre.
Notiamo quindi che il moto si svolge unicamente nella direzione dell'asse $Y$.

\subsubsection{Caduta da un'altezza}
Supponiamo che il corpo cada da un'altezza $h$ da fermo (cioe' $v_0 = 0$).
Avremo:
\begin{subequations}
\begin{align}
    &y(t) = h - \frac{1}{2}gt^2 \\
    &v(t) = -gt \label{v_caduta_libera}\\
    &a(t) = -g
\end{align}    
\end{subequations}

Da queste equazioni possiamo ricavare il tempo di caduta e la velocita' di impatto del corpo con il suolo.
Infatti quando il corpo tocca il suolo all'istante $t_f$, avremo che
\begin{alignat}{2}
          y(t_f) &= h - \frac{1}{2}gt_f^2 = 0    \nonumber\\
    \implies t_f &= \sqrt{\frac{2h}{g}}             &\rlap{\text{(Tempo di caduta)}}\\
    \intertext{dunque sostituendo $t_f$ nell'equazione della velocita' \ref{v_caduta_libera}:}
          v(t_f) &= -\sqrt{2gh}                  &\rlap{\text{(Velocita' finale)}}
\end{alignat}    


\subsubsection{Lancio verso l'alto}
Supponiamo ora che il corpo venga lanciato verso l'alto con una velocita' iniziale $v_0 \neq 0$.
Avremo:
\begin{subequations}
\begin{align}
    &y(t) = y_0 + v_0t - \frac{1}{2}gt^2 \label{y_lancio}\\
    &v(t) = v_0 - gt \\
    &a(t) = -g
\end{align}    
\end{subequations}

Possiamo calcolare il punto di altezza massima $y_M$ e il tempo necessario per raggiungerlo $t_M$ da queste equazioni.
Infatti al punto di altezza massima la velocita' sara' nulla, dunque avremo che    
\begin{alignat}
    {2}
          v(t_M) &= v_0 - gt_M = 0       \nonumber \\
    \implies t_M &= \frac{v_0}{g}               \\
    \intertext{dunque sostituendo $t_M$ nell'equazione della posizione \ref{y_lancio}}
    y_M     &= y_0 + v_0t_M - \frac{1}{2}gt_M^2 \nonumber \\
            &= y_0 + \frac{v_0^2}{2g}               
\end{alignat}

\section{Moto in due dimensioni}
Consideriamo ora moti che avvengono in due dimensioni.
Siano $\bvv{r}$, $\bvv{v}$ e $\bvv{a}$ i vettori che descrivono la posizione, la velocita' e l'accelerazione del corpo nel tempo.
Allora il moto e' rettilineo se $\bvv{a} \parallel \bvv{v}$ oppure se $\bvv{a} = \bvv{0}$, altrimenti il moto e' bidimensionale.
Le leggi del moto sono le stesse del caso unidimensionale
\begin{subequations}
\begin{align}
    &\bvv{a}(t) = \bvv{a_0} \\
    &\bvv{v}(t) = \bvv{v_0} + \bvv{a_0}t \\
    &\bvv{r}(t) = \bvv{r_0} + \bvv{v_0}t + \frac{\bvv{a_0}}{2}t^2
\end{align}
\end{subequations}
ma possono essere separate in due equazioni che si riferiscono al moto sui due assi
\begin{subequations}
    \begin{align}
        &\bvv{r}\text{:}
        \begin{cases}{}
            x(t) = x_0 + (v_0\cos{\theta})t + \frac{1}{2}(a_0\cos{\psi})t^2 \\
            y(t) = y_0 + (v_0\sin{\theta})t + \frac{1}{2}(a_0\sin{\psi})t^2
        \end{cases} \\
        &\bvv{v}\text{:}
        \begin{cases}{}
            v_x(t) = v_0\cos{\theta} + (a_0\cos{\psi})t \\
            v_y(t) = v_0\sin{\theta} + (a_0\sin{\psi})t
        \end{cases} \\
        &\bvv{a}\text{:}
        \begin{cases}{}
            a_x(t) = a_0\cos{\psi} \\
            a_y(t) = a_0\sin{\psi}
        \end{cases}
    \end{align}
\end{subequations}
dove $v_0$ e' il modulo del vettore $\bvv{v_0}$, $\theta$ e' l'angolo formato da $\bvv{v_0}$ con l'asse $X$, 
$a_0$ e' il modulo del vettore $\bvv{a_0}$ e $\psi$ e' l'angolo formato da $\bvv{a_0}$ con l'asse $X$.

\subsection{Moto del proiettile}
Consideriamo un caso particolare del moto accelerato bidimensionale in cui $\bvv{a} = -g\bh{j}$.
Se sostituiamo nelle equazioni precedenti otteniamo
\begin{subequations}
    \begin{align}
        &\bvv{r}\text{:}
        \begin{cases}{}
            x(t) = x_0 + (v_0\cos{\theta})t\\
            y(t) = y_0 + (v_0\sin{\theta})t - \frac{1}{2}gt^2
        \end{cases} \label{proj_pos}\\
        &\bvv{v}\text{:}
        \begin{cases}{}
            v_x(t) = v_0\cos{\theta} \\
            v_y(t) = v_0\sin{\theta} - gt
        \end{cases} \\
        &\bvv{a}\text{:}
        \begin{cases}{}
            a_x(t) = 0 \\
            a_y(t) = -g
        \end{cases}
    \end{align}
\end{subequations}
Possiamo notare che, considerando i moti sui due assi separatamente, il moto del punto sull'asse $X$ e' rettilineo uniforme, 
mentre quello sull'asse $Y$ e' uniformemente accelerato.

\subsubsection{Traiettoria del proiettile}
Se ricaviamo $t$ dalla formula di $x(t)$ da \ref{proj_pos} (ottenendo $t = \frac{x-x_0}{v_0\cos{\theta}}$)
e lo sostituiamo nella formula di $y(t)$ otteniamo la traiettoria tracciata dal proiettile, 
cioe' una curva di secondo grado del tipo
\begin{equation}
    y(x) = y_0 + \tan{\theta}(x-x_0) - \frac{g}{2(v\cos{\theta})^2}(x-x_0)^2 \label{proj_traj}
\end{equation}
che rappresenta la traiettoria del proiettile al variare della $x$.

\subsubsection{Punto di altezza massima}
Come nel caso del corpo lanciato verticalmente, il punto di altezza massima viene raggiunto nell'istante
di tempo $t_h$ tale che $v_y(t_h) = 0$.
Sostituendo nelle equazioni otteniamo:   
\begin{alignat}
    {2}
        v_y(t_h) &= v_0\sin{\theta} - gt_h = 0             \nonumber \\       
    \implies t_h &= \frac{v_0\sin{\theta}}{g}                   \\
    \intertext{dunque sostituendo $t_h$ nell'equazione della posizione \ref{proj_pos}}
    y(t_h)   &= y_0 + (v_0\sin{\theta})t_h - \frac{1}{2}gt_h^2  \nonumber \\
             &= y_0 + \frac{(v_0\sin{\theta})^2}{g} - \frac{(v_0\sin{\theta})^2}{2g} \nonumber \\
             &= y_0 + \frac{(v_0\sin{\theta})^2}{2g}            \label{h_max_proj_theta}
\end{alignat}
che e' massimo quando $\theta = \frac{\pi}{2}$, ed e' dunque uguale a
\begin{equation}
    y_M = \frac{v_0^2}{2g} \label{h_map_proj}
\end{equation}

\subsubsection{Gittata}
Per calcolare la gittata del proiettile ci bastera' capire in che punto esso raggiunge 
l'altezza che aveva all'inizio del lancio;
bastera' cioe' trovare $\Delta x = x(t_g)-x_0$, dove $t_g > 0$ e' tale che $y(t_g) = y_0$.
\begin{alignat}
    {2}
             y(t_g) &= y_0 + (v_0\sin{\theta})t_g - \frac{1}{2}gt_g^2 = y_0 \nonumber \\
    \implies t_g &= \frac{2v_0\sin{\theta}}{g}                   \\
    \intertext{dunque sostituendo $t_g$ nell'equazione della posizione \ref{proj_pos}}
    \Delta x &= (v_0\sin{\theta})t_g  \nonumber \\
             &= \frac{v_0^2\sin{2\theta}}{g}            \label{gittata_proj_theta}
\end{alignat}
che e' massimo quando $\theta = \frac{\pi}{4}$ ed e' dunque uguale a
\begin{equation}
    x_{g_M} = \frac{x_0^2}{g}   \label{gittata_proj}
\end{equation}

\subsubsection{Impatto col suolo}
Invece per calcolare la distanza percorsa per impattare il suolo e' sufficiente trovare l'intersezione con l'asse $X$;
cioe' bastera' trovare $\Delta x = x(t_s)-x_0$, dove $t_s > 0$ e' tale che $y(t_s) = 0$.
\begin{alignat}
    {2}
             y(t_s) &= y_0 + (v_0\sin{\theta})t_s - \frac{1}{2}gt_s^2 = 0               \nonumber \\
    \implies t_s &= \frac{1}{g}\left(v_0\sin{\theta} + \sqrt{(v_0\sin{\theta})^2 + 2gy_0}\right)   \\
    \intertext{dunque sostituendo $t_s$ nell'equazione della posizione \ref{proj_pos}}
    \Delta x &= (v_0\sin{\theta})t_s   \label{impatto_proj_theta}
\end{alignat}
\chapter{Dinamica}

La dinamica e' la branca della fisica che si occupa di studiare le cause delle variazioni del moto. Ci occuperemo inizialmente della dinamica del punto materiale, cioe' il moto del corpo e le forze che agiscono su di esso sono riferiti ad un punto $(x, y, z)$ dello spazio; passeremo poi a studiare la dinamica di sistemi con piu' gradi di liberta'.

\begin{definition}
    Si dice massa inerziale di un corpo la resistenza che il corpo oppone al cambiamento di velocita' risultante dall'azione di una forza.    
\end{definition}
La massa e' una grandezza scalare e additiva, la cui unita' di misura e' il chilogrammo.

\begin{definition}
    La forza e' una quantita' vettoriale che descrive le interazioni fra corpi.
\end{definition}
La forza e' rappresentata da un vettore applicato, cioe' viene descritta da intensita', direzione, verso e punto di applicazione. Da un punto di vista pratico, per misurare una forza si sfrutta la proprieta' che esa ha di deformare gli oggetti.

Vi sono due tipi di forze, distinte dal loro raggio di azione.
\begin{enumerate}
    \item Le forze a distanza (o forze a lungo raggio) non richiedono che i corpi su cui agiscono siano a contatto tra loro. Gli esempi piu' comuni sono le 4 iterazioni fondamentali:
        \begin{itemize}
            \item iterazione gravitazionale (mediata dal gravitone (forse));
            \item iterazione elettromagnetica (mediata dai fotoni);
            \item iterazione nucleare debole (mediata dai bosoni W e Z);
            \item iterazione nucleare forte (mediata dai gluoni);
        \end{itemize}
    \item Le forze a contatto (o forze a corto raggio) sono forze che agiscono a contatto tra i corpi macroscopici, e derivano dalle interazioni elettromagnetiche fra atomi e molecole che costituiscono la materia. Alcuni esempi sono:
        \begin{itemize}
            \item forze esplicate dai vincoli (tensione di fili; forza normale associata ad una superficie che si oppone alla deformazione);
            \item forze di attrito (dinamico e statico);
            \item forze elastiche (interazioni elettromagnetiche che si oppongono alle deformazioni dei corpi).
        \end{itemize}
\end{enumerate}

\section{Principi della dinamica}

\begin{principle}[Primo principio della dinamica]
    Sia $\bvv{R}$ la risultante delle forze che agiscono su un punto materiale. Allora se $\bvv{R} = \bvv{0}$ allora il corpo permane nel suo stato di quiete o moto rettilineo uniforme.
\end{principle}

\begin{definition}
    Si dice che un sistema di riferimento e' inerziale se dato un corpo e appurato che la risultante delle forze che agiscono su quel corpo e' nulla, allora il corpo e' in quiete o si muove di moto rettilineo uniforme rispetto al sistema.
\end{definition}

Segue dalla definizione che un sistema in moto rettilineo uniforme rispetto a un sistema inerziale e' anch'esso inerziale.

\begin{principle}[Secondo principio della dinamica]
    Sia $\bvv{R}$ la risultante delle forze che agiscono su un punto materiale. Allora se $\bvv{R} \neq \bvv{0}$ allora il corpo subisce un'accelerazione che e' direttamente proporzionale alla risultante delle forze. La costante di proporzionalita' e' la massa inerziale, secondo la formula:
    \begin{equation} \label{second_principle}
        \bvv{R} = m\bvv{a}
    \end{equation}
\end{principle}

Facciamo ora delle considerazioni sui primi due principi.
I primi due principi della dinamica sono validi soltanto in sistemi di riferimento inerziali. Essi ci danno un modo per studiare il moto di un corpo a partire dalle forze che agiscono su di esso. Studiando il diagramma di corpo libero, cioe' il diagramma delle forze che agiscono su un corpo possiamo calcolarne la risultante e stabilire, a seconda del risultato, se il corpo ha un'accelerazione nulla o meno.

\begin{principle}[Terzo principio della dinamica]
    Siano $A$, $B$ due corpi puntiformi di massa $m_A, m_B$. Allora se il corpo $A$ esercita sul corpo $B$ una forza $\bvv{F}_{B,A}$, allora il corpo $B$ esercitera' necessariamente una forza $\bvv{F}_{A, B}$ sul corpo $A$, tale che
    \begin{equation} \label{third_principle}
        \bvv{F}_{B, A} = -\bvv{F}_{A, B}   
    \end{equation}
\end{principle}

Non bisogna confondere il terzo principio della dinamica con le forze vincolari: le forze vincolari sono applicate allo stesso corpo che esercita la forza, mentre il secondo principio coinvolge due corpi.

\section{Iterazione gravitazionale}
L'iterazione gravitazionale e' l'iterazione fondamentale che spiega la caduta dei gravi e il moto dei pianeti.

\begin{definition}
    Siano $A$, $B$, due corpi di massa $m_A$, $m_B$. Allora i due corpi si attraggono con forze che sono:
    \begin{itemize}
        \item dirette lungo la congiungente dei centri di massa;
        \item attrattive;
        \item di intensita' uguali, proporzionali al prodotto delle masse e inversamente proporzionali al quadrato della distanza dei centri di massa, secondo la formula
        \begin{equation} \label{forza_grav}
            \abs{\bvv{F}_{A, B}} = \abs{\bvv{F}_{B, A}} = G\frac{m_Am_b}{r^2}
        \end{equation}
        dove $G = 6,64 \times 10^{-11} \text{Nm}^2\text{/kg}$ e' la costante di gravitazione universale.
    \end{itemize}
\end{definition}

Dato che il valore di $G$ e' molto piccolo la forza gravitazionale ha un effetto trascurabile a meno che la massa dei corpi in esame sia grande (almeno $10^{10}$ kg) ed essi sono relativamente vicini.

In teoria la massa inerziale e la massa gravitazionale, cioe' la grandezza scalare proporzionale alla forza di gravita', rappresentano due concetti distinti di massa.

Se consideriamo un corpo sulla superficie terrestre oppure ad un'altezza trascurabile, la forza peso e' costante in modulo e in direzione (radiale, diretta verso il centro di massa della Terra). Dunque possiamo approssimarla con:
\begin{equation}
    \vmag{P} = G\frac{m_Tm_A}{(R_T + h)^2} \approx m_AG\frac{m_T}{R^2_T} = m_Ag
\end{equation}
dove $g = 9,81$ N/kg e' l'accelerazione di gravita' terrestre.

Dagli esperimenti e' stato poi dimostrato che la massa gravitazionale e' equivalente alla massa inerziale: dunque tutti i corpi in caduta libera hanno la stessa accelerazione quando si trovano sullo stesso pianeta e ad altitudini comparabili, sotto l'azione della sola forza gravitazionale.
L'accelerazione di un corpo in caduta libera e' dunque
\begin{equation}
    \bvv{a} = \frac{\bvv{P}}{m_A} = \bvv{g} = -g\bh{j}
\end{equation}

\section{Forze di contatto}
Quando due corpi macroscopici sono a contatto tra di loro agiscono delle forze che derivano dalle interazioni elettromagnetiche della materia. Il primo tipo che studieremo sono le forze legate ai vincoli
Possono essere legate a vincoli di due tipi:
\begin{itemize}
    \item vincoli di superfici (come forze di attrito, forze normali);
    \item vincoli unidimensionali (tensioni di corde, funi, fili).
\end{itemize}

\subsection{Forze legate a vincoli di superfici}
Le forze legate ai vincoli di superfici possono essere
\begin{enumerate}
    \item forze di attrito statico o dinamico: esse sono parallele alla superficie di contatto e opposte al moto relativo dei due corpi;
    \item forze normali: esse sono perpendicolari alla superficie di contatto e impediscono ai corpi di compenetrarsi.
\end{enumerate}
Se la superficie di contatto e' piana allora le forze normali bilanciano il peso del corpo sulla superficie, dunque 
\begin{equation}
    \left(\sum_i \bvv{F}_i \right)_{\perp} = \bvv{0} \implies \bvv{a}_{\perp} = \bvv{0}.
\end{equation}
Se la superficie di contatto non e' piana allora le forze normali non bilanciano il peso del corpo sulla superficie, dunque 
\begin{equation}
    \left(\sum_i \bvv{F}_i \right)_{\perp} \neq \bvv{0} \implies \bvv{a}_{\perp} \neq \bvv{0}.
\end{equation}

\begin{example}[Piano inclinato liscio]
    Supponiamo di avere un piano inclinato con angolo alla base $\theta$. Per descrivere il moto del corpo sul piano inclinato, scegliamo come sistema di riferimento un sistema $XY$ dove l'asse $X$ e' parallelo al piano inclinato, l'asse $Y$ perpendicolare ad esso, e l'origine degli assi sia nel punto dove si trova il corpo al tempo $t_0 = 0$s.

    Se disegnamo il diagramma del corpo libero notiamo che le forze in gioco sono il peso $\bvv{P}$ e la reazione vincolare del piano $\bvv{N}$.
    Sia $\bvv{R}$ la forza risultante; allora
    \begin{equation}
        \bvv{R} = \begin{cases}
            R_x = mg\sin\theta \\
            R_y = N - mg\cos\theta
        \end{cases}
    \end{equation}
    Per il secondo principio della dinamica vale allora
    \begin{align}
        \begin{cases}
            R_x = mg\sin\theta = ma_x \\
            R_y = N - mg\cos\theta = ma_y = 0 
        \end{cases}
        &\implies
        \begin{cases}
            a_x = g\sin\theta \\
            N = mg\cos\theta
        \end{cases}
    \end{align}

    Supponendo che la rampa sia lunga $L$ e che il corpo si muova inizialmente con una velocita' $v_0 = 0$ possiamo calcolare la velocita' con cui il corpo giunge alla fine e il tempo che impiega per percorrerla $t_f$. Dalla legge oraria del moto uniformemente accelerato otteniamo
    \begin{alignat*}{1}
        L &= \frac{1}{2}g\sin\theta t_f^2 \\
        \intertext{da cui possiamo ricavare $t_f$}
        \implies t_f &= \sqrt{\frac{2L}{g\sin\theta}} \\
        \intertext{Sostituendo $h = L\sin\theta$}
              &= \sqrt{\frac{2h}{g\sin^2\theta}} \\
              &= \frac{1}{\sin\theta}\sqrt{\frac{2h}{g}}
        \intertext{Sostituendo $t_f$ nella formula per la velocita' otteniamo infine}
        \implies v_f &= a_xt_f\\
              &= g\sin\theta t_f\\
              &= \sqrt{2gh}
    \end{alignat*}
\end{example}

\begin{example}
    [Piano inclinato liscio e forza orizzontale]
    Supponiamo di avere un piano inclinato con angolo alla base $\theta$ e un corpo di massa $m$ che viene spinto su per il piano inclinato tramite una forza orizzontale $\bvv{F_e}$. Sappiamo inoltre che la velocita' del corpo e' costante. Calcoliamo quanto vale la forza orizzontale e la reazione vincolare $\bvv{N}$.

    Dato che per ipotesi il corpo si muove a velocita' costante, allora $\bvv{R} = \bvv{0}$, cioe' $R_x = 0$ e $R_y = 0$. Dunque disegnando il diagramma del corpo libero:
    \begin{alignat*}{1}
        &\begin{cases}
            R_x = mg\sin\theta - F_e\cos\theta = 0\\
            R_y = N - mg\cos\theta - F_e\sin\theta = 0
        \end{cases} \\
        \implies &\begin{cases}
            F_e\cos\theta = mg\sin\theta\\
            N = mg\cos\theta + F_e\sin\theta
        \end{cases} \\
        \implies &\begin{cases}
            F_e = mg\tan\theta\\
            N = mg\cos\theta + mg\tan\theta \sin\theta = mg\cos\theta (1 + \tan^2 \theta)
        \end{cases}
    \end{alignat*}
\end{example}

Le forze di attrito sono esercitate parallelamente alla superficie di contatto. Esse si dividono in due categorie: \begin{itemize}
    \item attrito statico: si oppone all'inizio del moto del corpo;
    \item attrito dinamico: si oppone al moto di un corpo che non e' in quiete.
\end{itemize}
Le forze di attrito dinamico sono inferiori a quelle di attrito statico.

\subsubsection{Attrito statico}

L'attrito statico si oppone al moto, e ha un'intensita' tale che l'accelerazione e' nulla. La forza di attrito ha direzione tangente alla superficie e ha verso opposto alla forza applicata. Finche' il corpo rimane in uno stato di quiete si ha che $\vmag{f_s} = \vmag{F_{\text{ext}}}$ (dove $\bvv{f_s}$ e' la forza di attrito e $\bvv{F_{\text{ext}}}$ e' la forza esterna). 
In generale si ha che $\vmag{f_s} \leq \mu_s N$, dove $\mu_s$ e' il coefficiente di attrito statico, dunque al massimo $\vmag{f_s} = \mu_s N$.

\subsubsection{Attrito dinamico}

Se la forza esterna supera il valore $\mu_s N$ il corpo comincia a muoversi e l'attrito diventa attrito dinamico, che ha la stessa direzione del moto, verso opposto e modulo \[
    \vmag{f_d} = \mu_d N    
\]
dove $\mu_d$ e' il coefficiente di attrito dinamico, e risulta sempre $\mu_d \leq \mu_s$.

\begin{example}
    Un corpo e' lanciato con velocita' $\bvv{v_0}$ lungo un piano scabro con attrito dinamico $\mu_d$. Dopo quanto tempo si ferma? Che tratto percorre prima di fermarsi?

    Consideriamo il modo nelle due dimensioni. Sull'asse $X$ il corpo si muove di moto accelerato (con accelerazione negativa data dall'attrito dinamico), mentre sull'asse $Y$ il corpo e' fermo dunque la somma delle forze sara' $0$. Dunque:
    \begin{equation*}
        \begin{cases}
            R_x = -f_d \\
            R_y = N - mg = 0            
        \end{cases}
        \implies \begin{cases}
            -f_d = -\mu_d N = -\mu_d mg = ma_x\\
            N = mg
        \end{cases} 
    \end{equation*}
    Quindi $a_x = -\mu_d g$. Sostituendolo nelle equazioni del moto otteniamo\[
        t_f = \frac{v_0}{-a_x} = \frac{v_0}{\mu_d g}
    \] e inoltre \[
        x_f = v_0t_f + \frac{1}{2}a_xt_f^2 = \frac{v_0^2}{\mu_d g} - \frac{1}{2}\mu_d g\frac{v_0^2}{\mu_d^2g^2} = \frac{v_0^2}{2\mu_d g}.
    \]
\end{example}


\begin{example}
    Un corpo e' lanciato con velocita' $\bvv{v_0}$ lungo un piano scabro con attrito dinamico $\mu_d$. Se volessi mantenere il corpo a velocita' costante, che forza esterna dovrei fornire? Ci vuole meno forza a tirare verso l'alto ($\theta > 0$) oppure a spingere verso il basso ($\theta < 0$)?

    Consideriamo il modo nelle due dimensioni includendo una forza $F$ inclinata con un angolo $\theta$: per il primo principio della dinamica la risultante delle forze sul corpo dovra' essere nulla. Dunque:
    \begin{equation*}
        \begin{cases}
            R_x = -f_d + F\cos\theta = -\mu_d N + F\cos\theta = 0\\
            R_y = N - mg + F\sin\theta = 0       
        \end{cases}
        \implies \begin{cases}
            -\mu_d N + F\cos\theta = -\mu_d (mg - F\sin\theta) + F\cos\theta = -\mu_d mg + F(\cos\theta + \mu_d\sin\theta) = 0\\
            N = mg - F\sin\theta
        \end{cases} 
    \end{equation*}
    dunque \[
        F = \frac{\mu_d mg}{\cos\theta + \mu_d \sin\theta}.    
    \]    
\end{example}

\begin{example}[Piano inclinato con attrito]
    Supponiamo di avere un piano inclinato con attrito e che il corpo con massa $m$ sia in uno stato di quiete.

    Questo significa che la forza di attrito $\bvv{f_a}$ riesce a bilanciare le altre forze, in modo che la risultante lungo l'asse $X$ sia nulla. Al massimo vale quindi $f_a = \mu_s N$.
    \begin{equation*}
        \begin{cases}
            R_x = N - mg\cos\theta = 0\\
            R_y = -f_a + mg\sin\theta = 0
        \end{cases} \implies
        \begin{cases}
            N = mg\cos\theta\\
            f_a = mg\sin\theta
        \end{cases} 
    \end{equation*}
    
    Definiamo pendenza critica l'angolo $\theta_c$ oltre il quale la forza di attrito statica non riesce a controbilanciare la forza di gravita'. Quindi quando $\theta > \theta_c$ segue che NON HO CAPITO E NON MI VA DI CAPIRE
\end{example}

\subsection{Tensioni}

Quando non ci sono altre forze applicate in una qualsiasi sua parte una fune ideale (cioe' inestensibile e di massa trascurabile) esercita sui corpi fissati ai suoi estremi due forze che hanno la direzione della fune, stessa intensita' e versi opposti.

\section{Moto circolare}

Si dice moto circolare uniforme un moto su una traiettoria circolare di raggio $R$ con velocita' costante in modulo. Dato che la velocita' cambia direzione avremo che $\bvv a \neq 0$.

Chiamiamo $s$ lo spazio percorso sulla circonferenza, $\theta = \frac{s}{R}$ l'angolo spazzato. Allora valgono le seguenti: \[
    \begin{cases}
        \Delta \bvv r = \bvv{r_f} - \bvv{r_i}\\
        \Delta \theta= \theta_f - \theta_i\\
        \Delta s = R\Delta \theta
    \end{cases}    
\]

\begin{definition}
    Diciamo velocita' angolare media di un moto circolare la grandezza $\ang{\omega} = \frac{\Delta \theta}{\Delta t}$. Dunque la velocita' angolare istantanea sara' $\omega = \lim_{\Delta t \to 0} \ang{\omega} = \frac{d\theta}{dt}$.
\end{definition}

\begin{definition}
    Diciamo accelerazione angolare media di un moto circolare la grandezza $\ang{\alpha} = \frac{\Delta \omega}{\Delta t}$. Dunque la velocita' angolare istantanea sara' $\alpha = \lim_{\Delta t \to 0} \ang{\alpha} = \frac{d\omega}{dt}$.
\end{definition}

\end{document}
\chapter{Cinematica del punto materiale}

\section{Definizioni fondamentali}
\begin{definition}
    Si dice raggio vettore o \textbf{vettore posizione} il vettore $\bvv{r}$ che descrive la posizione del punto materiale rispetto ai tre assi al variare del tempo.
    \[\bvv{r}(t) = x(t)\bh{i} + y(t)\bh{j} + z(t)\bh{k}\]
\end{definition}

\begin{definition}
    Si dice \textbf{vettore spostamento} il vettore $\bvv{s}$ che descrive lo spostamento del punto materiale tra due istanti di tempo.
    \[\bvv{s} = \Delta\bvv{r} = \bvv{r_f} - \bvv{r_i} = \bvv{r}(t_f) - \bvv{r}(t_i)\]
\end{definition}

\begin{definition}
    Si dice \textbf{velocita' media} il vettore $\ang{\bvv{v}}$ che descrive la velocita' media del punto materiale tra due istanti di tempo.
        \begin{equation} \label{def_vel_media}
            \ang{\bvv{v}} = \bvv{v_m}(t_1, t_2) = \frac{\Delta\bvv{r}}{\Delta t} = \frac{\bvv{r}(t_2) - \bvv{r}(t_1)}{t_2-t_1}
        \end{equation}
    Si dice invece \textbf{velocita' istantanea} il vettore $\bvv{v}$ che descrive la velocita' del punto materiale in ogni istante di tempo.
    La velocita' istantanea e' definita come il limite della velocita' media quando $t_2 \to t_1$, o equivalentemente se supponiamo
    $t_2 = t_1 + \Delta t$ la velocita' istantanea e' il limite della velocita' media quando $\Delta t \to 0$.
    \[\bvv{v} = \lim_{t_2 \to t_1} \frac{\bvv{r}(t_2) - \bvv{r}(t_1)}{t_2-t_1} = \lim_{\Delta t \to 0} \frac{\bvv{r}(t_1 + \Delta t) - \bvv{r}(t_1)}{\Delta t} = \dot{\bvv{r}}(t)\]
\end{definition}

Notiamo che la velocita' istantanea e' un vettore parallelo allo spostamento infinitesimo, e quindi e' in particolare tangente alla traiettoria
del punto materiale. Scrivendola come derivata delle componenti otteniamo:
\[\bvv{v} = \dot{x}(t)\bh{i} + \dot{y}(t)\bh{j} + \dot{z}(t)\bh{k} = v_x\bh{i} + v_y\bh{j} + v_z\bh{k}\]

\begin{definition}
    Si dice \textbf{accelerazione media} il vettore $\ang{\bvv{a}}$ che descrive il cambiamento medio della velocita' del punto materiale tra due istanti di tempo.
    \[\ang{\bvv{a}} = \bvv{a_m}(t_1, t_2) = \frac{\Delta\bvv{v}}{\Delta t} = \frac{\bvv{v}(t_2) - \bvv{v}(t_1)}{t_2-t_1}\] 
    Si dice invece \textbf{accelerazione istantanea} il vettore $\bvv{a}$ che descrive l'accelerazione del punto materiale in ogni istante di tempo.
    \[\bvv{a} = \lim_{t_2 \to t_1} \frac{\bvv{v}(t_2) - \bvv{v}(t_1)}{t_2-t_1} = \lim_{\Delta t \to 0} \frac{\bvv{v}(t_1 + \Delta t) - \bvv{v}(t_1)}{\Delta t} = \dot{\bvv{v}}(t) = \ddot{\bvv{r}}(t)\]
\end{definition}

\theoremstyle{plain}
\begin{remark}
    L'accelerazione e' diversa da 0 se la velocita' cambia in modulo, ma anche se cambia in direzione!
\end{remark}

\section{Moto ad una dimensione}
\theoremstyle{definition}
\begin{definition}
    Si dice \textbf{legge oraria del moto} la legge che associa ad ogni istante di tempo la posizione del corpo
    sull'asse di riferimento:
    \[x(t) = f(t)\]
\end{definition}

Dalla legge oraria possiamo ricavare la velocita' e l'accelerazione del corpo tramite la derivata:
\[x(t) = f(t) \implies v(t) = \dot{x}(t) \implies a(t) = \dot{v}(t) = \ddot{x}(t)\]

\subsection{Stato di quiete}
Si dice che un punto materiale e' in stato di quiete se vale che $x(t) = x_0$ costante $\forall t > 0$.
Da questa relazione si ricavano la velocita' e l'accelerazione:
\begin{subequations}
\begin{align}
    &x(t) = x_0 \\
    &v(t) = \dot{x}(t) = 0 \\
    &a(t) = \dot{v}(t) = 0 
\end{align}
\end{subequations}

\subsection{Moto a velocita' costante}
Si dice che un punto materiale si muove di moto rettilineo uniforme se vale che $v(t) = v_0$ costante $\forall t > 0$.
Da questa relazione si ricavano la posizione e l'accelerazione:
\begin{subequations}
\begin{align}
    &x(t) = \int_{0}^t v(t) dt = x_0 + v_0t \\
    &v(t) = v_0 \\
    &a(t) = \dot{v}(t) = 0 
\end{align}    
\end{subequations}
dove $x_0 = x(0)$.
Se consideriamo il vettore posizione e il vettore velocita', otteniamo che
\[\bvv{v}(t) = \ang{\bvv{v}} = \frac{\Delta \bvv{r}}{\Delta t} = \frac{\bvv{r}(t) - \bvv{r_0}}{\Delta t}\]
da cui segue
\[\bvv{r}(t) = \bvv{r_0} + \ang{\bvv{v}}(t - t_0)\]
dove $\ang{\bvv{v}}$ e' il vettore velocita' media (che e' sempre uguale alla velocita' 
istantanea nel caso di moto a velocita' costante). 
Da questo segue il fatto che il moto sia lungo una traiettoria rettilinea.

\subsection{Moto ad accelerazione costante}
Si dice che un punto materiale si muove di moto rettilineo uniformemente accelerato
se vale che $a(t) = a_0$ costante $\forall t > 0$.
Da questa relazione si ricavano la posizione e la velocita':
\begin{subequations}
\begin{align}
&a(t) = a_0 \\
&v(t) = \int_{0}^t a(t) dt = v_0 + a_0t \\
&x(t) = \int_{0}^t v(t) dt = x_0 + v_0t + \frac{a_0}{2}t^2
\end{align}
\end{subequations}

dove $x_0 = x(0)$ e $v_0 = v(0)$.
Dalla seconda possiamo ricavare
\begin{numcases}{v(t) = v_0 + a_0t \implies}
    t = \frac{v(t) - v_0}{a_0} \label{time} \\
    a_0 = \frac{v(t) - v_0}{t} \label{accel}
\end{numcases}
Sostituendo la \ref{time} nell'espressione per $x(t)$ e riordinando otteniamo
\begin{equation}
    v^2(t) = v_0^2 + 2a_0(x-x_0) \label{MUA_senza_t}
\end{equation}
Sostituendo invece la \ref{accel} nell'espressione per $x(t)$ e riordinando otteniamo
\begin{equation}
    x(t) = x_0 + \frac{1}{2}(v(t)+v_0)t \label{MUA_senza_a}
\end{equation}

Se consideriamo il vettore velocita' e il vettore accelerazione, otteniamo che
\[\bvv{a}(t) = \ang{\bvv{a}} = \frac{\Delta \bvv{v}}{\Delta t} = \frac{\bvv{v}(t) - \bvv{v_0}}{\Delta t}\]
da cui segue
\[\bvv{v}(t) = \bvv{v_0} + \ang{\bvv{a}}(t - t_0)\]
dove $\ang{\bvv{a}}$ e' il vettore accelerazione media (che e' sempre uguale all'accelerazione 
istantanea nel caso di moto ad accelerazione costante).
Da questo segue il fatto che il moto sia lungo una traiettoria rettilinea.


\subsection{Moto a caduta libera}
E' un caso particolare di un moto uniformemente accelerato. Consideriamo un sistema ortogonale $XY$ e un corpo
che si trova inizialmente nel punto $(x_0, y_0) = (0, y_0)$ e che si muove verso il basso con una velocita' di modulo iniziale $v_0$. 
I vettori che rappresentano lo stato del corpo saranno quindi:
\begin{subequations}
\begin{align}
    &\bvv{r}(0) = y_0\bh{j}\\
    &\bvv{v}(0) = v_0\bh{j}\\
    &\bvv{a}(0) = -g\bh{j}
\end{align}    
\end{subequations}
dove $g$ e' l'accelerazione di gravita' terrestre.
Notiamo quindi che il moto si svolge unicamente nella direzione dell'asse $Y$.

\subsubsection{Caduta da un'altezza}
Supponiamo che il corpo cada da un'altezza $h$ da fermo (cioe' $v_0 = 0$).
Avremo:
\begin{subequations}
\begin{align}
    &y(t) = h - \frac{1}{2}gt^2 \\
    &v(t) = -gt \label{v_caduta_libera}\\
    &a(t) = -g
\end{align}    
\end{subequations}

Da queste equazioni possiamo ricavare il tempo di caduta e la velocita' di impatto del corpo con il suolo.
Infatti quando il corpo tocca il suolo all'istante $t_f$, avremo che
\begin{alignat}{2}
          y(t_f) &= h - \frac{1}{2}gt_f^2 = 0    \nonumber\\
    \implies t_f &= \sqrt{\frac{2h}{g}}             &\rlap{\text{(Tempo di caduta)}}\\
    \intertext{dunque sostituendo $t_f$ nell'equazione della velocita' \ref{v_caduta_libera}:}
          v(t_f) &= -\sqrt{2gh}                  &\rlap{\text{(Velocita' finale)}}
\end{alignat}    


\subsubsection{Lancio verso l'alto}
Supponiamo ora che il corpo venga lanciato verso l'alto con una velocita' iniziale $v_0 \neq 0$.
Avremo:
\begin{subequations}
\begin{align}
    &y(t) = y_0 + v_0t - \frac{1}{2}gt^2 \label{y_lancio}\\
    &v(t) = v_0 - gt \\
    &a(t) = -g
\end{align}    
\end{subequations}

Possiamo calcolare il punto di altezza massima $y_M$ e il tempo necessario per raggiungerlo $t_M$ da queste equazioni.
Infatti al punto di altezza massima la velocita' sara' nulla, dunque avremo che    
\begin{alignat}
    {2}
          v(t_M) &= v_0 - gt_M = 0       \nonumber \\
    \implies t_M &= \frac{v_0}{g}               \\
    \intertext{dunque sostituendo $t_M$ nell'equazione della posizione \ref{y_lancio}}
    y_M     &= y_0 + v_0t_M - \frac{1}{2}gt_M^2 \nonumber \\
            &= y_0 + \frac{v_0^2}{2g}               
\end{alignat}

\section{Moto in due dimensioni}
Consideriamo ora moti che avvengono in due dimensioni.
Siano $\bvv{r}$, $\bvv{v}$ e $\bvv{a}$ i vettori che descrivono la posizione, la velocita' e l'accelerazione del corpo nel tempo.
Allora il moto e' rettilineo se $\bvv{a} \parallel \bvv{v}$ oppure se $\bvv{a} = \bvv{0}$, altrimenti il moto e' bidimensionale.
Le leggi del moto sono le stesse del caso unidimensionale
\begin{subequations}
\begin{align}
    &\bvv{a}(t) = \bvv{a_0} \\
    &\bvv{v}(t) = \bvv{v_0} + \bvv{a_0}t \\
    &\bvv{r}(t) = \bvv{r_0} + \bvv{v_0}t + \frac{\bvv{a_0}}{2}t^2
\end{align}
\end{subequations}
ma possono essere separate in due equazioni che si riferiscono al moto sui due assi
\begin{subequations}
    \begin{align}
        &\bvv{r}\text{:}
        \begin{cases}{}
            x(t) = x_0 + (v_0\cos{\theta})t + \frac{1}{2}(a_0\cos{\psi})t^2 \\
            y(t) = y_0 + (v_0\sin{\theta})t + \frac{1}{2}(a_0\sin{\psi})t^2
        \end{cases} \\
        &\bvv{v}\text{:}
        \begin{cases}{}
            v_x(t) = v_0\cos{\theta} + (a_0\cos{\psi})t \\
            v_y(t) = v_0\sin{\theta} + (a_0\sin{\psi})t
        \end{cases} \\
        &\bvv{a}\text{:}
        \begin{cases}{}
            a_x(t) = a_0\cos{\psi} \\
            a_y(t) = a_0\sin{\psi}
        \end{cases}
    \end{align}
\end{subequations}
dove $v_0$ e' il modulo del vettore $\bvv{v_0}$, $\theta$ e' l'angolo formato da $\bvv{v_0}$ con l'asse $X$, 
$a_0$ e' il modulo del vettore $\bvv{a_0}$ e $\psi$ e' l'angolo formato da $\bvv{a_0}$ con l'asse $X$.

\subsection{Moto del proiettile}
Consideriamo un caso particolare del moto accelerato bidimensionale in cui $\bvv{a} = -g\bh{j}$.
Se sostituiamo nelle equazioni precedenti otteniamo
\begin{subequations}
    \begin{align}
        &\bvv{r}\text{:}
        \begin{cases}{}
            x(t) = x_0 + (v_0\cos{\theta})t\\
            y(t) = y_0 + (v_0\sin{\theta})t - \frac{1}{2}gt^2
        \end{cases} \label{proj_pos}\\
        &\bvv{v}\text{:}
        \begin{cases}{}
            v_x(t) = v_0\cos{\theta} \\
            v_y(t) = v_0\sin{\theta} - gt
        \end{cases} \\
        &\bvv{a}\text{:}
        \begin{cases}{}
            a_x(t) = 0 \\
            a_y(t) = -g
        \end{cases}
    \end{align}
\end{subequations}
Possiamo notare che, considerando i moti sui due assi separatamente, il moto del punto sull'asse $X$ e' rettilineo uniforme, 
mentre quello sull'asse $Y$ e' uniformemente accelerato.

\subsubsection{Traiettoria del proiettile}
Se ricaviamo $t$ dalla formula di $x(t)$ da \ref{proj_pos} (ottenendo $t = \frac{x-x_0}{v_0\cos{\theta}}$)
e lo sostituiamo nella formula di $y(t)$ otteniamo la traiettoria tracciata dal proiettile, 
cioe' una curva di secondo grado del tipo
\begin{equation}
    y(x) = y_0 + \tan{\theta}(x-x_0) - \frac{g}{2(v\cos{\theta})^2}(x-x_0)^2 \label{proj_traj}
\end{equation}
che rappresenta la traiettoria del proiettile al variare della $x$.

\subsubsection{Punto di altezza massima}
Come nel caso del corpo lanciato verticalmente, il punto di altezza massima viene raggiunto nell'istante
di tempo $t_h$ tale che $v_y(t_h) = 0$.
Sostituendo nelle equazioni otteniamo:   
\begin{alignat}
    {2}
        v_y(t_h) &= v_0\sin{\theta} - gt_h = 0             \nonumber \\       
    \implies t_h &= \frac{v_0\sin{\theta}}{g}                   \\
    \intertext{dunque sostituendo $t_h$ nell'equazione della posizione \ref{proj_pos}}
    y(t_h)   &= y_0 + (v_0\sin{\theta})t_h - \frac{1}{2}gt_h^2  \nonumber \\
             &= y_0 + \frac{(v_0\sin{\theta})^2}{g} - \frac{(v_0\sin{\theta})^2}{2g} \nonumber \\
             &= y_0 + \frac{(v_0\sin{\theta})^2}{2g}            \label{h_max_proj_theta}
\end{alignat}
che e' massimo quando $\theta = \frac{\pi}{2}$, ed e' dunque uguale a
\begin{equation}
    y_M = \frac{v_0^2}{2g} \label{h_map_proj}
\end{equation}

\subsubsection{Gittata}
Per calcolare la gittata del proiettile ci bastera' capire in che punto esso raggiunge 
l'altezza che aveva all'inizio del lancio;
bastera' cioe' trovare $\Delta x = x(t_g)-x_0$, dove $t_g > 0$ e' tale che $y(t_g) = y_0$.
\begin{alignat}
    {2}
             y(t_g) &= y_0 + (v_0\sin{\theta})t_g - \frac{1}{2}gt_g^2 = y_0 \nonumber \\
    \implies t_g &= \frac{2v_0\sin{\theta}}{g}                   \\
    \intertext{dunque sostituendo $t_g$ nell'equazione della posizione \ref{proj_pos}}
    \Delta x &= (v_0\sin{\theta})t_g  \nonumber \\
             &= \frac{v_0^2\sin{2\theta}}{g}            \label{gittata_proj_theta}
\end{alignat}
che e' massimo quando $\theta = \frac{\pi}{4}$ ed e' dunque uguale a
\begin{equation}
    x_{g_M} = \frac{x_0^2}{g}   \label{gittata_proj}
\end{equation}

\subsubsection{Impatto col suolo}
Invece per calcolare la distanza percorsa per impattare il suolo e' sufficiente trovare l'intersezione con l'asse $X$;
cioe' bastera' trovare $\Delta x = x(t_s)-x_0$, dove $t_s > 0$ e' tale che $y(t_s) = 0$.
\begin{alignat}
    {2}
             y(t_s) &= y_0 + (v_0\sin{\theta})t_s - \frac{1}{2}gt_s^2 = 0               \nonumber \\
    \implies t_s &= \frac{1}{g}\left(v_0\sin{\theta} + \sqrt{(v_0\sin{\theta})^2 + 2gy_0}\right)   \\
    \intertext{dunque sostituendo $t_s$ nell'equazione della posizione \ref{proj_pos}}
    \Delta x &= (v_0\sin{\theta})t_s   \label{impatto_proj_theta}
\end{alignat}
\section{Errori di approssimazione}

Abbiamo iniziato ad introdurre il concetto di approssimazione nella \Cref{sez:machine_nums}. Definiamolo ora formalmente.

\begin{definition}
    {Troncamento e arrotondamento}{}
    Sia $x \in \R$, $x \neq 0$, tale che $\omega < \abs{x} < \Omega$. Consideriamo inoltre la rappresentazione normalizzata di $x$: \[
        x = \sgn{x}\beta^p\sum_{i=1}^{\infty} d_i\beta^{-i}.
    \] 
    Allora definiamo \begin{enumerate}[(1)]
        \item il \strong{troncamento} di $x$ alla $t$-esima cifra, indicato con $\trun{x}$, dato da \[
            \trun{x} \deq \sgn{x}\beta^p\sum_{i=1}^t d_i\beta^{-i};
        \]
        \item l'\strong{arrotondamento} di $x$, indicato con $\arr{x}$, dato da \[
            \arr{x} \deq \begin{cases}
                \trun{x}, &\text{se } 0 \leq d_{t+1} < \nicefrac{\beta}{2}\\
                \trun{x} + \beta^{p-t}, &\text{se } d_{t+1} \geq \nicefrac{\beta}{2}.
            \end{cases}
        \] 
    \end{enumerate} 
\end{definition}

\begin{remark}
    Aggiungere $\beta^{p-t}$ ad un numero significa aggiungere $1$ alla sua $t$-esima cifra. Infatti (considerando lo stesso $x$ della definizione) si ha che \[
        \beta^p\sum_{i=1}^\infty d_i\beta^{-i} + \beta^{p-t} = \beta^p\parens*{\sum_{i=1}^\infty d_i\beta^{-i} + \beta^{-t}}.    
    \] 
\end{remark}

\begin{example}
    Consideriamo il numero reale \[
        \pi = 3.141592653\dots = 10^1 \cdot (0.3141592653\dots)_{10}.
    \] Il suo troncamento alla quinta cifra sarà $3.1415$, mentre il suo arrotondamento alla stessa cifrà (siccome la sesta cifra è $9 \geq \nicefrac{10}{2}$) sarà \[
        3.1415 + 10^{1 - 5} = 3.1415 + 0.0001 = 3.1416.
    \]
\end{example}

D'ora in avanti quando parleremo dell'approssimazione di un numero reale $x$ senza voler necessariamente specificare il metodo (troncamento o arrotondamento) usato scriveremo $\fl{x}$ oppure $\tilde{x}$.  

Approssimando numeri reali a numeri di macchina si genera un errore, detto \strong{errore di rappresentazione}.

\begin{definition}
    {Errore di rappresentazione}{}
    Sia $x \in \R$, $x \neq 0$ tale che $\omega \leq \abs{x} \leq \Omega$. Chiameremo \strong{errore assoluto} la quantità \[
        \eta_x \deq \tilde{x} - x,
    \] mentre chiameremo \strong{errore relativo} o \strong{errore di rappresentazione} la quantità \[
        \eps_x \deq \frac{\tilde{x} - x}{x}.
    \]
\end{definition}

La qualità più importante dell'errore relativo è il fatto che è sempre \strong{superiormente limitato}, come dimostreremo attraverso un lemma ed un teorema.

\begin{lemma}
    {Limite superiore all'errore assoluto}{lim_sup_abserr} 
    Sia $x \in \R$, $x \neq 0$, $\omega \leq x \leq \Omega$. Sia inoltre $x = \sgn{x}\beta^p\sum_{i=1}^\infty d_i\beta^{-i}$ la rappresentazione in base di $x$. Si ha che 
    \begin{enumerate}[(i)]
        \item $\abs[\big]{\trun{x} - x} < \beta^{p-t}$
        \item $\abs[\big]{\arr{x} - x} \leq \frac12\beta^{p-t}$.  
    \end{enumerate}
\end{lemma}
\begin{proof}
    Se fosse $x = \Omega$ oppure $x = \omega$  allora $\trun{x} = \arr{x} = x$, da cui l'errore assoluto $\tilde x - x$ è nullo.  

    Supponiamo quindi $\omega < x < \Omega$ (il caso $-\Omega < x < -\omega$ è speculare). Siccome valgono le ipotesi del \Cref{th:rapp_base} possiamo scrivere \[
        x = \beta^p\sum_{i=1}^\infty d_i\beta^{-i}.
    \] Esisteranno dunque due numeri di macchina $a, b \in \MSet$ tali che $a \leq x < b$ e $b$ è il numero di macchina successivo rispetto ad $a$, ovvero $a$ è il numero di macchina appena più piccolo di $x$ e $b$ è quello appena più grande.

    Allora dovrà essere necessariamente \[
        a = \beta^p\sum_{i=1}^t d_i\beta^{-i} = \trun{x},
    \] mentre $b$ dovrà essere \[
        b = \beta^p\parens*{\sum_{i=1}^t d_i\beta^{-i} + \beta^{-t}} = a + \beta^{p-t},
    \] da cui segue anche che $b - a = \beta^{p-t}$. 

    Si ha quindi \[
        \abs[\big]{\trun{x} - x} 
            = \abs*{a - x}
            < \abs*{a - b} 
            = \beta^{p-t},
    \] dove il minore centrale deriva dal fatto che $x < b$.
    
    Per quanto riguarda l'arrotondamento, osserviamo invece che \begin{itemize}
        \item se $d_{t+1} < \nicefrac{\beta}{2}$ (ovvero se $x < \nicefrac12(a + b)$) allora $\arr{x} = \trun{x} = a$;
        \item altrimenti $\arr{x} = \trun{x} + \beta^{p-t} = a + \beta^{p-t} = b$.
    \end{itemize}

    Nel primo caso si ha \[
        \abs[\big]{\arr{x} - x} 
            =    \abs[\big]{x - a}
            \leq \abs*{\frac{a+b}{2} - a} 
            =    \frac12\abs*{b-a} 
            =    \frac12\beta^{p-t},
    \] mentre nel secondo si ha 
    \[
        \abs[\big]{\arr{x} - x} 
            =    \abs[\big]{b - x}
            \leq \abs*{b - \frac{a+b}{2}} 
            =    \frac12\abs*{b - a} 
            =    \frac12\beta^{p-t},
    \] da cui la tesi.
\end{proof} 

\begin{remark}
    \label{rem:abs_err_arr} Si ha l'uguaglianza $\abs[\big]{\arr x - x} = \frac12\beta^{p-t}$ se e solo se $x = \frac12(a + b)$, ovvero se e solo se $x$ è esattamente il punto medio dell'intervallo $\interval[{a, b}]$.
\end{remark}

\begin{theorem}
    {Esistenza della precisione di macchina}{machine_prec}
    Esiste un numero $u \in \R$, detto \strong{precisione di macchina}, tale che per ogni $x \in \R$, $x \neq 0$, $\omega \leq \abs*{x}\leq \Omega$ vale che \[
        \abs*{\eps_x} \leq u.
    \] In particolare \begin{enumerate}[(i)]
        \item se $\fl{x} = \trun{x}$ si ha $u = \beta^{1-t}$;
        \item se $\fl{x} = \arr{x}$ si ha $u = \frac12\beta^{1-t}$. 
    \end{enumerate}
\end{theorem}
\begin{proof}
    Osserviamo che se $x = \pm\beta^p \sum_{i=1}^\infty d_i\beta^{-i}$ allora varrà sicuramente che \[
        \abs*{x} \geq \beta^p (0.10\dots)_\beta = \beta^p \cdot (1 \beta^{-1}) = \beta^{p-1}.
    \] Infatti scegliendo il numero di macchina con lo stesso esponente di $x$ ma la minima mantissa possibile otterremo necessariamente un numero più piccolo (o uguale) del valore assoluto di quello di partenza.

    Consideriamo allora i due casi.
    \begin{description}
        \item[Troncamento] Per il \Cref{lem:lim_sup_abserr} sappiamo che $\abs[\big]{\trun{x} - x} < \beta^{p-t}$. Si ha quindi che \[
                \abs*{\eps_x} 
            = \frac{\abs[\big]{\trun{x} - x}}{\abs x} 
            < \frac{\beta^{p-t}}{\beta^{p-1}}
            = \beta^{1-p}.
        \] 
        \item[Arrotondamento] Per il \Cref{lem:lim_sup_abserr} sappiamo che $\abs[\big]{\arr{x} - x} \leq \frac12\beta^{p-t}$. Si ha quindi che \[
                 \abs*{\eps_x} 
            =    \frac{\abs[\big]{\arr{x} - x}}{\abs x} 
            \leq \frac{\nicefrac12\beta^{p-t}}{\beta^{p-1}}
            =    \frac12\beta^{1-p}. \qedhere
        \]
    \end{description}
\end{proof}

Osserviamo che il \Cref{th:machine_prec} ci dice che $u$ è un limite superiore \emph{debole} (nel senso che $\abs*{\eps_x} \leq u$ e non $\abs*{\eps_x} < u$), tuttavia almeno nel caso del troncamento si vede dalla dimostrazione che vale anche la disuguaglianza stretta. 

In realtà la disuguaglianza stretta vale anche nel caso dell'arrotondamento: la dimostrazione completa è lasciata all'appendice (in particolare nella \Cref{sez:machine_prec_>}).
 
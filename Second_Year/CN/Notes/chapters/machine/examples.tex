\section{Esempi di condizionamento e stabilità}

In questa sezione faremo alcuni esempi dello studio del condizionamento di problemi e della stabilità di algoritmi di macchina.

\begin{example}
    Consideriamo la funzione $f(x) = \frac{x-1}{x}$ e studiamone il condizionamento per $x \neq 0$.

    Osserviamo che la funzione è derivabile (almeno) due volte nell'intervallo $\interval({0, +\infty})$, dunque possiamo calcolare il coefficiente di amplificazione di un dato $x \in \interval({0, +\infty})$. \[
        c_x = \frac{x}{f(x)}f'(x) = \frac{1}{x-1}.
    \]  

    Il problema è ben condizionato se e solo se $\abs*{c_x}$ è limitato superiormente per ogni $x \in \interval*({0, +\infty})$: tuttavia possiamo osservare che \[
        \lim_{x \to 1} \abs*{\frac1{x-1}} = +\infty,
    \] dunque il problema è mal condizionato in un intorno di $1$.
\end{example}

\begin{example}
    Consideriamo l'algoritmo $g_1(x) = (x \ominus 1) \oslash x$ e studiamone la stabilità. Il grafo di $g_1$ è
    \begin{center}
        \tikzsetnextfilename{algo_err_graph_es1}
        \begin{tikzpicture}
            [
                scale=.5,auto=left,
                important/.style={ellipse, draw=red, inner sep=2pt, minimum size=12pt}
                % arrow/.style={-{Stealth[]}}
            ]
            \node[important] 
                (n1) at (1,0)  {$x$};
            \node[important] 
                (n2) at (1,2)  {$x$};
            \node[important, label={$\mu_1$}] 
                (n3) at (4,1)  {$x - 1$};
            \node[important] 
                (n4) at (4,5)  {$x$};
            \node[important, label={$\mu_2$}] 
                (n5) at (9,3)  {$\frac{x-1}{x}$};
        
            \foreach \from/\to/\coeff in {n1/n3/{},n3/n5/{$1$}}
                \draw [->] (\from) to node[below, sloped] {\coeff} (\to);
            
            \foreach \from/\to/\coeff in {n2/n3/{},n4/n5/{$-1$}}
                \draw [->] (\from) to node[above, sloped] {\coeff} (\to);
        \end{tikzpicture}
    \end{center}

    L'errore algoritmico totale è quindi dato da \[
        \epsalg = \mu_2 + 1\cdot(\mu_1 + 0 + 0) + 0 = \mu_2 + \mu_1.
    \] Dimostriamo ora formalmente che l'algoritmo è stabile, ovvero che $\abs*{\epsalg}$ è sempre limitato superiormente.

    In effetti si ha
    \begin{align*}
        \abs*{\epsalg} = \abs*{\mu_1 + \mu_2} \stackrel{(1)}{\leq} \abs*{\mu_1} + \abs*{\mu_2} \stackrel{(2)}{<} 2u
    \end{align*} dove la disuguaglianza (1) è la disuguaglianza triangolare dei moduli, mentre la (2) viene dal fatto che gli errori locali sono sempre minori della precisione di macchina $u$.
\end{example}

\subsection{Problema del calcolo della somma}

Studiamo la funzione $f : \R^n_{> 0} \to \R$ data da \[
    f(x_1, \dots, x_n) \deq \sum_{i=1}^n x_i.  
\] Osserviamo che la funzione ha come dominio $\R^n_{>0}$, dunque tutti gli $x_i$ devono essere strettamente positivi. Inoltre per semplicità scriveremo $\vec x = \parens*{x_1, \dots, x_n}$.

\newthought{Condizionamento} Studiamo inizialmente il condizionamento del problema studiando l'errore inerente.

\begin{align*}
    \epsin &\doteq \sum_{i=1} c_{x_i}\eps_{x_i} \\
    &= \sum_{i=1}^n \frac{x_i}{f(\vec{x})} \pdv{f}{x_i} \cdot \eps_{x_i}\\
    &= \sum_{i=1}^n \frac{x_i}{f(\vec{x})} \cdot 1 \cdot \eps_{x_i}\\
    &= \sum_{i=1}^n \frac{x_i}{f(\vec x)} \eps_{x_i}.
\end{align*}

Per mostrare che il problema è ben condizionato dimostriamo che il valore assoluto dell'errore inerente è limitato superiormente.
In effetti si ha
\begin{align*}
    \abs*{\epsin} &\doteq \abs*{\sum_{i=1}^n \frac{x_i}{f(\vec x)}\eps_{x_i}}\\[1.2em]
    &\stackrel{(1)}{\leq} \sum_{i=1}^n \frac{\abs{x_i}}{\abs*{f(\vec x)}}\abs*{\eps_{x_i}}\\
    &\stackrel{(2)}{<} \frac{1}{\abs*{f(\vec x)}} \sum_{i=1}^n \abs{x_i}\cdot u\\
    &= \frac{u}{\abs*{f(\vec x)}} \sum_{i=1}^n \abs*{x_i} \\
    &\stackrel{(3)}{=} u,
\end{align*} dove (1) è la disuguaglianza triangolare, (2) deriva dal fatto che $\abs*{\eps_{x_i}} < u$ per ogni $x_i$ e (3) segue dall'ipotesi che gli $x_i$ siano strettamente positivi: in questo caso infatti il modulo di $f(\vec x) = x_1 + \dots + x_n$ è uguale alla somma dei moduli degli $x_i$.

Si ha quindi che il problema è ben condizionato.

\newthought{Stabilità} Consideriamo l'algoritmo che consiste nell'effettuare le somma da sinistra a destra, ovvero \[
    g_1(\vec x) = ((((x_1 + x_2) + x_3) + \dots + x_{n-1}) + x_n).
\]

Nel caso (ad esempio) di $n = 4$ l'algoritmo può essere rappresentato col seguente grafo:    
    \begin{center}
        \tikzsetnextfilename{algo_err_graph_es2}
        \begin{tikzpicture}
            [
                scale=.5,auto=left,
                important/.style={ellipse, draw=red, inner sep=2pt, minimum size=12pt}
                % arrow/.style={-{Stealth[]}}
            ]
            \node[important] 
                (n1) at (1,0)  {$x_1$};
            \node[important] 
                (n2) at (1,2)  {$x_2$};
            \node[important, label={$\mu_1$}] 
                (n3) at (4,1)  {$s_2$};
            \node[important] 
                (n4) at (4,4)  {$x_3$};
            \node[important, label={$\mu_2$}] 
                (n5) at (8,2.5)  {$s_3$};
            \node[important] 
                (n6) at (8,6.5)  {$x_4$};
            \node[important, label={$\mu_2$}] 
                (n7) at (12,4.5)  {$s_4$};
        
            \foreach \from/\to/\coeff in {n1/n3/{},n3/n5/{$\frac{s_2}{s_2+x_3}$}, n5/n7/{$\frac{s_3}{s_3+x_4}$}}
                \draw [->] (\from) to node[below, sloped] {\coeff} (\to);
            
            \foreach \from/\to/\coeff in {n2/n3/{},n4/n5/{}, n6/n7/{}}
                \draw [->] (\from) to node[above, sloped] {\coeff} (\to);
        \end{tikzpicture}
    \end{center}
dove $s_i = s_{i-1} + x_i = x_1 + \dots + x_i$. 

Calcoliamo ad esempio l'errore accumulatosi sul nodo $s_3$: si ha \[
    \eps_3 = \mu_3 + \frac{s_2}{s_2 + x_3}\underbrace{(\mu_2 + 0 + 0)}_{\eps_2}.
\] Stessa cosa sul nodo $s_4$: infatti \[
    \eps_4 = \mu_4 + \frac{s_3}{s_3 + x_4}\eps_3.
\] Più in generale si avrà quindi \[
    \eps_k = \mu_k + \frac{s_{k-1}}{s_{k-1}+ x_k}\eps_{k-1}.
\] Osserviamo che per definizione di $s_i$ possiamo riscrivere $\frac{s_{k-1}}{s_{k-1} + x_k}$ come $\frac{s_{k-1}}{s_k}$ e questa frazione è sempre minore di $1$: in questo modo riusciamo a limitare superiormente il modulo dell'errore algoritmo. Infatti \begin{align*}
    \abs*{\epsalg} = \abs*{\eps_n} &= \abs*{\mu_n + \frac{s_{n-1}}{s_n}\eps_{n-1}}\\
    &\leq \abs*{\mu_n} + \abs*{\frac{s_{n-1}}{s_n}}\abs*{\eps_{n-1}}\\
    &< u + 1 \cdot \abs*{\eps_{n-1}} \\
    &< u + u + \abs*{\eps_{n-2}} \\
    &< \dots \\
    &< (n-1)u. 
\end{align*}  L'algoritmo è dunque stabile, ma solo se $n$ non è troppo grande.
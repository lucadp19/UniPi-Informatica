\section{"$9$-periodico"}

\begin{proposition}
    [$9$-periodico]
    \label{prop:9_periodico}
    In base $10$ i numeri $0.\bar{9}$ e $1$ sono uguali. 
\end{proposition}
Forniamo due diverse dimostrazioni di questa proposizione.
\begin{proof}
    [Prima dimostrazione]
    Dalle formule per trasformare i numeri periodici in frazioni sappiamo che $0.\bar{9} = \nicefrac{9}{9} = 1$. 
\end{proof}
\begin{proof}
    [Seconda dimostrazione]
    Espandendo la definizione di numero periodico otteniamo che \[
        0.\bar{9} = 0.999\dots = 9\cdot10\inv + 9\cdot10^{-2} + 9\cdot 10^{-3} + \dots = \sum_{i=1}^\infty 9\cdot10^{-i}.
    \] Sfruttando la formula della serie geometrica si ottiene che \begin{align*}
        \sum_{i=1}^\infty 9\cdot10^{-i} &= 9 \cdot \sum_{i=1}^\infty \parens*{\frac{1}{10}}^i\\
        &= 9 \cdot \parens*{\parens*{\sum_{i=0}^\infty \parens*{\frac{1}{10}}^i} - \parens*{\frac{1}{10}}^0}\\
        &= 9 \cdot \parens*{\frac{1}{1-\nicefrac{1}{10}} - 1}\\
        &= 9 \cdot \parens*{\frac{10}{9} - 1}\\
        &= 9 \cdot \frac{1}{9}\\
        &= 1. \qedhere
    \end{align*}
\end{proof}

La proposizione vale in generale in una base $\beta$ qualsiasi ($\beta \geq 2$).
\begin{proposition}
    In base $\beta$ ($\beta \geq 2$) vale che $0.\overline{(\beta-1)} = 1$.  
\end{proposition}
\begin{proof}
    La dimostrazione è uguale alla seconda dimostrazione della \autoref{prop:9_periodico}.
\end{proof}

\section{La precisione di macchina è un limite superiore stretto}
\label{sez:machine_prec_>}

In questa sezione mostriamo che la precisione di macchina $\mathbold{u}$ è sempre strettamente maggiore all'errore relativo $\eps_x$ per ogni $x \in \R$ tale che $x \neq 0$ e $\omega \leq \abs x \leq \Omega$, anche usando l'arrotondamento come metodo di approssimazione.

Dalla dimostrazione del \autoref{th:machine_prec} si ha che \[
    \abs*{\eps_x} = \frac12\beta^{1-p} = \mathbold{u}
\] se e solo se la disuguaglianza \[
    \frac{\abs[\big]{\arr{x} - x}}{\abs x} 
    \leq \frac{\nicefrac12\beta^{p-t}}{\beta^{p-1}}
\] è in realtà un'uguaglianza, ovvero se e solo se valgono le seguenti due condizioni:
\begin{itemize}
    \item $\abs[\big]{\arr{x} - x} = \frac12\beta^{p-t}$
    \item $\abs x = \beta^{p-1}$.
\end{itemize}
Tuttavia, come abbiamo notato nella \autoref{rem:abs_err_arr}, la prima condizione vale se e solo se \[
    \abs x = \frac12(a+b) = \frac12(a + a + \beta^{p-t}) = a + \frac12\beta^{p-t}.
\] È quindi necessario che \begin{align*}
    &\beta^{p-1} 
        \begin{aligned}[t] 
            &= a + \frac12\beta^{p-t} \\
            &= \beta^p \parens*{\sum_{i=1}^t d_i\beta^{-i}} + \frac12\beta^{p-t} \\
            &= \beta^p \parens*{\sum_{i=1}^t d_i\beta^{-i} + \frac12\beta^{-t}}
        \end{aligned} \\
    \iff &\beta^{-1} = \sum_{i=1}^t d_i\beta^{-i} + \frac12\beta^{-t}.
\end{align*} Tuttavia ciò è assurdo: infatti il termine al primo membro ha una singola cifra decimale diversa da zero, ovvero quella di $\beta^{-1}$, mentre il termine del secondo membro ha anche delle cifre non nulle nelle posizioni $\beta^{-t}$ e/o $\beta^{-t-1}$.  

Segue quindi che $\abs{\eps_x} < u$ in qualsiasi caso.

\section{Dimostrazione dell'equivalenza topologica tra norme}
\label{sec:eq_topo_norm}

In questa sezione dimostreremo il \autoref{th:eq_topo}, che per comodità enunciamo nuovamente.

\eqTopo*

Per dimostrare il Teorema sfruttiamo il seguente lemma.
\begin{lemma}
    Sia $\norm$ una norma su $\F^n$. Allora la funzione \[
        \vec x \mapsto \norm{\vec x}
    \] è una funzione continua.
\end{lemma}
\begin{proof}
    Sia $\vec x_0 \in \F^n$ un punto qualsiasi e dimostriamo che la norma è continua in $\vec x_0$.

    Sia quindi $\eps > 0$ qualsiasi: vogliamo determinare un $\delta > 0$ tale che per ogni $\vec x \in \F^n$ con $\norm{\vec x - \vec x_0} < \delta$ si ha $\abs[\big]{\norm{\vec x} - \norm{\vec x_0}} < \eps$.

    Osserviamo che \[
        \norm{\vec x} = \norm{\vec x - \vec x_0 + \vec x_0} \leq \norm{\vec x_0} + \norm{\vec x - \vec x_0},
    \] da cui $\norm{\vec x} - \norm{\vec x_0} \leq \norm{\vec x - \vec x_0}$. Scambiando i ruoli di $\vec x$ e $\vec x_0$ otteniamo che $\norm{\vec x_0} - \norm{\vec x} \leq \norm{\vec x_0 - \vec x} = \norm{\vec x - \vec x_0}$, quindi combinando le due disuguaglianze si ha \[
        \abs[\Big]{\norm{\vec x} - \norm{\vec x_0}} \leq \norm{\vec x - \vec x_0}.
    \]
    
    Poniamo dunque $\delta = \eps$. Allora \[
        \abs[\Big]{\norm{\vec x} - \norm{\vec x_0}} \leq \norm{\vec x - \vec x_0} < \delta = \eps,
    \] cioè la norma è continua in $\vec x_0$, da cui segue (per generalità di $\vec x_0$) che la norma è una funzione continua.
\end{proof}

Possiamo quindi dimostrare il \autoref{th:eq_topo}.
% \begin{proof}
    % Consideriamo la funzione $h : \F^n \setminus \set{\vec 0} \to \R$ data da \[
        % h(\vec x) = \frac{f(\vec x)}{g(\vec x)}.
    % \]
% \end{proof}
\section{Fattorizzazione LU}

Abbiamo già iniziato a parlare di sistemi lineari e della loro risoluzione precedentemente. Dopo averne studiato il condizionamento studiamo degli algoritmi per trovare efficientemente le soluzioni di sistemi della forma $A\vec x = \vec b$, dove $A$ è una matrice, $\vec b$ è il vettore dei termini noti e $\vec x$ è il vettore delle incognite.

\subsection{Risoluzione di un sistema triangolare}

Se la matrice $A$ è triangolare superiore è particolarmente semplice risolvere il sistema, in quanto possiamo usare la tecnica di \strong{sostituzione all'indietro}.

\[
    \begin{pmatrix}
        a_{11}  &\dots  &a_{1n} \\
                &\ddots &\vdots \\
                &       &a_{nn}
    \end{pmatrix} \begin{pmatrix}
        x_1 \\ \vdots \\ x_n
    \end{pmatrix} = \begin{pmatrix}
        b_1 \\ \vdots \\ b_n
    \end{pmatrix}
\]

Infatti l'ultima riga ci fornisce l'equazione $a_{nn}x_n = b_n$, da cui ricaviamo $x_n = \dfrac{b_n}{a_{nn}}$ (a patto che $a_{nn}$ sia diverso da $0$); usando questo risultato possiamo risolvere l'equazione della penultima riga, cioè \[
    a_{n-1, n-1}x_{n-1}+a_{n-1,n}x_n = b_{n-1},
\] che diventa \[
    x_{n-1} = \frac{b_{n-1} - a_{n-1,n}x_n}{a_{n-1,n-1}}.
\]

In generale, tramite questo metodo, la formula per calcolare $x_k$ è \[
    x_k = \frac{\displaystyle b_k - \sum_{j=k+1}^n a_{kj}x_j}{a_{kk}}.
\]

Un programma MatLab che esprime questo algoritmo è il seguente:
\begin{minted}{MatLab}
    % given an upper triangular matrix A of order n
    % and a n-dimensional vector b
    % it calculates the solution of the system |$A\vec x = \vec b$|
    function x = upp_triang_solve(A, b)
        x(n) = b(n) / A(n, n)
        for k = n-1 : -1 : 1
            s = 0
            for j = k+1 : n
                s = s + A(k, j)*x(j)
            end
            x(k) = (b(k) - s) / A(k, k)
        end
    end
\end{minted}

Una veloce analisi del costo computazionale dell'algoritmo ci dice che esso ha costo, al caso pessimo, di $\OO(n^2)$: infatti il ciclo interno ha costo $\OO(n - k)$, e viene ripetuto per $k = 1, \dots, n-1$, da cui viene il costo \[
    \sum_{k=1}^{n-1} \OO(n-k) = \OO(n^2).
\] Questo costo non è migliorabile: infatti la matrice ha $\OO(n^2)$ elementi, da cui non è possibile diminuire il numero di operazioni fatte.

Se la matrice è triangolare inferiore possiamo usare la stessa tecnica ma partendo dalla prima riga, tramite la \strong{sostituzione all'avanti}.

\subsection{Mosse di Gauss e fattorizzazione LU}

Se la matrice $A$ non è triangolare il metodo più usato per risolvere il sistema $A\vec x = \vec b$ è l'\strong{algoritmo di Gauss}, che sfrutta le cosiddette mosse di Gauss.

\begin{definition}
    {Mosse di Gauss}{mosse_gauss}
    Sia $A \in \Mat{\F, n, n}$ una matrice. Una \strong{mossa di Gauss per riga} è un'operazione sulle righe della matrice con una delle seguenti forme:
    \begin{enumerate}[(1)]
        \item scambiare la riga $i$ con la riga $j$;
        \item moltiplicare la riga $i$ per un coefficiente $\alpha \in \F$, $\alpha \neq 0$;
        \item sottrarre alla riga $i$ la riga $j$ moltiplicata per un coefficiente $\alpha$.   
    \end{enumerate} 
\end{definition}

La strategia è quindi quella di applicare ripetutamente le mosse di Gauss (secondo il cosiddetto \strong{algoritmo di Gauss}) per ridurre la matrice $A$ ad una matrice $U$ triangolare superiore: \[
    [A | \vec b] = [A^{(0)} | \vec b^{(0)}] \leadsto [A^{(1)} | \vec b^{(1)}] \leadsto \dots \leadsto [A^{(n-1)} | \vec b^{(n-1)}] = [U | \vec y].
\]

Mostreremo nella prossima sezione che in alcuni casi particolari possiamo descrivere una singola mossa di Gauss come un prodotto tra matrici \[
    A^{(k+1)} = E^{(k+1)}A^{(k)}
\] dove $E^{(k+1)}$ sarà una matrice invertibile e triangolare inferiore.

In tal caso segue quindi che \[
    U = A^{(n-1)} = E^{(n-1)}A^{(n-2)} = E^{(n-1)}E^{(n-2)}A^{(n-3)} = \dots = E^{(n-1)}\cdots E^{(1)}A.
\] Siccome abbiamo detto che tutte le matrici $E^{(i)}$ sono invertibili anche il loro prodotto lo sarà: chiamiamo dunque \[
    L\inv \deq E^{(n-1)}\cdots E^{(1)},
\] da cui segue che $U = L\inv A$, ovvero $A = LU$.

Il sistema $A\vec x = \vec b$ diventa quindi $LU\vec x = \vec b$, che può essere risolto come \[
    \left\{ 
        \;
        \begin{aligned}
            &L\vec y = \vec b \\
            &U\vec x = \vec y
        \end{aligned}
    \right.
\] 
Sappiamo che $U$ è triangolare superiore, dunque possiamo risolvere il sistema $U\vec x = \vec y$ usando i metodi studiati precedentemente.
Allo stesso modo, dato che $L$ è data dal prodotto di matrici triangolari inferiori anch'essa è triangolare inferiore, dunque possiamo risolvere anche il sistema $L\vec y = \vec b$ efficientemente, trovando quindi una soluzione al sistema originale.
% Tuttavia per ricavare $\vec y$ dobbiamo risolvere il primo sistema: mostrando che anche $L$ è triangolare riusciremmo a trovare molto velocemente una soluzione alla prima equazione, e quindi all'intero sistema.

Siamo quindi interessati a studiare in che modo e in quali casi si può scrivere $A$ come prodotto di due matrici, di cui una è triangolare inferiore ($L$, da \emph{lower triangular}) e una è triangolare superiore ($U$, da \emph{upper triangular}). Diamo dunque la seguente definizione.

\begin{definition}
    {Fattorizzazione LU}{LU-fact}
    Sia $A \in \Mat{\F, n, n}$ una matrice. Si dice che $A$ è \strong{fattorizzabile LU} se esistono due matrici $L, U \in \Mat{\F, n, n}$ tali che \begin{enumerate}[(1)]
        \item $L$ è triangolare inferiore ed ha tutti $1$ sulla diagonale principale;
        \item $U$ è triangolare superiore;
        \item vale che \[
            A = LU.
        \]
    \end{enumerate} 
\end{definition}

% Il nostro scopo sarà quindi quello di trovare condizioni per cui una matrice $A$ può essere fattorizzata in modo LU, fattorizzarla e risolvere il sistema separando la parte triangolare superiore e quella triangolare inferiore.

Il seguente Teorema ci dà delle condizioni sufficienti sulla fattorizzabilità.

\begin{theorem}
    {Esistenza e unicità della fattorizzazione LU}{exist-unique-LU-fact}
    Sia $A \in \Mat{\F, n, n}$. Se le sottomatrici principali di testa di $A$ di ordine $k$, con $k = 1, \dots, n-1$, sono invertibili, allora $A$ è fattorizzabile LU e tale fattorizzazione è unica.  
\end{theorem}
\begin{proof}
    Dimostriamo la tesi per induzione su $n$.
    \newthought{Caso base} Supponiamo $n = 1$. Allora la matrice $A = \begin{bmatrix}
        a_{11}
    \end{bmatrix}$ non ha sottomatrici, dunque la condizione è vacuamente verificata. Inoltre \begin{itemize}
        \item una possibile fattorizzazione di $A$ è data da \[
            L = \begin{bmatrix} 1 \end{bmatrix}, \qquad
            U = \begin{bmatrix} a_{11} \end{bmatrix}
        \]
        \item tale fattorizzazione è unica: infatti $L$ deve essere la matrice con un singolo $1$ poiché sulla diagonale non può avere valori diversi, da cui segue che $U = A$.
    \end{itemize}
    La tesi è quindi verificata nel caso base.

    \newthought{Passo induttivo} Supponiamo che la tesi sia vera per matrici di ordine $m$, per ogni $m < n$, e mostriamola per $n$.

    Supponiamo quindi che $A$ abbia sottomatrici di testa non singolari. Per mostrare che $A$ sia fattorizzabile LU dobbiamo trovare due matrici $L, U$ che rispettano le condizioni sulla fattorizzazione. Scrivendole a blocchi, si ha \[
        % A = \left( \begin{array}{@{}c|c@{}}
        %     A_{n-1} & \vec z \\\hline
        %     \vec x\trans & a_{nn}
        % \end{array}\right)
        % A = \begin{array}{@{}c@{}c@{}c|c@{}}
        %     \hphantom{a_{ii}} &             &\hphantom{a_{ii}}   & \\
        %     \hphantom{a_{ii}} &A_{n-1}      &\hphantom{a_{ii}}   &\vec z \\
        %     \hphantom{a_{ii}} &             &\hphantom{a_{ii}}   & \\ 
        %         \hline \rule{0pt}{1.1\normalbaselineskip}
        %     \hphantom{a_{ii}} &\vec x\trans &\hphantom{a_{ii}}   &a_{nn}
        % \end{array}
        A = 
        % \begin{pNiceArray}{ccc|c}
        %     \Block{3-3}<\Large>{A_{n-1}} &             &   & \\
        %      &      &   &\vec z \\
        %      &             &   & \\ 
        %         \hline \rule{0pt}{1.1\normalbaselineskip}
        %      &\vec x\trans &   &a_{nn}
        % \end{pNiceArray}
        \left( \begin{array}{@{}ccc|c}
            \; &             &\;   & \\
            \; &A_{n-1}      &\;   &\vec z \\
            \; &             &\;   & \\ 
                \hline \rule{0pt}{1.1\normalbaselineskip}
            \; &\vec x\trans &\;   &a_{nn}\;
        \end{array}\right)
        =
        \left( \begin{array}{@{}ccc|c}
            \; &             &\;   & \\
            \; &\widehat{L}  &\;   &\vec 0 \\
            \; &             &\;   & \\ 
                \hline \rule{0pt}{1.1\normalbaselineskip}
            \; &\vec w\trans &\;   &1\;
        \end{array}\right)
        \left( \begin{array}{@{}ccc|c}
            \; &             &\;   & \\
            \; &\widehat{U}  &\;   &\vec y \\
            \; &             &\;   & \\ 
                \hline \rule{0pt}{1.1\normalbaselineskip}
            \; &\vec 0\trans &\;   &\beta\;
        \end{array}\right)
    \] dove $A_{n-1}, \vec z, \vec x\trans, a_{nn}$ sono noti, mentre $\widehat{L}, \widehat{U}, \vec w\trans, \vec y$ e $\beta$ sono incognite.  

    Svolgendo il prodotto a blocchi devono quindi valere le seguenti condizioni: \[
        \left\{
            \!\!\!\!\!\!\!
            \begin{aligned}
                &&A_{n-1} &= \widehat{L}\widehat{U} + \vec 0\vec 0\trans = \widehat{L}\widehat{U} \\
                &&\vec z &= \widehat{L}\vec y + \beta\vec 0 = \widehat{L}\vec y\\
                &&\vec x\trans &= \vec w\trans \widehat{U} + 1\cdot \vec 0\trans = \vec w\trans \widehat{U}\\
                &&a_{nn} &= \vec w\trans \vec y + \beta.
            \end{aligned}
        \right.
    \]

    \begin{enumerate}[(1)]
        \item Osserviamo che la prima equazione corrisponde alla fattorizzazione LU di $A_{n-1}$. 
        Siccome $A_{n-1}$ è la sottomatrice di testa di $A$ di ordine $n-1$ le sue sottomatrici di testa sono uguali alle sottomatrici di testa di $A$, dunque sono invertibili per ipotesi.   

        Per l'ipotesi induttiva segue quindi che la fattorizzazione LU di $A_{n-1}$ esiste ed è unica, dunque $\widehat{L}, \widehat{U}$ esistono e sono uniche.
        \item Siccome la matrice $\widehat{L}$ è triangolare inferiore segue che il suo determinante è il prodotto degli elementi della diagonale, ovvero $\det \widehat{L} = 1$. Segue quindi che il sistema $\widehat{L}\vec y = \vec z$ ammette una e una sola soluzione, dunque $\vec y$ è univocamente determinato.  
        \item Applicando l'operatore di trasposizione ad entrambi i membri dell'equazione si ottiene il sistema equivalente \[
            \widehat{U}\trans \vec w = \vec x.
        \] La matrice $\widehat{U}\trans$ è invertibile: infatti $A_{n-1} = \widehat{L}\widehat{U}$ e $\widehat{L}$ invertibile implica che $\widehat{U} = \widehat{L}\inv A_{n-1}$. Per il Teorema di Binet segue quindi che \[
            \det \widehat{U} = \det \widehat{L}\inv \cdot \det A_{n-1}.
        \] $\widehat{L}\inv$ è certamente invertibile, mentre $A_{n-1}$ è invertibile in quanto è la sottomatrice di testa di $A$ di ordine $n-1$, e per ipotesi tutte le sottomatrici di testa di $A$ sono invertibili.

        Segue quindi che $\widehat{U}$ è invertibile, da cui anche $\widehat{U}\trans$ è invertibile, dunque il vettore $\vec w$ esiste ed è unico.
        \item L'ultima equazione è equivalente a \[
            \beta = a_{nn} - \vec w\trans\vec y.
        \] Abbiamo dimostrato che $\vec w$ e $\vec y$ esistono e sono unici, dunque anche $\beta$ esiste ed è univocamente determinato.
    \end{enumerate}

    Segue quindi che la matrice $A$ è fattorizzabile LU, da cui la tesi.
\end{proof}
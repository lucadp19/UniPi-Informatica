\section{"$9$-periodico"}

\begin{proposition}
    {$9$-periodico}
    {9_periodico}
    In base $10$ i numeri $0.\bar{9}$ e $1$ sono uguali. 
\end{proposition}
Forniamo due diverse dimostrazioni di questa proposizione.
\begin{proof}
    [Prima dimostrazione]
    Dalle formule per trasformare i numeri periodici in frazioni sappiamo che $0.\bar{9} = \nicefrac{9}{9} = 1$. 
\end{proof}
\begin{proof}
    [Seconda dimostrazione]
    Espandendo la definizione di numero periodico otteniamo che \[
        0.\bar{9} = 0.999\dots = 9\cdot10\inv + 9\cdot10^{-2} + 9\cdot 10^{-3} + \dots = \sum_{i=1}^\infty 9\cdot10^{-i}.
    \] Sfruttando la formula della serie geometrica si ottiene che \begin{align*}
        \sum_{i=1}^\infty 9\cdot10^{-i} &= 9 \cdot \sum_{i=1}^\infty \parens*{\frac{1}{10}}^i\\
        &= 9 \cdot \parens*{\parens*{\sum_{i=0}^\infty \parens*{\frac{1}{10}}^i} - \parens*{\frac{1}{10}}^0}\\
        &= 9 \cdot \parens*{\frac{1}{1-\nicefrac{1}{10}} - 1}\\
        &= 9 \cdot \parens*{\frac{10}{9} - 1}\\
        &= 9 \cdot \frac{1}{9}\\
        &= 1. \qedhere
    \end{align*}
\end{proof}

La proposizione vale in generale in una base $\beta$ qualsiasi ($\beta \geq 2$).
\begin{proposition}{}{}
    In base $\beta$ ($\beta \geq 2$) vale che $0.\overline{(\beta-1)} = 1$.  
\end{proposition}
\begin{proof}
    La dimostrazione è uguale alla seconda dimostrazione della \Cref{prop:9_periodico}.
\end{proof}

\section{La precisione di macchina è un limite superiore stretto}
\label{sez:machine_prec_>}

In questa sezione mostriamo che la precisione di macchina $\mathbold{u}$ è sempre strettamente maggiore all'errore relativo $\eps_x$ per ogni $x \in \R$ tale che $x \neq 0$ e $\omega \leq \abs x \leq \Omega$, anche usando l'arrotondamento come metodo di approssimazione.

Dalla dimostrazione del \Cref{th:machine_prec} si ha che \[
    \abs*{\eps_x} = \frac12\beta^{1-p} = \mathbold{u}
\] se e solo se la disuguaglianza \[
    \frac{\abs[\big]{\arr{x} - x}}{\abs x} 
    \leq \frac{\nicefrac12\beta^{p-t}}{\beta^{p-1}}
\] è in realtà un'uguaglianza, ovvero se e solo se valgono le seguenti due condizioni:
\begin{itemize}
    \item $\abs[\big]{\arr{x} - x} = \frac12\beta^{p-t}$
    \item $\abs x = \beta^{p-1}$.
\end{itemize}
Tuttavia, come abbiamo notato nella \Cref{rem:abs_err_arr}, la prima condizione vale se e solo se \[
    \abs x = \frac12(a+b) = \frac12(a + a + \beta^{p-t}) = a + \frac12\beta^{p-t}.
\] È quindi necessario che \begin{align*}
    &\beta^{p-1} 
        \begin{aligned}[t] 
            &= a + \frac12\beta^{p-t} \\
            &= \beta^p \parens*{\sum_{i=1}^t d_i\beta^{-i}} + \frac12\beta^{p-t} \\
            &= \beta^p \parens*{\sum_{i=1}^t d_i\beta^{-i} + \frac12\beta^{-t}}
        \end{aligned} \\
    \iff &\beta^{-1} = \sum_{i=1}^t d_i\beta^{-i} + \frac12\beta^{-t}.
\end{align*} Tuttavia ciò è assurdo: infatti il termine al primo membro ha una singola cifra decimale diversa da zero, ovvero quella di $\beta^{-1}$, mentre il termine del secondo membro ha anche delle cifre non nulle nelle posizioni $\beta^{-t}$ e/o $\beta^{-t-1}$.  

Segue quindi che $\abs{\eps_x} < u$ in qualsiasi caso.

\section{Dimostrazione dell'equivalenza topologica tra norme}
\label{sec:eq_topo_norm}

In questa sezione dimostreremo il \Cref{th:eq_topo}: per far ciò sfrutteremo due semplici lemmi.

\begin{lemma}{}{norm_continuous}
    Sia $\norm$ una norma su $\F^n$. Allora la funzione \[
        \vec x \mapsto \norm{\vec x}
    \] è una funzione continua rispetto alla norma $\infty$.
\end{lemma}
\begin{proof}
    Sia $\vec x \in \F^n$ un punto qualsiasi e dimostriamo che la norma è continua in $\vec x$. 
    Sia quindi $\eps > 0$ qualsiasi: vogliamo determinare un $\delta > 0$ tale che per ogni $\vec y \in \F^n$ con $\norm{\vec x - \vec y}_{\infty} < \delta$ si ha $\abs[\big]{\norm{\vec x} - \norm{\vec y}} < \eps$. Dimostriamo la tesi in due passi.

    \newthought{Primo step} Osserviamo che \[
        \norm{\vec x} = \norm{\vec x - \vec y + \vec y} \leq \norm{\vec y} + \norm{\vec x - \vec y},
    \] da cui \[
        \norm{\vec x} - \norm{\vec y} \leq \norm{\vec x - \vec y}.
    \] Scambiando i ruoli di $\vec x$ e $\vec y$ otteniamo che \[
        \norm{\vec y} - \norm{\vec x} \leq \norm{\vec y - \vec x} = \norm{\vec x - \vec y},
    \] quindi combinando le due disuguaglianze si ha \begin{equation*}
        \abs[\Big]{\norm{\vec x} - \norm{\vec y}} \leq \norm{\vec x - \vec y}.
    \end{equation*}

    \newthought{Secondo step} Scriviamo \[
        \vec x = \sum_{i=1}^n x_i\vec e_i, \qquad \vec y = \sum_{i=1}^n y_i\vec e_i,
    \] dove $\parens*{\vec e_i}_{i=1, \dots, n}$ è una base di $\F^n$ tale che $\norm{\vec e_i} = 1$: osserviamo che una tale base esiste sempre in quanto (per il \Cref{lem:vec_over_norm}) basta prendere una base qualsiasi $\BB \deq \parens*{\vec e_i}_{i=1, \dots, n}$ di $\F^n$ e considerare la base \[
        \BB^{\prime} \deq \parens*{\frac{\vec e_i}{\norm{\vec e_1}}}_{i=1, \dots, n}.
    \] Tale insieme di vettori è ancora una base e ogni vettore ha norma $1$ (rispetto alla norma che stiamo considerando).
    
    Allora \begin{align*}
        \norm{\vec x - \vec y} 
        &=    \norm*{\sum_{i=1}^n (x_i - y_i)\vec e_i} \\
        &\leq \sum_{i=1}^n \norm*{(x_i - y_i)\vec e_i}\\
        &=    \sum_{i=1}^n \abs*{x_i - y_i}\norm{\vec e_i}\\
        &=    \sum_{i=1}^n \abs*{x_i - y_i}\\
        &\leq n\norm{\vec x - \vec y}_{\infty}.
    \end{align*}

    \newthought{Conclusione} Poniamo quindi $\delta = \nicefrac{\eps}{n}$ e supponiamo che $\norm{\vec x - \vec y}_\infty < \delta = \nicefrac{\eps}{n}$. Segue quindi che \[
        \abs[\Big]{\norm{\vec x} - \norm{\vec y}} \leq \norm{\vec x - \vec y} \leq n\norm{\vec x - \vec y}_{\infty} < n\frac{\eps}{n} = \eps,
    \] ovvero $\norm$ è continua.
\end{proof}

\begin{lemma}{}{norm_equiv_infty}
    Siano $\norm_{a}, \norm_b$ due norme su $\F^n$. Se $\norm_{a}$ e $\norm_{b}$ sono entrambe equivalenti a $\norm_{\infty}$, allora $\norm_a$ e $\norm_b$ sono equivalenti.  
\end{lemma}
\begin{proof}
    Sia $\vec x \in \R^n$. Per definizione di equivalenza tra norme, esistono $\alpha_1, \alpha_2, \beta_1, \beta_2 \in \interval[{0, +\infty})$ tali che \[
        \alpha_1\norm{\vec x}_{\infty} \leq \norm{\vec x}_a \leq \beta_1\norm{\vec x}_\infty,
        \qquad
        \alpha_2\norm{\vec x}_{\infty} \leq \norm{\vec x}_b \leq \beta_2\norm{\vec x}_\infty.
    \] Osserviamo che dalla prima disuguaglianza segue che \[
        \norm{\vec x}_{\infty} \leq \frac{1}{\alpha_1} \norm{\vec x}_a, 
        \qquad
        \norm{\vec x}_{\infty} \geq \frac{1}{\beta_1} \norm{\vec x}_a,
    \] dunque segue che \[
        \frac{\alpha_2}{\beta_1} \norm{\vec x}_a \leq \alpha_2\norm{\vec x}_{\infty} \leq \norm{\vec x}_b \leq \beta_2\norm{\vec x}_\infty \leq \frac{\beta_2}{\alpha_1} \norm{\vec x}_a,
    \] ovvero $\norm_a$ e $\norm_b$ sono equivalenti.  
\end{proof}

Possiamo quindi dimostrare il \Cref{th:eq_topo}.
\begin{proof}
    Per il \Cref{lem:norm_equiv_infty} è sufficiente mostrare che la norma $\norm$ è equivalente alla norma infinito, ovvero che esistano $\alpha, \beta \in \interval[{0, +\infty})$ tali che \[
        \alpha\norm{\vec x}_{\infty} \leq \norm{\vec x} \leq \beta \norm{\vec x}_{\infty}.
    \] 
    
    Sicuramente la tesi vale per $\vec x = \vec 0$, dunque supponiamo $\vec x \neq \vec 0$. Allora dividendo i membri della disequazione per $\norm{\vec x}_{\infty}$ la tesi è equivalente a dimostrare che esistono $\alpha, \beta \in \interval[{0, +\infty})$ tali che \[
        \alpha \leq \frac{\norm{\vec x}}{\norm{\vec x}_{\infty}} = \norm*{\frac{\vec x}{\norm{\vec x}_{\infty}}} = \norm*{\vec u} \leq \beta,
    \] dove $\vec u \deq \frac{\vec x}{\norm{\vec x}_{\infty}}$ è un vettore tale che $\norm{\vec u}_{\infty} = 1$ (per il \Cref{lem:vec_over_norm}).

    È quindi sufficiente mostrare che la norma $\norm$ di tutti i vettori con norma infinito uguale a $1$ è limitata superiormente e inferiormente.
    
    Consideriamo quindi la sfera unitaria rispetto alla norma infinito \[
        S_\infty \deq \set*{\vec v \in \F^n \given \norm{\vec v}_{\infty} = 1}.
    \] Siccome per il \Cref{lem:norm_continuous} la funzione $\norm$ è continua e $S_{\infty}$ è un compatto in $\F^n$ (poiché chiuso e limitato) vale il Teorema di Weierstrass: esistono dunque $\alpha, \beta \in \interval[{0, +\infty})$ tali che \[
        \alpha \deq \min_{\vec v \in S_\infty} \norm{\vec v}, \qquad \beta \deq \max_{\vec v \in S_\infty} \norm{\vec v},
    \] ovvero per ogni $\vec u \in \F^n$ tale che $\norm{\vec u}_{\infty} = 1$ si ha che \[
        \alpha \leq \norm{\vec u} \leq \beta,
    \] da cui la tesi.
\end{proof}
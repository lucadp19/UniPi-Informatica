\section{Operazioni aritmetico-logiche}

Nel caso delle operazioni aritmetico-logiche la destinazione e il primo operando devono essere necessariamente dei registri (eventualmente anche lo stesso), mentre il secondo operando può essere un registro oppure un immediato (ovvero una costante numerica, come ad esempio \mintinline{ARM}|#5|).

\paragraph{Somme}
\begin{ARMcode}
    ADD  |<dest>|, |<op1>|, |<op2>|
    ADC  |<dest>|, |<op1>|, |<op2>|
    ADDS |<dest>|, |<op1>|, |<op2>|
\end{ARMcode}
Calcola la somma tra <op1> e <op2> e la salva nel registro <dest>. La variante \ARMinline{ADC} aggiunge eventualmente il riporto contenuto nel flag \strong{C}; la variante \ARMinline{ADDS} setta i flag dopo l'operazione.

\paragraph{Sottrazioni}
\begin{ARMcode}
    SUB  |<dest>|, |<op1>|, |<op2>|
    SBC  |<dest>|, |<op1>|, |<op2>|
    SUBS |<dest>|, |<op1>|, |<op2>|
\end{ARMcode}
Calcola la differenza tra <op1> e <op2> e la salva nel registro <dest>. La variante \ARMinline{ADC} aggiunge eventualmente il riporto contenuto nel flag \strong{C}; la variante \ARMinline{ADDS} setta i flag dopo l'operazione.

\begin{ARMcode}
    RSB |<dest>|, |<op1>|, |<op2>|
\end{ARMcode}
Calcola la differenza tra <op2> ed <op1> (notare l'ordine): la sua utilità è nel fatto che con la \ARMinline{SUB} non possiamo calcolare differenze della forma "costante - registro" poiché solo il secondo operando può contenere immediati.

\paragraph{Moltiplicazioni}
\begin{ARMcode}
    MUL  |<dest>|, |<op1>|, |<op2>|
    UMUL |<dest1>|, |<dest2>|, |<op1>|, |<op2>|
    SMUL |<dest1>|, |<dest2>|, |<op1>|, |<op2>|
\end{ARMcode}
\ARMinline{MUL} calcola il prodotto tra <op1> e <op2>: siccome essi sono a 32 bit e il prodotto di due interi a 32 bit è un intero con al massimo 64 bit, \ARMinline{MUL} mette in <dest> i 32 bit meno significativi del risultato. \ARMinline{UMUL} e \ARMinline{SMUL} invece calcolano il risultato a 64 bit e mettono i 32 bit più significativi in <dest2> e i 32 bit meno significativi in <dest1>: la differenza tra le due è che \ARMinline{UMUL} considera gli operandi come interi \emph{unsigned} (ovvero senza segno), mentre \ARMinline{SMUL} li considera con segno.

\paragraph{"Move"}
\begin{ARMcode}
    MOV |<dest>|, |<src>|
    MVN |<dest>|, |<src>|
\end{ARMcode}
\ARMinline{MOV} copia il contenuto di <src> dentro <dest>. \ARMinline{MVN} copia il contenuto \emph{negato} di <src> dentro <dest>.

\paragraph{"Compare"}
\begin{ARMcode}
    CMP |<op1>|, |<op2>|
\end{ARMcode}
\ARMinline{CMP} opera un confronto tra <op1> e <op2>: in particolare calcola la differenza tra i due valori e setta i flag.

\paragraph{Operazioni logiche}
\begin{ARMcode}
    AND |<dest>|, |<op1>|, |<op2>|
    ORR |<dest>|, |<op1>|, |<op2>|
    EOR |<dest>|, |<op1>|, |<op2>|
\end{ARMcode}
Calcolano rispettivamente l'AND bit a bit, l'OR bit a bit e lo XOR bit a bit di <op1> e <op2>, mettendo il risultato in <dest>.

\paragraph{"Bit clear"}
\begin{ARMcode}
    BIC |<dest>|, |<op1>|, |<op2>|
\end{ARMcode}
Effettua un'operazione di \emph{bit clear}: mette a 0 tutti i bit di <op1> che sono settati ad 1 in <op2> e mette il risultato in <dest>.

\paragraph{Shift binario}
\begin{ARMcode}
    LSL |<dest>|, |<op1>|, |<op2>|
    LSR |<dest>|, |<op1>|, |<op2>|
    ASL |<dest>|, |<op1>|, |<op2>|
    ASR |<dest>|, |<op1>|, |<op2>|
\end{ARMcode}
Calcolano gli shift logici (L) o arimetici (A) a sinistra (L) o destra (R) di <op1> (il numero di posizioni da shiftare è dato da <op2>) e mettono il risultato in <dest>.

\paragraph{Rotazione binaria}
\begin{ARMcode}
    ROR |<dest>|, |<op1>|, |<op2>|
\end{ARMcode}
Ruota a destra i bit di <op1> di <op2> posizioni e mette il risultato in <dest>.

\paragraph*{Operazioni condizionali}
Possiamo aggiungere a tutte le operazioni precedenti due lettere finali che indicano sotto quali condizioni (dei flag) l'operazione viene svolta. Le condizioni sono:
\begin{itemize}
    \item \ARMinline{EQ}: sta per $Z = 1$;
    \item \ARMinline{NE}: sta per $Z = 0$;
    \item \ARMinline{GE}: sta per $N = V$;
    \item \ARMinline{GT}: sta per $Z = 0$, $N = V$;
    \item \ARMinline{LE}: sta per $Z = 1$, $N \neq V$;
    \item \ARMinline{LT}: sta per $N \neq V$.
\end{itemize}

Quindi ad esempio il codice
\begin{ARMcode}
    CMP   R0, #1
    ADDEQ R0, R0, #1
\end{ARMcode}
aggiunge $1$ al registro \ARMinline{R0} solo se la \ARMinline{CMP} ha dato come risultato \ARMinline{EQ}, ovvero se la differenza tra \ARMinline{R0} e $1$ è $0$ (e quindi sono uguali).
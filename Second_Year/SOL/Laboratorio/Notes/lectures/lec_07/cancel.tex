\section{Cancellazione di un thread}

Introduciamo ora un'ulteriore funzione utile per gestire i thread, chiamata \inlinec{pthreaad_cancel}.

\begin{minted}{c}
    #include <pthread.h>

    int pthread_cancel(
        pthread_t *tid, // thread id
    );
    // 0 on success, error value otherwise
\end{minted}

Questa funzione fa terminare un thread appena raggiunge un \emph{punto di cancellazione}, che in genere è una chiamata di funzione \emph{safe}, nel senso che la funzione chiamata non è una mutex, malloc, o cose simili.

Tuttavia la cancellazione esplicita dei thread deve essere vista come un'ultima spiaggia: nella maggior parte dei casi va evitata in quanto termina brutalmente il thread.

Per renderla più \emph{pulita} possiamo usare i \strong{cleanup handlers}, che sono una versione specifica per i thread della funzione \inlinec{atexit}: servono a chiamare delle funzioni specifiche alla terminazione di un thread, che sia per \inlinec{pthread_exit} o \inlinec{return} dalla funzione principale o per cancellazione.

\begin{minted}{c}
    #include <pthread.h>
    void pthread_cleanup_push(
        void (*handler) (void*), // cleanup function
        void *arg, // arguments to the handler
    );
\end{minted}

Come nel caso di \inlinec{atexit} possiamo registrare più funzioni, e queste vengono messe \emph{in pila} e quindi eseguite in ordine inverso di registrazione. Per rimuovere funzioni dalla pila possiamo usare la seguente funzione:
\begin{minted}{c}
    void pthread_cleanup_pop(
        int execute, // do we want to execute the function?
    );
\end{minted}
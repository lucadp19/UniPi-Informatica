\section{Interferenze tra thread}

Abbiamo già detto che non abbiamo alcun controllo sull'ordine di esecuzione dei thread: questo può portare a dei problemi di \emph{concorrenza} tra thread chiamati \strong{race conditions}.

Consideriamo il \Cref{lst:threads_1}. I due thread di quel programma operano sulla variabile condivisa \inlinec{x} contemporaneamente, sommando $1$ e poi stampando il risultato.

Tuttavia in alcuni casi sfortunati il comportamento potrebbe non essere quello che ci aspettiamo. Immaginiamo ad esempio la seguente sequenza di operazioni (chiamando i due thread $T_1$ e $T_2$): \begin{enumerate}
    \item \inlinec{x} è inizialmente uguale a $5$; 
    \item il thread $T_1$ legge \inlinec{x}, ma prima di poterla incrementare viene \emph{deschedulato};
    \item $T_2$ legge \inlinec{x}, che è ancora uguale a $5$, incrementa la variabile e la scrive (stampando $6$), poi viene sospeso;
    \item $T_1$ viene rieseguito e scrive anche lui $6$ in \inlinec{x}.     
\end{enumerate}

In questo caso abbiamo solamente perso un incremento, ma in problemi più seri possiamo causare grandi danni a causa di problemi di concorrenza di questo tipo.

Vogliamo quindi fare in modo che un thread abbia momentaneamente un accesso esclusivo a dei dati condivisi: in questo modo non possono esserci problemi di concorrenza tra thread.

Introduciamo quindi alcuni concetti fondamentali:
\begin{itemize}
    \item si usa il termine \strong{regione critica} per riferirci al frammento di codice che contiene la lettura/scrittura/modifica della risorsa condivisa;
    \item vogliamo fare in modo che l'accesso alla regione critica sia sempre \strong{mutuamente esclusivo}: due thread diversi non possono accedere ad una regione critica contemporaneamente. 
\end{itemize}

Il meccanismo base per risolvere il problema delle \emph{race conditions} è il meccanismo delle \strong{lock} (in italiano \emph{lucchetti}): creiamo una variabile speciale, condivisa dai thread, che ha lo scopo di bloccare l'accesso ad una regione critica se qualche altro thread sta già usando la risorsa condivisa.

La libreria \inlinec{pthread.h} usa il tipo \inlinec{pthread_mutex_t} per rappresentare un \strong{mutex} (che è un modo per implementare le lock) insieme alle seguenti due funzioni: 
\begin{minted}{c}
    #include <pthread.h>

    int pthread_mutex_lock(
        pthread_mutex_t *mutex  // mutex to lock
    );

    int pthread_mutex_unlock(
        pthread_mutex_t *mutex  // mutex to unlock
    );
\end{minted}

Le due funzioni servono ovviamente a richiedere l'uso di un lucchetto (tramite la \inlinec{pthread_mutex_lock}) o a rilasciarlo (tramite la \inlinec{pthread_mutex_unlock}) e ritornano $0$ in caso di successo, il codice dell'errore altrimenti.

\subsection{Uso dei mutex}

Per usare correttamente i mutex bisogna seguire questo schema:
\begin{itemize}
    \item si crea un mutex per ogni risorsa condivisa a cui vogliamo accedere in mutua esclusione;
    \item prima di accedere ai dati gestiti dal mutex \inlinec{mutex_data} si chiama la funzione \inlinec{pthread_mutex_lock(&mutex_data)};
    \item alla fine della sezione critica si chiama \inlinec{pthread_mutex_unlock(&mutex_data)}. 
\end{itemize} La \inlinec{pthread_mutex_lock} controlla se il lucchetto è occupato da qualche altro thread: se è libero lo prende, lo setta e il thread è libero di entrare nella zona in mutua esclusione; invece se è occupato il thread si \strong{sospende} fino a quando il thread che ha bloccato il mutex non lo rilascia.

\newthought{Inizializzazione} Bisogna sempre inizializzare i mutex quando li dichiariamo! Se sono globali possiamo usare una macro: 
\begin{minted}{c}
    pthread_mutex_t mtx = PTHREAD_MUTEX_INITIALIZER;
\end{minted}
mentre se non sono globali dobbiamo usare la funzione 
\begin{minted}{c}
    int pthread_mutex_init(
        pthread_mutex_t *mtx, // mutex to initialize
        const pthread_mutexattr_t *attr // attributes (NULL is default)
    ); // always returns 0
\end{minted}
Noi useremo sempre la macro.


Il \Cref{lst:threads_1} può dunque essere migliorato come si vede in \Cref{lst:mtx_exmp}: abbiamo circondato la linea di codice che si occupa di interagire con \inlinec{x} con il mutex. In questo modo otteniamo la mutua esclusione e impediamo le race conditions che erano presenti nella prima versione.

Osserviamo che abbiamo incluso nella regione critica \emph{soltanto} il codice strettamente necessario: se avessimo incluso, ad esempio, anche \inlinec{sleep(1)} avremmo rallentato il programma inavvertitamente. Infatti in tal caso mentre $T_1$ è in possesso del lock $T_2$ è bloccato, anche se $T_1$ è in \inlinec{sleep}: abbiamo eliminato il parallelismo fra i thread e il programma viene eseguito esattamente come se fosse un singolo thread.

\subsection{Versioni "safe" di lock/unlock}

Inoltre come al solito le funzioni \inlinec{pthread_mutex_lock} e \inlinec{pthread_mutex_unlock} possono restituire degli errori che vanno sempre controllati: è buona pratica quindi creare delle funzioni ausiliarie per farlo automaticamente.

\begin{minted}{c}
    void safe_pthread_mutex_lock(pthread_mutex_t *mtx){
        int err;
        if( (err = pthread_mutex_lock(&mtx)) != 0) {
            errno = err;
            perror("lock");
            pthread_exit((void*) errno); // can be changed
        } else {
            printf("locked\n"); // can be commented when the program is finished
        }
    }

    void safe_pthread_mutex_unlock(pthread_mutex_t *mtx){
        int err;
        if( (err = pthread_mutex_unlock(&mtx)) != 0) {
            errno = err;
            perror("unlock");
            pthread_exit((void*) errno); // can be changed
        } else {
            printf("unlocked\n"); // can be commented when the program is finished
        }
    }
\end{minted}

\subsection{Restringere l'area d'uso}

A questo punto possiamo evitare la duplicazione del codice creando una funzione che incrementi il valore di \inlinec{x}: in questo modo non serve gestire in due punti diversi il mutex ma basta chiamare la funzione appena definita.

\begin{minted}{c}
    static int atomic_incr_x(int incr){
        static int x = 0; // the shared variable
        static pthread_mutex_t mtx = PTHREAD_MUTEX_INITIALIZER;
        int res;

        safe_pthread_mutex_lock(&mtx);
        res = (x += incr);
        safe_pthread_mutex_unlock(&mtx);

        return res;
    }
\end{minted}

Notiamo come abbiamo dovuto usare un'altra variabile \inlinec{res}, in quanto altrimenti staremmo accedendo a $x$ al di fuori dell'area delimitata dal mutex: se restituissimo il valore di \inlinec{x} direttamente lo scheduler potrebbe toglierci il controllo del processore appena prima della \inlinec{return}, dunque qualche altro thread potrebbe modificare $x$ e il valore restituito sarebbe scorretto.
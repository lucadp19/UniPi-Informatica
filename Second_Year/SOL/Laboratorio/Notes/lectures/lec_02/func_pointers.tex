\section{Puntatori a funzione}

Il C permette di definire puntatori a diversi tipi di oggetti: puntatori a \inlinec{int}, a \inlinec{char}, puntatori a puntatori, ecc...
In particolare, consente anche la definizione di \strong{puntatori a funzione}:

\begin{minted}{c}
    int somma (int a, int b) {
        return a + b;
    }
    /*
    - - - - > return type
    |           - - - - -> parameter types
    |           |    |     */
    int (*fun) (int, int);
    fun = somma;

    int a = fun(3, 6); // a <- 9
\end{minted}

A cosa servono i puntatori a funzione? Dopotutto, il codice precedente poteva essere semplificato usando direttamente \inlinec{somma} invece di \inlinec{fun}.

Un possibile caso d'uso dei puntatori a funzione è nel caso delle cosiddette \emph{funzioni di ordine superiore}, ovvero funzioni che prendono come argomento una funzione. Un classico esempio di funzione di ordine superiore è \inlinec{map}: matematicamente, data una collezione di dati $S$ (che sia un insieme, una lista, un array, ecc...) e una funzione $f$, l'applicazione \inlinec{map(|$f$|, |$S$|)} mi restituisce la collezione $S$ dove ad ogni elemento $x \in S$ sostituisco il numero $f(x)$.

\begin{example}
    Dato l'insieme $S = \{1, 2, 3, 4\}$ e la funzione $f(x) = 2x + 1$, il risultato di \inlinec{map(|$f$|, |$s$|)} è $\{3, 5, 7, 9\}$: ho \emph{mappato} la funzione $f$ su tutti gli elementi della collezione $S$.
\end{example}

Una possibile implementazione della \inlinec{map} (di funzioni \inlinec{int |$\to$| int} su array di interi) in C è la seguente.

\begin{minted}{c}
        /*    fun has type 
                int -> int
            ------------------  */
    void map(int (*fun)(int), int arr[], size_t len){
        for(int i = 0; i < len; i++)
            arr[i] = fun(arr[i]);
}
\end{minted}

Dato che i tipi di puntatori a funzione possono essere poco chiari è possibile definire degli \emph{alias} tramite la \inlinec{typedef}:
\begin{minted}{c}
    // defines a new type, called myFuncType
    typedef int (*myFuncType)(int);
    // now we can define map like this
    void map(myFunctType fun, int arr[], size_t len){
        for(int i = 0; i < len; i++)
            arr[i] = fun(arr[i]);
    }
\end{minted}
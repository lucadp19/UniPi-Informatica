\section{Funzioni rientranti}

% \begin{definition}
%     {Funzione rientrante}{reentrant}
%     Una funzione si dice \strong{rientrante} se può essere interrotta in un punto e poi ripresa da quel punto senza che questo produca un risultato diverso all'esecuzione senza interruzione.
% \end{definition}

Ad esempio la seguente funzione non è rientrante:
\begin{minted}{c}
    int x = 5;

    int func(){
        printf("Beginning: x = %d\n", x);
        x = x+2;
        printf("End: x = %d\n", x);
    }
\end{minted}

Infatti se interrompiamo la funzione dopo l'assegnamento \inlinec{x = x+2} ed eseguiamo qualche altra cosa, il contenuto della variabile globale \inlinec{x} può essere modificato e la funzione non stampa più il numero \inlinec{x+2}.

Una funzione rientrante non può fare uso di variabili globali o statiche.

Nel resto del corso varemo sempre e solo uso di funzioni rientranti, spesso indicate con il suffisso \inlinec{_r}.

\subsection*{Tokenizzazione di stringhe}
\emph{Tokenizzare} una stringa significa spezzarla ad ogni occorrenza di un determinato carattere. Per far ciò può essere utile usare le funzioni \inlinec{strtok} e la sua corrispondente versione rientrante, \inlinec{strtok_r}.

La versione non rientrante è la seguente:
\begin{minted}{c}
    int main(){
        char string[] = "Hello crazy world";
        // token represents the single portion of the string
        char* token = strtok(string, " ");
                    /*          |     | - -> separator, here it's a space
                                | - -> string to tokenize
                    */
        // keep reading single tokens
        while(token){
            printf("%s\n", token);
            // calling strtok with NULL uses the string tokenized before,
            // using a hidden global variabile
            token = strtok(NULL, " ");
        }
    }
\end{minted}

Per renderla rientrante dobbiamo eliminare la variabile globale nascosta: la versione rientrante \inlinec{strtok_r} ha infatti un parametro aggiuntivo, usato nel seguente modo.
\begin{minted}{c}
    int main(){
        char string[] = "Hello crazy world";
        // save variable, used by strtok_r
        char* save = NULL;
        // token represents the single portion of the string
        char* token = strtok_r(string, " ", &save);
                    //                        | - -> local state!
        // keep reading single tokens
        while(token){
            printf("%s\n", token);
            // calling strtok with NULL uses the string tokenized with the state saved in "save"
            token = strtok(NULL, " ", &save);
        }
    }
\end{minted}

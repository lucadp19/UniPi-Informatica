\section{Terminazione}

Un processo UNIX può terminare solo in $4$ modi:
\begin{itemize}
    \item chiamando \inlinec{exit()};
    \item chiamando \inlinec{_exit()} (UNIX) oppure \inlinec{_Exit()} (standard C), che sono chiamate di livello più basso rispetto alla \inlinec{exit()};
    \item ricevendo un segnale (li vedremo in seguito);
    \item per un crash del sistema (spegnimento forzato del PC, bug dell'OS, ecc).
\end{itemize}

Le tre funzioni \inlinec{exit()}, \inlinec{_exit()} e \inlinec{_Exit()} sono molto simili, a partire dalla \emph{signature}:
\begin{minted}{c}
    #include <unistd.h>

    void _exit(
        int status  // exit status
    );
    
    void _Exit(
        int status  // exit status
    );
    
    void exit(
        int status  // exit status
    );
\end{minted}

Nessuna delle $3$ funzioni ha un valore di ritorno, in quanto terminano il processo corrente.

La \inlinec{exit()} fa tutto quello che fa la \inlinec{_exit()}, ma inoltre \begin{itemize}
    \item chiama la funzione \inlinec{atexit()}, se esiste;
    \item esegue il flush dei buffer di \textsf{I/O} tramite \inlinec{fflush} e \inlinec{fclose}.
\end{itemize}

La \inlinec{exit} viene chiamata automaticamente quando il programma termina con \inlinec{return} dal \inlinec{main}.

\newthought{Funzione \inlinec{atexit()}} La funzione \inlinec{atexit()} serve a \emph{registrare} le azioni da compiere alla chiusura del programma. Ha la seguente \emph{signature}:
\begin{minted}{c}
    #include <stdlib.h>

    int atexit(
        void *function (void); 
    )
\end{minted}

Se la \inlinec{atexit} ha successo ritorna $0$, altrimenti ritorna un valore diverso da $0$ ma \strong{NON} setta \inlinec{errno}.

La funzione \inlinec{function} deve essere una funzione che non prende argomenti e non restituisce alcun valore, il cui codice conterrà le operazioni da compiere alla chiusura del programma. Le tipiche operazioni che vengono inserite in questa funzione sono \begin{itemize}
    \item cancellazione di file temporanei;
    \item stampa di messaggi sull'esito della computazione;
    \item \emph{pipes} (le vedremo più avanti);
    \item eccetera.
\end{itemize}

Inoltre possiamo registrare più di una funzione, tramite chiamate multiple alla \inlinec{atexit}: alla terminazione del programma le funzioni registrate saranno chiamate in ordine \strong{inverso}.

\newthought{Funzione \inlinec{_exit()}} 
La funzione \inlinec{_exit(status)} svolge le seguenti operazioni: \begin{itemize}
    \item termina il processo;
    \item chiude tutti i descrittori di file;
    \item libera lo spazio di indirizzamento;
    \item invia un segnale \textsf{SEGCHILD} al padre;
    \item salva il byte meno significativo di \inlinec{status} nella tabella dei processi, in attesa che il padre lo accetti tramite le funzioni \inlinec{wait}/\inlinec{waitpid} (prossima sezione);
    \item i figli, diventati a questo punto orfani (\emph{orphans}), vengono \emph{adottati} da \textsf{init}, ovvero il loro \textsf{PPID} diventa uguale a $1$. 
\end{itemize}
\section{Exec}

Spesso quello che vogliamo fare quando creiamo un nuovo processo non è avere due copie dello stesso processo, ma eseguire un programma diverso. 
Per far ciò sfruttiamo una sequenza di due funzioni:
prima usiamo la \inlinec{fork} per creare un nuovo \textsf{PCB} e dopo usiamo una funzione della famiglia delle \inlinec{exec*} per \emph{cambiare eseguibile}.

In generale una qualsiasi funzione della famiglia delle \inlinec{exec*} segue il seguente procedimento:
\begin{itemize}
    \item trova il file eseguibile da eseguire;
    \item usa i contenuti di quel file per sovrascrivere lo spazio di indirizzamento del processo che ha invocato la \inlinec{exec};
    \item carica nel program counter \textsf{PC} l'indirizzo iniziale.
\end{itemize}

Le funzioni della famiglia \inlinec{exec*} sono $6$ e sono \inlinec{execl}, \inlinec{execlp}, \inlinec{execle}, \inlinec{execv}, \inlinec{execvp}, \inlinec{execve}.

\newthought{\inlinec{execl}}
Vediamo la \inlinec{execl}:
\begin{minted}{c}
    #include <unistd.h>

    int execl(
        char* path, // path to the executable
        char* arg0, // first arg: name of the file
        char* arg1, // second arg
        |\dots|
        char* argN, // Nth argument
        (char*) NULL   // the list must be NULL-terminated! 
    );
\end{minted}

Se la \inlinec{execl} fallisce ritorna $-1$ e setta \inlinec{errno}, invece se ha successo \strong{non ritorna}: infatti l'eseguibile è cambiato, quindi il vecchio eseguibile non esiste più e non potrebbe vedere il risultato restituito.

La lista di argomenti di lunghezza variabile passata ad \inlinec{execl} deve essere terminata da un valore \inlinec{NULL} e potrà essere usata dal nuovo programma attraverso \inlinec{argv}.

Infine, il \emph{path} specificato come primo argomento della \inlinec{execl} deve essere un file eseguibile e deve avere i permessi per l'esecuzione per l'effective-user-ID del processo che invoca la \inlinec{execl}.

\newthought{Altre exec}
Le altre funzioni della famiglia delle \inlinec{exec*} possono essere distinte sulla base delle lettere che seguono "exec":
\begin{itemize}
    \item se vogliamo passare gli argomenti al nuovo programma come una lista \inlinec{NULL}-terminated usiamo la lettera \inlinec{l};
    \item se vogliamo passarli come array di argomenti che rispetta il formato di \inlinec{argv} (ovvero l'ultimo argomento deve essere \inlinec{NULL}) possiamo usare la lettera \inlinec{v};
    \item se vogliamo che la \inlinec{exec} cerchi il file nelle directory specificate dalla variabile di ambiente \textsf{PATH} possiamo usare la lettera \inlinec{p};
    \item se vogliamo passare un array di stringhe che descrivono l'ambiente (\emph{environment}) possiamo usare la lettera \inlinec{e}; se non lo facciamo, l'ambiente viene preservato.
\end{itemize}

In generale il processo rimane lo stesso, quindi diversi attributi (come il \textsf{PID}, \textsf{PPID}, descrittori dei file aperti, ecc...) rimangono invariati. Gli attributi che possono cambiare sono \begin{itemize}
    \item la gestione dei segnali (la vedremo più avanti);
    \item effective-user-ID e effective-group-ID, se il nuovo eseguibile li ha settati;
    \item le funzioni registrate in \inlinec{_atexit};
    \item i segmenti di memoria condivisa;
    \item i semafori POSIX (che vengono resettati).
\end{itemize}

\newthought{Flush dei buffer}
Osserviamo che se cambiamo eseguibile con la \inlinec{exec} non stiamo terminando il programma precedente, quindi non c'è il \emph{flush} automatico delle funzioni della libreria \inlinec{stdio.h}. 

Ad esempio il programma \textsf{execl-test.c}:
\begin{minted}{c}
    int main () {
        printf("The quick brown fox jumped over");
        // calling "echo", the program that prints things
        execl("/bin/echo", "echo", "the", "lazy", "dogs", (char*)NULL); 
        // if execl returns then an error occurred!
        perror("execl");
        return 1;
    }
\end{minted}
ha un effetto indesiderato:
\begin{minted}{shell-session}
    $ ./execl-test
    the lazy dogs
    $ 
\end{minted}

La chiamata a \inlinec{printf} non ha avuto alcun effetto.

Infatti come abbiamo visto la \inlinec{printf} (come molte delle funzioni della libreria \inlinec{stdio.h}) è \emph{bufferizzata}: le stringhe da scrivere non vengono subito inviate allo \inlinec{stdout}, ma vengono memorizzate in un buffer e inviate quando il buffer è pieno oppure quando viene fatto un \emph{flush} del buffer.

Generalmente il \emph{flush} viene fatto ogni volta che un programma termina, ovvero quando chiama la funzione \inlinec{exit}, tuttavia in questo caso a causa dell'\inlinec{execl} la \inlinec{exit} non viene chiamata.

La soluzione è quindi fare un flush manuale tramite la funzione \inlinec{fflush}:
\begin{minted}{c}
    int main () {
        printf("The quick brown fox jumped over");
        fflush(stdout);
        
        execl("/bin/echo", "echo", "the", "lazy", "dogs", (char*)NULL); 
        // if execl returns then an error occurred!
        perror("execl");
        return 1;
    }
\end{minted}
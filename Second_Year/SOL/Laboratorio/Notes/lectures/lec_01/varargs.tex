\section{Funzioni con un numero variabile di argomenti}

Alcune funzioni del C hanno un numero variabile di argomenti: ad esempio la \inlinec{printf} può accettare un singolo argomento (solo una stringa), ma anche $2$, $3$, $\dots$, a seconda di quanti placeholder (come \mintinline{c}{%d}, \mintinline{c}{%s}, eccetera) ci sono.

\begin{minted}{c}
    char* str = "Mario";
    int a = 3;
    int b = 5;

    printf("Ciao!");
    printf("Ciao %s!", str); // prints "Ciao Mario!"
    printf("%d + %d = %d", a, b, a + b) // prints "3 + 5 = 8"
\end{minted}

Per dichiarare funzioni con un numero variabile di argomenti bisogna usare l'\emph{include file} \inlinec{stdarg.h}.

Facciamo un esempio con una funzione che prende un intero \inlinec{count} che rappresenta il numero di argomenti passati alla funzione e fa la somma di tutti gli argomenti passati successivamente.

\inputminted{c}{./c_files/va_sum.c}

In questo programma abbiamo usato diverse funzionalità forniteci da \inlinec{stdarg.h}:
\begin{itemize}
    \item il tipo \inlinec{va_list} rappresenta la lista di tutti gli argomenti variabili;
    \item la macro \inlinec{va_start} che serve ad inizializzare una lista e prende due argomenti: una lista di argomenti variabili (di tipo \inlinec{va_list}) e \strong{l'ultimo argomento non-variabile};
    \item la macro \inlinec{va_arg} serve a ottenere il prossimo argomento: prende la \inlinec{va_list} e un \strong{tipo}, che deve essere conosciuto a tempo di compilazione;
    \item la macro \inlinec{va_end} che serve a liberare la memoria usata dalla \inlinec{va_list}.
\end{itemize}
\section{Type-system}

Cerchiamo ora di comprendere come deve essere implementato un \emph{type-system} per un linguaggio di programmazione.

Un \strong{type system} è una funzione che associa ad ogni oggetto un \emph{tipo}. Esaminando il flusso dei valori calcolati, il sistema ci assicura che il programma non contenga errori di tipo.

Per tenere traccia del tipo di ogni identificatore nel nostro programma dobbiamo usare un \strong{ambiente dei tipi}. Indicheremo con \[
    \Gamma = x_1 : \tau_1,\ x_2 : \tau_2,\ \dots,\ x_n : \tau_n
\] l'ambiente dei tipi che associa all'espressione $x_i$ il valore $\tau_i$. Inoltre useremo il simbolo \[
    (\Gamma,\ x : \tau)    
\] per indicare l'ambiente dei tipi che associa $x$ al tipo $\tau$ e associa ogni espressione $y$ diversa da $x$ il tipo ad essa associato precedentemente da $\Gamma$.
Infine, il judgment \[
    \Gamma \proves x : \tau    
\] indica che nell'ambiente dei tipi $\Gamma$ l'espressione $x$ ha tipo $\tau$.

Definiamo ora un semplice sistema dei tipi per le nostre espressioni.
\begin{gather*}
    \Gamma \proves n : \mt{int}\\
    \Gamma \proves b : \mt{bool}\\
    \infer{
        \Gamma \proves x : \tau
    }{
        \Gamma(x) = \tau
    }\\
    \infer{
        \Gamma \proves \mt{let x = e1 in e2} : \tau
    }{
        \Gamma \proves \mt{e1} : \tau' &
        (\Gamma, x : \tau') \proves \mt{e2} : \tau
    }\\[5pt]
    \infer{
        \Gamma \proves \mt{e1 |$\otimes$| e2} : \tau
    }{
        \Gamma \proves \mt{e1} : \tau_1 &
        \Gamma \proves \mt{e2} : \tau_2 &
        \Gamma \proves \otimes : \tau_1 \times \tau_2 \to \tau
    }\\[5pt]
    \infer{
        \Gamma \proves \mt{(if cond then e1 else e2)} : \tau
    }{
        \Gamma \proves \mt{cond} : \mt{bool} &
        \Gamma \proves \mt{e1} : \tau &
        \Gamma \proves \mt{e2} : \tau
    }
\end{gather*}

Nel caso delle funzioni il type system deve diventare necessariamente più complicato. Definiamo il tipo di una funzione come il \emph{tipo freccia} $\tau_1 \to \tau_2$, dove $\tau_1$, $\tau_2$ sono due tipi del nostro type system.

Abbiamo quindi le seguenti due regole di inferenza, una per le definizioni di funzione e una per le applicazioni:
\begin{gather*}
    \infer{
        \Gamma \proves \mt{fun (|x : $\tau_1$|) = e} : \tau_1 \to \tau_2
    }{
        (\Gamma,\ x : \tau_1) \proves \mt{e} : \tau_2
    }\\[5pt]
    \infer{
        \Gamma \proves \mt{App(e1, e2)} : \tau_2
    }{
        \Gamma \proves \mt{e1} : \tau_1 \to \tau_2 &
        \Gamma \proves \mt{e2} : \tau_1
    }
\end{gather*}

Queste due regole possono tuttavia risultare insufficienti nel caso di \emph{scoping dinamico}.
Parleremo in seguito del concetto di scoping statico e dinamico, ma per il momento accontentiamoci di questo esempio.

\begin{OCaml}
    let x = 10 in
    let f = fun k -> k + x in
    let x = 5 in
        f 10;;
\end{OCaml}

Se analizziamo questo codice abbiamo due possibili interpretazioni:
\begin{itemize}
    \item la \InlineOCaml{x} contenuta nella chiamata a funzione \InlineOCaml{f x} si riferisce alla variabile \InlineOCaml{x} legata al valore $10$ dal primo operatore \InlineOCaml{let}: infatti essa è l'unica variabile chiamata \InlineOCaml{x} al momento della definizione della funzione;
    \item la \InlineOCaml{x} contenuta nella chiamata a funzione \InlineOCaml{f x} si riferisce alla variabile \InlineOCaml{x} legata al valore $5$ dal terzo operatore \InlineOCaml{let}.
\end{itemize}

Nel primo caso (chiamato \emph{scoping statico}) il valore dell'espressione complessiva è $20$; nel secondo (chiamato \emph{scoping dinamico}) il valore è invece $15$. Tuttavia nel secondo caso le regole del nostro type-system non possono essere controllate staticamente: l'espressione \begin{OCaml}
    let x = 10 in
    let f = fun k -> k + x in
    let x = false in
        f 10;;
\end{OCaml}
conterrebbe un ovvio errore di tipo che non si manifesterebbe fino al momento dell'esecuzione.

\subsection{Implementazione in OCaml di un type system}
Innanzitutto dobbiamo decidere quali sono i tipi del nostro linguaggio:
\begin{OCaml}
    type tval =
        | TInt
        | TBool
        | FunT of tval * tval
\end{OCaml}
A questo punto dobbiamo creare l'ambiente dei tipi:
\begin{OCaml}
    type tenv = ide -> tval
\end{OCaml}

Le funzioni \InlineOCaml{bind_t} e \InlineOCaml{empty_tenv} ci consentono rispettivamente di aggiungere un binding ad un ambiente dei tipi dato e di ottenere l'ambiente vuoto.
\begin{OCaml}
    let bind_t (env : tenv) (id : ide) (v : tval) : tenv =
        fun x -> if x = id then v else env x
    let empty_tenv : tenv = fun (x : ide) -> raise EmptyEnv
\end{OCaml}

Per poter sfruttare il typechecker dobbiamo modificare il tipo delle espressioni che usiamo, aggiungendo dove necessario i tipi dei parametri.
\begin{OCaml}
    type texp =
        | EInt of int
        | EBool of bool
        | Den of ide
        | Add of texp * texp
        | ...
        | Let of ide * texp * texp
        | Fun of ide * tval * texp
        | Apply of texp * texp
\end{OCaml}

Osserviamo che l'unica modifica fatta è al costruttore delle funzioni anonime: in questo caso dobbiamo sapere di che tipo è il parametro per poter dedurre il tipo della funzione.

Il typechecker rispetta precisamente le regole semantiche date in precedenza:
\begin{OCaml}
    let rec teval (e : texp) (env : tenv) : tval =
        match e with
        | EInt a -> TInt
        | EBool b -> TBool
        | Den s -> env s
        | Let (id, e1, e2) -> 
            let t = teval e1 tenv in
                teval e2 (bind tenv id t)
        | Add (e1, e2) ->
            let t1 = teval e1 tenv in
            let t2 = teval e2 tenv in
                ( match t1, t2 with
                    | (TInt, TInt) -> TInt
                    | _ -> raise WrongType)
        | ...
        | If (cond, e1, e2) ->
            let tcond = teval cond tenv in
                ( match tcond with 
                    | TBool -> 
                        let t1 = teval e1 tenv in
                        let t2 = teval e2 tenv in
                            ( match t1, t2 with
                                | TInt, TInt -> TInt
                                | TBool, TBool -> TBool
                                | _ -> raise WrongType ))
        | Fun (id, tpar, body) ->
            let tenv1 = bind tenv id tpar in
            let t2 = teval body tenv1 in
                FunT (t1, t2)
        | Apply (e1, e2) ->
            let f = teval e1 tenv
                ( match f with
                    | Fun(t1, t2) -> 
                        if (teval e2 tenv = t1) then t2
                        else raise WrongType
                    | _ -> raise WrongType )
\end{OCaml}

Possiamo anche estendere il type system definito in precedenza con il supporto ad altri tipi, ad esempio con il supporto alle funzioni ricorsive, oppure alle coppie di valori, eccetera.

% TODO : do this

% Ad esempio cerchiamo di estendere il linguaggio ed il typechecker con un costrutto per definire funzioni ricorsive. Il costrutto sarà \[
%     \mt{rec}\ f\ (x : \tau_1) : \tau_2 = e
% \] dove $e$ è un'espressione che può contenere la funzione $f$. La regola operazionale per descrivere il tipo di $f$ sarà quindi \[
%     \infer {
%         \Gamma \proves (\mt{rec}\ f\ (x : \tau_1) : \tau_2 = e) : \tau_1 \to \tau_2
%     } {
%         (\Gamma,\ f : \tau_1 \to \tau_2,\ x : \tau_1) \proves e : \tau_2
%     }
% \]

% La corrispondente regola in OCaml è \begin{OCaml}
%     let rec teval (e : texp) (env : tenv) : tval =
%         match e with
%         | ...
%         | Rec(Den funName, Den x, )
% \end{OCaml}
\section{Garbage collection}

Abbiamo visto studiando la JVM che durante l'esecuzione di un programma Java si usano tre tipi di memoria:
\begin{itemize}
    \item una \emph{static area} dove vengono allocati contenuti definiti a tempo di compilazione;
    \item un \emph{run-time stack}, che è di dimensione variabile e viene usato per allocare i vari record di attivazione dei sottoprogrammi e dei blocchi;
    \item uno \emph{heap}, di dimensione fissa o variabile, usato per allocare oggetti e strutture dati dinamiche tramite il costrutto \InlineJava{new}.
\end{itemize}

Nella prima area vi sono generalmente le variabili globali, le variabili locali dei sottoprogrammi (non ricorsivi), le costanti determinate a tempo di compilazione, le tabelle usate dal supporto a runtime per effettuare ad esempio il type checking e in generale tutte le entità che hanno un indirizzo di memoria assoluto che viene mantenuto per tutta l'esecuzione del programma.

Nello stack, come abbiamo visto diverse volte, abbiamo un record di attivazione per ogni istanza di un sottoprogramma con le informazioni relative a quella determinata istanza.

Lo \strong{heap} invece è la regione di memoria in cui possiamo allocare e deallocare blocchi di memoria in momenti arbitrari. Esso è necessario quando il linguaggio permette \begin{itemize}
    \item allocazione esplicita di memoria a run-time (come con il costrutto \InlineJava{new} o con la primitiva \mintinline{C}{malloc} del C);
    \item oggetti di dimensioni variabili;
    \item oggetti con vita non esprimibile da una struttura dati LIFO come una pila.
\end{itemize}
Tuttavia una corretta gestione dello heap non è banale: vediamo innanzitutto delle possibili implementazioni.

\paragraph{Implementazione con blocchi di dimensione fissa}
La prima implementazione dello heap può essere la seguente: scegliamo un'area di memoria da usare come heap e suddividiamola in blocchi di dimensione fissa (abbastanza limitata). Questi blocchi vengono usati per formare una lista (detta \emph{lista libera}) contenente tutti i blocchi che non sono stati allocati dal programma. 

Inizialmente la lista contiene tutti i blocchi dello heap, ma ogni volta che viene allocato un oggetto tramite la \InlineJava{new} alcuni di questi blocchi vengono rimossi dalla lista libera e usati per contenere le informazioni dell'oggetto creato. Quando la memoria occupata dall'oggetto viene liberata i corrispondenti blocchi vengono restituiti alla lista libera e possono essere riusati per successive allocazioni.

Questo metodo è semplice, ma porta ad una grande frammentazione dello heap poiché la dimensione fissa dei blocchi non è necessariamente la più adatta in ogni circostanza.

\paragraph{Heap con blocchi variabili} In questa seconda implementazione inizialmente lo heap è formato da un unico blocco. Ad ogni richiesta di allocazione si cerca il blocco di dimensione opportuna usando una delle seguenti due strategie:
\begin{itemize}
    \item \strong{first fit}: si sceglie il primo blocco grande abbastanza per contenere l'oggetto da allocare;
    \item \strong{best fit}: tra i blocchi grandi a sufficienza per contenere l'oggetto da allocare si sceglie il blocco più piccolo (con il minor spreco di spazio quindi).
\end{itemize}
Se il blocco scelto è molto più grande del necessario viene diviso in due e la parte inutilizzata viene restituita alla lista libera; stessa cosa succede quando il contenuto di un blocco viene deallocato, ed in particolare se un blocco adiacente è libero i due vengono fusi in un unico blocco.

\subsection*{Garbage collection}

In alcuni linguaggi (come ad esempio il C/C++) la gestione della memoria viene lasciata interamente al programmatore: ogni blocco di memoria allocato dinamicamente va deallocato dinamicamente. Seppure questa scelta garantisca la massima flessibilità ed efficienza, un uso scorretto della memoria fa emergere diversi problemi, come ad esempio
\begin{itemize}
    \item memory leaks, che avvengono quando strutture vengono allocate e mai deallocate;
    \item dangling reference, quando dei puntatori continuano a puntare ad aree di memoria deallocate;
    \item heap fragmentation, che si ha quando la lista libera dello heap è molto frammentata a seguito di diverse allocazioni e deallocazioni.
\end{itemize}
In più, da un punto di vista più concettuale, la gestione esplicita della memoria viola il principio di astrazione: è necessario quindi che questo processo sia automatizzato.

Il \strong{Garbage Collector} è un componente della macchina virtuale che esegue un programma di un dato linguaggio: ad esempio linguaggi come Lisp, Scheme, Java, Haskell contengono un garbage collector nella loro macchina virtuale, mentre le macchine virtuali del C e C++ generalmente no.

Vediamo i principali problemi da risolvere.
\paragraph{Frammentazione} Esistono due tipi di frammentazione: la \strong{frammentazione interna} e la \strong{frammentazione esterna}. 

Si parla di \strong{frammentazione interna} quando viene allocato un blocco di dimensione maggiore di quella strettamente necessaria: una parte dello spazio allocato è sprecato.

Si parla invece di \strong{frammentazione esterna} quando vogliamo allocare un blocco di una certa dimensione $x$ e la quantità di memoria libera ce lo consente, ma la memoria libera è frammentata e ognuno dei blocchi liberi non è abbastanza grande per allocare un blocco di dimensione $x$.

\paragraph{Gestione dell'allocazione} Dobbiamo inoltre decidere quale tipo di strategia usare per allocare i blocchi: la strategia first fit è più veloce ma spreca più memoria, mentre la strategia best fit è più lenta ma più efficiente nella gestione della memoria.

\paragraph{Identificazione dei blocchi da deallocare} Un'altra scelta che dobbiamo fare è come decidere quali blocchi vanno deallocati. Nel caso di una deallocazione esplicita è semplice, in quanto basta seguire il puntatore passato come parametro alla funzione \mintinline{C}{free} (nel caso del C): tuttavia questo meccanismo può causare i problemi visti precedentemente e quindi si preferisce avere una gestione automatica.

Nel caso di deallocazione automatica diciamo che una porzione di memoria è recuperabile (nel senso che può essere deallocata) se non è più \emph{raggiungibile} dal resto del programma. Dobbiamo quindi capire come definire precisamente il concetto di raggiungibilità.

\subsection*{Tecniche di garbage collection}

Studiamo ora le principali tecniche di garbage collection.
\chapter{Sintassi e semantica}

\section{Sintassi}

Per descrivere un linguaggio di programmazione abbiamo bisogno di diversi strumenti:
\begin{itemize}
    \item una \strong{grammatica libera dal contesto} per descrivere la sintassi;
    \item un metodo per descrivere le regole di scoping e il sistema dei tipi, chiamato \strong{semantica statica};
    \item un metodo per descrivere i comportamenti del linguaggio, detto \strong{semantica dinamica}.
\end{itemize}

Per analizzare la sintassi di un linguaggio abbiamo bisogno quindi di studiarne la grammatica: tuttavia questo può essere scomodo in alcune circostanze, in quanto possono presentarsi problemi di ambiguità tra espressioni.

Si ricorre quindi all'uso dei cosiddetti \strong{Alberi di Sintassi Astratta} (abbreviati in AST, dall'inglese Abstract Syntax Tree): ogni nodo dell'albero rappresenta un'operazione e i suoi nodi figli rappresentano gli operandi dell'operazione (e possono essere a loro volta altre operazioni).

\section{Meccanismi di inferenza}

Per asserire proprietà (statiche o dinamiche) dei nostri programmi abbiamo bisogno di un meccanismo di inferenza. 

\begin{definition}
    [judgment] Si dice \strong{judgment} (o anche \strong{sentenza}) un enunciato che asserisca una proprietà di un oggetto.
\end{definition}

Ad esempio, la frase "Il numero $3$ è dispari" è un judgment.

\begin{definition}
    [Regola di inferenza] Siano $J_1, \dots, J_n$, $J$ delle sentenze. Una \strong{regola di inferenza} è un'implicazione della forma \[
        \infer{J}{
            J_1 & \dots & J_n
        },
    \] il cui significato è che se $J_1, \dots, J_n$ sono sentenze derivabili dal sistema assiomatico, allora lo è anche $J$.

    I judgment $J_1, \dots, J_n$ vengono detti \emph{premesse} o \emph{precondizioni}, mentre il judgment $J$ viene detto \emph{conclusione} o \emph{postcondizione}.

    Una regola di inferenza senza premesse si dice \emph{assioma}.
\end{definition}

Ad esempio la seguente regola è un assioma: \[
    \infer{0 : \text{nat}}{},    
\] mentre la seguente è una regola con premesse: \[
    \infer{
        s(n) : \text{nat}
    }{
        n : \text{nat}
    },
\] dove con $s(n)$ intendiamo il successore di $n$.

La strategia che useremo per dimostrare sentenze della forma $s : A$ (il cui significato è "l'oggetto "s" soddisfa la proprietà $A$") è la seguente:
\begin{itemize}
    \item troviamo una regola $R$ la cui conclusione corrisponde alla sentenza $s : A$;
    \item dimostriamo tutte le precondizioni con la stessa strategia.
\end{itemize}

\section{Semantica statica}

La semantica statica è il meccanismo che ci permette di descrivere proprietà di un programma che si manifestano senza doverlo eseguire. Queste proprietà sono spesso controllate dal compilatore o da strumenti esterni e ci permettono di avere un controllo statico sui programmi che vogliamo eseguire.

Per mostrare l'uso della semantica statica studiamo un semplice linguaggio di programmazione che può solo fare calcoli. La grammatica di questo linguaggio è la seguente:
\begin{BNF}
E ::= Const | Ide | (Times E1 E2) | (Plus E1 E2) | (Let Ide E1 E2)
Const ::= 0 | 1 | 2 | ...
Ide ::= x | y | z | ...
\end{BNF}

Consideriamo il seguente judgment: $E : \text{ok}$ se tutti gli identificatori contenuti in $E$ sono definiti correttamente tramite un espressione di tipo Let.

Le regole per le costanti e per le operazioni di addizione e moltiplicazione sono molto semplici:
\begin{align*}
    \infer{\mt{Const : ok}}{} 
    & \qquad
    \infer{
        \mt{(Times E1 E2) : ok}
    }{
        \mt{E1 : ok} & \mt{E2 : ok}
    } 
    & \qquad
    \infer{
        \mt{(Plus E1 E2) : ok}
    }{
        \mt{E1 : ok} & \mt{E2 : ok}
    }
\end{align*}
Tuttavia non possiamo al momento descrivere una regola per verificare la correttezza delle singole variabili e delle espressioni \mt{Let}: data un'espressione del tipo \mt{Ide} non abbiamo abbastanza informazione per decidere se essa è corretta oppure no, poiché potrebbe essere sia libera nell'espressione generale (e in tal caso sarebbe sbagliata), sia legata da qualche \mt{Let} precedente. 

Abbiamo quindi bisogno di introdurre una struttura di supporto per recuperare le informazioni relative agli identificatori. Chiameremo questa struttura \strong{tabella dei simboli}, ed essa conterrà tutti i nomi delle variabili che abbiamo dichiarato nel programma. Inoltre se $\Gamma$ è una tabella dei simboli, il judgment $\Gamma \proves \mt{e : A}$ signfica che considerando l'\emph{ambiente} $\Gamma$ l'espressione \mt{e} ha la proprietà \mt{A}.

Possiamo quindi esprimere completamente la condizione di correttezza di un programma del nostro linguaggio:
\begin{gather*}
    \infer{\Gamma \proves \mt{Const : ok}}{} 
    \\[4pt]
    \infer{
        \Gamma \proves \mt{(Times E1 E2) : ok}
    }{
        \Gamma \proves \mt{E1 : ok} & 
        \Gamma \proves \mt{E2 : ok}
    } 
    \\[4pt]
    \infer{
        \Gamma \proves \mt{(Plus E1 E2) : ok}
    }{
        \Gamma \proves \mt{E1 : ok} & 
        \Gamma \proves \mt{E2 : ok}
    } \\[4pt]
    \infer{
        \Gamma \proves \mt{x : ok}
    }{
        \mt{x} \in \Gamma
    }
    \\[4pt]
    \infer{
        \Gamma \proves \mt{(Let x E1 E2)}
    }{
        \Gamma \proves \mt{E1 : ok} & \Gamma \union \set{\mt{x}} \proves \mt{E2 : ok}
    }
\end{gather*}

\subsection*{Simulazione in OCaml}

Cerchiamo ora di simulare le regole dell'analisi statica in OCaml. Introduciamo innanzitutto le definizioni di tipo per le espressioni e gli identificatori:
\begin{OCaml}
    type ide = string;;
    type expr = 
        | Int   of int
        | Den   of ide * int
        | Plus  of expr * expr
        | Times of expr * expr
        | Let   of ide * expr * expr;;
\end{OCaml}

Dato un ambiente e un identificatore, controlliamo se l'identificatore esiste nell'ambiente:
\begin{OCaml}
    let rec lookup st x =
        match st with
        | []    -> false
        | y::ys -> if x = y then true
                            else lookup ys x;;
\end{OCaml}

La funzione che controlla se l'espressione è ben formata è quindi la seguente:
\begin{OCaml}
    let rec check (e : expr) (st : string list) =
        match e with
        | Int n -> true
        | Den id -> lookup st x
        | Plus (e1, e2) -> (check e1 st) && (check e2 st)
        | Times (e1, e2) -> (check e1 st) && (check e2 st)
        | Let (id, e1, e2) -> (check e1 st) && (check e2 (x::st))
\end{OCaml}

\section{Semantica dinamica}

Vogliamo ora studiare la \strong{semantica dinamica}, che ci dà le regole di esecuzione di un programma e ci permette di calcolarne il risultato. Vi sono diversi stili di semantica (in particolare \emph{denotazionale}, \emph{operazionale} e \emph{assiomatica}), ma noi ci concentreremo sulla semantica operazionale ed in particolare su due tipi particolari:
\begin{itemize}
    \item la \strong{Structural Operational Semantics}, anche chiamata \emph{semantica Small Steps}, che descrive ogni passo dell'esecuzione di un programma;
    \item la \strong{Natural Semantics}, che descrive il risultato dell'esecuzione di un programma completo.
\end{itemize}

\subsubsection{Semantica Small Steps}

L'idea di base della semantica small steps è che ogni passo di valutazione ci porta da un'espressione ad un'espressione più semplice, fino ad arrivare ad un'espressione che non può più essere semplificata (un valore).

Si dice quindi \strong{sistema di transizione} una tupla $(S, I, F, \sstep)$ dove:
\begin{itemize}
    \item $S$ è l'insieme dei possibili stati della macchina astratta del linguaggio;
    \item $I$ è l'insieme degli stati iniziali;
    \item $F$ è l'insieme degli stati finali;
    \item $\sstep {}\subseteq S \times S$ è la \emph{relazione di transizione}, che descrive l'effetto di un singolo passo di valutazione.
\end{itemize}

Nel caso del nostro semplice linguaggio possiamo definire:
\begin{itemize}
    \item $S$ come l'insieme delle espressioni aritmetiche sintatticamente corrette (in qualche ambiente), ovvero \[
        S \deq \set*{\mt{e} \given \exists\Gamma \text{ tale che } \Gamma \proves \mt{e : ok}};    
    \]
    \item $I$ come l'insieme delle espressioni aritmetiche "chiuse", ovvero che non contengono variabili libere: \[
        I \deq \set*{\mt{e} \given \varnothing \proves \mt{(e : ok)}};    
    \]
    \item $F$ come l'insieme dei valori interi: \[
        F \deq \set*{\mt{Int n} \given n \in \N}.
    \]
\end{itemize}

Iniziamo a studiare le regole di valutazione.
\paragraph{Times} Facciamo l'esempio della regola del costrutto \mt{Times}:
\begin{gather*}
    \infer{
        \mt{(Times (Int n) (Int m))} \sstep \mt{Int (n * m)}
    }{}\\
    \infer{
        \mt{(Times e1 e2)} \sstep \mt{(Times e1' e2)}
    }{
        e1 \sstep e1'
    }\\
    \infer{
        \mt{(Times (Int n) e2)} \sstep \mt{(Times (Int n) e2')}
    }{
        e2 \sstep e2'
    }
\end{gather*}

Questa regola composta ci dice che \begin{itemize}
    \item se stiamo cercando di calcolare il risultato di \mt{Times} applicato a due valori, il risultato è un valore (in particolare è \mt{Int (n * m)});
    \item se entrambi gli operandi non sono valori, eseguiamo uno step sul primo operando (che quindi può diventare un valore, ma può anche dover essere semplificato ancora) e rivalutiamo l'espressione;
    \item se il primo operando è un valore ma il secondo non lo è, eseguiamo uno step sul secondo operando e rivalutiamo l'espressione.
\end{itemize}
Facciamo alcune osservazioni:
\begin{itemize}
    \item questa regola di moltiplicazione è \strong{eager}: prima valuta completamente entrambi gli operandi, poi valuta l'espressione completa;
    \item inoltre anche l'ordine in cui vengono eseguite le valutazioni è evidente: prima viene eseguita la valutazione del primo operando, poi viene eseguita la valutazione del secondo.
\end{itemize}

\paragraph{Let} Studiamo ora la regola di valutazione del \mt{Let}:
\begin{gather*}
    \infer{
        \mt{(Let x (Int n) e2)} \sstep \mt{e2[x} \deq \mt{(Int n)]}
    }{}\\
    \infer{
        \mt{(Let x e1 e2)} \sstep \mt{(Let x e1' e2)}
    }{
        e1 \sstep e1'
    }\\
\end{gather*}

Questa regola ci dice che \begin{itemize}
    \item se abbiamo valutato completamente \mt{e1} al valore \mt{(Int n)} allora lo step consiste nel restituire l'espressione \mt{e2} dove tutte le occorrenze di \mt{x} vengono sostituite da occorrenze di \mt{(Int n)};
    \item altrimenti eseguiamo uno step di valutazione sull'espressione \mt{e1} e rivalutiamo l'espressione.
\end{itemize}

\subsubsection{Semantica naturale}

Nel caso della semantica naturale non siamo più interessati a semplificare un'espressione \emph{un passo alla volta}, ma vogliamo valutare in un passo solo un'intera espressione. Per far ciò avremmo bisogno di
\begin{itemize}
    \item un insieme $E$ di espressioni valutabili,
    \item un insieme $V$ di valori (che non è necessariamente un sottoinsieme di $E$),
    \item una relazione $\evaluatesTo {}\subseteq E \times V$ (anche indicata con $\Downarrow$), detta \emph{relazione di valutazione}.
\end{itemize}

Ad esempio nel caso del nostro semplice linguaggio possiamo dare una semantica naturale nel seguente modo:
\begin{itemize}
    \item $E$ è l'insieme di tutte le espressioni ben formate, ovvero \[
        E = \set*{\mt{e} \given \Gamma \proves \mt{(e : ok)}};
    \]
    \item $V$ è l'insieme dei valori interi,
    \item alcune regole di valutazione sono le seguenti: \begin{gather*}
        \infer{
            \mt{Int n} \evaluatesTo \mt{Int n}
        }{}\\[5pt]
        \infer{
            \mt{Times e1 e2} \evaluatesTo \mt{Int (n*m)}
        }{
            \mt{e1} \evaluatesTo \mt{Int n} & \mt{e2} \evaluatesTo \mt{Int m}
        }\\[5pt]
        \infer{
            \mt{(Let x e1 e2)} \evaluatesTo \mt{Int m}
        }{\mt{e1} \evaluatesTo \mt{(Int n)} & \mt{e2[x := (Int n)]} \evaluatesTo \mt{(Int m)}
        }
    \end{gather*}
\end{itemize}

Osserviamo che la regola data sopra per il \mt{Let} è una regola di valutazione \emph{eager}, nel senso che valuta tutti gli operandi prima di passare alla valutazione dell'operazione principale. Una regola di valutazione \emph{lazy} per il \mt{Let} è ad esempio la seguente: \[
    \infer{\mt{Let x e1 e2} \evaluatesTo \mt{(Int m)}}{\mt{e2[x := e1]} \evaluatesTo \mt{(Int m)}}.    
\]

Tuttavia per definire correttamente un interprete abbiamo bisogno di un metodo per recuperare a tempo di esecuzione le informazioni relative alle variabili legate dai \mt{Let} (e successivamente dalle funzioni): definiamo quindi il concetto di \emph{ambiente} e di \emph{binding}.

Si dice \strong{binding} un'associazione tra un nome (di variabile) e un valore; si dice \strong{ambiente} una funzione \[
    \env : \mt{Ide} \to \mt{Values} {}\union{} \set*{\mt{Unbound}}
\] dove \mt{Ide} è l'insieme degli identificatori di variabili, \mt{Values} è l'insieme dei valori e \mt{Unbound} è il valore particolare assunto da una variabile che non è legata ad alcun valore. Per aggiungere un nuovo binding ad un ambiente useremo la notazione $\env\mt{[|$x := v$|]}$, che indica la funzione che si comporta esattamente come $\env$ per ogni $y \in \mt{Ide} \setminus \set{\mt{x}}$, mentre $\env{\mt{x}} = \mt{v}$.

Una possibile prima implementazione di questo tipo di ambiente in OCaml è la seguente:
\begin{OCaml}
    let empty_env = [];;

    let rec lookup env x = 
        match env with
        | []         -> failwith ("not found")
        | (a, v)::as -> if x = a then v
                                 else lookup as x;;
    
    let bind x v env = (x, v)::env;;
\end{OCaml}

Possiamo quindi specificare meglio la nostra costruzione della semantica naturale: abbiamo bisogno di \begin{itemize}
    \item un insieme di stati \[
        S = \set*{\rho \envTo e \given \rho \in \mt{Env},\ e \in \mt{Exp},\ \varnothing \proves \mt{(|$e$| : ok)}}.
    \] Ogni stato rappresenta una coppia formata da un ambiente e da un'espressione da valutare in quell'ambiente.
    \item un insieme di valori, che nel nostro caso sono i numeri naturali,
    \item una relazione di valutazione $\evaluatesTo \subseteq S \times V$.
\end{itemize}

Le regole di valutazione dell'interprete diventano quindi ad esempio:
\begin{gather*}
    \infer{\env \envTo \mt{(Den id)} \evaluatesTo v}{\env{\mt{id}} = v \in V}\\[5pt]
    \infer{
        \env \envTo \mt{(Times |$e_1$| |$e_2$|)} \evaluatesTo v
    }{
        \env \envTo e_1 \evaluatesTo v_1 &
        \env \envTo e_2 \evaluatesTo v_2 &
        v_1 \cdot v_2 = v
    }\\[5pt]
    \infer{
        \env \envTo \mt{(let |$x = e$| in body)} \evaluatesTo v
    }{
        \env \envTo e \evaluatesTo v' &
        \env\mt{[|$x := v'$|]} \envTo \mt{body} \evaluatesTo v 
    }
\end{gather*}

Una possibile ottimizzazione di questo processo viene data dall'uso degli \strong{indici di De Bruijn}: a tempo di compilazione sostituiamo il nome di ogni variabile con il numero di \mt{Let} che bisogna attraversare per raggiungere il \mt{Let} in cui è stata definita tale variabile. Quindi ad esempio l'espressione \[
    \mt{Let ("x", (Int 5), (Let ("z", (Int 17), (Times (Den "z", Den "x"))))}    
\] può essere trasformata nell'espressione \[
    \mt{Let ((Int 5), (Let ((Int 17), (Times (Den 0, Den 1)))))}    
\]


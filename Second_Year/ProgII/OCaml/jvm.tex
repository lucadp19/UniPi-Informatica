\section{Java Virtual Machine}

Studiamo ora i meccanismi di funzionamento del Java ed in particolare la Java Virtual Machine (JVM).

Un'applicazione Java viene eseguita da un programma chiamato \strong{Java Run-Time}, contenente la JVM e la Java Class Library (JCL), che contiene tutte le librerie del Java.

La Java Virtual Machine è la macchina astratta del Java: essa descrive il \emph{class file format}, ovvero il formato e i vincoli (strutturali e sintattici) che devono essere rispettati da tutti i file \code{.class} a prescindere dalla macchina su cui vengono compilati.

I file prodotti dal compilatore \code{.javac} non sono quindi specifici per un solo tipo di macchina, ma sono universali: è la JVM che si occupa di compilarli nel linguaggio \strong{bytecode}, che è il linguaggio macchina usato dalla JVM. In particolare quindi la JVM è formata da \begin{itemize}
    \item il \emph{loader}, usato per caricare i file;
    \item il \emph{verifier}, usato per controllare che i file \code{.class} caricati rispettino le specifiche e non siano quindi malevoli/malformati;
    \item in \emph{linker}, che collega le librerie e i vari file tra loro, restituendo il bytecode;
    \item il \emph{bytecode interpreter}, che esegue il programma.
\end{itemize}

Perché il doppio livello di compilazione?
\begin{itemize}
    \item Innanzitutto perché la specifica della JVM non prescrive come deve essere implementata, quindi in questo modo possono essere implementate forme diverse della macchina virtuale.
    \item Inoltre le macchine hardware non possono avere a che fare con alberi di sintassi astratta, ma operano direttamente sui registri: astraendo sulla macchina hardware possiamo generare un linguaggio vicino alla macchina che però opera su strutture superiori all'architettura.
\end{itemize}

In particolare il bytecode Java sfrutta uno stack invece dei registri: le operazioni sono fatte sullo stack e lo modificano in modo incrementale, nel senso che ogni operazione viene eseguita sugli elementi che si trovano in testa allo stack e il risultato viene messo in testa allo stack. Lo stack degli operandi viene utilizzato per \begin{itemize}
    \item trasmettere i parametri ai metodi;
    \item restituire il risultato di un metodo;
    \item memorizzare i risultati intermedi delle operazioni;
    \item memorizzare le variabili locali.
\end{itemize}

La JVM può quindi essere descritta come una macchina astratta, stack-based e multi-threaded. In particolare ogni thread della JVM contiene \begin{itemize}
    \item un \strong{program counter}, contenente l'indirizzo dell'istruzione corrente del metodo in esecuzione;
    \item uno stack chiamato \strong{ambiente}, contenente i record di attivazione (chiamati \strong{frame}) dei metodi. Ogni frame contiene \begin{itemize}
        \item un array con le variabili locali (\strong{local variable array});
        \item uno spazio per il valore di ritorno;
        \item lo stack degli operandi;
        \item un riferimento alle informazioni di runtime, contenute nella \strong{constant pool}.
    \end{itemize}
\end{itemize} In particolare il local variable array contiene tutte le variabili locali del metodo e tutti i parametri, incluso \InlineJava{this} nel caso di metodi non statici. Nel caso di metodi non statici quindi la posizione $0$ è riservata al puntatore \InlineJava{this}, mentre nel caso di metodi statici le variabili locali e i parametri partono dall'indice $0$.

Ogni frame inoltre contiene un puntatore ad una struttura particolare, denominata \strong{constant pool}: essa contiene il descrittore di tipo associato alla classe dove è definito il metodo in esecuzione, e viene usato per fare il linking dinamico. Infatti quando una classe Java viene compilata, tutti i riferimenti a variabili e metodi nella classe sono memorizzati nella constant pool come riferimenti simbolici (nel senso che non indicano effettivamente una locazione fisica di memoria, ma solo logica).

I riferimenti simbolici così introdotti possono essere risolti con le seguenti strategie:
\begin{itemize}
    \item possono essere risolti quando le classi vengono caricate, con una strategia \emph{eager};
    \item possono essere risolti la prima volta che durante l'esecuzione si incontra il riferimento, con una strategia chiamata \emph{lazy resolution}; in particolare se il riferimento è per una classe che ancora non è stata caricata, il programma si occupa di caricarla con una strategia lazy (\strong{lazy class loading});
    \item ogni riferimento risolto diventa un offset rispetto alla struttura di memorizzazione a runtime.
\end{itemize}

Lo spazio dedicato alle classi nella JVM viene chiamato \strong{Class Area}. Per ogni classe essa contiene:
\begin{itemize}
    \item riferimento al classloader;
    \item la constant pool;
    \item i dati relativi alle variabili di istanza;
    \item i dati relativi ai metodi;
    \item il codice dei metodi.
\end{itemize}
Il bytecode generato dal compilatore \code{javac} viene memorizzato in un file \code{.class} contenente il bytecode dei metodi della classe e la constant pool della classe: quest'ultima viene usata nel class loading per risolvere i riferimenti simbolici nominati precedentemente.

Gli offset di accesso ai metodi non possono invece essere determinati staticamente: essi vengono determinati dinamicamente la prima volta che si trova un riferimento all'oggetto, mentre per eventuali accessi successivi si utilizza l'offset già calcolato. 
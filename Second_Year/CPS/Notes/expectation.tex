\section{Valore atteso e momenti}

Nel primo capitolo abbiamo introdotto il concetto di media empirica: data una collezione di dati $\vec x = (x_1, \dots, x_n)$ possiamo calcolarne il valore medio con la formula \[
    \frac1n \sum_{i = 1}^n x_i. 
\] Potremmo generalizzare questa formula come la somma pesata dei valori assunti da una variabile aleatoria $X$, ognuno moltiplicato per il valore della funzione di massa in quel punto: \[
    \sum_{x_i} x_i p(x_i). 
\] Tuttavia nel caso di variabili che prendono infiniti valori dobbiamo prima assicurarci che la serie converga. Diamo quindi la seguente definizione.

\begin{definition}
    [Valore atteso] Sia $X$ una variabile aleatoria.
    \paragraph{Caso discreto} Se $X$ è discreta e vale che \[
        \sum_{x_i} \abs{x_i}p(x_i) < +\infty
    \] si dice che $X$ ha \emph{valore atteso} (oppure \emph{speranza matematica}, oppure ancora \emph{momento primo}) e il valore atteso di $X$ è \[
        \Expect{X} = \sum_{x_i} x_ip(x_i).    
    \]
    \paragraph{Caso discreto} Se $X$ è con densità e vale che \[
        \int_\R \abs{x}f_X(x)dx < +\infty
    \] si dice che $X$ ha \emph{valore atteso} e il valore atteso di $X$ è \[
        \Expect{X} = \int_\R xf_X(x)dx.    
    \]
\end{definition}

\begin{remark}
    Notiamo che nel caso in cui una variabile aleatoria prenda solo valori non-negativi $\Expect{X}$ è sempre definito, anche se potrebbe essere uguale a $+\infty$.
    Inoltre, siccome la funzione di massa di una variabile aleatoria è sempre non-negativa, allora \[
        \Expect[\big]{\abs{X}} = \sum_{x_i} \abs*{x_ip(x_i)} = \sum_{x_i} \abs*{x_i}p(x_i).   
    \] Segue quindi che $X$ ha momento primo se e solo se $\Expect[\big]{\abs{X}} < +\infty$. Un risultato analogo ovviamente vale per le variabili con densità.
\end{remark}

\begin{proposition}
    [Valore atteso di una trasformazione]
    Sia $X$ una variabile aleatoria e sia $h$ una trasformazione.
    \paragraph{Caso discreto} Se $X$ è discreta allora $h \circ X$ ammette valore atteso se e solo se \[
        \sum_{x_i} \abs*{h(x_i)}p(x_i) < +\infty.    
    \] Il valore atteso di $h \circ X$ è quindi \[
        \Expect{h \circ X} = \sum_{x_i} h(x_i)p(x_i).    
    \]
    \paragraph{Caso con densità} Se $X$ è con densità allora $h \circ X$ ammette valore atteso se e solo se \[
        \int_{\R} \abs*{h(x)}f_X(x)dx < +\infty.    
    \] Il valore atteso di $h \circ X$ è quindi \[
        \Expect{h \circ X} = \int_{\R} h(x)f_X(x)dx.
    \]
\end{proposition}

\begin{example}
    Consideriamo $X$ uniforme in $\interval[{0, 1}]$ e calcoliamo $\Expect*{X^2}$. Usando la proposizione precedente otteniamo che \[
        \Expect*{X^2} = \int_0^1 x^2dx = \EvaluateAt*{\frac{x^3}{3}}_0^1 = \frac13.
    \]

    Tuttavia potremmo anche considerare la variabile $Y = X^2$, calcolarne la densità e infine il valore atteso: verifichiamo che il valore ottenuto è lo stesso.
    Siccome la funzione $h(x) = x^2$ è invertibile su $\interval[{0, 1}]$, è derivabile e anche la sua inversa è derivabile, vale la \autoref{prop:trasf_rand_var}, per cui \[
        f_Y(y) = \begin{cases}
            0, &y \leq 0\\
            f_X(\sqrt{y}) \abs*{\frac{1}{2\sqrt{y}}} = \frac{1}{2\sqrt{y}}, &y \in \interval*({0, 1}]\\
            0, &y > 1.
        \end{cases}
    \] Segue quindi che \[
        \Expect*{Y} = \int_0^1 y\frac{1}{2\sqrt y}dy = \int_0^1 \frac12\sqrt{y}dy = \EvaluateAt*{\frac12 \cdot \frac23 y^{\frac32}}_0^1 = \frac13,
    \] che è lo stesso risultato ottenuto precedentemente.
\end{example}

\begin{proposition}
    [Proprietà del valore atteso]
    \label{prop:prop_expect}
    Siano $X$, $Y$ variabili aleatorie. Valgono i seguenti fatti.
    \begin{enumerate}[label={(\roman*)}]
        \item Se $X$ e $Y$ ammettono momento primo, allora anche $X + Y$ ammette momento primo e vale che \[
            \Expect*{X+ Y} = \Expect*{X} + \Expect*{Y}.    
        \]
        \item Se $X$ ammette momento primo, allora per ogni $a, b \in \R$ la variabile $Y \deq aX + b$ ammette momento primo e vale che \[
            \Expect*{aX + b} = a\Expect*{X} + b.    
        \]
        \item Se $X \geq 0$ allora $\Expect*{X} \geq 0$. In particolare da ciò segue che se $X \geq Y$ allora $\Expect*{X} \geq \Expect*{Y}$.
        \item Se $X$ e $Y$ ammettono momento primo e sono indipendenti, allora $XY$ ammette momento primo e \[
            \Expect*{XY} = \Expect*{X}\Expect*{Y}.    
        \]
    \end{enumerate}
\end{proposition}

\begin{definition}
    [Momento di ordine $n$] Sia $X$ una variabile aleatoria. Si dice che $X$ \emph{ammette momento $n$-esimo} (oppure \emph{di ordine n}) se vale che $\Expect[\big]{\abs*{X^n}}$, e in questo caso il momento $n$-esimo è il numero $\Expect[\big]{X^n}$.
\end{definition}

La seguente proposizione mostra che se una variabile aleatoria ha momento $n$-esimo, allora ammette tutti i momenti di ordine inferiore.

\begin{proposition}
    Sia $X$ una variabile aleatoria tale che $\Expect[\big]{\abs*{X^n}} < +\infty$ (ovvero $X$ ammette momento $n$-esimo). Allora per ogni $m = 1, \dots, n$ vale che \[
        \Expect[\big]{\abs*{X^m}} < +\infty,    
    \] ovvero $X$ ammette momento $m$-esimo.
\end{proposition}
\begin{proof}
    Sia $1 \leq m \leq n$ qualsiasi. Osserviamo che per qualunque $t \in \R$ vale che \[
        \abs*{t}^m \leq \abs*{t}^n + 1.
    \] Infatti se $\abs{t} \in \interval*[{1, +\infty})$ vale che $\abs*{t}^m \leq \abs*{t}^n$; invece se $\abs{t} \in \interval*[{0, 1}]$ vale che $\abs*{t}^m \leq 1$.

    Da questa disuguaglianza segue che (supponendo $X$ discreta) \begin{align*}
        \sum_{x_i} \abs*{x_i}^m p(x_i)
        &\leq \sum_{x_i} \parens[\big]{\abs*{x_i}^n + 1}p(x_i)\\
        &= \sum_{x_i} \abs*{x_i}^np(x_i) + \sum_{x_i} p(x_i)\\
        &= \Expect[\big]{\abs*{X}^n} + 1\\
        &< +\infty. \qedhere
    \end{align*}
\end{proof}

\subsection{Disuguaglianze relative ai momenti}

\begin{proposition}
    [Disuguaglianza di Markov] 
    \label{prop:dis_Markov}
    Sia $Y$ una variabile aleatoria a valori positivi e $a > 0$. Allora vale la disuguaglianza \[
        a\Prob\set[\big]{Y \geq a} \leq \Expect*{Y}.    
    \]
\end{proposition}

\begin{proposition}
    [Disuguaglianza di Schwartz] 
    \label{prop:dis_Schwartz}
    Siano $X, Y$ due variabili aleatorie. Allora \[
        \Expect[\big]{\abs{XY}} \leq \sqrt{\Expect*{X^2}}\sqrt{\Expect*{Y^2}}
    \]
\end{proposition}

Introduciamo ora il concetto di \emph{varianza} per variabili aleatorie.

\begin{definition}
    [Varianza e scarto quadratico medio di una v.a.] Sia $X$ una variabile aleatoria. Si definisce la \emph{varianza di $X$} come \[
        \Var{X} = \Expect[\big]{\parens*{X-\Expect*{X}}^2}.    
    \] Si dice \emph{scarto quadratico medio} la quantità \[
        \sd{X} = \sqrt{\Var{X}}.     
    \]
\end{definition}

\begin{proposition}
    Sia $X$ una variabile aleatoria. Vale che \[
        \Var{X} = \Expect[\big]{X^2} - \Expect{X}^2.
    \]
\end{proposition}
\begin{proof}
    È sufficiente sviluppare i calcoli: \begin{align*}
        \Var{X} &= \Expect[\big]{\parens*{X-\Expect*{X}}^2}\\
        &= \Expect[\big]{X^2 - 2X\Expect*{X} + \Expect*{X}^2}.
        \intertext{Siccome $\Expect*{X}$ è una costante vale allora}
        &= \Expect[\big]{X^2} - 2\Expect*{X}\Expect*{X} + \Expect*{X}^2\\
        &= \Expect[\big]{X^2} - \Expect*{X}^2. \qedhere
    \end{align*}
\end{proof}

Notiamo inoltre che la varianza di una trasformazione affine è data da \[
    \Var{aX + b} = a^2\Var{X}.    
\]

\begin{proposition}
    [Disuguaglianza di Chebyshev]
    \label{prop:dis_Chebyshev}
    Sia $X$ una variabile aleatoria. Allora per ogni $d > 0$ si ha \[
        \Prob\set[\big]{\abs*{X - \Expect*{X}} > d} \leq \frac{\Var{X}}{d^2}.
    \]
\end{proposition}
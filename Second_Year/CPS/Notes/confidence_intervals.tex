\section{Intervalli di fiducia}

Consideriamo un campione statistico la cui distribuzione dipende da un parametro $\theta \in \Theta$, dove $\Theta$ è un sottoinsieme di $\R$.
\begin{definition}
    [Intervallo di fiducia]
    Siano $\alpha \in \interval*({0, 1})$, $I \subseteq \Theta$ un intervallo. Si dice che $I$ è un \strong{intervallo di fiducia} per $\theta$ al livello $1 - \alpha$ se per ogni $\theta$ vale che \[
        \Prob_\theta[\big]{\theta \in I} \geq 1 - \alpha.    
    \]
\end{definition}

Osseriamo che talvolta può essere più comodo verificare che vale la proprietà complementare, ovvero \[
    \Prob_\theta[\big]{\theta \notin I} \leq \alpha.    
\]

Studieremo gli intervalli di fiducia solo in casi particolari e non in generale.

\subsection*{Intervallo di fiducia per la media di un campione gaussiano con varianza nota}

Supponiamo di avere un campione $X_1, \dots, X_n$ di variabili aleatorie gaussiane $\NormalDist*{m, \sigma^2}$ con varianza $\sigma^2$ nota: vogliamo trovare un intervallo di fiducia per la media $m$. 
Siccome $\bar X$ è la stima corretta della media, è naturale scegliere come intervallo un sottoinsieme di $\R$ della forma \[
    I \deq \interval*[{\bar X(\omega) - d, \bar X(\omega) + d}][\Big],    
\] dove $d$ è un'incognita da determinare. 

\begin{remark}
    Con $\bar X(\omega)$ indichiamo il valore assunto dalla variabile aleatoria che rappresenta la media campionaria nell'esperimento compiuto: essa equivale quindi alla media dei dati $\mean{x} = \nicefrac{1}{n}\sum_{i=1}^n x_i$.
\end{remark}

Osseriamo che 
\begin{align*}
    &m \in I\\
    \iff {}& m \in \interval[{\bar X - d, \bar X + d}][\Big]\\
    \iff {}& \bar X - d \leq m \leq \bar X + d\\
    \iff {}& -d \leq m - \bar X \leq d \\
    \iff {}& \abs*{\bar X - m} \leq d.
\end{align*}

Imponiamo ora che $I$ sia un intervallo di fiducia al livello $1-\alpha$: deve valere che \[
    \Prob\set*{m \in I} = \Prob_m\set[\Big]{\abs*{\bar X - m} \leq d} \geq 1 - \alpha    
\] con $d$ più piccolo possibile (poiché vogliamo avere un intervallo di fiducia molto preciso). Questo significa imporre che \[
    \Prob_m\set[\Big]{\abs*{\bar X - m} \leq d} \approx 1 - \alpha.
\] 
Osserviamo ora che per il \autoref{th:campione_gaussiano} la variabile aleatoria $\bar X$ è ha densità $N\parens*{m, \nicefrac{\sigma^2}{n}}$.
Sottraendo a $\bar X$ la sua media e dividendo per la radice della varianza otteniamo quindi la variabile $\nicefrac{\sqrt{n}}{\sigma} \parens*{\bar X - d}$, che è una gaussiana standard. Siccome conosciamo i quantili della variabile gaussiana standard possiamo svolgere il seguente calcolo: \begin{align*}
    \Prob_m\set[\Big]{\abs*{\bar X - m} \leq d} &= \Prob_m\set*{\frac{\sqrt{n}}{\sigma}\abs*{\bar X - m} \leq \frac{\sqrt{n}}{\sigma}d}\\
    \intertext{Chiamando $Z$ la variabile gaussiana appena definita}
    &= \Prob_m\set*{\abs*{Z} \leq \frac{\sqrt{n}}{\sigma}d}\\
    &= \Prob_m\set*{-\frac{\sqrt{n}}{\sigma}d \leq Z \leq \frac{\sqrt{n}}{\sigma}d}\\[5pt]
    &= \Phi\parens*{\frac{\sqrt{n}}{\sigma}d} - \Phi\parens*{-\frac{\sqrt{n}}{\sigma}d}\\[5pt]
    &= 2\Phi\parens*{\frac{\sqrt{n}}{\sigma}d} - 1.
\end{align*} Allora porre la probabilità uguale ad $1 - \alpha$ equivale a dire che \begin{align*}
    &2\Phi\parens*{\frac{\sqrt{n}}{\sigma}d} - 1 = 1 - \alpha\\
    \iff {}&\Phi\parens*{\frac{\sqrt{n}}{\sigma}d} = 1 - \frac{\alpha}{2}\\
    \iff {}&\frac{d\sqrt{n}}{\sigma} = q_{1 - \nicefrac{\alpha}{2}}
\end{align*} per definizione di quantile della variabile gaussiana.

L'intervallo di fiducia assume quindi la forma \[
    \interval*[{\bar X  - \frac{\sigma}{\sqrt{n}}q_{1-\nicefrac{\alpha}{2}}, \bar X  + \frac{\sigma}{\sqrt{n}}q_{1-\nicefrac{\alpha}{2}}}].
\] Il numero $\dfrac{\sigma}{\sqrt{n}}q_{1-\nicefrac{\alpha}{2}}$ viene detto \emph{precisione della stima} e il numero \[
    \frac{\frac{\sigma}{\sqrt{n}}q_{1-\nicefrac{\alpha}{2}}}{\bar X}    
\] viene detto \emph{precisione relativa}.

\subsection*{Intervallo di fiducia per la media di un campione gaussiano con varianza sconosciuta}

In questo caso si sostituisce $\sigma^2$ (che è sconosciuto) con la sua stima corretta, ovvero $S^2$. Tuttavia in questo caso la variabile \[
    \sqrt{n}\frac{\bar X - m}{S}    
\] non è più gaussiana, ma è una variabile di Student con densità $T(n-1)$ per il \autoref{th:campione_gaussiano}. L'intervallo di fiducia risulterà quindi della forma \[
    \interval*[{\bar X - \frac{S}{\sqrt{n}}\tau_{\parens*{1 - \nicefrac{\alpha}{2},\ n - 1}}, \bar X + \frac{S}{\sqrt{n}}\tau_{\parens*{1 - \nicefrac{\alpha}{2},\ n - 1}}}].    
\] Quando $n$ è sufficientemente grande ($n \geq 60$) possiamo approssimare il quantile della variabile di Student $\tau$ con il quantile della variabile gaussiana.

Il quantile più usato frequentemente è $0.05$.

\subsection{Intervalli unilateri e bilateri}

Finora gli intervalli studiati erano \emph{bilateri}: a volte può essere utile considerare intervalli unilateri che saranno della forma \[
    \interval*({-\infty, \bar X + \frac{\sigma}{\sqrt{n}}q_{1-\alpha}}], \qquad
    \interval*[{\bar X - \frac{\sigma}{\sqrt{n}}q_\alpha, +\infty}).
\] I calcoli vengono seguendo direttamente lo stesso procedimento della parte precedente. Inoltre, nel caso di varianza sconosciuta basta sostituire $\sigma$ con la varianza campionaria $S$ e i quantili della variabile gaussiana con i quantili della variabile di Student $T(n-1)$: gli intervalli di fiducia unilateri saranno quindi
\[
    \interval*({-\infty, \bar X + \frac{S}{\sqrt{n}}\tau_{\parens*{1-\alpha,\ n-1}}}], \qquad
    \interval*[{\bar X - \frac{S}{\sqrt{n}}\tau_{\parens*{\alpha,\ n-1}}, +\infty}).
\]

\subsection*{Intervallo di fiducia per la varianza}

Consideriamo ancora una volta un campione di $n$ variabili aleatorie $X_1, \dots, X_n \sim \NormalDist*{m, \sigma^2}$: questa volta vogliamo trovare un intervallo di fiducia per la varianza $\sigma^2$. La media $m$ è sconosciuta: noteremo infatti che non ha alcuna rilevanza nel corso della costruzione dell'intervallo.

Ricordiamo che per il \autoref{th:campione_gaussiano} la variabile \[
    \frac{1}{\sigma^2} \sum_{i=1}^n \parens[\big]{X_i - \bar X}^2 = (n-1)\frac{S^2}{\sigma^2} 
\] ha densità $\chi^2(n-1)$.

Siccome costruire intervalli bilateri è complicato e poco utile, ci concentreremo direttamente su intervalli unilateri ed in particolare sull'unilatero sinistro: la costruzione dell'intervallo destro è equivalente.

Innanzitutto l'intervallo deve essere della forma \[
    I = \interval*({0, (n-1)\frac{S^2}{a}}],
\] dove la costante $a$ determina l'ampiezza dell'intervallo. Osserviamo inoltre che la condizione $\sigma^2 \in I$ è equivalente a $\sigma^2 \leq (n-1)\frac{S^2}{a}$, in quanto la varianza sicuramente è positiva. Allora la condizione per il livello diventa:
\begin{align*}
    &\Prob\set*{\sigma^2 \in I} \\
    = {}&\Prob_{\sigma^2}\set[\Big]{\sigma^2 \leq (n-1)\frac{S^2}{a}} \\ 
    = {}&\Prob_{\sigma^2}\set[\Big]{a \leq (n-1)\frac{S^2}{\sigma^2}} \geq 1 - \alpha. \\
\end{align*}
Per avere l'intervallo più piccolo possibile poniamo la probabilità \emph{esattamente} uguale a $1 - \alpha$, ottenendo che \begin{align*}
    &\Prob_{\sigma^2}\set[\Big]{a \leq (n-1)\frac{S^2}{\sigma^2}} = 1 - \alpha \\
    \iff {}&1 - \Prob_{\sigma^2}\set[\Big]{(n-1)\frac{S^2}{\sigma^2} \leq a} = 1 - \alpha \\
    \iff {}&\Prob_{\sigma^2}\set[\Big]{(n-1)\frac{S^2}{\sigma^2} \leq a} = \alpha \\
    \iff {}&F_{\chi^2(n-1)}(a) = \alpha \\
    \iff {}&a = \chi^2_{\parens*{\alpha, n-1}}, \\
\end{align*} dove $\chi^2_{\parens*{\alpha, n-1}}$ indica l'$\alpha$-quantile della variabile $\chi^2(n-1)$.

L'intervallo di fiducia unilatero sinistro è quindi della forma \[
    \interval*({0, (n-1)\frac{S^2}{\chi^2_{\parens*{\alpha, n-1}}}}].
\] L'intervallo destro si ricava analogamente al sinistro, ed è della forma \[
    \interval*[{(n-1)\frac{S^2}{\chi^2_{\parens*{\alpha, n-1}}}, +\infty}).
\]
\chapter{Prerequisiti}

\section{Serie}

\begin{definition}
    [Successioni delle somme parziali] Sia $\seqn{a_n}_{n \in \N}$ una successione. Si dice \emph{successione delle somme parziali} la successione $\seqn{s_n}_{n \in \N}$ tale che \[
        s_n \deq a_0 + a_1 + \dots + a_n = \sum_{i=0}^n a_i.    
    \]
\end{definition}

\begin{definition}
    [Serie] Sia $\seqn{a_n}_{n \in \N}$ una successione e $\seqn{s_n}_{n \in \N}$ la sua successione delle somme parziali. Allora si dice serie il limite per $n \to +\infty$ delle somme parziali, e lo si indica con \[
        \sum_{i=0}^\infty a_n \deq \lim_{n\to +\infty} s_n.    
    \]
\end{definition}

Spesso si usa anche la notazione $\sum a_n$ quando l'indice di partenza è sottointeso oppure non è importante nel calcolo della serie.

\begin{definition}
    [Comportamento di una serie] Sia $\seqn{a_n}$ una successione. Allora si dice che \begin{enumerate}[label={(\arabic*)}]
        \item $\sum a_n$ converge se $\displaystyle\lim_{n \to +\infty} s_n = l$ per qualche $l \in \R$;
        \item $\sum a_n$ diverge positivamente se $\displaystyle\lim_{n \to +\infty} s_n = +\infty$;
        \item $\sum a_n$ diverge negativamente se $\displaystyle\lim_{n \to +\infty} s_n = -\infty$;
        \item $\sum a_n$ è indeterminata se $\displaystyle\lim_{n \to +\infty} s_n$ non esiste.
    \end{enumerate}
\end{definition}

\begin{proposition}
    [Condizione necessaria per la convergenza]
    Sia $\seqn{a_n}$ una successione. Allora se $\sum a_n$ converge segue che \[
        \lim_{n\to +\infty} a_n = 0. 
    \]
\end{proposition}
\begin{proof}
    Per definizione della successione delle somme parziali \begin{gather*}
        s_n \deq a_0 + \dots + a_{n-1} + a_n\\
        s_{n-1} \deq a_0 + \dots + a_{n-1}
    \end{gather*} dunque \[
        a_n = s_n - s_{n-1}.    
    \] Supponiamo che $\sum a_n$ converga al valore reale $l$: il limite della successione $\seqn{a_n}$ sarà quindi \begin{align*}
           \lim_{n\to +\infty} a_n
        &= \lim_{n\to +\infty} s_n - s_{n-1} \\
        &= l - l\\
        &= 0.
    \end{align*}
\end{proof}

\begin{proposition}
    \label{prop:serie_succ_cresc_nonneg}
    Sia $\seqn{a_n}$ una successione crescente e a termini non negativi (ovvero $a_n \geq 0$ per ogni $n \in \N$): allora la serie $\sum a_n$ è convergente oppure divergente positivamente. 
\end{proposition}
\begin{proof}
    La successione delle somme parziali è debolmente crescente (ad ogni passo aggiungiamo un numero positivo o nullo), dunque non può divergere negativamente o essere indeterminata.
\end{proof}

\begin{definition}
    [Serie assolutamente convergente] Sia $\seqn{a_n}$ una successione. Se $\sum \abs{a_n}$ converge, allora la serie $\sum a_n$ si dice assolutamente convergente.
\end{definition}

\begin{remark}
    Siccome $\abs{a_n}$ è una successione a termini positivi o nulli, la serie $\sum \abs*{a_n}$ (in virtù della \autoref{prop:serie_succ_cresc_nonneg}) può soltanto convergere o divergere positivamente.
\end{remark}

\begin{proposition}
    [Proprietà delle serie assolutamente convergenti]
    Sia $\seqn{a_n}$ una successione la cui serie converge assolutamente. Allora valgono le seguenti affermazioni: \begin{enumerate}[label={(\roman*)}]
        \item la serie $\sum a_n$ converge;
        \item se cambio l'ordine dei termini della successione, la serie converge allo stesso valore della serie relativa alla successione originale;
        \item data una partizione di $\N$ della forma $A_1, A_2, \dots$ vale che \[
            \sum_{n=1}^\infty a_n = \sum_{n=1}^\infty \parens*{\sum_{k \in A_n} a_k}.
        \]
    \end{enumerate}
\end{proposition}

\subsubsection{Serie geometrica}

Dato $a \in \R$ tale che $\abs*{a}< 1$ si dice \emph{serie geometrica} la serie \[
    \sum_{n=0}^\infty a^n = 1 + a + a^2 + \dots    
\]

\begin{proposition}
    Sia $a \in \R$, $\abs*{a} < 1$. Allora \begin{equation}\label{eq:geometric_series}
        \sum_{n = 0}^\infty a^n = \frac{1}{1-a}.
    \end{equation}
\end{proposition}
\begin{proof}
    Mostriamo innanzitutto per induzione su $n$ che \[
        s_n = \frac{a^{n+1} - 1}{a - 1}.    
    \]
    \begin{description}
        \item[Caso base] Se $n = 0$ allora $s_0 = a^0 = 1$.
        \item[Passo induttivo] Supponiamo che la formula valga per $n-1$ e dimostriamo che vale per $n$.
        \begin{align*}
            s_{n} &= s_{n - 1} + a^{n}\\
            &= \frac{a^n - 1}{a - 1} + a^n\\
            &= \frac{a^n - 1 + a^n(a - 1)}{a - 1}\\
            &= \frac{a^n - 1 + a^{n+1} - a^n}{a-1}\\
            &= \frac{a^{n+1} - 1}{a-1}.
        \end{align*}  
    \end{description}
    Dunque la formula vale per ogni $n \in \N$: quando $n$ tende a $+\infty$ allora avremo che \[
        \sum a^n = \lim_{n \to +\infty} s_n = \lim_{n \to +\infty} \frac{a^{n+1} - 1}{a-1} = \frac{-1}{a-1} = \frac{1}{1-a},   
    \] dove abbiamo usato il fatto che $a^n \to 0$ se $\abs{a} < 1$.
\end{proof}

\subsubsection{Serie esponenziale}
Vale la seguente formula per l'esponenziale: per ogni $x \in \R$ \begin{equation}
    \label{eq:exp_series}
    e^x = \sum_{n=0}^\infty \frac{x^n}{n!}.
\end{equation}
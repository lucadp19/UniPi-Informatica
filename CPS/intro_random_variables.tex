\section{Variabili aleatorie}

Per introdurre il concetto di variabili aleatorie e la loro utilità useremo il seguente esempio:
\begin{example}
    Supponiamo di giocare alla roulotte e di aver puntato 1£ sul numero $28$ e 1£ sull'uscita di un numero pari. Possiamo ad esempio domandarci quanto sia la probabilità di vincere più di 10£, oppure quanto sia la probabilità di perdere soldi.

    Lo spazio più naturale per questo problema è l'insieme dei possibili risultati della roulotte, ovvero $\Omega = \set{0, 1, \dots, 36}$ munito della distribuzione uniforme di probabilità; tuttavia gli eventi che vogliamo considerare ("vincere più di 10£", "perdere soldi") non sono sottoinsiemi di $\Omega$, bensì di $\R$.
    
    Possiamo quindi definire una funzione $X : \Omega \to \R$ che indichi nel seguente modo la "vittoria netta": \[
        X(\omega) = \begin{cases}
            36, &\text{se } \omega = 28\\
            0, &\text{se } \omega \neq 28, \omega \text{ è pari}\\
            -1, &\text{se } \omega = 0\\
            -2, &\text{se } \omega \text{ è dispari}.\\
        \end{cases}  
    \]
    Usando la funzione $X$ possiamo \emph{trasportare} la probabilità da $\Omega$ a $\R$: la risposta alla prima domanda è dunque \[
        \Prob[\big]{\set{\omega \in \Omega \given X(\omega) > 10}} 
        = \Prob*{X^{-1}\parens[\big]{\interval[{0, +\infty})}} 
        = \frac{1}{37};    
    \] la risposta alla seconda domanda è invece \[
        \Prob[\big]{\set{\omega \in \Omega \given X(\omega) < 0}} 
        = \Prob*{X^{-1}\parens[\big]{\interval({-\infty, 0})}} 
        = \frac{19}{37}.    
    \]
\end{example}

\begin{definition}
    [Variabili aleatorie] Sia $(\Omega, \FF, \P)$ uno spazio di probabilità. Una funzione $X : \Omega \to \R$ si dice \emph{variabile aleatoria} e la funzione \[
        \Prob_{X}{A} = \Prob{X\inv(A)}  
    \] viene detta \emph{legge di probabilità} associata alla variabile aleatoria $X$.
\end{definition}

Dato $A \subseteq \R$, introduciamo la notazione $\set{X \in A}$ per indicare l'insieme di tutti i valori dello spazio fondamentale $\Omega$ la cui immagine cade in $A$, ovvero \[
    \set{X \in A} \deq \set{\omega \in \Omega \given X(\omega) \in A} = X\inv(A).    
\] 

\begin{definition}
    [Variabili equidistribuite] Siano $X, Y : \Omega \to \R$. $X$ e $Y$ si dicono equidistribuite se hanno la stessa legge di probabilità.
\end{definition}

\begin{definition}
    [Variabili discrete e continue] Sia $X : \Omega \to \R$ una variabile aleatoria. Allora $X$ si dice: \begin{enumerate}
        \item \emph{discreta} se la sua distribuzione è discreta, ovvero se l'immagine di $X$ è un sottoinsieme finito o numerabile di $\R$;
        \item \emph{con densità} se la sua distribuzione è con densità, ovvero se esiste una densità $f : \R \to \interval[{0, +\infty})$ tale che \[
            \Prob_{X}{A} = \Prob\set[\big]{X \in A} = \int_A f(x)dx.    
        \]
    \end{enumerate}
\end{definition}

Questi due tipi non sono tutti i tipi possibili di variabili aleatorie, tuttavia ci occuperemo solo di questi due casi (anche perché sono i casi più rilevanti nelle applicazioni).

\begin{definition}
    [Funzione di ripartizione]
    Data una variabile aleatoria $X : \Omega \to \R$ si dice \emph{funzione di ripartizione} di $X$ (o anche c.d.f., da \emph{cumulative distribution function}) la funzione $F_{X} : \R \to \interval[{0,1}]$ tale che \[
        F_{X}(x) = \Prob\set[\big]{X \leq x} = \Prob\set[\big]{X \in \interval({-\infty, x}]}.   
    \]
\end{definition}

Notiamo che la funzione di ripartizione ha due comportamenti diversi a seconda del tipo di variabile aleatoria che stiamo considerando: \begin{description}
    \item[Discreta] La funzione di ripartizione è data da \[
        F_X(x) = \sum_{x_i \leq x} p(x_i).    
    \]
    \item[Con densità] La funzione di ripartizione è data da \[
        F_X(x) = \int_{-\infty}^x f(t)dt,   
    \] dove $f$ è la densità associata alla variabile $X$.
    Notiamo che siccome $f$ è integrabile, allora la corrispondente $F_X$ è necessariamente continua.

    Non vale invece il viceversa: è possibile trovare una funzione $F_X$ continua la cui variabile aleatoria associata $X$ non è con densità.
\end{description}

La funzione di ripartizione ha delle interessanti proprietà che racchiuderemo nella prossima proposizione.
\begin{proposition}
    [Proprietà della funzione di ripartizione]
    \label{prop:proprietà_cdf}
    Sia $X : \Omega \to \R$ una variabile aleatoria e sia $F_X$ la sua funzione di ripartizione. Valgono le seguenti affermazioni:
    \begin{itemize}
        \item $F_X$ è crescente.
        \item $\displaystyle\lim_{x \to -\infty} F_X(x) = 0$, $\displaystyle\lim_{x\to +\infty} F_X(x) = 1.$
        \item $F_X$ è sempre continua a destra; ovvero per ogni $x_0 \in \R$ vale che \[
            \lim_{x \to x_0^+} F_X(x) = F_X(x_0).    
        \]
    \end{itemize}
\end{proposition}
\begin{proof}
    Dimostriamo ad esempio che $\displaystyle\lim_{x \to -\infty} F_X(x) = 0$.

    Sia $\seqn{x_n}$ una successione decrescente che diverge negativamente (ovvero $x_n \to -\infty$).
    Sia $A_n$ l'insieme definito da \[
        A_n \deq \set{X \leq x_n} = X\inv\parens[\big]{\interval({-\infty, x_n}]}.
    \] Siccome $x_n$ è decrescente segue che $A_n \supseteq A_{n-1}$; dunque (sfruttando il fatto che $x_n \to -\infty$) \[
        \biginters_{n = 1}^\infty X\inv\parens[\big]{\interval({-\infty, x_n}]}
        =  X\inv\parens*{ \biginters_{n = 1}^\infty A_n }
        = X\inv\parens*{\varnothing}
        = \varnothing.
    \] Vale quindi che \[
        \lim_{x \to -\infty} F_X(x) 
        \begin{alignedat}[t]{1}
            &= \lim_{n \to +\infty} F_X(x_n)
            = \lim_{n \to +\infty} \Prob{A_n}\\
            &= \Prob*{\lim_{n\to +\infty} A_n}
            = \Prob*{\varnothing}
            = 0.    
        \end{alignedat} \tag*{\qedhere}
    \]
\end{proof}

Un fatto importante è che la proposizione precedente può essere in qualche modo "invertita": data una funzione $F : \R \to \interval[{0, 1}]$ che rispetta le tre proprietà della \autoref{prop:proprietà_cdf} esiste una e una sola probabilità $\Prob$ sui sottoinsiemi di $\R$ tale che $F$ sia la funzione di ripartizione di una variabile aleatoria con legge di probabilità $\Prob$.

Da ciò segue che tutte le variabili aleatorie che hanno la stessa funzione di ripartizione hanno a maggior ragione la stessa legge di probabilità, ovvero sono \emph{equidistribuite}. Nel caso generale è difficile ricavare la legge di probabilità dalla c.d.f., ma se ci limitiamo al caso delle variabili discrete e con densità è fattibile. Possiamo notrare infatti che in generale vale che \[
    F_X(x_0) - \lim_{x \to x_0^-} F_X(x) = \Prob\set[\big]{X = x_0}.    
\]

Nel caso discreto questo implica che $\Prob\set[\big]{X = x_i} = p(x_i)$: in ogni punto di salto l'\emph{ampiezza del salto} è data proprio dal valore della funzione di massa nel punto.

Nel caso con densità possiamo ricavare la legge di probabilità invertendo l'operazione di integrazione: in ogni punto in cui la densità $f$ è continua vale che \[
    f(x) = \frac{dF_X(x)}{dx}.    
\]

Vediamo ora la definizione di quantile:
\begin{definition}
    [Quantile]
    Sia $X$ una variabile aleatoria con legge di probabilità $\Prob{}$. Sia inoltre $\beta \in \interval({0, 1})$. Si dice \emph{$\beta$-quantile} un numero $r_\beta$ tale che \begin{align*}
        \Prob\set[\big]{X \leq r_\beta} \geq \beta,\quad &\Prob\set[\big]{X \geq r_\beta} \geq 1 - \beta.   
    \end{align*}
\end{definition}

Nel caso in cui la variabile sia discreta ci sono due possibilità:
\begin{enumerate}
    \item La funzione non assume mai il valore $\beta$ poiché $\beta$ è compreso tra gli estremi di un punto di salto.
    In questo caso l'unico $r_\beta$ possibile è il punto di salto, il cui valore è il minimo valore assunto dalla funzione superiore a $\beta$.

    \item La funzione assume il valore $\beta$ su tutto un intervallo. In questo caso per convenzione si sceglie $r_\beta$ come il punto medio dell'intervallo, anche se tutti i punti dell'intervallo andrebbero ugualmente bene.
\end{enumerate}
\begin{figure}[H]
    \begin{center}
        \includestandalone{Figures/quantile_discr}
    \end{center}
\end{figure}

Se invece la variabile $X$ ha densità, \emph{solitamente} esiste uno e un solo $\beta$-quantile, ed è il valore $r_\beta \in \R$ tale che $F_X(r_\beta) = \beta$.

\begin{example}
    Sia $X$ una variabile aleatoria con densità esponenziale, ovvero \[
        f(x) = \begin{cases}
            0, &\text{se } x \leq 0\\
            e^{-x}, &\text{se } x > 0.
        \end{cases}    
    \] La sua funzione di ripartizione è \[
        F_X(x) = \begin{cases}
            0, &\text{se } x \leq 0\\
            \int_0^x e^{-t}dt = 1 - e^{-x}, &\text{se } x > 0.
        \end{cases}    
    \] Il $\beta$-quantile di $X$ è quindi quel numero $r_\beta$ tale che \begin{align*}
        F_X(r_\beta) = 1 - e^{-r_\beta} = \beta \iff r_\beta = -\log\parens*{1 - \beta}.
    \end{align*} 
\end{example}

\section{Tipi più comuni di variabili aleatorie}
Definiamo ora i tipi più comuni di variabili aleatorie.

\subsection{Variabile binomiale}
La situazione concreta da cui nasce questa variabile aleatoria è ad esempio la seguente: vogliamo ripetere $n$ volte, in condizioni di indipendenza, un esperimento che può avere come risultato o il successo (con probabilità $p_0$), o l'insuccesso (con probabilità $1 - p_0$).
La variabile $X$ deve quindi contare il numero di successi in $n$ tentativi.

È chiaro quindi che i possibili valori che la variabile $X$ può assumere sono $0, 1, \dots, n$: negli $n$ tentativi posso avere $0$, $1$, $\dots$, oppure $n$ successi.

Calcoliamo ora la funzione di massa $p$: dato un qualsiasi $k \in \set{0, \dots, n}$ vogliamo calcolare la probabilità che la variabile aleatoria valga esattamente $k$, ovvero che su $n$ tentativi $k$ di essi siano successi.

La funzione di massa è la seguente: \[
    p(k) = \Prob\set[\big]{X = k} = \binom{n}{k}p_0^k(1-p_0)^{n-k}.
\] 
\begin{proof}
    Consideriamo una stringa di successi/insuccessi di lunghezza $n$. Esistono esattamente $\binom{n}{k}$ stringhe con $k$ successi; inoltre la probabilità ad essa associata è $p_0^k(1-p_0)^{n-k}$, in quanto ogni successo ha probabilità $p_0$ e ogni insuccesso ha probabilità $1-p_0$, da cui la tesi.
\end{proof}

Per esser certo che $p$ sia effettivamente una funzione di massa verifico che \[
    \sum_{k = 0}^n p(k) = 1.
\]
\begin{proof}
    Per definizione di $p$:
    \[
        \sum_{k = 0}^n p(k) = \sum_{k = 0}^n \binom{n}{k}p_0^k(1-p_0)^{n-k} = \parens[\Big]{p_0 + (1 - p_0)}^n = 1,
    \] dove il terzo passaggio è giustificato dal binomio di Newton.
\end{proof}

Se $n = 1$ la variabile $X$ viene detta \emph{di Bernoulli} di parametro $p_0$.

\subsection{Variabile geometrica}
Vogliamo ripetere, in condizioni di indipendenza, un esperimento che ha come risultati il successo (con probabilità $p_0$) oppure l'insuccesso (con probabilità $1 - p_0$) finché l'esperimento non ha successo. 

Come nel caso della variabile binomiale, la variabile aleatoria $X$ conta il numero di tentativi necessari: in questo caso però i valori possibili sono tutti i possibili valori naturali positivi insieme a $\set{+\infty}$, in quanto a priori il successo potrebbe non arrivare mai.

Calcoliamo ora la funzione di massa $p$: dato un qualsiasi $k \in \set{1, 2, \dots, +\infty}$ vogliamo calcolare la probabilità che la variabile aleatoria valga esattamente $k$, ovvero che i primi $k-1$ tentativi siano insuccessi e il $K$-esimo sia un successo. Nel caso $k = +\infty$, questo significa che tutti i tentativi sono insuccessi.

La funzione di massa legata a questa variabile è la seguente: \[
    p(k) = \Prob\set[\big]{X = k} = (1-p_0)^{k-1} \cdot p_0.
\]
\begin{proof}
    Siccome la probabilità di ogni insuccesso è di $1 - p_0$ e abbiamo $k-1$ insuccessi, la probabilità di ogni successo è $p_0$ e ne abbiamo uno solo, e gli eventi sono tutti indipendenti segue la tesi.
\end{proof}

Verifichiamo che $\sum p(k) = 1$:
\begin{align*}
    \sum_{k=1}^{+\infty} p(k) 
    &= \sum_{k=1}^{+\infty} \Prob{\set{X = k}}\\
    &= \sum_{k=1}^{+\infty} (1 - p_0)^{k-1} \cdot p\\
    &= p \cdot \sum_{k=1}^{+\infty} (1 - p_0)^{k-1}\\
    \intertext{Per ricondurmi alla serie geometrica pongo $h \deq k -1$, ottenendo}
    &= p \cdot \sum_{h=0}^{+\infty} (1 - p_0)^{h}\\
    &= p \cdot \frac{1}{1 - (1-p)}\\
    &= 1.
\end{align*}

La variabile geometrica ha un'altra interessante proprietà, detta \emph{proprietà dell'assenza di memoria}.
\begin{proposition}
    [Assenza di memoria della variabile geometrica]
    Siano $n, h \in \N$, $n, h > 0$ e sia $X$ una variabile aleatoria geometrica. Allora vale che \[
        \Prob[\Big]{\set[\big]{X = n + h} \given \set[\big]{X > n}} = \Prob\set[\big]{X = h}.    
    \]
\end{proposition}
\begin{proof}
    Per definizione di probabilità condizionata \begin{equation}
        \label{eq:tmp_abs_mem}
        \Prob[\Big]{\set[\big]{X = n + h} \given \set[\big]{X > n}}
        = \frac{\Prob\set[\big]{X = n + h}}{\Prob\set[\big]{X > n}}.
    \end{equation}
    Calcoliamo il denominatore di questa espressione: \begin{align*}
        \Prob\set[\big]{X > n}
        &= \sum_{h = n + 1}^\infty p(h)\\
        &= \sum_{h = n + 1}^\infty (1-p_0)^{h-1}\cdot p_0\\
        \intertext{Ponendo $k \deq h - n$, ovvero $h = k + n$:}
        &= \sum_{k = 1}^\infty (1-p_0)^{k+n-1} \cdot p_0\\
        &= (1 - p_0)^n \cdot \sum_{k = 1}^\infty (1-p_0)^{k-1}\cdot p_0\\
        &= (1 - p_0)^n,
    \end{align*}
    dove l'ultimo passaggio è giustificato dai calcoli fatti nel verificare la correttezza della funzione di massa della variabile binomiale.

    Sostituendo nella \eqref{eq:tmp_abs_mem}: \begin{align*}
        \frac{\Prob\set[\big]{X = n + h}}{\Prob\set[\big]{X > n}}
        &= \frac{(1+p_0)^{n+h-1} \cdot p_0}{(1-p_0)^n}\\
        &= (1-p_0)^{h-1}\cdot p_0\\
        &= \Prob\set[\big]{X = h}. \qedhere
    \end{align*}
\end{proof}

\subsection{Variabile di Poisson}
Si dice \emph{variabile di Poisson di parametro $\lambda$} (con $\lambda \in \R$, $\lambda > 0$) la variabile aleatoria con dominio $\N$ definita dalla seguente funzione di massa: \[
    p(h) = \Prob[\big]{\set{X = h}} \deq e^{-\lambda}\frac{\lambda^h}{h!}.    
\]
Verifichiamo che $\sum p(h) = 1$: \begin{align*}
    \sum_{h = 0}^\infty p(h)
    &= \sum_{h = 0}^\infty e^{-\lambda}\frac{\lambda^h}{h!}\\
    &= e^{-\lambda} \sum_{h = 0}^\infty \frac{\lambda^h}{h!}\\
    &= e^{-\lambda}e^{\lambda}\\
    &= 1.
\end{align*}

\subsection{Densità esponenziale}
Si definisce \emph{densità esponenziale di parametro $\lambda$} la funzione $f : \R \to \interval[{0, +\infty})$ tale che \[
    f(x) \deq \begin{cases}
        \lambda e^{-\lambda x}, &x \geq 0\\
        0, &x < 0.
    \end{cases} 
\]

Questa funzione è effettivamente una densità: infatti \begin{align*}
    \int_{-\infty}^{+\infty} f(x)dx
    &= \int_{-\infty}^0 0dx + \int_0^{+\infty} \lambda e^{-\lambda x}dx\\
    \intertext{Sostituendo $t \deq \lambda x$ (da cui $dt = \lambda dx$) l'integrale è equivalente a}
    &= \int_0^{+\infty} e^{-t}dt\\
    &= 1.
\end{align*}

La funzione di ripartizione corrispondente a questa densità è data da \begin{equation}
    \label{eq:cdf_exponential}
    F_X(x) = \int_{-\infty}^x f(t)dt = \begin{cases}
        0, &x < 0 \\[5pt]
        \displaystyle\int_0^x \lambda e^{-\lambda t}dt = 1 - e^{-\lambda x}, &x \geq 0.
    \end{cases}    
\end{equation}

La variabile esponenziale è \emph{senza memoria}, esattamente come la variabile binomiale, ovvero per ogni $s, t > 0$ vale che \[
    \Prob[\big]{\set{X \leq s + t} \given \set{X > s}} = \Prob[\big]{\set{X \leq t}}.    
\]
\begin{proof}
    Per definizione di probabilità condizionata:
    \begin{align*}
        \Prob[\big]{\set{X \geq s + t} \given \set{X > s}}
        &= \frac{\Prob[\big]{\set{X \leq s + t} \inters \set{X > s}}}{\Prob[\big]{\set{X > s}}}\\
        &= \frac{\Prob[\big]{\set{s < X \leq s + t}}}{\Prob[\big]{\set{X > s}}}.
    \end{align*}

    Facciamo alcune osservazioni generali:
    \begin{alignat*}{2}
        &(1):\; &&\Prob[\big]{\set{X > s}} = 1 - \Prob[\big]{\set{X \leq s}} = 1 - F_X(s).\\
        &(2):\; &&\begin{alignedat}[t]{1}
                \Prob[\big]{\set{s < X \leq s + t}} 
                &= \Prob[\big]{\set{X \leq s + t}} - \Prob[\big]{\set{X \leq s}} \\
                &= F_X(s+t) - F_X(s).
            \end{alignedat}
    \end{alignat*}

    Sostituendo nel nostro caso si ha \begin{align*}
        \frac{\Prob[\big]{\set{s < X \leq s + t}}}{\Prob[\big]{\set{X > s}}}
        &= \frac{F_X(s+t) - F_X(s)}{1 - F_X(s)}\\[3pt]
        &= \frac{1 - e^{-\lambda (s+t)} - (1 - e^{-\lambda s})}{1 - (1 - e^{-\lambda s})}\\[3pt]
        &= \frac{-e^{-\lambda s}e^{-\lambda t} + e^{-\lambda s}}{e^{-\lambda s}}\\[3pt]
        &= \frac{e^{-\lambda s}\parens*{1 - e^{-\lambda t}}}{e^{-\lambda s}}\\[3pt]
        &= 1 - e^{-\lambda t}\\[3pt]
        &= \Prob[\big]{X \leq t}. \qedhere
    \end{align*}
\end{proof}

\subsection{Variabile gaussiana}

Consideriamo la funzione $f(x) = e^{-\frac{x^2}{2}}$: anche se non ha una primitiva (esprimibile tramite funzioni elementari), si può calcolare il valore del suo integrale sulla retta reale, e si ha che \[
    \int_\R f(x) = \int_{-\infty}^{+\infty} e^{-\frac{x^2}{2}} = \sqrt{2\pi}.    
\] La conseguenza immediata di questo fatto è che dividendo la funzione per $\sqrt{2\pi}$ si ottiene una densità.

\begin{definition}
    [Densità Gaussiana standard] Si dice \emph{densità gaussiana standard} (o anche \emph{densità normale standard}) la funzione \[
        \phi(x) = \frac{1}{\sqrt{2\pi}}e^{\frac{x^2}{2}}.    
    \] La funzione di ripartizione ad essa associata è \[
        \Phi(x) = \frac{1}{\sqrt{2\pi}}\int_{\infty}^x e^{\frac{t^2}{2}}dt.
    \] La densità gaussiana standard si indica spesso con $N(0, 1)$; inoltre indicheremo con $q_\alpha$ lo $\alpha$-quantile della variabile $N(0, 1)$.
\end{definition}

\begin{remark}
    La densità $\phi$ è una funzione pari: infatti \[
        \phi(-x) = \frac{1}{\sqrt{2\pi}}e^{\frac{(-x)^2}{2}} =  \frac{1}{\sqrt{2\pi}}e^{\frac{x^2}{2}} = \phi(x).
    \] Da questo segue immediatamente che $\Phi(0) = \nicefrac{1}{2}$.
\end{remark}

\begin{proposition}
    Vale la seguente proprietà per la funzione di ripartizione della $N(0, 1)$: \[
        \Phi(-x) = 1 - \Phi(x).    
    \]
\end{proposition}
\begin{proof} Siccome $\Phi$ è pari:
    \begin{align*}
        \Phi(-x) &= \int_{-\infty}^{-x} \phi(t)dt \\
        &= \int_x^{+\infty} \phi(t)dt \\
        &= \int_x^{+\infty} \phi(t)dt + \int_{\infty}^{x} \phi(t)dt - \int_{\infty}^{x} \phi(t)dt\\
        &= \int_{-\infty}^{+\infty} \phi(t)dt - \int_{\infty}^{x} \phi(t)dt\\
        &= 1 - \Phi(x). \qedhere
    \end{align*}
\end{proof}
\begin{proposition}
    Vale la seguente proprietà per l'$\alpha$-quantile $q_\alpha$ della funzione di ripartizione della variabile gaussiana: \[
        q_{-\alpha} = 1 - q_{\alpha}.
    \]
\end{proposition}

La variabile $N(0, 1)$ può essere generalizzata (tramite una trasformazione) ad una variabile di parametri $m$ e $\sigma^2$, detta variabile $N(m, \sigma^2)$.

\begin{definition}
    [Variabile gaussiana $N(m, \sigma^2)$] Si dice variabile gaussiana $N(m, \sigma^2)$ la variabile $Y$ ottenuta a partire dalla variabile $X = N(0, 1)$ tramite la trasformazione \[
        Y = \sigma X + m.     
    \]
\end{definition}

Facciamo alcune osservazioni. \begin{itemize}
    \item La funzione di ripartizione di $Y$ è data da \[
        F_Y(y) = \Phi\parens*{\frac{y-m}{\sigma}}.    
    \] Infatti \begin{align*}
        F_Y(y) &= \Prob\set[\big]{Y \leq y} \\
        &= \Prob\set[\big]{\sigma X + m \leq y}\\
        &= \Prob\set[\Big]{X \leq \frac{y-m}{\sigma}}\\
        &= F_X\parens*{\frac{y-m}{\sigma}}\\
        &= \Phi\parens*{\frac{y-m}{\sigma}}.
    \end{align*}
    \item La densità di $Y$ è data da \[
        f_Y(y) = \frac{1}{\sigma\sqrt{2\pi}} e^{-\frac{(y-m)^2}{2\sigma^2}}.
    \] Infatti \begin{align*}
        f_Y(y) &= \frac{dF_Y(y)}{dy}\\
        &= \Phi^\prime\parens*{\frac{y-m}{\sigma}}\frac{1}{\sigma}\\
        &= \frac{1}{\sigma}\phi\parens*{\frac{y-m}{\sigma}}\\
        &= \frac{1}{\sigma\sqrt{2\pi}} e^{-\frac{(y-m)^2}{2\sigma^2}}.
    \end{align*}
\end{itemize}

Il principio fondamentale per descrivere le variabili gaussiane non standard è il seguente: se $Y$ è una variabile $N(m, \sigma)$, allora la variabile $\frac{Y-m}{\sigma}$ è standard.


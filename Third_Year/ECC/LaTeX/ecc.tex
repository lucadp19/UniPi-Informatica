\documentclass[
    oneside,
    10pt,
    language=italian,
    a4paper
]{notes}
    % My packages
    % \usepackage{boxedthm}
    \usepackage{code}
    \usepackage{minted}
    \usepackage{ecc}

\usemintedstyle{manni}
\setminted{
    % bgcolor=lightgray,
    fontsize=\small,
    breaklines=true,
    % escapeinside=||,
    mathescape=true,
    frame=single
}

\let\oldtilde\tilde
% \renewcommand{\tilde}[1]{\widetilde{#1}}
\includeonly{
    chapters/intro-calcolability.tex, 
    chapters/calc-problems.tex,
    chapters/intro-complexity.tex,
    chapters/compl-problems.tex,
    chapters/appendices.tex
}

\begin{document}

\author{Luca De Paulis}
\title{Elementi di Calcolabilità e Complessità \\
    \normalsize Appunti del corso di ECC del Professor Pierpaolo Degano, A.A. 2021/2022
}
\maketitle

\frontmatter{}
\section*{Introduzione}
Questo file contiene i miei appunti (rielaborati) del corso di \textsf{Elementi di Calcolabilità e Complessità} tenuto dal Professor Pierpaolo Degano nell'anno 2021/2022.
Dato che questi appunti sono stati scritti per me stesso, potrebbero essere pieni di errori o ragionamenti incomprensibili a tutti tranne che a me: in ogni caso ogni segnalazione è ben accetta!
\begin{flushright}
    Luca De Paulis
\end{flushright}

\newpage
\tableofcontents*

\mainmatter{}
\chapter{Introduzione alla Calcolabilità}

Nella prima parte del corso, dedicata alla \sstrong{Teoria della Calcolabilità}, cercheremo di studiare cosa significhi \emph{calcolare} qualcosa e quali siano i limiti delle \emph{procedure} a disposizione degli esseri umani per calcolare.

Per far ciò bisogna innanzitutto definire il concetto di \sstrong{algoritmo} oppure procedura: lo faremo definendo dei vincoli che ogni algoritmo deve soddisfare per esser ritenuto tale.

\begin{enumerate}
    \item Dato che gli uomini possono calcolare solo seguendo procedure finite, un algoritmo deve essere \strong{finito}, ovvero deve essere costituito da un numero finito di istruzioni.
    \item Inoltre devono esserci un numero \strong{finito} di istruzioni distinte, e ognuna deve avere un \strong{effetto limitato} su \strong{dati discreti} (nel senso di non continui).
    \item Una \strong{computazione} è quindi una sequenza finita di passi discreti con durata finita, né analogici né continui.
    \item Ogni passo dipende solo dai \strong{passi precedenti} e viene scelto in modo \strong{deterministico}: se ripetiamo due volte la stessa esatta computazione nelle stesse condizioni dobbiamo ottenere lo stesso risultato e la stessa sequenza di passi.
    \item Non imponiamo un limite al numero di passi e alla memoria a disposizione.
\end{enumerate}

Questi vincoli non definiscono precisamente cosa sia un algoritmo, anzi, vedremo che vi sono diversi modelli di computazione che soddisfano questi 5 requisiti. Le domande a cui vogliamo rispondere sono: \begin{itemize}
    \item Modelli diversi che rispettano questi vincoli risolvono gli stessi problemi?
    \item Un tale modello risolve necessariamente tutti i problemi?
\end{itemize}

\section{Macchine di Turing}

Il primo modello di computazione che vedremo è stato proposto da Alan Turing nel 1936, ed è pertanto chiamato in suo nome.

\begin{definition}{Macchina di Turing}{mdt}
    Una \sstrong{Macchina di Turing} (MdT per gli amici) è una quadrupla $(Q, \Sigma, \delta, q_0)$ dove
    \begin{itemize}
        \item $Q$ è un insieme finito, detto \strong{insieme degli stati}. In particolare assumiamo che esista uno stato $h \notin Q$, detto stato terminatore o \strong{halting state}.
        \item $\Sigma$ è un insieme finito, detto \strong{insieme dei simboli}. In particolare \begin{itemize}
            \item esiste $\# \in \Sigma$ e lo chiameremo \strong{simbolo vuoto};
            \item esiste $\resp \in \Sigma$ e lo chiameremo \strong{respingente}.
        \end{itemize}
        \item $\delta$ è una funzione \[
            \delta : Q \times \Sigma \to (Q \union \set{h}) \times \Sigma \times \set{\Left, \Right, -}
        \] detta \strong{funzione di transizione}. È soggetta al vincolo \[
            \forall q \in Q: \exists q' \in Q : \quad \delta(q, \resp) = (q', \resp, \Right).
        \]
        \item $q_0$ è un elemento di $Q$ detto \strong{stato iniziale}.
    \end{itemize}
\end{definition}

La definizione formale di Macchina di Turing può sembrare complicata, ma l'idea alla base è molto semplice: 
abbiamo una macchina che opera su un \strong{nastro illimitato} (a destra) su cui sono scritti simboli (ovvero elementi di $\Sigma$).
In ogni istante di tempo, la \emph{testa} della macchina legge una casella del nastro, contenente il \strong{simbolo corrente}. 
La macchina mantiene inoltre al suo interno uno stato (ovvero un elemento di $Q$), inizialmente settato allo stato iniziale $q_0$.

Un singolo passo di computazione è il seguente: \begin{itemize}
    \item la macchina legge il simbolo corrente $\sigma$;
    \item la macchina usa la funzione di transizione $\delta$ per effettuare la mossa: in particolare calcola $\delta(q, \sigma)$, dove $q$ è lo stato corrente, e ne ottiene una tripla $(q', \sigma', \texttt{M})$;
    \item la macchina cambia stato da $q$ a $q'$;
    \item la macchina scrive al posto di $\sigma$ il simbolo $\sigma'$;
    \item la macchina si sposta nella direzione indicata da $\texttt{M}$: se $\texttt{M} = \Left$ si sposta di un posto a sinistra, se $\texttt{M} = \Right$ si sposta di un posto a destra, se $\texttt{M} = -$ rimane ferma.           
\end{itemize}

Per formalizzare questi concetti abbiamo bisogno di altre definizioni.

\begin{definition}
    {Monoide libero, o Parole su un Alfabeto}{free_monoid}
    Dato un insieme finito $\Sigma$, il \sstrong{monoide libero} su $\Sigma$, anche chiamato \sstrong{insieme delle parole su $\Sigma$}, è l'insieme $\Sigma^\ast$ così definito: \[
        \Sigma^\ast \deq \bigunion_{n \in \N} \Sigma^n
    \] dove \begin{itemize}
        \item $\Sigma^0 \deq \set*{\eps}$, dove $\eps$ è la parola vuota;
        \item $\Sigma^{n+1} \deq \set*{\sigma \cdot w \given \sigma \in \Sigma, w \in \Sigma^n}$ è l'insieme delle parole di lunghezza $n+1$, ottenute preponendo ad una parola di lunghezza $n$ (ovvero $w \in \Sigma^n$) un simbolo $\sigma \in \Sigma$. 
    \end{itemize}
    Tale insieme ammette un'operazione, ovvero la \strong{concatenazione} di parole, e la parola vuota $\eps$ è l'identità destra e sinistra di tale operazione. 
\end{definition}

\begin{remark}
    Un elemento di $\Sigma^\ast$ è una stringa di caratteri di $\Sigma$ di lunghezza arbitraria, ma sempre finita, in quanto ogni elemento di $\Sigma^\ast$ deve essere contenuto in un qualche $\Sigma^n$.
\end{remark}

Il nastro di una MdT può quindi essere formalizzato come un elemento di $\Sigma^\ast$. Questo tuttavia ancora non ci soddisfa per alcuni motivi:
\begin{itemize}
    \item gli elementi di $\Sigma^\ast$ sono illimitati a destra, ma non a sinistra, dunque la MdT potrebbe muoversi a sinistra ripetutamente fino a "cadere fuori dal nastro";
    \item non stiamo memorizzando da alcuna parte la posizione del cursore della MdT.
\end{itemize}

Per risolvere il primo problema possiamo assumere che ogni nastro inizi con il simbolo speciale $\resp$:
per il vincolo sulla funzione di transizione ogni volta che la MdT si troverà nella casella più a sinistra (contenente $\resp$) sarà costretta a muoversi verso destra lasciando scritto il respingente.

Per quanto riguarda il secondo invece possiamo dividere il nastro infinito in tre parti: \begin{itemize}
    \item la porzione a sinistra del simbolo corrente, che è una stringa di lunghezza arbitraria che inizia per $\resp$ e quindi un elemento di $\resp\Sigma^\ast$;
    \item il simbolo corrente, che è un elemento di $\Sigma$;
    \item la porzione a destra del simbolo corrente, che è una stringa e quindi un elemento di $\Sigma^\ast$.   
\end{itemize}

Quest'ultima porzione è una stringa che potrebbe terminare con un numero infinito di caratteri vuoti ($\#$): dato che non siamo interessati (per il momento) a tenere tutti i simboli vuoti a destra dell'ultimo simbolo non-vuoto del nastro, considereremo la porzione a destra "eliminando" tutti i \emph{blank} superflui.

In particolare indicando sempre con $\eps \in \Sigma^\ast$ la stringa vuota e convenendo che \begin{itemize}
    \item $\#\eps = \eps\# = \eps$ (la concatenazione della stringa vuota con il \emph{blank} dà ancora la stringa vuota);
    \item $\sigma\eps = \eps\sigma = \sigma$ per ogni $\sigma \neq \#$ (la concatenazione della stringa vuota con un simbolo non-\emph{blank} dà il simbolo) 
\end{itemize}
possiamo considerare l'insieme \[
    \Sigma^{F} \deq \parens[\Big]{\Sigma^\ast \cdot (\Sigma \setminus \set{\#})} \union \set{\eps},
\] ovvero l'insieme delle stringhe in $\Sigma$ che finiscono con un carattere non-\emph{blank}, più la stringa vuota.

Usando queste convenzioni, la stringa che definisce il nastro è finita: siamo pronti a definire la \emph{configurazione} di una MdT in un dato istante.

\begin{definition}
    {Configurazione di una MdT}{configuration}
    Sia $M = (Q, \Sigma, \delta, q_0)$ una MdT. Una \sstrong{configurazione} è una quadrupla \[
        (q, u, \sigma, v) \in Q \times \resp\Sigma^\ast \times \Sigma \times \Sigma^F.
    \] Più nel dettaglio:
    \begin{itemize}
        \item $q$ è lo stato corrente,
        \item $u$ è la porzione del nastro che precede il simbolo corrente, ed inizia per $\resp$,
        \item $\sigma$ è il simbolo corrente,
        \item $v$ è la porzione del nastro che segue il simbolo corrente, ed è vuota oppure termina per un simbolo diverso da $\#$. 
    \end{itemize}
\end{definition}

Osserviamo che:
\begin{itemize}
    \item il simbolo corrente può essere $\#$;
    \item è possibile che il simbolo corrente sia $\#$ e $v = \eps$ (cioè vuota),
    \item è possibile che $u$ sia vuota solo nel caso in cui il simbolo corrente è $\resp$, poiché significherebbe trovarsi all'inizio del nastro.    
\end{itemize}

Spesso indicheremo la quadrupla $(q, u, \sigma, v)$ con $(q, u\ul{\sigma}v)$: la sottolineatura ci indicherà il simbolo corrente. In contesti in cui non sia necessario sapere la posizione del cursore scriveremo semplicemente $(q, w)$ per risparmiare tempo.

\begin{example}
    Ad esempio la configurazione \[
        (q_0, \resp ab\#\#b\#a\ul{\#}b\#a)
    \] indica che la MdT è nello stato $q_0$, sta leggendo il carattere $\#$, a sinistra del simbolo letto ha la stringa $\resp ab\#\#b\#a$ e a destra $b\#a$.
\end{example}
\section{Linguaggi FOR e WHILE}

Introduciamo ora un secondo paradigma per il calcolo di algoritmi, ovvero quello dato dai linguaggi \texttt{FOR} e \texttt{WHILE}. In effetti anche se le MdT rispondono ai nostri requisiti formali per un algoritmo e sono il modello teorico delle \strong{macchine di Von Neumann}, al giorno d'oggi non costruiamo una nuova macchina per ogni algoritmo che dobbiamo risolvere: usiamo dei \strong{linguaggi di programmazione} che verranno interpretati o compilati e restituiranno il risultato del calcolo.

I linguaggi \texttt{FOR} e \texttt{WHILE} sono quindi la base teorica dei moderni linguaggi imperativi, e anche se sembrano mancare di espressività rispetto ad essi, vedremo che in realtà il linguaggio \texttt{WHILE} riesce a risolvere tutti e soli i problemi risolvibili da un linguaggio moderno.

\subsection*{Sintassi astratta}

\begin{definition}
    [Sintassi astratta di \texttt{FOR} e \texttt{WHILE}]
    \begin{align}
        \code{EXPR} &\Coloneqq n \mid x \mid E_1 + E_2 \mid E_1 \cdot E_2 \mid E_1 - E_2 \tag*{Espr. aritmetiche}\\
        \code{BEXPR} &\Coloneqq b \mid E_1 < E_2 \mid \neg b \mid B_1 \lor B_2 \tag*{Espr. booleane} \\
        \code{CMD} &\Coloneqq \code{skip} \mid x \deq E \mid C_1;C_2 \mid \code{if}\ B\ \code{then}\ C_1\ \code{else}\ C_2 \tag*{Comandi} \\
        &\;\;\mid \code{for}\ x = E_1\ \code{to}\ E_2\ \code{do}\ C \mid \code{while}\ B\ \code{do}\ C \tag*{}
    \end{align}
    dove $n \in \N$, $x \in \code{Var}$ (che è un insieme \emph{numerabile} di variabili), $b \in \B \deq \set{\TT, \FF}$. 

    Il linguaggio \texttt{FOR} contiene solo il comando \code{for}, il linguaggio \texttt{WHILE} contiene solo il comando \code{while}.
\end{definition}

\subsection*{Semantica}

Per definire la \sstrong{semantica} dei linguaggi \texttt{FOR} e \texttt{WHILE} abbiamo bisogno di alcuni costrutti ausiliari. In particolare ogni nostro programma conterrà delle variabili che possono essere valutate oppure aggiornate (tramite il comando di assegnamento): dobbiamo \emph{memorizzare} il loro valore.

\begin{definition}
    [Funzione memoria e funzione di aggiornamento]
    La funzione \sstrong{memoria} è una funzione \[
        \sigma : \code{Var} \to \N
    \] definita solo per un sottoinsieme finito di \code{Var}.

    La funzione di \sstrong{aggiornamento} è una funzione \[
        \blank[\blank/\blank] : (\code{Var} \times \N) \times \N \times \code{Var} \to (\code{Var} \times \N)
    \] definita da \[
        \sigma[n/x](y) \deq \begin{cases}
            n &\text{se } y = x,\\
            \sigma(y) &\text{altrimenti.}
        \end{cases}
    \]
\end{definition}

\begin{remark}
    La funzione di aggiornamento prende una memoria ($\sigma : \code{Var} \to \N$), un valore intero ($n \in \N$) e una variabile ($x \in \code{Var}$) e produce una nuova memoria $\sigma[n/x] : \code{Var} \to \N$ che si comporta come $\sigma$ su tutte le variabili diverse da $x$, ma restituisce $n$ quando l'input è $x$. 
\end{remark}

Tramite la memoria possiamo definire la funzione di valutazione delle espressioni aritmetiche, ovvero la loro \sstrong{semantica}.

\begin{definition}
    [Funzione di valutazione semantica (aritmetica)]
    La \sstrong{funzione di valutazione semantica (aritmetica)} è una funzione \[
        \eval{\blank, \blank} : \code{EXPR} \times (\code{Var} \to \N) \to \N 
    \]
    definita per induzione strutturale a partire da 
    \begin{alignat*}{2}
        &\eval{n, \sigma} &&\;\;\deq{}\;\; n \tag{val. dei naturali}\\
        &\eval{x, \sigma} &&\;\;\deq{}\;\; \sigma(x) \tag{val. delle variabili}\\
        &\eval{E_1 + E_2, \sigma} &&\;\;\deq{}\;\; \eval{E_1, \sigma} + \eval{E_2, \sigma} \tag{val. della somma}\\
        &\eval{E_1 \cdot E_2, \sigma} &&\;\;\deq{}\;\; \eval{E_1, \sigma} \cdot \eval{E_2, \sigma}  \tag{val. del prodotto}\\
        &\eval{E_1 - E_2, \sigma} &&\;\;\deq{}\;\; \eval{E_1, \sigma} - \eval{E_2, \sigma} \tag{val. della sottrazione}
    \end{alignat*}
\end{definition}

Dato che il nostro linguaggio modella solo numeri naturali (quindi positivi), l'operazione di sottrazione sarà quella data dal \strong{meno limitato}: \[
    a - b \deq \begin{cases}
        a - b, \text{se } a > b\\
        0, \text{altrimenti.}
    \end{cases}
\]

Analogamente possiamo definire la semantica delle espressioni booleane.

\begin{definition}
    [Funzione di valutazione semantica (booleana)]
    La \sstrong{funzione di valutazione semantica (booleana)} è una funzione \[
        \bval{\blank, \blank} : \code{BEXPR} \times (\code{Var} \to \N) \to \N
    \] definita per induzione strutturale a partire da
    \begin{alignat*}{2}
        &\bval{t, \sigma} &&\;\;\deq{}\;\; \TT \tag{val. del \texttt{true}}\\
        &\bval{f, \sigma} &&\;\;\deq{}\;\; \FF \tag{val. del \texttt{false}}\\
        &\bval{E_1 < E_2, \sigma} &&\;\;\deq{}\;\; \eval{E_1, \sigma} < \eval{E_2, \sigma} \tag{val. del minore}\\
        &\bval{\neg B, \sigma} &&\;\;\deq{}\;\; \neg \bval{B, \sigma}  \tag{val. del \texttt{not}}\\
        &\bval{B_1 \lor B_2, \sigma} &&\;\;\deq{}\;\; \bval{B_1, \sigma} \lor \bval{B_2, \sigma} \tag{val. della sottrazione}
    \end{alignat*}
\end{definition}

Osserviamo che i simboli usati nel linguaggio (come $+, <, \neg$, ed altri) sono solo \strong{simboli formali}: per essere più precisi dovremmo differenziarli dalle funzioni effettive (ovvero quelle che compaiono a destra del $\deq$).

\begin{remark}
    Le funzioni $\eval$ e $\bval$ si comportano come un \sstrong{interprete}: ad esempio $\eval$ prende un'espressione aritmetica, una memoria e restituisce la valutazione dell'espressione nella memoria data.

    Tuttavia tramite il \sstrong{currying} possiamo esprimere $\eval$ come una funzione \[
        \eval{\blank, \blank} : \code{EXPR} \to \parens[\Big]{(\code{Var} \to \N) \to \N}
    \] ovvero come una funzione che prende un'espressione aritmetica e restituisce una \emph{funzione} che a sua volta prenderà una memoria per restituire finalmente la valutazione dell'espressione nella memoria.

    Anche se le due modalità in pratica ci portano allo stesso risultato, la seconda modella più l'azione di un \sstrong{compilatore}: infatti nella seconda versione $\eval$ prende un'espressione, cioè del codice, e restituisce un \emph{eseguibile} che avrà bisogno dei dati (cioè della memoria) per dare il suo risultato. 
\end{remark}

Lo stile usato per definire la semantica delle espressioni viene chiamato \sstrong{semantica denotazionale}: in questo stile cerchiamo di associare ad ogni costrutto del linguaggio una funzione che ne dà la semantica (ad esempio abbiamo associato al $+$ del linguaggio la funzione che somma due naturali).

Per quanto riguarda i comandi adopereremo un altro stile, detto \sstrong{semantica operazionale}. Come si evince dal nome, cercheremo di definire una \emph{macchina astratta} che modifica il proprio \emph{stato interno} valutando a piccoli passi il comando da eseguire.

\begin{definition}
    [Sistema di transizioni]
    Si dice \sstrong{sistema di transizioni} una coppia $(\Gamma, \to)$ dove \begin{itemize}
        \item $\Gamma$ è l'insieme delle \sstrong{configurazioni} oppure stati;
        \item $\to : \Gamma \to \Gamma$ è una funzione, detta \sstrong{funzione di transizione}. 
    \end{itemize} 
\end{definition}

Nel caso della nostra macchina astratta, le configurazioni saranno delle coppie \[
    \ang{c, \sigma} \in \code{CMD} \times (\code{Var} \to \N)
\] ovvero delle coppie "comando da valutare", "memoria".

Per definire la semantica operazionale dei comandi useremo un approccio \sstrong{small-step}, in cui ogni transizione \[
    \ang{c, \sigma} \to \ang*{c', \sigma'}
\] rappresenta un singolo passo dell'esecuzione del programma. Una \sstrong{computazione} diventa allora una sequenza di passi, ovvero un elemento della chiusura transitiva e riflessiva di $\to$, che indicheremo come al solito come $\to^\ast$.

Analogamente alle MdT, una computazione \sstrong{termina con successo} se \[
    \ang{c, \sigma} \to^\ast \sigma',
\] ovvero se esauriamo la valutazione del comando $c$ in un numero finito (anche se arbitrario) di passi.

La semantica operazionale dei comandi è dunque data attraverso una serie di assiomi e regole di inferenza, che insieme ci permettono di valutare ogni comando per induzione strutturale.

\begin{align*}
    &\infer{\ang{\code{skip}, \sigma} \to \sigma}{\blank} \tag*{Assioma dello \code{skip}}\\[8pt]
    &\infer{\ang{x \deq E, \sigma} \to \sigma[n/x]}{\blank} \quad \text{ se } \eval{E, \sigma} = n \tag*{Assioma dell'assegnamento}\\[5pt]
    &\infer{\ang{C_1;C_2, \sigma} \to \ang{C_1';C_2, \sigma'}}{\ang{C_1, \sigma} \to \ang{C_1', \sigma'}} \tag*{Regola della sequenza 1}\\[8pt]
    &\infer{\ang{C_1;C_2, \sigma}\to \sigma'}{\ang{C_1, \sigma}  \to \ang{C_2, \sigma'}} \tag*{Regola della sequenza 2}\\[8pt]
    &\infer{\ang{\code{if}\ B\ \code{then}\ C_1\ \code{else}\ C_2, \sigma} \to \ang{C_1, \sigma}}{\blank} \quad \text{se } \bval{B, \sigma} = \TT \tag*{Assioma cond. 1}\\[8pt]
    &\infer{\ang{\code{if}\ B\ \code{then}\ C_1\ \code{else}\ C_2, \sigma} \to \ang{C_2, \sigma}}{\blank} \quad \text{se } \bval{B, \sigma} = \FF \tag*{Assioma cond. 2}\\[8pt]
    &\infer{\ang{\code{for}\ i = E_1\ \code{to}\ E_2\ \code{do}\ C, \sigma} \to \ang{i \deq n; C; \code{for}\ i = n_1 + 1\ \code{to}\ n_2\ \code{do}\ C, \sigma}}{\blank} \tag*{Assioma del \code{for}\ 1}\\
    & \qquad\qquad\text{se } \bval{E_2 < E_1, \sigma} = \FF, \eval{E_1, \sigma} = n_1, \eval{E_2, \sigma} = n_2\\[8pt]
    &\infer{\ang{\code{for}\ i = E_1\ \code{to}\ E_2\ \code{do}\ C, \sigma} \to \sigma}{\blank} \qquad \text{se } \bval{E_2 < E_1, \sigma} = \TT \tag*{Assioma del \code{for}\ 2}\\[8pt]
    &\infer{\ang{\code{while}\ B\ \code{do}\ C, \sigma} \to \ang{\code{if}\ B\ \code{then}\ (C; \code{while}\ B\ \code{do}\ C)\ \code{else skip}, \sigma}}{\blank} \tag*{Assioma del \code{while}}
\end{align*}

Dalla regola di transizione del \texttt{for} segue una proprietà fondamentale del linguaggio \texttt{FOR}: ogni suo programma termina in tempo finito. Infatti prima della prima iterazione la semantica ci impone di valutare le espressioni $E_1$ ed $E_2$, che verranno valutate a dei naturali $n_1, n_2$. A questo punto il ciclo \texttt{for} verrà eseguito esattamente $n_2 - n_1$ volte (dove il $-$ è sempre limitato, per cui se $n_1 > n_2$ non eseguiremo mai il ciclo) in quanto gli estremi di iterazione non possono essere modificati dai comandi del corpo del \texttt{for}.

Questo ci dimostra immediatamente che il linguaggio \texttt{FOR} \strong{non è equivalente} alle macchine di Turing, ovvero esistono macchine di Turing che codificano algoritmi non risolvibili dal linguaggio \texttt{FOR}. (Studieremo in seguito cosa significa \emph{codificare algoritmi}.)

Il linguaggio \texttt{WHILE} invece può codificare algoritmi che non terminano. Ad esempio si vede subito che la configurazione \[
    \ang*{\code{while}\ \TT\ \code{do skip}, \sigma}
\] diverge a prescindere da $\sigma$. 
\section{Calcolabilità di funzioni}

Dopo aver definito le computazioni per le MdT e per i linguaggi \code{FOR} e \code{WHILE}, vogliamo spiegare cosa significa che una MdT o un comando \emph{calcola} una funzione. 

\subsection*{Funzioni}

Prima di tutto, ricordiamo le definizioni di base sulle funzioni.

\begin{definition}
    {Funzione}{func}
    Dati $A, B$ insiemi, una \sstrong{funzione} $f$ da $A$ in $B$ è un sottoinsieme di $A \times B$ tale che \[
        (a, b), (a, b') \in f \;\implies\; b = b'.
    \] Scriveremo \begin{itemize}
        \item $f : A \to B$ per indicare una funzione da $A$ in $B$
        \item $b = f(a)$ per dire $(a, b) \in f$.
    \end{itemize}
\end{definition}

 Notiamo inoltre che non abbiamo fatto assunzioni sulla \strong{totalità} di $f$: per qualche valore di $a \in A$ potrebbe non esistere un valore $b \in B$ tale che $f(a)$, cioè $f$ potrebbe \emph{non essere definita} in $a$.

\begin{definition}
    {Funzioni totali e parziali}{total_partial_func}
    Sia $f : A \to B$. \begin{itemize}
        \item $f$ \sstrong{converge su} $a \in A$ (e lo si indica con $f(a)\conv$) se esiste $b \in B$ tale che $f(a) = b$;
        \item $f$ \sstrong{diverge su} $a \in A$ (e lo si indica con $f(a)\divg$) se $f$ non converge su $a$;
        \item $f$ è \sstrong{totale} se $f(a)\conv$ per ogni $a \in A$;
        \item $f$ è \sstrong{parziale} se non è totale.     
    \end{itemize}   
\end{definition}

In generale le nostre funzioni saranno parziali.

\begin{definition}
    {Dominio ed immagine}{dom_im_func}
    Sia $f : A \to B$. Si dice \sstrong{dominio} di $f$ l'insieme \[
        \dom f \deq \set*{a \in A \given f(a)\conv}.
    \]  Si dice \sstrong{immagine} di $f$ l'insieme \[
        \Imm f \deq \set*{b \in B \given b = f(a) \text{ per qualche} a \in A}.
    \]
\end{definition}

\begin{definition}
    {Iniettività/surgettività/bigettività}{}
    Sia $f : A \to B$ una funzione. 
    \begin{itemize}
        \item $f$ è \sstrong{iniettiva} se per ogni $a, a' \in A$, $a \neq a'$, allora $f(a) \neq f(a')$.
        \item $f$ è \sstrong{surgettiva} se $\Imm f = B$.
        \item $f$ è \sstrong{bigettiva} se è iniettiva e surgettiva.     
    \end{itemize}
\end{definition}

\subsection*{Calcolare funzioni}

Definiamo ora quando una macchina/un comando \emph{implementa} una funzione.

\begin{definition}
    {Turing-calcolabilità}{t-calc}
    Siano $\Sigma, \Sigma_0, \Sigma_1$ alfabeti, $\resp, \# \notin \Sigma_0 \union \Sigma_1 \subsetneq \Sigma$.

    Sia inoltre $f : \Sigma_0 \to \Sigma_1$, $M = (Q, \Sigma, \delta, q_0)$ una MdT.

    Si dice allora che $M$ \sstrong{calcola} $f$ (e che $f$ è \sstrong{Turing-calcolabile}) se per ogni $v \in \Sigma_0$ \[
        w = f(v) \;\;\text{se e solo se}\;\; (q_0, \ul{\resp}v) \to^\ast (h, \resp w \ul{\#}).
    \]
\end{definition}

Indicando con $M(v)$ il risultato della computazione della macchina $M$ sulla configurazione iniziale $(q_0, \ul{\resp}v)$, questa definizione ci dice che gli output della funzione e della MdT sono esattamente gli stessi. 
In particolare, dato che le funzioni possono essere \emph{parziali} e le macchine di Turing possono \emph{divergere}, $M(v)\conv$ se e solo se $f$ è definita su $v$, cioè se esiste $w$ tale che $f(v) = w$.

\begin{definition}
    {\code{WHILE}-calcolabilità}{w-calc}
    Sia $C$ un comando \code{WHILE}, $g : \code{Var} \to \N$. 
    Si dice allora che $C$ \sstrong{calcola} $g$ (e che $g$ è \sstrong{\code{WHILE}-calcolabile}) se per ogni $\sigma : \code{Var} \to \N$ \[
        n = g(x) \;\;\text{se e solo se}\;\; (C, \sigma) \to^\ast \sigma^\ast \text{ e } \sigma^\ast(x) = n.
    \]   
\end{definition}

Analogamente possiamo definire il concetto di funzione \code{FOR}-calcolabile.

È vero che le funzioni \code{WHILE}-calcolabili sono tutte e sole le funzioni \code{FOR}-calcolabili? \strong{No}: infatti dato che non abbiamo fatto assunzioni sulla totalità di $g$, essa può essere \code{WHILE}-calcolabile ma non \code{FOR}-calcolabile.

Un possibile problema nella definizione di calcolabilità data è che abbiamo supposto che le funzioni abbiano una specifica forma: sono funzioni $\Sigma_0^\ast \to \Sigma_1^\ast$ nel caso delle MdT, $\code{Var} \to \N$ nel caso dei comandi. Scegliendo altri insiemi con le stesse caratteristiche (quindi di cardinalità numerabile) cambiano le funzioni calcolabili?

Fortunatamente la risposta è \strong{no}. Consideriamo una funzione $f : A \to B$: se gli insiemi $A, B$ sono numerabili possiamo scegliere delle \sstrong{codifiche} $A \to \N$, $\N \to B$. A questo punto possiamo \begin{itemize}
    \item trasformare l'input $a \in A$ in un naturale tramite la prima codifica;
    \item fare il calcolo tramite una funzione $\N \to \N$,
    \item trasformare l'output in un elemento di $B$ tramite la seconda codifica. 
\end{itemize}

In questo modo possiamo limitarci a solo funzioni $\N \to \N$, a patto che la codifica sia \strong{effettiva}, cioè sia calcolabile anch'essa.
\section{Funzioni ricorsive}

Introduciamo ora un ultimo formalismo per rappresentare un modello di calcolo, ovvero quello delle funzioni ricorsive.

Per semplificare la notazione useremo la $\lambda$-notazione per le funzioni anonime: la funzione $\lambda x.\ f(x)$ è la funzione che prende un unico parametro di ingresso $x$ e restituisce $f(x)$.   

\begin{definition}
    {Funzioni primitive ricorsive}{prim_rec}
    La classe delle \sstrong{funzioni primitive ricorsive} $\PR$ è la minima classe di funzioni che contenga
    \newthought{Zero} $\lambda x_1, \dots, x_n.\ 0$ \quad per ogni $n \in \N$ 
    \newthought{Successore} $\lambda x.\ x+1$
    \newthought{Proiezione} $\lambda x_1, \dots, x_n.\ x_i$ \quad per ogni $n \in \N$, $i = 1, \dots, n$

    \medskip
    e che sia chiusa per le seguenti operazioni:
    \newthought{Composizione} se $g_1, \dots, g_k : \N^n \to \N$, $h : \N^k \to \N$ appartengono a $\PR$, allora \[
        \lambda x_1, \dots, x_n.\ h\parens[\Big]{g_1(x_1, \dots, x_n), \dots, g_k(x_1, \dots, x_n)}
    \] appartiene ancora a $\PR$;
    \newthought{Ricorsione Primitiva} se $h : \N^{n+1} \to \N$, $g : \N^{n-1} \to \N$ appartengono a $\PR$, allora \begin{gather*}
        f : \N^n \to \N \\
        f(x_1, \dots, x_n) \deq \begin{cases}
            f(0, x_2, \dots, x_n) = g(x_2, \dots, x_n)\\
            f(n+1, x_2, \dots, x_n) = h(n, f(n, x_2, \dots, x_n), x_2, \dots, x_n)
        \end{cases}
    \end{gather*}   
    appartiene ancora a $\PR$. 
\end{definition}
\chapter{Calcolabilità di problemi}

\section{Problemi di decisione}

Finora abbiamo studiato i vari formalismi per esprimere algoritmi e le loro caratteristiche principali, insieme ai diversi teoremi che ne seguono. Vogliamo ora studiare i \emph{problemi} che possono essere risolti da una determinata classe di funzioni.

I nostri problemi sono \sstrong{problemi di decisione}: dato un insieme $I \subseteq \N^k$ vogliamo stabilire se un dato elemento $x \in \N^k$ appartenga o no a $I$. In particolare ogni problema è identificato da un insieme.

Per parlare di \emph{appartenenza} ad un insieme conviene definire due funzioni che saranno di grande rilevanza in seguito.

\begin{definition}
    [Funzione caratteristica e semicaratteristica di un insieme]
    Sia $I \subseteq \N^k$. 
    Si dice \sstrong{funzione caratteristica} di $I$ la funzione 
    $\charf{I} : \N^k \to \set*{0,1}$ definita da \[
        \charf{I}(x) \deq \begin{cases}
            1, &\text{se } x \in I,\\
            0, &\text{se } x \notin I.
        \end{cases}
    \] Si dice inoltre \sstrong{funzione semicaratteristica} di $I$ la funzione parziale $\scharf{I} : \N_k \to \set*{0, 1}$ definita da \[
        \scharf{I}(x) \deq \begin{cases}
            1, &\text{se } x \in I,\\
            \bot, &\text{se } x \notin I.
        \end{cases}
    \]
\end{definition}

Possiamo ora definire le due principali classi di problemi che analizzeremo.

\begin{definition}
    [Insiemi ricorsivi e ricorsivamente enumerabili]
    Sia $I \subseteq \N^k$. \begin{itemize}
        \item $I$ si dice \sstrong{ricorsivo} oppure \sstrong{decidibile} se $\charf{I}$ è calcolabile totale.
        \item $I$ si dice \sstrong{ricorsivamente enumerabile} (in breve r.e.) oppure \sstrong{semidecidibile} se esiste un indice $i$ tale che $I = \dom{\phi_i}$.   
    \end{itemize}
    Chiameremo $\R$ la classe degli insiemi ricorsivi, $\RE$ la classe degli insiemi ricorsivamente enumerabili.
\end{definition}

Intuitivamente $I$ è decidibile se è possibile \emph{decidere} in tempo finito se un elemento appartiene o meno all'insieme. Per quanto riguarda gli insiemi semidecidibili, facciamo un'osservazione iniziale.

\begin{remark}
    $I$ è r.e. se e solo se $\scharf{I}$ è calcolabile (parziale).
    \begin{proof}
        Se $\scharf{I}$ è calcolabile, allora esiste un indice $i$ con $\phi_i = \scharf{I}$. In particolare $\dom{\phi_i} = \dom{\scharf{I}} = I$, dunque $I$ è r.e.
        
        Viceversa, se $I$ è r.e. esiste un indice $i$ tale che $\dom{\phi_i} = I$. Allora la funzione $\scharf{I}$ è calcolabile: dato $x$ iniziamo a calcolare $\phi_{i}(x)$; se il procedimento termina poniamo $\scharf{I}(x) = 1$, altrimenti continueremo all'infinto e quindi la computazione di $\scharf{I}$ non terminerà.   
    \end{proof} 
\end{remark}

Quindi un insieme $I$ è semidecidibile se per ogni elemento $x \in I$ possiamo controllare in tempo finito l'appartenenza, mentre per gli elementi $x \notin I$ il procedimento non termina mai.

Il fatto che gli insiemi r.e. si chiamano proprio in questo modo deriva da un'altra particolare caratterizzazione.

\begin{proposition}
    $I$ è r.e. se e solo se $I$ è vuoto oppure esiste una funzione $f$ calcolabile totale tale che $I = \Imm{f}$.  
\end{proposition}
\begin{proof}
    % TODO: proof
\end{proof}

Quali sono le relazioni tra insiemi ricorsivi e insiemi r.e.? Vediamone alcune che seguono immediatamente dalle definizioni.

\begin{proposition}
    [$\R \subseteq \RE$]
    Se $I$ è ricorsivo, allora $I$ è ricorsivamente enumerabile.
\end{proposition}
\begin{proof}
    Infatti se $I$ è ricorsivo la funzione $\scharf{I}$ è calcolabile: in effetti dato $x$, se $\charf{I}(x) = 1$ allora $\scharf{I}(x) = 1$, altrimenti $\scharf{I}(x)$ è indefinito. 
    Segue che $I$ è r.e. poiché $I = \dom*{\scharf{I}}$. 
\end{proof}

Per la prossima proposizione abbiamo bisogno di definire il \sstrong{complementare} di un problema.

\begin{definition}
    [Complementare di un problema]
    Dato un insieme $I$, il suo complementare $\compl{I}$ è definito da \[
        \compl{I} \deq \set*{x \given x \notin I}.
    \] 
\end{definition}

\begin{proposition}[ ][I_complI_re=>both_rec]
    Se $I$ e $\compl{I}$ sono entrambi r.e., allora sono entrambi ricorsivi.
\end{proposition}
\begin{proof}
    Osserviamo che basta mostrare che $I$ sia ricorsivo: a questo punto replicando il ragionamento su $\compl{I}$ e $\compl{\compl{I}} = I$ si ottiene che anche $\compl{I}$ è ricorsivo.
    
    Per definizione di insieme r.e., esistono due indici $i, j$ tali che \[
        I = \dom{\phi_i}, \qquad \compl{I} = \dom{\phi_j}. 
    \] Per calcolare $\charf{I}$, dato $x$ eseguiamo questa seguenza di passi:
    \begin{itemize}
        \item eseguiamo un passo di computazione di $\phi_i(x)$: se converge (cioè $x \in \dom{\phi_i} = I$) allora $\charf{I}(x) = 1$, altrimenti continuiamo;
        \item eseguiamo un passo di computazione di $\phi_j(x)$: se converge (cioè $x \in \dom{\phi_j} = \compl{I}$ allora $\charf{I}(x) = 0$, altrimenti continuiamo;    
        \item eseguiamo due passi di computazione di $\phi_i(x)$... 
    \end{itemize}
    e così via. Ma $x$ deve appartenere ad uno tra $I$ e $\compl{I}$, dunque questo procedimento ad un certo punto termina. Segue che $\charf{I}$ è calcolabile.  
\end{proof}

\section{Separazione di $\R$ e $\RE$}

Dai risultati ottenuti nella sezione precedente potremmo essere indotti a sperare che tutti i problemi semidecidibili siano anche decidibili. Purtoppo ciò non è vero, e lo dimostriamo attraverso un particolare insieme, chiamato tradizionalmente $K$:

\begin{equation}
    \boxed{K \deq \set[\Big]{n \given \phi_n(n)\conv}}
\end{equation}

\begin{theorem}
    {}{}
    $K$ è r.e. ma non è ricorsivo.
\end{theorem}

Per chiarezza dividiamo in due parti la dimostrazione.

\begin{proof}[Dimostrazione che $K$ è r.e.]
    Per quanto detto precedentemente, per mostrare che $K$ è r.e. basta far vedere che la sua funzione semicaratteristica \[
        \scharf{K} \deq n \mapsto \begin{cases}
            1, &\text{se } \phi_n(n)\conv \\
            \bot, &\text{altrimenti}
        \end{cases}
    \] è calcolabile. Ma questa funzione è intuitivamente calcolabile: \begin{itemize}
        \item si esegue un passo del calolo di $\phi_0(0)$: se converge allora $\scharf{K}(0)$ vale $1$, altrimenti si continua;
        \item si esegue un passo del calcolo di $\phi_1(1)$: se converge allora $\scharf{K}(0)$ vale $1$, altrimenti si continua;
        \item si eseguono due passi del calcolo di $\phi_0(0)$...       
    \end{itemize} e così via. Questo procedimento ad un certo punto termina per tutti i valori di $n$ che sono in $K$, e per gli altri invece non termina mai.
\end{proof}

\begin{proof}[Dimostrazione che $K$ non è ricorsivo]
    Supponiamo per assurdo che $K$ sia ricorsivo: per definizione allora $\charf{K}$ è calcolabile totale. Definiamo allora $f : \N \to \N$ data da \[
        f(n) \deq \begin{cases}
            \phi_n(n) + 1, &\text{se } \charf{K}(n) = 1\\
            0, &\text{altrimenti.}
        \end{cases}
    \] Dato che $\charf{K}$ è calcolabile totale possiamo calcolare $\charf{K}(n)$; inoltre se $\charf{K}(n) = 1$ (ovvero $n \in K$) per definizione di $K$ si ha che $\phi_n(n)$ converge.
    
    Per la Tesi di Church-Turing allora esiste un indice $i$ tale che $\phi_i = f$,e quindi in particolare $\phi_i(i) = f(i)$. Ma ciò è assurdo: \begin{itemize}
        \item se $\phi_i(i)$ converge allora $f(i) = \phi_i(i) + 1 \neq \phi_i(i)$;
        \item se $\phi_i(i)$ diverge allora $f(i)$ converge (a $0$).   
    \end{itemize}
    Segue in particolare che $K$ non può essere ricorsivo. 
\end{proof}

\newthoughtpar{Piccola parentesi sul \emph{bootstrapping}} L'insieme $K$ può risultare abbastanza contorto e quindi è facile farsi venire l'idea che la dimostrazione funzioni solo perché abbiamo scelto un insieme "innaturale". In realtà l'auto-applicazione non è strana, e compare anche nel mondo reale, ad esempio quando si parla di compilatori.

Infatti dato un compilatore $C_L^{L \to A}$ scritto nel linguaggio $L$ e che compila codice $L$ (alto livello) in codice scritto nel linguaggio $A$ (più a basso livello), potrebbe essere necessario dover usare il compilatore in una macchina che nativamente non sa far girare il codice $L$, ma sa far girare solo codice nel linguaggio $A$. Vogliamo perciò un modo efficiente per ottenere un compilatore $L \to A$ scritto in $A$.

Il metodo più semplice è detto \strong{bootstrapping}: in pratica si dà in pasto al compilatore $C_L^{L \to A}$ il suo stesso codice, ovvero si dà il comando $C_L^{L\to A}\parens[\bigg]{C_L^{L \to A}}$. Il risultato è un codice scritto in $A$ che mantiene la semantica originale, ovvero è un compilatore $C_A^{L \to A}$.

\bigskip
\bigskip

Tuttavia il bootstrapping non è l'unico motivo per cui il problema $K$ è importante anche da un punto di vista applicativo: $K$ è strettamente legato al \sstrong{problema della fermata}, che ora definiremo formalmente.

\begin{definition}
    {Problema della fermata}{}
    Dato un indice $i$ e un input $x$, dire se $\phi_i(x)$ converge.   
\end{definition}

Come ogni problema decisionale, il problema della fermata può essere rappresentato tramite un insieme. In questo caso l'insieme è indicato con \[
    \boxed{K_0 \deq \set[\Big]{(i, x) \given \phi_i(x)\conv}}. 
\]

Questo problema è intuitivamente molto importante: se fosse decidibile, potremmo decidere in tempo finito se una funzione $\phi_i$ si arresta sull'input $x$ oppure continua in eterno.

\begin{theorem}
    {$K_0$ non è ricorsivo}{K_0-not-R}
    L'insieme $K_0$ non è ricorsivo.
\end{theorem}
\begin{proof}
    Osserviamo che $(x, x) \in K_0$ se e solo se $x \in K$: se $K_0$ fosse decidibile lo sarebbe anche $K$, ma ciò è assurdo.    
\end{proof}
\section{Riduzioni di classi di problemi}

Nella dimostrazione del \Cref{th:K_0-not-R} abbiamo sfruttato una tecnica comune in matematica: abbiamo \emph{ridotto} il problema $K_0$ al problema $K$ e in questo modo abbiamo dimostrato che $K_0$ non può essere decidibile.

Vogliamo ora generalizzare questo concetto.

\begin{definition}
    [Riduzione secondo una funzione]
    Dati due problemi $A$, $B$, si dice che $A$ \sstrong{si riduce secondo $f$} a $B$ (e si scrive $A \leq_f B$) se \[
        x \in A \;\iff\; f(x) \in B.
    \] 
\end{definition}

\begin{remark}
    $K \leq_f K_0$ secondo la funzione $f : \N \to \N^2$ definita da $f(x) \deq (x, x)$.   
\end{remark}

\begin{remark}
    $A \leq_f B$ se e solo se $\compl{A} \leq_f \compl{B}$.  
\end{remark}

Spesso non ci interessa quale sia la funzione che permette la riduzione di $A$ a $B$, ma solo a quale classe di funzioni appartiene.

\begin{definition}
    [Riduzione secondo una classe di funzioni]
    Siano $A$, $B$ problemi, $\FF$ insieme di funzioni. Allora si dice che $A$ \sstrong{si riduce secondo $\FF$} a $B$ (e si scrive $A \leq_{\FF} B$, o anche $A \leq B$ se l'insieme $\FF$ è deducibile dal contesto) se esiste una $f \in \FF$ tale che $A \leq_f B$.   
\end{definition}

Come scegliamo l'insieme $\FF$? In generale dipende dalle classi di problemi che vogliamo confrontare: infatti per poter studiare queste classi è importante che la riduzione rispetti alcune proprietà.

\begin{definition}
    [Riduzione che classifica due classi][red-classifies]
    Date $\DD, \EE$ classi di problemi con $\DD \subseteq \EE$, $\FF$ insieme di funzioni, si dice che la relazione $\leq_{\FF}$ \sstrong{classifica} le classi $\DD$, $\EE$ se \begin{enumerate}
        \item $A \leq_{\FF} A$,
        \item se $A \leq_{\FF} B$ e $B \leq_{\FF} C$, allora $A \leq_{\FF} C$,
        \item se $A \leq_{\FF} B$ e $B \in \DD$, allora $A \in \DD$,
        \item se $A \leq_{\FF} B$ e $B \in \EE$, allora $A \in \EE$.       
    \end{enumerate}   
\end{definition}

\begin{remark}
    Sfruttando la definizione di riduzione, possiamo riscrivere le proprietà in forma \emph{algebrica}:
    \begin{enumerate}
        \item $\id_A \in \FF$,
        \item se $f$ e $g$ appartengono a $\FF$, allora anche la composizione $gf$ appartiene ad $\FF$,
        \item se $f \in \FF$ e $B \in \DD$, allora $f\inv[B] \in \DD$,
        \item se $f \in \FF$ e $B \in \EE$, allora $f\inv[B] \in \EE$.            
    \end{enumerate}
    Questo in particolare mi dice che se $\leq_{\FF}$ classifica $\DD, \EE$, allora $\leq_{\FF}$ è un \strong{preordine} su $\DD$ e su $\EE$.
\end{remark}

\newthought{Intuizione} Qual è il significato intuitivo di una relazione che classifica due classi? Possiamo leggere $\leq_{\FF}$ come "è al più difficile quanto":
\begin{itemize}
    \item $A$ è al più difficile quanto se stesso,
    \item se $A$ è al più difficile quanto $B$ e $B$ è al più difficile quanto $C$, allora $A$ è al più difficile quanto $C$,
    \item i problemi difficili al più quanto $B$ si trovano tutti nel più piccolo insieme contenente $B$: se $A$ è al più difficile quanto $B$ e $B$ appartiene a $\DD$ (risp. $\EE$), allora anche $A$ appartiene a $\DD$ (risp. $\EE$), cioè $A$ \strong{non sta fuori $\DD$ (risp. $\EE$)}.    
\end{itemize} 

Fissiamo ora due classi $\DD$, $\EE$ con $\DD \subseteq \EE$ e una classe di funzioni $\FF$.    

\begin{definition}
    [Grado di un problema]
    Definiamo il \sstrong{grado} di un problema $A$ è \[
        \deg A \deq \set*{B \given A \leq_{\FF} B \text{ e } B \leq_{\FF} A},
    \] ovvero è l'insieme di tutti i problemi con la stessa difficoltà di $A$. 
\end{definition}

\begin{definition}
    [Problemi ardui e completi]
    Sia $H$ un problema. \begin{itemize}
        \item $H$ si dice \sstrong{$\leq_{\FF}$-arduo per $\EE$} se per ogni $A \in \EE$ si ha $A \leq_{\FF} H$, ovvero se $H$ è almeno difficile quanto tutti i problemi di $\EE$.
        \item $H$ si dice \sstrong{$\leq_{\FF}$-completo per $\EE$} se è $\leq_{\FF}$-arduo per $\EE$ e $H \in \EE$.    
    \end{itemize} 
\end{definition}

In particolare possiamo studiare la relazioni tra $\DD$ e $\EE$ tramite i problemi $\leq_{\FF}$-ardui/completi per $\EE$.

\begin{proposition}
    Se $C$ è $\leq_{\FF}$-completo per $\EE$ e $C \in \DD$, allora $\EE = \DD$.    
\end{proposition}
\begin{proof}
    Dato che $\DD \subseteq \EE$, basta mostrare che $\EE \subseteq \DD$. Sia allora $B \in \EE$: dato che $C$ è $\leq_{\FF}$-completo per $\EE$ segue che $B \leq_{\FF} C$. Ma $C \in \DD$, dunque anche $B \in \DD$.      
\end{proof}

\begin{proposition}[][A-hard=>B-hard]
    Se $A$ è $\leq_{\FF}$-arduo per $\EE$ e $A \leq_{\FF} B$, allora $B$ è $\leq_{\FF}$-arduo per $\EE$. In particolare se $B \in \E$ allora $A$, $B$ sono $\leq_{\FF}$-completi per $\EE$.          
\end{proposition}
\begin{proof}
    Se $H \in \EE$. Dato che $A$ è $\leq_{\FF}$-arduo per $\EE$ si ha $H \leq_{\FF} A$; inoltre $A \leq_{\FF} B$ dunque per transitività $H \leq_{\FF} B$, ovvero $B$ è $\leq_{\FF}$-arduo.
    
    Inoltre se $B \in \EE$ segue anche che $A$ lo è, poiché $A \leq_{\FF} B$. Allora per definizione $A, B$ sono $\leq_{\FF}$-completi per $\EE$.
\end{proof}
\section{Studio di $\R$ e $\RE$ tramite riduzioni}

Vogliamo studiare ora le classi di problemi $\R$ e $\RE$ tramite riduzioni: come prima cosa dobbiamo identificare una classe di funzioni. 

\begin{definition}
    {Classe \rec}{}
    Indicheremo con \[
        \rec \deq \set*{\phi_i \given \dom{\phi_i} = \N}
    \] la classe di tutte le funzioni calcolabili totali, che quindi chiameremo (per motivi storici) \sstrong{ricorsive}.
\end{definition}

\begin{proposition}
    {}{}
    $\leq_\rec$ classifica $\R \subseteq \RE$. 
\end{proposition}
\begin{proof}
    Basta mostrare le quattro condizioni date dalla definizione. In particolare lo faremo attraverso le condizioni algebriche equivalenti.
    \begin{enumerate}
        \item L'identità è ricorsiva.
        \item Se $f, g$ sono ricorsive, allora anche la loro composizione $gf$ lo è.
        \item Supponiamo $A \leq_\rec B$ con $B \in \R$ (cioè $\charf{B}$ è ricorsiva). Vogliamo dimostrare che $A \in \R$, cioè che $\charf{A}$ è ricorsiva.
        
        Per definizione di $\leq_\rec$ esiste una funzione $f \in \rec$ con $A \leq_f B$, cioè $x \in A$ se e solo se $f(x) \in B$, ovvero $\charf{A}(x) = 1$ se e solo se $\charf{B}(f(x)) = \charf{B} \circ f(x) = 1$. Ma allora $\charf{A} = \charf{B}f$ e dunque $\charf{A}$ è ricorsiva poiché composizione di funzioni ricorsive.
        \item Supponiamo $A \leq_\rec B$ con $B \in \RE$, ovvero con $\scharf{B}$ calcolabile. Vogliamo dimostrare che $A$ è r.e., cioè che $\scharf{A}$ è calcolabile.
        
        Per definizione di $\leq_\rec$ esiste una funzione $f \in \rec$ con $A \leq_f B$, ovvero $x \in A$ se e solo se $f(x) \in B$.
        Mostriamo che $\scharf{A} = \scharf{B}f$:
        \begin{itemize}
            \item se $x \in A$ (e quindi $\scharf{A}(x) = 1$) allora $f(x) \in B$ e quindi $\scharf{B}(f(x)) = \scharf{B}f(x) = 1$;
            \item se $x \notin A$ (cioè $\scharf{A}(x)$ diverge) allora $f(x) \notin B$, e quindi $\scharf{B}(f(x)) = \scharf{B} \circ f(x)$ diverge.
        \end{itemize}
        
        Segue che $\scharf{A} = \scharf{B}f$. In particolare $\scharf{A}$ è calcolabile, poiché composizione di funzioni calcolabili.  \qedhere
    \end{enumerate}
\end{proof}

Dunque in questa sezione diremo che un problema è arduo/completo per $\RE$ sottointendendo la relazione $\leq_\rec$.   

\begin{remark} 
    Osserviamo che $\compl{K} \nleq_\rec K$ e $K \nleq_\rec \compl{K}$: \begin{itemize}
        \item se per assurdo $\compl{K} \leq_\rec K$, allora $\compl{K}$ sarebbe r.e. (perché $K$ lo è), ma questo implicherebbe (per la \Cref{prop:I_complI_re=>both_rec}) che sia $K$ che $\compl{K}$ sono ricorisvi, il che è assurdo (poiché $K$ non è ricorsivo);
        \item come osservato in precedenza, $A \leq_\FF B$ se e solo se $\compl{A} \leq_\FF \compl{B}$, dunque $K \leq_\rec \compl{K}$ se e solo se $\compl{K} \leq_\rec K$, che abbiamo dimostrato essere falso.   
    \end{itemize}  
\end{remark}

\newthought{Problemi $\coRE$} 
I problemi tali che i loro complementari sono in $\RE$ formano una classe chiamata $\coRE$. Dato che ogni problema ricorsivo ha anche un complementare ricorsivo, $\R \subseteq \coRE$; tuttavia esistono anche problemi al di fuori di $\coRE$.   

\medskip

Per studiare le relazioni tra gli insiemi $\R$ e $\RE$ vorremo trovare un problema completo per $\RE$.

\begin{theorem}
    {}{}
    $K$ è completo per $\RE$. 
\end{theorem}
\begin{proof}
    Dato che $K$ è r.e., basta dimostrare che $K$ è arduo per $\RE$. Sia allora $A \in \RE$, ovvero tale che esista un indice $i$ tale che $\dom{\phi_i} = A$, ovvero $A = \set*{x \given \phi_i(x)\conv}$.
    
    Definiamo allora la funzione $\psi : \N^2 \to \N$ data da $\psi(x, y) \deq \phi_i(x)$. (Il parametro $y$ non fa niente.)
    Dato che $\psi$ è calcolabile dovrà esistere un indice $j$ con $\psi = \phi_j$. In particolare $A = \set*{x \given \phi_j(x, y)\conv}$.
    
    Per il \nameref{th:s-1-1} e l'\Cref{rem:s-1-1-one-var}, possiamo considerare la funzione $f \deq \lam{x}{s(j, x)}$: il Teorema ci garantisce che $\phi_j(x, y) = \phi_{f(x)}(y)$ per ogni $x, y$. Osserviamo in particolare che $f$ è calcolabile totale.

    Ma $y$ non ha un effetto sulla computazione ($\phi_j(x, y) = \phi_j(x, y')$ per ogni $y, y'$), ovvero $\phi_{f(x)}$ è \strong{costante}. Segue che \[
        A = \set*{x \given \phi_{f(x)}(y)\conv} = \set*{x \given \phi_{f(x)}(f(x))\conv} = \set*{x \given f(x) \in K},
    \] ovvero $A \leq_f K$. Dato che $f \in \rec$, segue la tesi.
\end{proof}

\subsection{Altri problemi completi per $\RE$}

Il fatto che $K \leq_\rec K_0$ mostra (per il \Cref{prop:A-hard=>B-hard}) che anche $K_0$ è completo per $\RE$. Cerchiamo altri problemi completi per $\RE$.

\newthoughtpar{\TOT{} è completo per $\RE$} 

Consideriamo il problema \[
    \boxed{\TOT \deq \set*{i \given \dom{\phi_i} = \N} = \set*{i \given \phi_i \in \rec}}.
\]

\begin{proposition}{}{TOT-is-RE-complete}
    Vale la riduzione \[
        K \leq_\rec \TOT.
    \] In particolare $\TOT$ è completo per $\RE$. 
\end{proposition}
\begin{proof}
    Consideriamo la funzione $\psi : \N^2 \to \N$ definita da \[
        \psi(x, y) \deq \begin{cases}
            1, &\text{se } x \in K \\
            \bot, &\text{se } x \notin K.
        \end{cases}
    \] Dato che $K$ è semidecidibile $\psi$ è calcolabile: basta calcolare $\scharf{K}(x)$ (che è calcolabile); se la computazione termina (resitutendo $1$) poniamo $\psi(x, y) = 1$, altrimenti la computazione di $\scharf{K}(x)$ non termina e quindi anche $\psi(x, y)$ diverge.

    Siccome $\psi$ è calcolabile esisterà un indice $i$ tale che $\phi_i = \psi$. Per il \nameref{th:s-1-1} (insieme all'\Cref{rem:s-1-1-one-var}) esisterà $f$ calcolabile totale, $f \deq \lam{x}{s(i, x)}$ tale che \[
        \phi_i(x, y) = \phi_{f(x)}(y).
    \] Ora abbiamo due casi: \begin{itemize}
        \item se $x \in K$ allora per ogni $y$ si ha \[
            \phi_{f(x)}(y) = \psi(x, y) = 1,
        \] dunque $f(x)$ è indice di una funzione calcolabile totale, ovvero $f(x) \in \TOT$;
        \item se $x \notin K$ allora per ogni $y$ si ha \[
            \phi_{f(x)}(y) = \psi(x, y) = \bot,
        \] dunque $f(x)$ è non è indice di una funzione calcolabile totale (è sempre indefinita!) e quindi $f(x) \notin \TOT$.   
    \end{itemize}

    Segue che $x \in K$ se e solo se $f(x) \in \TOT$, ovvero (dato che $f$ è calcolabile totale) $K \leq_\rec \TOT$.  
\end{proof}


\chapter{Teoria della Complessità}

\section{Struttura generale del progetto}

La specifica del progetto Microblog richiede di progettare dei tipi di dato astratti in grado di simulare un social network. In particolare, la specifica si divide in tre punti:
\begin{itemize}
    \item definire un tipo di dato in grado di rappresentare un post;
    \item definire un tipo di dato per rappresentare il social network;
    \item estendere il social network per implementare un meccanismo di segnalazione dei post.
\end{itemize}

La descrizione dei dettagli implementativi dei tre punti sono rispettivamente nella \autoref{sec:posts}, nella \autoref{sec:sn} e nella \autoref{sec:reports} di questa relazione. Un'ultima sezione (la \autoref{sec:exceptions}) sarà dedicata a spiegare brevemente i tipi di eccezioni dichiarate.

Ogni interfaccia o classe dichiarata in questa implementazione è corredata da una \emph{overview}, che spiega lo scopo del tipo di dato astratto, e da un \emph{typical element}, che descrive un generico elemento tipico dell'interfaccia (o classe).

Inoltre ogni classe ha associato un invariante di rappresentazione (per distinguere gli elementi ben formati da quelli che non lo sono) e una funzione di astrazione (per ottenere il significato astratto di un elemento ben formato).

Infine, i vari metodi sono tutti dotati di un contratto d'uso, descritto attraverso commenti in stile \texttt{Javadocs} contenenti le clausole \texttt{requires}, \texttt{throws}, \texttt{modifies} e \texttt{effects}: il chiamante deve rispettare le precondizioni per esser certo che il metodo funzioni correttamente.

\section{Complessità deterministica}

Vogliamo studiare la complessità in tempo e spazio usando come modello
base di \emph{calcolatore} le Macchine di Turing.
Come prima cosa, facciamo una piccola estensione delle MdT normali.

\begin{definition}
    [Macchina di Turing a $k$ nastri]
    Una \sstrong{Macchina di Turing a $k$ nastri} è una quintupla 
    $M = (Q, \Sigma, \delta, q_0)$ tale che \begin{enumerate}[(1)]
        \item $\#, \resp \in \Sigma$, mentre $\Left, \Right, \blank \notin \Sigma$,
        \item $\SI, \NO$ sono due stati speciali che non appartengono a $Q$,
        \item la funzione di transizione ha la forma \[
            \delta : Q \times \Sigma^k \to 
                (Q \union \set*{\SI, \NO}) \times 
                    \parens*{\Sigma \times \set*{\Left, \Right, \blank}}^k
        \]   
    \end{enumerate} e inoltre valgono le altre condizioni delle MdT, 
    opportunamente riscritte.
\end{definition}

Una MdT a $k$ nastri quindi è una normale MdT che può leggere
contemporaneamente $k$ simboli diversi e la cui funzione di transizione
lavora considerando lo stato corrente e tutti i $k$ simboli letti.

In particolare una configurazione ha la forma \[
    (q, u_1\ul{\sigma}_1v_1, u_2\ul{\sigma}_2v_2, \dots, u_k\ul{\sigma}_kv_k).
\] Una macchina a $k$ nastri può essere quindi pensata come 
una macchina a \emph{$k$ processori}: ogni nastro contiene dell'informazione,
e i vari nastri possono eseguire calcoli aiutandosi l'un l'altro oppure
ognuno per i fatti suoi.

Perché non abbiamo diviso lo stato in $k$-uple? 
Perché in generale non cambia niente: possiamo immaginare l'insieme $Q$ 
composto da $k$-uple e ogni coordinata della $k$-upla rappresenta lo stato
del singolo nastro.

Introduciamo le macchine a $k$ nastri formalmente solo ora perché da un
punto di vista della calcolabilità non apportano nessun miglioramento:
un problema è decidibile/semidecidibile da una macchina a $k$ nastri se e
solo se è decidibile/semidecidibile da una macchina standard ad $1$ nastro.

\begin{definition}
    [Tempo deterministico richiesto]
    Sia $M$ una MdT a $k$ nastri che risolve il problema $I$. 
    Diremo che $t \in \N$ è \sstrong{tempo deterministico richiesto} per
    decidere l'istanza $x \in I$ se \[
        (q_0, \ul{\resp}x, \ul{\resp}, \dots, \ul{\resp}) \to^t 
        (h, u_1, u_2, \dots, u_t)
    \] 
    con $h \in \set*{\SI, \NO}$. 
\end{definition}

Il \emph{deterministico} nella definizione precedente evidenzia il fatto
che la MdT considerata è deterministica, ovvero che la sua $\delta$ è una
funzione di transizione e non semplicemente una relazione.
Questa differenza, finora immateriale, sarà fondamentale nello studio della
Teoria della Complessità.

Questo ci dà un modo per misurare il tempo richiesto per la decisione di
ogni istanza, ma noi vorremmo una funzione che varia in base alla taglia
$\abs*{x}$ del caso $x \in I$.

\begin{definition}
    [Tempo per la decisione]
    Sia $M$ una MdT a $k$ nastri che risolve il problema $I$.
    Sia inoltre $f : \N \to \N$ una funzione crescente.

    Diremo che $M$ \sstrong{decide $I$ in tempo deterministico $f$} se
    per ogni $x \in I$ il tempo deterministico $t_x$ richiesto da $M$ per
    decidere l'istanza $x \in I$ è tale che \[
        t \leq f\parens[\big]{\abs{x}}.
    \]  
\end{definition}

\begin{definition}
    [Classe delle funzioni decidibili in tempo det. $f$]
    Sia $f : \N \to \N$ monotona crescente. 
    La \sstrong{classe di complessità di tempo deterministico $f$} è la classe \[
        \TIME{f} \deq \set*{I \given \exists M \text{ che decide } I \text{in tempo det. } f}.
    \]
\end{definition}

Spesso ci converrà ragionare in termini di ordini di grandezza:
ricordiamo che date due funzioni $f, g : A \to \R$ diremo che $f$ è \sstrong{$\OO$-grande} di $g$
(e lo indichiamo impropriamente con $f = \OO(g)$) se esiste $c \in [0, +\infty]$ tale che \[
    f(x) \leq c \cdot g(x) \qquad \text{definitivamente},
\] o equivalentemente se \[
    \lim_{x \to +\infty} \frac{f(x)}{g(x)} \leq c.
\] Scriveremo quindi spesso $\TIME{f}$ per indicare in realtà $\TIME[\big]{\OO(f)}$, 
ovvero la classe di tutti gli algoritmi decidibili in tempo deterministico \emph{ordine di $f$}.

Vogliamo ora dimostrare che l'aggiunta di nastri è un procedimento \emph{effettivo},
ovvero che non altera in modo non-algoritmico la capacità di computazione delle macchine di Turing.

\begin{theorem}
    [Riduzione del numero di nastri]
    Sia $M$ a $k$ nastri che decide $I$ in tempo deterministico $f$. 
    Allora esiste $M'$ ad un nastro che decide $I$ in tempo deterministico $\OO(f^2)$. 
\end{theorem}
\begin{proof}
    Consideriamo una configurazione della macchina a $k$ nastri \[
        (q, \ul{\resp}w_1, \dots, \ul{resp}w_k)_M.
    \] Per farla diventare ad un nastro la rappresentiamo come \[
        (q', \resp\resp'w_1\eresp'\resp'w_2\eresp'\dots\resp'w_k\eresp')_{M'}.
    \] Per memorizzare il simbolo corrente, aggiungiamo a $\Sigma$ un simbolo $\overline{\sigma}$
    per ogni $\sigma \in \Sigma$.
    
    Il primo passo da fare è prendere l'input della macchina $M$ 
    e trasformarlo nel relativo input della macchina $M'$. L'input di $M$ sarà della forma \[
        (q, \resp x, \resp, \dots, \resp);
    \] per scriverlo in $M'$ eseguiamo i seguenti passi: 
    \begin{align*}
        (q, \resp x) 
        &\to^{\abs*{x} + 1} (q, \resp\resp' x) \\
        &\to^{\OO(\abs x)} (q, \ul{\resp}\resp'x\eresp'\resp'\eresp'\dots\resp'\eresp').
    \end{align*} Dunque per scrivere l'input su $M'$ abbiamo bisogno (al caso pessimo) 
    di $\OO(\abs x)$ passi.
    
    Vogliamo studiare ora il tempo (massimo) necessario per l'esecuzione di un passo di computazione: 
    data una configurazione di $M'$ \[
        (q, \ul{\resp} 
            \resp' u_1\overline{\sigma_1}v_1 \eresp' \cdots 
            \resp' u_k\overline{\sigma_k}v_k \eresp')
    \] con il cursore nel respingente "vero", per eseguire tutti i passi ci basta 
    \begin{enumerate}[(1)]
        \item scorrere il nastro verso destra per leggere tutti i caratteri correnti,
        \item tornare al respingente a sinistra,
        \item scorrere il nastro ed eseguire le computazioni,
        \item tornare al respingente.
    \end{enumerate} Dunque sono sufficienti $4$ letture complete del nastro di $M'$.
    
    Osserviamo che ogni \emph{sottonastro} di $M'$, ovvero ogni stringa 
    $u_i\overline{\sigma_i}v_i$, è lunga al più $f(\abs x)$: in effetti la macchina $M$ esegue
    al più $f(\abs x)$ passi, e in un passo può scrivere al più una casella del nastro.
    
    Segue che il nastro di $M'$ in totale è lungo al più \[
        \ell \deq k\parens[\big]{f(\abs x) + 2} + 1.\footnotemark
    \]
    \footnotetext{Il $+2$ è dovuto ai due respingenti finti $\resp'$ e $\eresp'$,
    mentre il $+1$ è dovuto al respingente vero $\resp$ di $M'$.}

    Segue che per ogni passo di computazione eseguiamo $\OO(\ell) = \OO(f)$ operazioni;
    dato che eseguiamo al più $f(\abs x)$ passi di computazione segue che $M'$
    decide $I$ in al più \[
        f \cdot \OO(f) = \OO(f^2)
    \] passi.
\end{proof}

\begin{theorem}
    [Teorema di Accelerazione Lineare][linear-accel]
    Se $I \in \TIME{f}$, allora per ogni $\eps > 0$ si ha che \[
        I \in \TIME{\eps \cdot f(n) + n + 2}.
    \] 
\end{theorem}

Possiamo finalmente definire una delle più importanti classi di complessità studiate.

\begin{definition}
    [Classe $\P$]
    La \sstrong{classe di complessità $\P$} è definita come \[
        \P \deq \bigunion_{k \in \N} \TIME[\big]{n^k},
    \] considerando gli ordini di grandezza.
\end{definition}
\section{Complessità in spazio}

Vogliamo ora definire le misure di complessità in spazio per i problemi decidibili.
Iniziamo discutendo una nuova estensione delle Macchine di Turing.

\begin{definition}
    [Macchina di Turing I/O]
    Una \sstrong{Macchina di Turing I/O} è una macchina di Turing a $k$ nastri ($k \geq 3$)
    $M = (Q, \Sigma, \delta, q_0)$ con i seguenti vincoli aggiuntivi: \begin{enumerate}[(1)]
        \item Se \[
            \delta(q, \sigma_1, \dots, \sigma_k) = 
            \parens[\big]{q', (\sigma_1', D_1), \dots, (\sigma_k', D_k)}
        \] allora $\sigma_1' = \sigma_1$ e $D_k \in \set*{\Right, \blank}$. 
        Inoltre se $D_k = \blank$ allora $\sigma_k' = \sigma_k$.
        \item Se $\sigma_1 = \#$ allora $D_1 \in \set*{\blank, \Left}$.      
    \end{enumerate}
\end{definition}

L'idea delle macchine I/O è che il primo e l'ultimo nastro hanno i ruoli privilegiati di 
\emph{lettore} e \emph{scrittore}. Infatti il primo nastro non può essere modificato 
grazie alla prima condizione, e inoltre arrivati alla fine (cioè alla porzione vuota) non possiamo
proseguire verso destra. Analogamente l'ultimo nastro può essere solo scritto: 
infatti non si può riscrivere un carattere già scritto e ci si può solo muovere verso destra.

Anche questa estensione delle MdT è effettiva, come mostrato dal seguente Teorema.

\begin{theorem}
    Per ogni MdT a $k$ nastri che decide $I$ in tempo deterministico $f$,
    esiste una MdT I/O a $(k+2)$-nastri che decide $I$ in tempo deterministico $c \cdot f$
    per qualche $c \in [0, +\infty]$.        
\end{theorem}
\begin{proof}
    Basta copiare il contenuto del nastro di input nei nastri $2, \dots, k+1$,
    eseguire gli stessi passi della macchina di partenza sui nastri \emph{di lavoro}
    e infine copiare l'output sul nastro $k+2$.  
\end{proof}

Il nostro scopo è misurare lo spazio necessario ad una macchina I/O per risolvere un'istanza
di un problema. Per farlo potremmo eseguire la macchina e infine contare il numero di caselle 
non bianche, ma questo non funzionerebbe poiché la macchina può cancellare pezzi scritti durante
la computazione.

Introduciamo quindi un'altra banale estensione, ovvero il simbolo di fine stringa $\eresp$:
ogni volta che abbiamo necessità di più spazio ci basterà spostare tale simbolo a destra,
ma avremo cura di non spostarlo mai verso sinistra o cancellarlo in modo da memorizzare la
quantità \emph{massima} di memoria usata.

\begin{definition}
    [Spazio richiesto per la decisione e classe $\SPACE$]
    Sia $M$ una MdT a $k$ nastri I/O che decide il problema $I$. 

    Data un'istanza $x \in I$ e la computazione di decisione \[
        (q_0, \ul{\resp} x\eresp, \ul{\resp}\eresp, \dots, \ul{\resp}\eresp)
        \to^{\ast} (\SI, w_1, \dots, w_k)
    \] diremo che lo \sstrong{spazio richiesto} per la decisione dell'istanza $x$ è \[
        \sum_{i = 2}^{k-1} \abs*{w_i}.
        \footnote{Qui $\abs{w_i}$ indica la lunghezza della stringa $w_i$.}
    \]

    Diremo inoltre che $M$ \sstrong{decide $I$ in spazio deterministico $f$} se per ogni $x \in I$
    lo spazio richiesto per la decisione di $x$ è minore o uguale a $f(\abs*{x})$.

    Infine la \sstrong{classe $\SPACE{f}$} è la classe \[
        \SPACE{f} \deq \set*{I \given \exists M 
            \text{ a $k$ nastri I/O che decide $I$ in spazio det. } f
        }.
    \]
\end{definition}

\begin{remark}[Alcune osservazioni sulle definizioni date] \leavevmode
    \begin{itemize}
        \item Lo "spazio richiesto" è dato dalla somma delle lunghezze di tutti i nastri di lavoro,
        ma non del nastro di lettura né di quello di scrittura. 
        
        In effetti il nastro di scrittura 
        non ha molto peso, in quanto contiene solamente la risposta $\SI$ oppure $\NO$. 
        Invece se includessimo il nastro di lettura otterremo che lo spazio è sempre almeno lineare
        nella dimensione dell'input, e questo \emph{appiattirebbe} la gerarchia che cercheremo di
        delineare.
        \item Talvolta lo spazio richiesto viene definito come \[
            \max \set*{\abs*{w_i} \given i = 2, \dots, k-1}.
        \] Siccome \[
            \sum_{i=2}^{k-1} \abs*{w_i} \leq (k-2) \cdot \max_{i=2, \dots, k-2} \abs{w_i}
        \] e noi siamo interessati solamente all'ordine di grandezza, 
        queste definizioni sono equivalenti. 
    \end{itemize}
\end{remark}

Enunciamo un analogo del \Cref{th:linear-accel}.

\begin{theorem}
    [Compressione lineare dello spazio]
    Sia $I \in \SPACE{f}$: allora per ogni $\eps \in [0, +\infty]$ si ha che \[
        I \in \SPACE{2 + \eps\cdot f}.
    \]  
\end{theorem}

Definiamo ora le classi di complessità in spazio che studieremo in questo corso.

\begin{definition}
    Le classi \sstrong{$\PSPACE$} e \sstrong{$\LOGSPACE$} sono definite come segue: \[
        \PSPACE \deq \bigunion_{k \in \N} \SPACE[big]{n^k} \qquad \quad
        \LOGSPACE \deq \bigunion_{k \in \N} \SPACE[\big]{k\log(n)}.
    \]
\end{definition}

Uno dei risultati fondamentali della Teoria della Complessità (che noi non dimostreremo) 
è il seguente: \begin{theorem}
    \[ \LOGSPACE \subsetneq \PSPACE\]
    ovvero esiste un problema risolvibile in spazio lineare che non può essere risolto
    in spazio logaritmico.
\end{theorem}

Per costruire la gerarchia vogliamo scoprire in che relazione è $\P$ con le classi $\LOGSPACE$ e $\PSPACE$.

\begin{theorem}
    \[
        \LOGSPACE \subseteq \P \subseteq \PSPACE.
    \]
\end{theorem}
\begin{proof}
    Il fatto che $\P$ sia un sottoinsieme di $\PSPACE$ è una conseguenza del principio
    generale che in $t$ unità di tempo una MdT può scrivere al più $t$ caselle del suo nastro,
    dunque in tempo polinomiale una MdT può scrivere al più un numero polinomiale di caselle.

    Dimostriamo che $\LOGSPACE \subseteq \P$: sia allora $M$ una MdT a $3$ nastri
    \footnote{Se fossero più nastri basterebbe dividere ad un certo punto per un'opportuna costante.}
    che risolve un problema $I$ in spazio logaritmico, ovvero la lunghezza massima del
    singolo nastro di lavoro è al più $k\log n$, dove $n$ è la taglia dell'input.
    
    Segue che il numero massimo di configurazioni attraversabili da $M$ è al più \[
        N \deq \abs*{\Sigma}^{k\log n} \cdot k\log n \cdot \abs*{Q} \cdot n;
    \] infatti ${\Sigma}^{k\log n}$ è il numero massimo di possibili stringhe scritte sul nastro,
    $k\log n$ è il numero massimo di posizioni del cursore,
    $\abs*{Q}$ è il numero massimo di stati e 
    $n$ è il numero massimo di posizioni del cursore nel nastro di input.

    Vogliamo dimostrare che esiste $t$ tale che $N \leq n^t$, ovvero tale che $\log{N} \leq t \log n$. Per definizione di $N$ però \begin{align*}
        \log N 
        &= \log \parens[\Big]{\abs*{\Sigma}^{k\log n} \cdot k\log n \cdot \abs*{Q} \cdot n}\\
        &= k\log n\log \abs*{\Sigma} + \log (k\log n) + \log \abs*{Q} + \log n,
    \end{align*} dunque dividendo per $\log n$ otteniamo un possibile valore di $t$.
\end{proof}
\section{Non-Determinismo e Complessità non-deterministica}

Studiamo infine un'ultima estensione delle macchine di Turing.

\begin{definition}
    [Macchine di Turing non deterministiche]
    Una \sstrong{macchina di Turing non deterministica} è una quadrupla 
    $N = (Q, \Sigma, \Delta, q_0)$ definita allo stesso modo di una macchina di Turing
    solita tranne per il fatto che \[
        \Delta  \subseteq 
            \parens[\big]{Q \times \Sigma} \times
            \parens[\big]{(Q \union \set*{h}) \times \Sigma \times \set*{\Left, \Right, \blank}}
    \] è una relazione, detta \sstrong{relazione di transizione}.
\end{definition}

La differenza è quindi che a partire da una configurazione vi sono più configurazioni
raggiungibili in output, e la macchina ogni volta ne sceglie una in modo arbitrario
e quindi \emph{non deterministico}.

Questa definizione può sembrare fuori dal mondo, ma è collegata strettamente a due tipi
di strategie di risoluzione dei problemi molto usate.

\newthought{Forza bruta}
Dato un problema, un possibile modo di risolverlo è sempre quello di generare tutte le
possibili soluzioni e controllarle una ad una: tale metodo viene detto \emph{bruteforce} o
metodo del \emph{forza bruta}, in quanto brutalmente tentiamo ogni possibilità.
Generando lo spazio delle soluzioni esplicitamente, ogni volta che ne tentiamo una stiamo in
realtà percorrendo un cammino su tale albero.
Una macchina non deterministica dà un risultato positivo se e solo se esiste
un cammino sull'albero delle computazioni che termini nello stato di accettazione.

\newthought{Guess and Try}
Un altro modo di risolvere problemi è \emph{tirare ad indovinare}: si prova 
una possibilità e si controlla se tale possibilità effettivamente risolve il
problema o meno.
Per quanto tale strategia sembri diversa dal bruteforce, in realtà stiamo 
implicitamente generando tutte le soluzioni.\footnote{specialmente se siamo sfortunati}
Potremmo allora considerare il non determinismo come un metodo per generare
automaticamente una soluzione del problema (assumendo che esista) e poi verificarla.

\begin{definition}
    [Decisione non deterministica e tempo richiesto]
    Una macchina di Turing non deterministica $N$ \sstrong{decide} un problema $I$ 
    se per ogni $x$ vale che $x \in I$ se e solo se esiste una computazione terminante \[
        N(x) \to^{\ast}_N (\SI, w).
    \] $N$ \sstrong{decide $I$ in tempo non deterministico $f$} se decide $I$ e per ogni
    $x \in I$ esiste una computazione terminante \[
        N(x) \to^t_N (\SI, w)
    \] con $t \leq f(\abs x)$. Infine denotiamo con $\NTIME{f}$ la classe di problemi \[
        \NTIME{f} \deq \set*{I \given \exists N \text{ non det. che decide $I$ in tempo non det. $f$}}.
    \]  
\end{definition}

Avendo definito $\NTIME$, vorremmo sapere in che relazione è con la classe
$\TIME$ definita in precedenza.   

\begin{theorem}[Relazione tra $\NTIME$ e $\TIME$][NTIME-subset-TIME-exp]
    $\NTIME{f} \subseteq \TIME*{c^f}$ per qualche costante $c$.  
\end{theorem}
\begin{proof}
    Sia $I$ un problema in $\NTIME{f}$ e sia $M$ una macchina che lo risolve
    in tempo non det. $f$: vogliamo trovare una macchina $M'$ che lo risolve in
    tempo $c^f$ per qualche costante $c$.
    
    Data la relazione di transizione $\Delta$, consideriamo una coppia 
    $(q, \sigma) \in Q \times \Sigma$ e definiamo \begin{gather*}
        \deg (q, \sigma) \deq \card*{\set*{(q', \sigma', D')} \given
            (q, \sigma, q', \sigma', D') \in \Delta} \\
        \deg \Delta \deq \max \set*{ \deg(q, \sigma) \given (q, \sigma) \in
            Q \times \Sigma }.
    \end{gather*} Intuitivamente, $\deg (q, \sigma)$ è il numero di possibili scelte
    che la macchina non deterministica può fare quando legge $\sigma$ nello stato
    $q$, cioè è il fattore di ramificazione dell'albero delle computazioni
    quando ci troviamo in $(q, \sigma)$, mentre $\deg \Delta$ è il massimo tra
    tutti i fattori di ramificazione.
    
    Per semplicità, chiamiamo $d \deq \deg \Delta$. Ordiniamo lessicograficamente
    le quintuple di $\Delta$: a questo punto ogni computazione consiste in una
    sequenza di scelte $(c_1, \dots, c_t)$ dove ognuna delle scelte è necessariamente
    compresa tra $0$ e $d - 1$.
    
    Per simulare la computazione di $M$ attraverso una macchina deterministica, 
    adottiamo una strategia di visita dell'albero delle computazioni a 
    \emph{profondità limitata}: fissiamo inizialmente $t = 1$, $c_1 = 0$ e
    controlliamo se la computazione codificata da $(c_1)$ termina. Se sì bene,
    altrimenti incrementiamo $c_1$ di $1$ e eseguiamo il controllo.
    Arrivati al punto in cui $c_1 = d - 1$ (e quindi non possiamo incrementarlo
    ulteriormente) aumentiamo $t$ e controlliamo tutte le computazioni lunghe $2$.
    
    Dato che $M$ decide $I$ in tempo non deterministico $f$, per ogni input $x$ 
    avremo che il tempo $t$ richiesto per decidere il caso $x \in I$ è al più
    $f(\abs x)$. In particolare ogni sequenza che rappresenta una computazione
    è al più lunga $f(\abs x)$.    
    Segue che la macchina $M'$ deve fare al più $d^{f(\abs x)}$ scelte, da cui
    $I \in \TIME{d^f}$.
\end{proof}

Introduciamo finalmente la classe dei problemi $\NP$ e la sua analoga versione
per lo spazio, $\NPSPACE$. 

\begin{definition}
    [Classe $\NP$]
    Definiamo la classe $\NP$ come \[
        \NP \deq \bigunion_{k \in \N} \NTIME{n^k}.
    \]
\end{definition}

Osserviamo che ovviamente \[
    \P \subseteq \NP
\] in quanto ogni problema risolvibile in tempo polinomiale deterministico
può anche essere risolto non deterministicamente.
La domanda da \emph{un miliardo di dollari}\footnote{Letteralmente.} è se il
contenimento è stretto, ovvero esiste un problema in $\NP$ che non appartiene a
$P$, o se le due classi sono in realtà le stesse.  

\begin{definition}
    [Classe $\NPSPACE$]
    Un problema $I$ si dice \sstrong{decidibile in spazio non deterministico $f$}
    se esiste una macchina di Turing I/O non deterministica $N$ tale che per ogni
    $x$ \emph{esista} una computazione \[
        N(x, \resp, \dots, \resp) \longrightarrow^{\ast}_N (\SI, w_1, \dots, w_k)
    \] tale che \[
        \sum_{i = 2}^{n-1} \abs*{w_i} \leq f(\abs x).
    \] Chiamiamo $\NSPACE{f}$ la classe \[
        \NSPACE{f} \deq \set*{I \given \exists N \text{ non det. che decide } 
        I \text{ in spazio non det. } f}.
    \] Infine, definiamo la classe $\NPSPACE$ come \[
        \NPSPACE \deq \bigunion_{k \in \N} \NSPACE{n^k}.
    \] 
\end{definition}

Analogamente al caso temporale, $\PSPACE \subseteq \NPSPACE$: potremmo dunque
porci lo stesso problema riguardo al contenimento dei due insiemi.
Tuttavia, al contrario del problema $\P \iseq \NP$, il problema $\PSPACE \iseq
\NPSPACE$ è stato risolto.

\begin{theorem}[Teorema di Savitch]
    \[
        \PSPACE = \NPSPACE.
    \]
\end{theorem}

Osserviamo infine che la classe $\NP$ può essere descritta in un modo più naturale
pensandola come un \emph{Guess and Try con i superpoteri}. Infatti, in
ottica Guess and Try, un problema è in $\NP$ se e solo se, avendo indovinato una
soluzione del problema, possiamo verificarla in tempo polinomiale.
Questo ci porta alla definizione dei \sstrong{certificati}: un problema è in $\NP$
se ammette un certificato polinomiale, ovvero se data una soluzione del problema
sappiamo verificare che è tale in tempo polinomiale deterministico.

Tale approccio è, per quanto detto precedentemente, equivalente a chiedere che 
una macchina non deterministica decida il problema in tempo polinomiale, e dunque, 
dato che è più semplice immaginare i certificati piuttosto che il non determinismo,
spesso dimostreremo l'appartenenza ad $\NP$ facendo vedere che i certificati
possono essere verificati in tempo polinomiale.
\section{Funzioni di misura appropriata}

Finora abbiamo studiato le classi $\TIME$ e $\SPACE$ (e le corrispettive
non deterministiche $\NTIME$, $\NSPACE$) con funzioni di misura $f$ qualsiasi.
In realtà per ottenere risultati significativi abbiamo bisogno di alcuni
vincoli in più sulla funzione di misura.

\begin{definition}
  [Funzione di misura appropriata]
  Una funzione $f : \N \to \N$ si dice \sstrong{di misura appropriata} se
  \begin{enumerate}[(1)]
    \item è (debolmente) monotona crescente,
    \item è calcolabile ed esiste una macchina $M$ che calcola $f$ sull'input
      $x$ in \begin{itemize}
        \item tempo $\OO\parens[\big]{f(\abs x) + \abs x}$,
        \item spazio $\OO\parens[\big]{f(\abs x)}$. 
      \end{itemize}  
  \end{enumerate} 
\end{definition}

Fortunatamente i polinomi e le funzioni esponenziali/logaritmiche rispettano
queste condizioni; inoltre si può dimostrare che somma/prodotto/composizione di
funzioni appropriate è ancora appropriata, dunque la teoria delle funzioni di
misura appropriate include i casi a cui siamo davvero interessati.

\begin{theorem}
  [Teorema di Gerarchia][hierarchy]
  Se $f$ è appropriata allora \begin{enumerate}[(1)]
    \item $\TIME{f(n)}  \subsetneq \TIME*{f(2n+1)^3}$,\footnotemark
    \item $\SPACE{f(n)} \subsetneq \SPACE{f(x) \cdot \log(f(x))}$.
  \end{enumerate}
\end{theorem}

\footnotetext{
Sorprendentemente, questo fatto si dimostra facendo vedere che
il sottoinsieme $\set*{x \given \phi_x(x)\conv 
\text{ in al più } f(\abs x) \text{ passi}}$ di $K$ appartiene 
a $\TIME*{f(2n+3)^3}$ ma non a $\TIME{f(n)}$.
} 

Come conseguenza otteniamo i seguenti due teoremi.

\begin{theorem}
  \[
    \P \subsetneq \EXP \deq \bigunion_{k \in \N} \TIME{2^{n^k}}.
  \]
\end{theorem}
\begin{proof}
  Innanzitutto $\P \subseteq \TIME*{2^n}$ poiché i polinomi sono sempre limitati
  superiormente dalle funzioni esponenziali (definitivamente). Per il \nameref{th:hierarchy} segue che $\TIME*{2^n} \subsetneq \TIME*{2^{(2n + 1)^3}}$, che
  è a sua volta un sottoinsieme di $\EXP$. Segue dunque la tesi.  
\end{proof}

\begin{theorem}
  \[
      \P \subseteq \NP \subseteq \EXP.
  \]
\end{theorem}
\begin{proof}
  Il fatto che $\P \subseteq \NP$ è già stato dimostrato; l'inclusione di $\NP$
  in $\EXP$ deriva dal \Cref{th:NTIME-subset-TIME-exp}.
\end{proof}

Le funzioni di misura appropriate ci consentono finalmente di descrivere la
gerarchia di cui parliamo dall'inizio del capitolo.

\begin{theorem}
  Sia $f$ appropriata, $k$ costante. Allora \begin{enumerate}[(1)]
    \item $\TIME{f} \subseteq \NTIME{f}$,
    \item $\SPACE{f} \subseteq \NSPACE{f}$,
    \item $\NTIME{f} \subseteq \TIME*{k^{\log n + f(n)}}$,
    \item vale la gerarchia \[
        \LOGSPACE \subseteq \P \subseteq \NP \subseteq \PSPACE = \NPSPACE.
    \] Inoltre $\NP \subseteq \EXP$. 
  \end{enumerate}
\end{theorem}

La gerarchia definita sopra tuttavia non è superiormente limitata, ovvero
non esiste una classe che include tutti i problemi. In effetti una tale classe
non può esistere, come garantito dal seguente teorema.

\begin{theorem}
  [Illimitatezza della gerarchia di complessità]
  Per ogni funzione $g : \N \to \N$ calcolabile totale\footnote{non 
  necessariamente appropriata} esiste una funzione $f : \N \to \N$ calc. tot.
  tale che \begin{enumerate}[(1)]
    \item esiste un problema $I$ tale che $I \in \TIME{f}$ ma 
      $I \notin \TIME{g}$,
    \item $f \geq g$ definitivamente, ovvero $f(n) \geq g(n)$ per tutti gli
      $n \in \N$ eccetto un numero finito.   
  \end{enumerate} 
\end{theorem}

Concludiamo la discussione delle funzioni di misura appropriate mostrando
quale è il prezzo da pagare se volessimo considerare tutte le funzioni
calcolabili totali, e non solo quelle appropriate. Lo facciamo enunciando
due teoremi, ma senza dimostrarli.

\begin{theorem}
  [Teorema di Accelerazione, Blum]
  Per ogni funzione $h : \N \to \N$ calcolabile totale, ma non appropriata,
  esiste un problema $I$ tale che, per ogni macchina $M$ che decide $I$ 
  in tempo $f$, esiste una macchina $M'$ che decide $I$ in tempo $f'$ con \[
      f \geq h \circ f'
  \] quasi ovunque.
\end{theorem}

Il Teorema di Accelerazione di Blum ci dice che data una funzione non appropriata
$h$ possiamo costruire un problema $I$ che \emph{non ammette algoritmo ottimo}:
il suo tempo di esecuzione può essere ridotto all'infinito seguendo la
successione di macchine $M, M', M'', \dots$ descritto nell'enunciato.

\begin{theorem}
  [Teorema della Lacuna, Borodin][borodin]
  Esiste una funzione calcolabile totale $f$ (non appropriata) tale che \[
      \TIME{f} = \TIME{2^f}.
  \]
\end{theorem}

Se il Teorema della Lacuna fosse ammissibile la gerarchia delineata verrebbe
distrutta: le classi $\P$ ed $\EXP$ collasserebbero e studiare la complessità
sarebbe inutile.

Un ultimo tentativo è quello di condensare le definizioni di complessità
in spazio e tempo usando il cosiddetto \sstrong{approccio assiomatico alla
complessità}, dovuto ancora una volta a Blum.

\begin{definition}
  [Funzioni che misurano la complessità]
  Una funzione $\phi$ \sstrong{misura la complessità} se è della forma \[
      \phi : \parens[\big]{(\N \to \N) \times \N} \to \N
      \footnote{
        Ovvero prende in input una funzione $\N \to \N$ 
        e un numero naturale e restituisce in output un altro numero naturale.
      }
  \] e soddisfa le seguenti due condizioni:
  \begin{enumerate}[(1)]
    \item per ogni $\psi : \N \to \N$, $x \in \N$ si ha che 
      $\phi(\psi, x)$ converge se e solo se $\psi(x)$ converge,
    \item per ogni $\psi : \N \to \N$, $x, k \in \N$ è possibile decidere 
      se $\phi(\psi, x) = k$.  
  \end{enumerate}
\end{definition}

Il primo assioma ci dice che $\phi$ misura la complessità del calcolo di $\psi$,
il secondo ci assicura che è possibile calcolare la complessità usando $\phi$.
Se $\phi$ contasse il numero di passi per calcolare $\psi(x)$ 
otterremmo la misura di complessità in tempo; se contasse lo spazio otterremmo
la complessità in spazio.

Le funzioni di misura appropriata soddisfano gli assiomi, tuttavia esistono
anche funzioni non appropriate che misurano la complessità e ancora una volta
ci portano a situazioni controintuitive (come il \Cref{th:borodin}).
\chapter{Complessità di problemi}

\section{Classificazione di \texorpdfstring{$\P$ e $\NP$}{P e NP}}

Vogliamo ora studiare più nel dettaglio le classi di complessità $\P$ e $\NP$:
esattamente come abbiamo fatto con $\R$ e $\RE$ vogliamo dunque trovare una
riduzione che \emph{classifichi} le classi $\P \subseteq \NP$ e dei problemi
completi per queste due classi.

Per capire perché ciò è interessante è sufficiente enunciare la cosiddetta
\sstrong{Tesi di Cook-Karp}, che rappresenta un analogo nel mondo della
complessità della Tesi di Church-Turing.

\begin{theorem}
  [Tesi di Cook-Karp]
  I problemi in $\P$ sono \emph{trattabili}, 
  i problemi di $\NP$ sono i problemi \emph{intrattabili}. 
\end{theorem}

Possiamo spiegarci la Tesi di Cook-Karp nel seguente modo. \begin{itemize}
  \item I problemi in $\P$ sono chiusi per \emph{cambiamento di
  modello}, ovvero se un problema $P \in \P$ può essere rappresentato in un'altra
  forma $P' \in \P$ allora possiamo sempre trovare un algoritmo di conversione 
  che abbia complessità polinomiale.

  In altre parole, la classe $\P$ è chiusa per \emph{composizione polinomiale
  sinistra}. 
  \item La classe $\P$ è chiusa rispetto a somma/prodotto e alle riduzioni 
  $\leq_{\LL}$ dove $\LL$ è una sottoclasse di problemi. Mostreremo questo fatto
  usando in particolare la sottoclasse $\LL = \LOGSPACE$.

  Riformulando, $\P$ è anche chiusa per \emph{composizione polinomiale destra}.
  \item Inoltre gli algoritmi polinomiali, specialmente se con costanti piccole
  e esponenti bassi, sono davvero (nella pratica) più \emph{trattabili} di
  quelli esponenziali.
\end{itemize}

Tuttavia non sempre gli algoritmi di $\P$ sono migliori di quelli esponenziali, 
soprattutto se le costanti sono molto grandi e i corrispettivi algoritmi 
esponenziali sono in realtà polinomiali nel caso medio, che spesso è quello più
interessante.

Per semplicità noi ci limiteremo comunque al caso pessimo e ignoreremo
tranquillamente i problemi pratici legati alle costanti grandi o agli esponenti
alti.

Come accennato in precedenza, invece di studiare riduzioni polinomiali ci 
limiteremo a studiare un gruppo di riduzioni più piccolo, ovvero quelle 
\emph{logaritmiche}.

\begin{definition}
  [Riduzione efficiente]
  Un problema $I$ si \sstrong{riduce efficientemente} ad un problema $J$ se \[
      I \leq_{\LOGSPACE} J,
  \] ovvero se esiste una funzione $f \in \LOGSPACE$ tale che \[
      x \in I \qquad\text{ se e solo se }\qquad f(x) \in J.
  \] 
\end{definition}

Per semplicità, pigrizia e motivi estetici\footnote{Soprattutto gli ultimi due.}
in seguito scriveremo $\leq_{\LL}$ invece di $\leq_{\LOGSPACE}$.

\begin{theorem}
  Siano $\DD, \EE$ due classi tra $\set*{\LOGSPACE, \P, \NP, \EXP, \PSPACE}$
  tali che $\DD \subseteq \EE$. 
  
  Allora $\leq_{\LL}$ (e quindi a maggior ragione $\leq_{\P}$) 
  classifica $\DD$ ed $\EE$.     
\end{theorem}
\begin{proof}
  Per mostrare che una riduzione classifica due classi di problemi dobbiamo
  mostrare gli assiomi dati in \Cref{def:red-classifies}.
  \begin{enumerate}[(1)]
    \item Certamente $A \leq_{\LL} A$ in quanto l'identità (che copia l'input 
    nell'output) è logaritmica nello spazio di lavoro.\footnote{Non lo usa!}
    \item Vorremmo dimostrare che la composizione di algoritmi logaritmici in
    spazio è ancora logaritmico in spazio, ma non possiamo semplicemente incollare
    le due MdT tra di loro, perché a questo punto l'output della prima diventa
    un nastro di lavoro, e quindi può rendere l'algoritmo polinomiale.

    Possiamo tuttavia operare con uno stile \emph{master-slave}: la seconda
    macchina (che fa la parte del \emph{master}) legge un carattere alla volta
    dalla prima (che fa la parte dello \emph{slave}) e quando ha letto tutto
    l'input esegue i suoi calcoli. In questo modo magari sprechiamo molto tempo,
    ma il risultato è logaritmico in spazio.
    \item Sia $f \in \LOGSPACE$ la funzione che riduce $A$ a $B$, che appartiene
    a $\DD$: per mostrare che $A$ appartiene a $\DD$ basta trasformare $A$ in $\B$
    tramite $f$ e poi risolvere $B$. Dato che $\LOGSPACE \subseteq \DD$ segue che
    questa composizione appartiene ancora a $\DD$, come volevamo.
    \item Analogo al punto precedente, ma usando l'ipotesi $B \in \EE$.           
  \end{enumerate}
\end{proof}

\subsection{Logica proposizionale}

Prima di introdurre i principali problemi che studieremo nell'ambito della
complessità abbiamo bisogno di introdurre alcune semplici nozioni di logica
proposizionale.

\begin{definition}
  [Espressione booleana] Un'\sstrong{espressione booleana} nell'insieme di
  variabili $X$ è una espressione della forma \[
      B \;\Coloneqq\;  
        \TT \mid \FF \mid x \mid \neg B \mid
        B_1 \lor B_2 \mid B_1 \land B_2
  \] dove $x \in X$. Le espressioni $\TT, \FF, x, \neg x$ sono dette 
  \sstrong{letterali}.
\end{definition}

\begin{definition}
  [Interpretazione e soddisfacibilità] 
  Dato un insieme di variabili $X$, un'\sstrong{interpretazione} o
  \sstrong{valutazione} è una funzione \[
      \VV : X \to \set*{\TT, \FF}.
  \] Se $B$ è un'espressione booleana su $X$, diremo che l'interpretazione
  $\VV$ \sstrong{soddisfa} $B$ se e solo se vale il predicato $\VV \models B$
  definito per induzione strutturale: \begin{align*}
    &\VV \models \TT\\
    &\VV \models x              &&\text{se } \VV(x) = \TT\\
    &\VV \models \neg B         &&\text{se non vale } \VV \models B\\
    &\VV \models B_1 \lor B_2 
        &&\text{se } \VV \models B_1 \text{ oppure } \VV \models B_2\\
    &\VV \models B_1 \land B_2 
        &&\text{se } \VV \models B_1 \text{ e } \VV \models B_2.
  \end{align*}  Infine una formula $B$ si dice \sstrong{soddisfacibile}
  se esiste un'interpretazione $\VV$ che soddisfa $B$.  
\end{definition}

\begin{definition}
  [Forma normale congiuntiva]
  Una formula booleana si dice \sstrong{in forma normale congiuntiva} se \[
      B = \bigwedge_i C_i
  \] dove i termini $C_i$ si dicono \sstrong{clausole} o \sstrong{vincoli}
  e sono della forma \[
      C_i = \bigvee_j c_{ij}
  \] e ogni $c_{ij}$ è un letterale.
\end{definition}

Più semplicemente, una formula è in forma normale congiuntiva se è scritta come
congiunzione (cioè \emph{and}) di formule disgiuntive (cioè contententi
solamente \emph{or}): \[
    (c_{11} \lor \dots \lor c_{1,n_1}) \land \dots \land 
    (c_{t1} \lor \dots \lor c_{t,n_t}).
\]
L'importanza delle formule in FNC viene dal seguente teorema.
\begin{theorem}
  Ogni formula può essere espressa equivalentemente in FNC, ovvero data $B$ 
  formula booleana sull'insieme di variabili $X$ esiste $B'$ su $X$ tale che 
  \begin{enumerate}
    \item $B'$ è in FNC
    \item per ogni interpretazione $\VV$ si ha che $\VV \models B$ 
    se e solo se $\VV \models B'$.  
  \end{enumerate}    
\end{theorem}

\subsection{Il problema \SAT}

Definiamo ora il più importante problema di $\NP$: il resto del corso
sarà dedicato a dimostrare che esso è in realtà $\NP$-completo.

\begin{definition}
  [Il problema \SAT]
  \[
      \SAT \deq \set*{B \text{ formula booleana} 
        \given \exists \VV \text{ valutazione tale che } \VV \models B}.
  \]
\end{definition}

\SAT{} appartiene ad $\NP$: infatti data una valutazione $\VV$ sono necessari
un numero polinomiale di passi per certificare che $\VV$ soddisfi $B$.

Confrontiamo \SAT{} con un altro classico problema di $\NP$, che questa volta
viene dalla teoria dei grafi.

\begin{definition}
  [Problema del Ciclo Hamiltoniano]
  Il problema \HAM{} è l'insieme di tutti i grafi $G \deq (V, E)$ che ammettono
  un \sstrong{ciclo hamiltoniano}, ovvero un ciclo che tocca tutti e soli i nodi
  del grafo una ed una sola volta. 
\end{definition}

\begin{theorem}
  \[
      \HAM \leq_{\LL} \SAT.
  \]
\end{theorem}
\begin{proof}
  Dimostrare che $\HAM$ si riduce a $\SAT$ in spazio logaritmico significa
  mostrare che possiamo codificare tutti e soli i grafi che ammettono un ciclo
  hamiltoniano in una formula soddisfacibile, e tale codifica deve essere
  logaritmica in spazio.

  Sia allora $G = (V, E)$ un grafo, $n \deq \card{V}$: costruiamo una
  formula booleana $B$ sull'insieme di variabili \[
      X \set*{ x_{ij} \given i, j = 1, \dots, n }.
  \] Osserviamo che un ciclo hamiltoniano è identificato univocamente da una
  sequenza di vertici $(v_1, \dots, v_n)$, che indica l'ordine di attraversamento 
  dei nodi nel grafo: in altre parole, ogni ciclo hamiltoniano
  è in realtà una funzione bigettiva \[
      \sigma : V \iso \set*{1, \dots, n}.
  \]  
  La variabile $x_{ij}$ sarà $\TT$ se e solo se il nodo $j$ occupa l'$i$-esimo
  posto nella permutazione $(v_1, \dots, v_n)$, ovvero se $\sigma(j) = i$.  

  Vogliamo costruire una formula che sia soddisfacibile se e solo se esiste una
  permutazione dei nodi di $V$ per cui ogni coppia di nodi consecutivi sia
  connessa da un arco: lo facciamo imponendo 5 tipi di vincoli, che verranno
  portati in forma normale congiuntiva.
  
  \begin{enumerate}[(1)]
    \item Per ogni $i \neq k$ aggiungiamo i vincoli \[
        \neg (x_{ij} \land x_{kj}) \;\equiv\; \neg x_{ij} \lor \neg x_{kj}
    \] al variare di $j = 1, \dots, n$. Questi vincoli ci dicono che il nodo $j$ 
    non può occupare contemporaneamente le posizioni $i$ e $k$ della 
    permutazione.
    \item Per ogni $j$ aggiungiamo il vincolo \[
        x_{1j} \lor x_{2j} \lor \dots x_{nj}.
    \] Ciò impone che il nodo $j$ debba essere presente nella
    permutazione in almeno una posizione.
    \item Per ogni $i$ aggiungiamo il vincolo \[
        x_{i1} \lor x_{i2} \lor \dots \lor x_{in}.
    \] In questo modo imponiamo la condizione che in posizione $i$ ci sia
    almeno un nodo. 
    \item Per ogni $j \neq k$ aggiungiamo i vincoli \[
        \neg (x_{ij} \land x_{ik}) \;\equiv\; \neg x_{ij} \lor \neg x_{ik}
    \] al variare di $i = 1, \dots, n$. Questi vincoli ci dicono che nella
    posizione $i$ non possono esserci contemporaneamente i nodi $j$ e $k$.
    \item Per ogni coppia $j, k$ tali che $(j, k) \notin E$ aggiungiamo i 
    vincoli \[
        \neg (x_{ij} \land x_{i+1, k}) \;\equiv\; 
          \neg x_{ij} \lor \neg x_{i+1, k},
    \] per ogni $i$,\footnote{Se $i = n$, allora interpretiamo $i+1$
    \emph{in modulo}, e quindi $i+1 = 1$: questo è necessario
    perché siamo interessati ai \emph{cicli} hamiltoniani.} 
    ovvero i nodi $j, k$ non possono comparire in posizioni
    consecutive della permutazione. 
  \end{enumerate}

  Osserviamo che le prime 4 condizioni in ordine impongono che $\sigma$ sia 
  univalente, totale, surgettiva e iniettiva, ovvero che $\sigma$ sia una
  bigezione. Inoltre se $\sigma$ esiste (cioè il grafo ammette un ciclo
  hamiltoniano) allora per costruzione la valutazione \[
      \VV(x_{ij}) \deq \begin{cases}
        \TT, &\text{se } \sigma(j) = i\\
        \FF, &\text{altrimenti}
      \end{cases}
  \] soddisfa la formula costruita sopra. Invece se $\sigma$ non esistesse
  certamente la formula risulterebbe insoddisfacibile.

  Rimane solo da mostrare che la funzione di riduzione $f$ sia in $\LOGSPACE$:
  tale $f$ prende sul nastro di lavoro il grafo, rappresentato come insieme dei
  nodi e insieme degli archi, e restituisce in output la formula booleana
  che corrisponde ai vincoli necessari per costruire la permutazione.
  Sui nastri di lavoro dobbiamo solo scorrere le variabili $i, j, k$ che
  servono alla costruzione dei vincoli e ricordare il valore di $n$.
  
  Ma allora rappresentando $n, i, j, k$ in binario ognuno di essi occupa al più 
  $\log n$ caselle, dunque per i nastri di lavoro abbiamo bisogno di al più
  $(4\log n)$ caselle, che è logaritmico nella taglia dell'input, come volevamo.
\end{proof}
\section{Circuiti booleani e il Teorema di Cook-Levin}

Per dimostrare che \SAT{} è $\NP$-completo definiremo e dimostreremo un problema
simile, che è il problema \CSAT{} della \emph{soddisfacibilità di un 
circuito booleano}.

\begin{definition}
  [Funzioni e circuiti booleani]
  Una \sstrong{funzione booleana} è una funzione $f : \set*{0, 1}^n \to \set*{0, 1}$.
  
  Un \sstrong{circuito booleano} su un insieme di variabili $X$ 
  è un grafo diretto aciclico $G = (V, E)$ insieme ad una funzione \[
    s : V \to \set*{0, 1, {\neg}, {\lor}, {\land}} \union X
  \] dove
  \begin{itemize}
    \item i nodi si chiamano \sstrong{porte},
    \item $X$ è un insieme di variabili,
    \item esiste un nodo speciale $u \in V$, detto \sstrong{uscita} del circuito, 
    \item la funzione $s$ assegna ad ogni nodo la sua \sstrong{sorta} e soddisfa
      la seguente condizione: per ogni $n \in V$ \begin{enumerate}[(1)]
        \item se $s(n) \in \set{0, 1} \union X$ allora $n$ si dice 
          \sstrong{porta costante} ed ha zero ingressi e un'uscita;
        \item se $s(n) = {\neg}$ allora $n$ ha un ingresso e un'uscita;
        \item se $s(n) \in \set*{{\land}, {\lor}}$ allora $n$ ha due ingressi
          e un'uscita.
      \end{enumerate} Tuttavia se $n = u$ è l'uscita del circuito, il nodo $n$
      ha $0$ uscite (anche se la sorta indicherebbe altrimenti).
  \end{itemize}

  Analogamente alle formule booleane, diremo che $G$ è \sstrong{chiuso} se
  non contiene variabili, cioè se $X = \varnothing$. 
\end{definition}

Osserviamo che, seppure le condizioni sul numero di ingressi siano importanti,
un nodo può avere più\footnote{Ma non meno.} uscite di quelle permesse dalla sua
sorta: possiamo immaginare di duplicare il filo, o di duplicare tutto il 
sottocircuito relativo al nodo da cui vogliamo più uscite.

\begin{definition}
  [Soddisfacibilità di un circuito]
  Sia $G = (V, E)$ un circuito booleano su $X$ con uscita $u \in V$;
  sia inoltre $\VV : X \to \set*{0, 1}$ un'interpretazione. 
  Possiamo estendere $\VV$ ad una funzione \[
      \VV_G : V \to \set*{0, 1}
  \] dove \begin{itemize}
    \item $\VV_G(0) = 0$ e $\VV_G(1) = 1$,
    \item $\VV_G(x) = \VV(x)$ per ogni $x \in X$,
    \item se $n = {\neg}$ e la sua unica entrata è $(n', n) \in E$, vale che
      $\VV_G(n) = 1$ se e solo se $\VV_G(n') = 0$,
    \item se $n = {\land}$ e le sue entrate sono $(n_1, n), (n_2, n) \in E$, vale che
      $\VV_G(n) = 1$ se e solo se $\VV_G(n_1) = 1$ e $\VV_G(n_2) = 1$,  
    \item se $n = {\lor}$ e le sue entrate sono $(n_1, n), (n_2, n) \in E$, vale che
      $\VV_G(n) = 1$ se e solo se $\VV_G(n_1) = 1$ oppure $\VV_G(n_2) = 1$.  
  \end{itemize}

  La valutazione del circuito $G$ sarà quindi $\VV_G(u)$. Diremo infine che
  $\VV$ \sstrong{soddisfa} $G$ ($\VV \models G$) se $\VV_G(u) = 1$.   
\end{definition}

Come nel caso delle formule booleane, sostituendo le variabili di un circuito con
le loro interpretazioni si ottiene un circuito chiuso: in particolare possiamo
parlare della \emph{valutazione di un circuito chiuso} senza dover pensare ad una
particolare interpretazione. In questi casi scriveremo perciò 
$\varnothing \models G$ per dire che il circuito chiuso è soddisfatto.

Questo motiva la prossima definizione.

\begin{definition}
  [\CVAL{} e \CSAT]
  Definiamo i problemi \CVAL{} e \CSAT{} come \begin{gather*}
      \CVAL = \set*{C \text{ circuito booleano chiuso} \given \varnothing \models C}\\
      \CSAT = \set*{C \text{ circuito booleano} \given \exists 
        \VV \text{ tale che } \VV \models C}.
  \end{gather*}
\end{definition}

Il problema \CVAL{} è ovviamente in $\P$: la taglia del problema è il numero di
porte, e per calcolare il valore della porta di uscita ci basta eseguire un'operazione
per porta. Analogamente è facile vedere che \CSAT{} è in $\NP$: dato un certificato,
ovvero una valuazione $\VV$, verificare che $\VV$ soddisfi $C$ è come verificare
che il circuito chiuso $\bar{C}$ ottenuto per sostituzione è soddisfatto, ovvero     
è come verificare che $\bar{C}$ appartenga a \CVAL, che si può fare in tempo
polinomiale perché \CVAL{} appartiene a $\P$.

Inoltre \CVAL{} è banalmente un caso particolare di \CSAT, dunque vale la seguente
riduzione.

\begin{proposition}
  $\CVAL \leq_{\LL} \CSAT$ e la funzione di riduzione è $\id \in \LOGSPACE$. 
\end{proposition}

Delineamo ora il piano per dimostrare l'$\NP$-completezza di \SAT{}.
\begin{enumerate}[(1)]
  \item Innanzitutto dimostreremo che $\CSAT \leq_{\LL} \SAT$: in questo modo
    ci basterà dimostrare che \CSAT{} è $\NP$-completo per ottenere l'analogo
    risultato per \SAT{} (grazie alla \Cref{prop:A-hard=>B-hard}).
  \item Mostreremo poi che \CVAL{} è $\P$-completo: nel far ciò costruiremo la
    \sstrong{tabella di computazione} di un problema, che ci sarà utile per 
    l'ultimo passo.
  \item Infine mostreremo che \CSAT{} è $\NP$-completo, per cui dal passo (1)
    seguirà l'$\NP$-completezza di \SAT. 
\end{enumerate} 

\subsection{\texorpdfstring{$\CSAT \leq_{\LL} \SAT$}
  {CIRCUIT-SAT si riduce efficientemente a SAT}}

\begin{theorem}
  \[
    \CSAT \leq_{\LL} \SAT.
  \]
\end{theorem}
\begin{proof}
  Vogliamo una funzione $f \in \LOGSPACE$ che trasformi circuiti in formule booleane
  e tale che per ogni circuito $C$ \[
      \exists \VV \text{ tale che } \VV \models C 
      \qquad \text{ se e solo se } \qquad
      \exists \VV' \text{ tale che } \VV' \models f(C).
  \]

  Dato un circuito $C$ con variabili $X$, costruiamo una formula booleana con
  variabili \[
      X' \deq X \union \set*{x_g \given g \text{ è una porta di } C}.
  \] Per costruire la formula $f(C)$, costruiamo delle clausole che rappresentano
  i vincoli del circuito booleano $C$. 
  
  In particolare aggiungiamo un vincolo per ogni porta, più uno per l'uscita.
  Sia allora $g \in V$ un nodo.
  \begin{itemize}
    \item Se $g$ ha sorta $1$ (risp. $0$) aggiungiamo la clausola $(x_g)$
      (risp. $(\neg x_g)$).
    \item Se $g$ ha sorta $x \in X$ (ovvero $g$ è una variabile), $x_g$ deve 
      essere vero se e solo se lo è $x$. Portando il "se e solo se" in forma a
      clausole otteniamo due clausole: \[
          (\neg x_g \lor x) \land (x_g \lor \neg x).
      \]
    \item Se $g$ ha sorta $\neg$ esiste un unico arco $(h, g) \in E$: vogliamo
      che $x_g$ sia vero se e solo se $x_h$ è falso. In forma a clausole: \[
          (x_g \lor x_h) \land (\neg x_g \lor \neg x_h).
      \]
    \item Se $g$ ha sorta ${\lor}$ esistono due archi distinti $(h, g), 
      (k, g) \in E$: vogliamo che $x_g$ sia vero se e solo se lo è almeno uno tra
      $x_k$ e $x_h$. Portiamolo in forma a clausole: \begin{align*}
        (x_g \iff x_h \lor x_k) 
          &\equiv (x_g \implies x_h \lor x_k) \land (x_h \lor x_k \implies x_g)\\
          &\equiv (\neg x_g \lor x_h \lor x_k) \land (\neg (x_h \lor x_k) \lor x_g)\\
          &\equiv (\neg x_g \lor x_h \lor x_k) \land 
                  ((\neg x_h \land \neg x_k) \lor x_g)\\
          &\equiv (\neg x_g \lor x_h \lor x_k) \land 
                  (\neg x_h) \land (\neg x_k \lor x_g).
      \end{align*} 
    \item Se $g$ ha sorta ${\land}$ esistono due archi distinti $(h, g), 
      (k, g) \in E$: vogliamo che $x_g$ sia vero se e solo se lo sono $x_k$ e $x_h$. 
      Portiamolo in forma a clausole: \begin{align*}
        (x_g \iff x_h \land x_k) 
          &\equiv (x_g \implies x_h \land x_k) \land (x_h \land x_k \implies x_g)\\
          &\equiv (\neg x_g \lor (x_h \land x_k)) \land (\neg (x_h \land x_k) \lor x_g)\\
          &\equiv (\neg x_g \lor x_h) \land (\neg x_g \lor x_k) \land 
                  (\neg x_h \lor \neg x_k \lor x_g).
      \end{align*} 
  \end{itemize}
  Infine se $g$ è anche l'uscita del circuito aggiungiamo la clausola $(x_g)$,
  in quanto siamo interessati solo alle interpretazioni che rendano vere il circuito.
  
  Tale costruzione è sicuramente ben definita ed è evidente che il circuito $C$
  è soddisfacibile se e solo se la formula $f(C)$ è soddisfacibile, in quanto 
  $f(C)$ è la \emph{riscrittura} di $C$ come formula booleana.
  
  Infine, la costruzione può essere compiuta in spazio logaritmico poiché abbiamo
  bisogno di memorizzare solamente i valori di $g, h, k, n$ (dove $n \deq \card{V}$) 
  che sono tutti logaritmici in $n$.  
\end{proof}

\subsection{Tabella delle computazioni e 
  \texorpdfstring{$\P$-completezza di \CVAL}{P-completezza di CIRCUIT-VALUE}}

Il secondo passo nel nostro "programma" è dimostrare la $\P$-completezza di \CVAL:
anche se questo non ha direttamente peso nella dimostrazione della
$\NP$-completezza di \CSAT, per farlo useremo uno strumento che ci tornerà molto
utile.

Data una macchina di Turing $M$ e un input $x$, vogliamo costruire una matrice
quadrata $T$, detta \sstrong{tabella di computazione}, che riassuma i passi fatti
nel calcolo di $M(x)$: in particolare definiamo $T$ in modo che la sua $i$-esima
riga contenga il nastro di $M$ all'$i$-esimo passo di computazione, e quindi
la posizione $T(i, j)$ contenga la $j$-esima casella del nastro all'$i$-esimo
passo.

Una tabella così definita non racchiude in sé tutta l'informazione necessaria
per capire come si è mossa la macchina: mancano almeno delle informazioni sul 
cursore e sullo stato. 
Iniziamo quindi ad aggiungere qualche vincolo che può tornarci utile. 

\begin{enumerate}
  \item Innanzitutto, siccome lavoreremo con problemi che sono in $\P$, sicuramente
    il numero di passi necessari alla macchina per decidere il problema è minore di
    $\abs{x}^k$ per qualche $k \in \N$. Allora scegliamo $k$ sufficientemente grande,
    in modo che $M$ si arresti in meno di $\abs{x}^k - 2$ passi, e imponiamo che
    $T$ sia di taglia $\abs{x}^k \times \abs*{x}^k$.
  \item Osserviamo che $T$ ha abbastanza colonne per contenere i nastri di $M$: 
    infatti in tempo $t$ si possono scrivere al più $t$ caselle, dunque 
    sicuramente non avremo problemi di spazio. In particolare riempiremo tutte 
    le caselle a destra dell'ultima casella scritta con caratteri vuoti ($\#$):
    siccome la macchina termina la sua computazione in meno di $\abs*{x}^k$ passi, 
    questo ci assicura subito che l'ultima colonna sarà sempre formata da tutti
    caratteri vuoti e non verrà mai raggiunta.
  \item Per memorizzare lo stato e la posizione del cursore cambiamo i simboli
    dell'alfabeto $\Sigma$ in \[
        \Sigma' \deq \Sigma \times (Q \union \set{\ast})
    \] dove $\ast$ è un simbolo a caso che non rappresenti uno stato. 
    Imponiamo allora che tutte le caselle di una riga di $T$ tranne esattamente
    una contengano un simbolo della forma $(\sigma, \ast)$, dove $\sigma \in \Sigma$.
    L'interpretazione che diamo a questa nuova scrittura è la seguente:
    \begin{itemize}
      \item se in una casella c'è un simbolo della forma $(\sigma, \ast)$,
        allora il cursore non si trova in questa posizione e la casella del nastro
        di $M$ contiene il simbolo $\sigma$;
      \item l'unica casella della forma $(\sigma, q)$ con $q \neq \ast$ è la casella
        del nastro contenente il cursore; inoltre tale casella ci indica che
        lo stato della macchina in tale passo è $q \in Q$.   
    \end{itemize}
  \item Facciamo in modo che il cursore della tabella di computazione non sia mai
    sul respingente, ma sia al massimo sulla casella appena a destra. Per far ciò
    potremmo introdurre un secondo respingente, oppure far partire la computazione
    dalla seconda casella e, ogni qualvolta il cursore di $M$ vada sul respingente,
    condensare due passi in uno e rispostarlo sulla casella appena dopo.
  \item Quando la macchina arriva nello stato di arresto ($\SI$ oppure $\NO$)
    aggiungiamo dei passi che la riportano nella seconda casella del nastro.
    Ciò comporta: \begin{itemize}
      \item dover potenzialmente aumentare $k$, poiché abbiamo bisogno di più passi;
      \item passare sopra il respingente in caso ve ne siano alcuni sul nastro.
    \end{itemize} In ogni caso non passeremo mai sul respingente in colonna $1$:
    questo implica che la prima colonna sia fatta soltanto di respingenti.
  \item Infine, una volta arrivati allo stato di arresto e aver portato il cursore
    in seconda colonna, riempiamo tutte le righe successive con il contenuto 
    di quest'ultima riga (per ottenere una matrice quadrata).
\end{enumerate}

Osserviamo ora che per come abbiamo definito la tabella \begin{itemize}
  \item la prima riga è completamente determinata dall'input: contiene infatti
    il respingente in posizione $T(1, 1)$, l'input dalla posizione $T(1, 2)$ e
    infine lo spazio rimanente è occupato da caratteri vuoti;  
  \item la prima colonna contiene solo respingenti;
  \item l'ultima colonna contiene solo caratteri vuoti.
\end{itemize}

Come determiniamo le righe successive alla prima? L'osservazione cruciale è che
per $1 < i, j, < \abs*{x}^k$ la casella $T(i, j)$ dipende solo dalle tre caselle
\[
    T(i-1, j-1), \;\; T(i-1, j) \;\; T(i-1, j+1)
\] e dalla funzione di transizione $\delta$. Infatti \begin{itemize}
  \item il simbolo $\sigma$ scritto nella posizione $j$ può cambiare solo se
    al passo precedente il cursore era già in posizione $j$;
  \item il cursore può arrivare in posizione $j$ solo se al passo precedente
    si trovava in posizione $j-1$, $j$ oppure $j+1$ (quindi in una posizione
    adiacente);
  \item se il cursore al passo $i-1$ si trova in una posizione diversa dalle tre
    appena nominate, sicuramente $T(i, j) = T(i-1, j)$.      
\end{itemize} Infine, se il cursore effettivamente è in una delle posizioni
$T(i-1, j-1)$, $T(i-1, j)$ o $T(i-1, j+1)$, automaticamente conosciamo lo stato
$q$ della macchina $M$ al passo $i-1$ e quindi sappiamo quale mossa deve compiere.

\begin{theorem}
  [$\P$-completezza di \CVAL][CVAL-P-compl]
  Il problema \CVAL{} è $\leq_{\LL}$-completo per $\P$.   
\end{theorem}
\begin{proof}
  Sia $I$ un problema in $\P$, $M$ la macchina che lo risolve in tempo polinomiale
  e $x$ un dato di ingresso: vogliamo una funzione $f$ che trasformi $x$ in un
  circuito chiuso $f(x)$ che sia soddisfatto (dall'assegnamento vuoto) se e solo
  se $x \in I$. Per semplicità chiamiamo anche $n \deq \abs*{x}$. 
  
  \medskip
  \textsf{\color{RoyalBlue} Costruzione della funzione $f$}

  Consideriamo la tabella di computazione di $M(x)$: essa avrà taglia
  $n^{\!k} \times n^{\!k}$ e avrà come simboli elementi di 
  $\Sigma' \deq \Sigma \times Q \union \set{\ast}$ come descritto precedentemente.
  Come primo passo codifichiamo ogni possibile simbolo come una stringa di bit
  \[
      (s_1, \dots, s_m) \in \set*{0, 1}^m,
  \] dove $m \deq \ceil[\big]{\log \abs*{\Sigma'}}$. In questo modo ogni riga
  della tabella di computazione diventa una stringa di bit lunga $m \cdot n^{\!k}$.
  Denotiamo dunque con $s_{i, j, k}$ il $k$-esimo bit del blocco alla riga $i$ e
  colonna $j$. Indicheremo inoltre \[
      S_{i, j} \deq (s_{i, j, 1}, \dots, s_{i, j, m})
  \] ovvero tutti i bit del blocco $T(i, j)$.  
  
  Grazie ai vincoli imposti sulla tabella di computazione, i bit di $S_{i, 1}$
  rappresentano il respingente, mentre i bit di $S_{i, n^{\!k}}$ rappresentano
  il carattere vuoto. Infine la prima riga di bit, ovvero $S_{1, j}$ al variare
  di $j$, è determinata dall'input.
  
  Per i ragionamenti fatti in precedenza sulla forma della tabella di computazione,
  quando $i, j$ rappresenta una posizione centrale della tabella (cioè escludendo
  la prima riga e la prima e l'ultima colonna) si ha che $S_{i, j}$ è univocamente
  determinato da $S_{i-1, j-1}$, $S_{i-1, j}$, $S_{i-1, j+1}$, ovvero esiste una
  funzione $F$ che prende $3m$ bit in ingresso e ne restituisce $m$ che calcola
  le varie posizioni della tabella a partire da quelle immediatamente precedenti.
  
  Tale funzione è booleana\footnote{Abbiamo definito le funzioni booleane solo
  quando l'output ha una sola uscita, ma combinando più funzioni booleane insieme
  otteniamo una versione analoga ma a più uscite.} e dunque può essere rappresentata
  come un circuito, che chiameremo $\bar C$.  
  Osserviamo che il circuito $\bar C$ costruito dipende solamente dalla funzione di
  transizione $\delta$ e \emph{non} dall'input $x$, pertanto ogni copia di $\bar C$
  ha una taglia fissata, \emph{costante} rispetto alla taglia dell'istanza!
  
  Inoltre combinando dei circuiti (rispettando il numero di 
  ingressi ed uscite) si ottiene un nuovo circuito: possiamo allora collegare
  le uscite di tre copie di $\bar C$ agli ingressi di un'ulteriore copia di
  $\bar C$ e ottenere un circuito composto. 
  
  Costruiamo allora $f(x)$ componendo circuiti in questo modo: \begin{itemize}
    \item la prima riga è formata da circuiti costanti che rappresentano la prima
      riga di $T$ (ovvero l'input seguito da $\#$);
    \item la prima colonna è formata da circuiti costanti che rappresentano il
      respingente;
    \item l'ultima colonna è formata da circuiti costanti che rappresentano il
      carattere vuoto $\#$;
    \item le altre posizioni contengono copie del circuito $\bar C$: in particolare
      la copia di $\bar C$ che si trova in posizione $(i, j)$ prende in input
      il risultato dei circuiti che si trovano in posizione $(i-1, j-1)$, $(i-1, j)$
      e $(i-1, j+1)$. In totale avremo quindi bisogno di
      $(n^{\!k} - 1) \times (n^{\!k} - 2)$ copie di $\bar C$.         
  \end{itemize} 
  Inoltre per semplicità chiameremo $C_{i, j}$ la copia di $\bar C$ che si trova
  in posizione $i, j$.   

  \medskip
  \textsf{\color{RoyalBlue} La funzione $f$ è una riduzione}

  Dobbiamo ora convincerci che $x \in I$ se e solo se $f(x) \in \CVAL$, ovvero
  se e solo se il circuito $C_{n^{\!k}, 2}$\footnote{La tabella di computazione dà
  il suo output in posizione $(n^{\!k}, 2)$.} dà come risultato la codifica 
  di $\SI$: dimostriamo allora che $C_{i, j}$ ha come uscita la codifica di $T(i, j)$
  per induzione.
  \begin{description}
    \item[Caso base] Per $i = 1$ è banale, in quanto per definizione abbiamo
      scelto come $C_{1, j}$ il circuito che rappresenta l'input,
      ovvero il simbolo $T(1, j)$. 
    \item[Passo induttivo] Vogliamo dimostrare che l'uscita del circuito
      $C_{i+1, j}$ è una sequenza di bit $r_1, \dots, r_m$ che codifica esattamente
      il simbolo $T_{i + 1, j}$, sapendo che $C_{i, j-1}$, $C_{i, j}$ e $C_{i, j+1}$
      codificano i simboli $T_{i, j-1}$, $T_{i, j}$ e $T_{i, j+1}$ rispettivamente.
      
      Ma le uscite $r_1, \dots, r_m$ di $C_{i+1, j}$ sono calcolate a partire
      dalle uscite di $C_{i, j-1}$, $C_{i, j}$ e $C_{i, j+1}$ tramite la funzione
      booleana $F$, che codifica il comportamento della $\delta$ della macchina
      $M$. Per ipotesi induttiva i tre circuiti al livello $i$ danno codifiche
      corrette dei simboli della tabella di computazione, dunque per correttezza
      di $F$ anche $C_{i+1, j}$ deve essere corretto.       
  \end{description}

  Segue in particolare che per $i = n^{\!k}$ il risultato di $C_{n^{\!k}, 2}$ è la codifica
  di $T_{n^{\!k}, 2}$, che è $\SI$ se e solo se $x \in I$. Dunque $f$ è una riduzione
  da $I$ a \CVAL.

  \medskip
  \textsf{\color{RoyalBlue} La funzione $f$ è logaritmica in spazio}
  
  Infine vogliamo mostrare che $f$ è logaritmica in spazio. Sui nastri di lavoro
  dobbiamo scrivere \begin{itemize}
    \item le porte di ingresso, ovvero l'input: ci basta contare in binario fino
      ad $n^{\!k}$ (che si può fare in spazio logaritmico) e scrivere $x$ seguito da
      caratteri vuoti;
    \item gli elementi della prima e ultima colonna, che sono $2n^{\!k}$ e quindi 
      ancora esprimibili in spazio logaritmico;
    \item le copie del circuito $\bar C$: ognuna di esse occupa spazio costante
      e quindi trascurabile, dunque il costo viene solamente da associare ad ognuna
      i propri indici $i, j$, che in binario possono essere espressi in spazio
      logaritmico in $n$.  
  \end{itemize}
  Segue che $f$ è calcolabile in spazio logaritmico.
\end{proof}


\subsection{Il Teorema di Cook-Levin}

Usando la tabella di computazione definita precedentemente possiamo finalmente
dimostrare che $\CSAT$ è $\NP$-completo.

\begin{theorem}
  [$\NP$-completezza di \CSAT]
  $\CSAT$ è $\leq_{\LL}$-completo per $\NP$.   
\end{theorem}
\begin{proof}
  Sia $I$ un problema in $\NP$, ovvero tale che esista una macchina non deterministica
  $N$ che decide $I$ in tempo polinomiale (non deterministico), ovvero tale che
  per ogni $x$ esista una sequenza di scelte $(s_1, \dots, s_t)$ che porta $N(x)$
  in $\SI$ se $x \in I$, in $\NO$ altrimenti. Inoltre tale $t$ è al più $n^k$, 
  dove $n = \abs x$.
  
  Supponiamo per semplicità che il \sstrong{grado di non determinismo} di $I$ 
  sia al più $2$, ovvero che ad ogni passo si possa scegliere al più tra $2$ 
  quintuple: allora ogni scelta $s_i$ può essere rappresentata con un singolo bit
  $s_i \in \set*{0, 1}$.
  
  Nella realizzazione della tabella di computazione come circuito
  \footnote{vista per bene nella dimostrazione del \Cref{th:CVAL-P-compl}}
  aggiungiamo ad ogni circuito $C_{i, j}$ un ulteriore ingresso, 
  ovvero il bit $s_{i-1}$ che rappresenta la scelta fatta al passo precedente.
  Otterremo quindi un circuito $\CC$ sulle variabili $s_1, \dots, s_{n^{\!k}}$
  che rappresenta il problema $I$. Analogamente a quanto fatto per \CVAL,
  tale trasformazione è una funzione logaritmica in spazio; inoltre è ovvio
  che il circuito $f(x)$ risultante da tale funzione sia soddisfacibile se e solo se
  esiste una sequenza di scelte che mostra che $x \in I$, dunque $f$ riduce
  $I$ a \CSAT.

  Infine se il grado di non determinismo di un nodo è $d > 2$ possiamo 
  ordinare le possibilità da $1$ a $d$ e indicare la scelta fatta 
  tramite una sequenza di $0$ terminata da un $1$: il numero $0$ indicherà 
  il numero di possibilità scartate, l'$1$ indicherà finalmente la scelta compiuta.
  \footnote{Questo ragionamento è equivalente a dire \emph{non questa, non questa,
  non questa, \dots, sì questa}.}
\end{proof}

Da tutto il lavoro fatto finora segue, come preannunciato, il Teorema di Cook-Levin.

\begin{theorem}
  [Teorema di Cook-Levin]
  \SAT{} è $\leq_{\LL}$-completo per $\NP$.   
\end{theorem}

\appendix
\chapter{Esercizi}

\section{Esercizi di calcolabilità}

In questa sezione raccogliamo alcuni esercizi sulla parte di Calcolabilità.

\subsection*{Nessun limite al tempo}

Esiste una $t$ calcolabile totale tale che $t(i, n)$ maggiora il tempo di calcolo di $\phi_i(n)$?

\begin{proof}[Soluzione]
    Supponiamo per assurdo esista una tale $t$ calcolabile totale. Essa dovrà essere necessariamente della forma \[
        t(i, n) \deq \begin{cases}
            k_{i, n}, &\text{se } \phi_i(n) \text{ converge}\\
            0, &\text{se } \phi_i(n) \text{ diverge.} 
        \end{cases}
    \] dove $k_{i, n}$ è maggiore o uguale del numero di passi di computazione effettuati nel calcolo di $\phi_i(n)$. Indicheremo il numero di passi esatti con $T_i(n)$.\footnote{Questa funzione non è necessariamente calcolabile!}
    
    Dato che $t$ è calcolabile possiamo definire \[
        f(x) \deq \begin{cases}
            \phi_x(x) + 1, &\text{se } T_x(x) \leq t(x, x)\\
            0, &\text{altrimenti}.
        \end{cases}
    \] Osserviamo che questa definizione ci dice che $f(x)$ è $\phi_x(x) + 1$ se $\phi_x(x)$ converge, $0$ altrimenti.
    
    Per la Tesi di Church-Turing esiste $i$ tale che $\phi_i = f$. Mostriamo che ciò è assurdo: in effetti $\phi_i(i) \neq f(i)$ in quanto \begin{itemize}
        \item se $\phi_i(i)$ converge, allora $f(i) = \phi_i(i) + 1 \neq \phi_i(i)$;
        \item se $\phi_i(i)$ diverge, allora $f(i) = 0$.    
    \end{itemize}
    Segue che $t$ non può essere calcolabile.  
\end{proof}

Osserviamo che la seconda parte della dimostrazione è identica alla dimostrazione della non-ricorsività di $K$: in alternativa possiamo quindi mostrare che se $t$ fosse calcolabile allora $K$ sarebbe ricorsivo.

\begin{proof}
    [Soluzione alternativa]
    Supponiamo come sopra che esista una $t$ calcolabile e della forma vista. Allora definiamo \[
        f(x) \deq \begin{cases}
            1, &\text{se } t(x, x) > 0\\
            0, &\text{se } t(x, x) = 0.
        \end{cases}
    \] Tale $f$ è certamente calcolabile, in quanto $t$ lo è.
    
    Ma $f$ è esattamente la funzione caratteristica di $K$: se $t(i, i)$ è diverso da $0$ allora la macchina $\phi_i$ calcolata su $i$ termina la sua computazione, e quindi $i \in K$; viceversa se $t(i, i)$ è $0$ allora $\phi_i(i)$ diverge, e quindi $i$ non appartiene a $K$.
    
    Tuttavia $K$ non è ricorsivo, dunque segue l'assurdo.
\end{proof}

\subsection*{Calcolatori con memoria finita in ciclo}

Esiste una funzione calcolabile totale che determina se $\phi_i(x)$ se uno specifico calcolatore $C$ con memoria finita è in ciclo?

% \begin{remark}
    
% \end{remark}

\begin{proof}
    [Soluzione]
    La risposta è sì: siano $Q_C$, $\Sigma_C$, $\delta_C$, $q_0$ le caratteristiche della MdT che modella $C$.
    
    Siano inoltre $m - 1 \deq \card{Q_C}$, $n \deq \card{\Sigma_C}$, $k$ la lunghezza del nastro di $C$. Allora il numero totale di configurazioni è \[
        l \deq k^n \cdot m \cdot k.\footnote{$k^n$ è il numero di possibili nastri, $m$ è il numero di possibiltà per lo stato (incluso il terminatore $h$), $k$ è il numero di possibilità per la testina.}
    \]

    Costruiamo allora una macchina universale $U$ che calcola la macchina $M_i$ su $x$ a partire dallo stato $q_0$ e con un campo aggiuntivo inizialmente inizializzato a $l$: ad ogni passo di $M_i$ diminuiamo di $1$ il valore di $l$.
    
    Alla fine del calcolo possiamo controllare il valore finale di $l$: se è maggiore di $0$ allora ci sono ancora degli stati non visitati, dunque $M_i$ non è in ciclo; altrimenti lo è. 
\end{proof}

\subsection*{Esercizio su insiemi non ricorsivi}

L'insieme $I \deq \set*{i \given \dom{\phi_i} = \set*{3}}$ è ricorsivo?

\begin{proof}
    [Soluzione 1]
    Dimostriamo che $I$ non è ricorsivo applicando l'\nameref{th:index-set-theorem}: abbiamo bisogno di mostrare che $\varnothing \neq I \neq \N$ e che $I$ è un iirf.
    \begin{enumerate}[(1)]
        \item $I$ non è vuoto perché la funzione \[
            \lam{x}{
                \begin{cases}
                    1, &\text{se } x = 3\\
                    \bot, &\text{alrimenti}
                \end{cases}
            }
        \] è calcolabile (ad esempio è facile scrivere un programma \WHILE{} che la calcola) ed ha come dominio esattamente $\set*{3}$.
        \item $I$ non è tutto $\N$ in quanto ad esempio la funzione costantemente uguale a $1$ è calcolabile ma non ha dominio $\set*{3}$.
        \item $I$ è un iirf: supponiamo che $x \in I$ e $y$ sia un altro indice tale che $\phi_x = \phi_y$. Ma allora \[
            \dom{\phi_y} = \dom{\phi_x} = \set*{3}
        \] e dunque $y \in I$. 
    \end{enumerate}

    Per l'\nameref{th:index-set-theorem} segue la tesi.
\end{proof}

\begin{proof}
    [Soluzione 2]
    Mostriamo che $K \leq_{\rec} I$: definiamo \[
        \psi(x, y) \deq \begin{cases}
            1, &\text{se } x \in K \text{ e } y = 3\\
            \bot, &\text{altrimenti.}
        \end{cases}
    \] Per la Tesi di Church-Turing insieme al \nameref{th:s-1-1} e all'\Cref{rem:s-1-1-one-var} esiste una funzione $f$ calcolabile totale tale che $\phi_{f(x)}(y) = \psi(x, y)$.
    
    Allora $K \leq_f I$: in effetti \begin{itemize}
        \item se $x \in K$ allora $\phi_{f(x)}$ vale $1$ quando l'input è $3$ ed è indefinita altrimenti, dunque $\dom{\phi_{f(x)}} = \set*{3}$, ovvero $f(x)$ appartiene ad $I$;
        \item se $x \notin K$ allora $\phi_{f(x)}$ è sempre indefinita e dunque ha dominio vuoto, ovvero $f(x)$ non appartiene a $I$.        
    \end{itemize}

    Segue quindi che $I$ non è ricorsivo.
\end{proof}

\end{document}
\subsection{Il Teorema di Cook-Levin}

Usando la tabella di computazione definita precedentemente possiamo finalmente
dimostrare che $\CSAT$ è $\NP$-completo.

\begin{theorem}
  [$\NP$-completezza di \CSAT]
  $\CSAT$ è $\leq_{\LL}$-completo per $\NP$.   
\end{theorem}
\begin{proof}
  Sia $I$ un problema in $\NP$, ovvero tale che esista una macchina non deterministica
  $N$ che decide $I$ in tempo polinomiale (non deterministico), ovvero tale che
  per ogni $x$ esista una sequenza di scelte $(s_1, \dots, s_t)$ che porta $N(x)$
  in $\SI$ se $x \in I$, in $\NO$ altrimenti. Inoltre tale $t$ è al più $n^k$, 
  dove $n = \abs x$.
  
  Supponiamo per semplicità che il \sstrong{grado di non determinismo} di $I$ 
  sia al più $2$, ovvero che ad ogni passo si possa scegliere al più tra $2$ 
  quintuple: allora ogni scelta $s_i$ può essere rappresentata con un singolo bit
  $s_i \in \set*{0, 1}$.
  
  Nella realizzazione della tabella di computazione come circuito
  \footnote{vista per bene nella dimostrazione del \Cref{th:CVAL-P-compl}}
  aggiungiamo ad ogni circuito $C_{i, j}$ un ulteriore ingresso, 
  ovvero il bit $s_{i-1}$ che rappresenta la scelta fatta al passo precedente.
  Otterremo quindi un circuito $\CC$ sulle variabili $s_1, \dots, s_{n^{\!k}}$
  che rappresenta il problema $I$. Analogamente a quanto fatto per \CVAL,
  tale trasformazione è una funzione logaritmica in spazio; inoltre è ovvio
  che il circuito $f(x)$ risultante da tale funzione sia soddisfacibile se e solo se
  esiste una sequenza di scelte che mostra che $x \in I$, dunque $f$ riduce
  $I$ a \CSAT.

  Infine se il grado di non determinismo di un nodo è $d > 2$ possiamo 
  ordinare le possibilità da $1$ a $d$ e indicare la scelta fatta 
  tramite una sequenza di $0$ terminata da un $1$: il numero $0$ indicherà 
  il numero di possibilità scartate, l'$1$ indicherà finalmente la scelta compiuta.
  \footnote{Questo ragionamento è equivalente a dire \emph{non questa, non questa,
  non questa, \dots, sì questa}.}
\end{proof}

Da tutto il lavoro fatto finora segue, come preannunciato, il Teorema di Cook-Levin.

\begin{theorem}
  [Teorema di Cook-Levin]
  \SAT{} è $\leq_{\LL}$-completo per $\NP$.   
\end{theorem}
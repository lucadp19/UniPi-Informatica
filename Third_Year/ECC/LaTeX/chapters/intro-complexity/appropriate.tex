\section{Funzioni di misura appropriata}

Finora abbiamo studiato le classi $\TIME$ e $\SPACE$ (e le corrispettive
non deterministiche $\NTIME$, $\NSPACE$) con funzioni di misura $f$ qualsiasi.
In realtà per ottenere risultati significativi abbiamo bisogno di alcuni
vincoli in più sulla funzione di misura.

\begin{definition}
  [Funzione di misura appropriata]
  Una funzione $f : \N \to \N$ si dice \sstrong{di misura appropriata} se
  \begin{enumerate}[(1)]
    \item è (debolmente) monotona crescente,
    \item è calcolabile ed esiste una macchina $M$ che calcola $f$ sull'input
      $x$ in \begin{itemize}
        \item tempo $\OO\parens[\big]{f(\abs x) + \abs x}$,
        \item spazio $\OO\parens[\big]{f(\abs x)}$. 
      \end{itemize}  
  \end{enumerate} 
\end{definition}

Fortunatamente i polinomi e le funzioni esponenziali/logaritmiche rispettano
queste condizioni; inoltre si può dimostrare che somma/prodotto/composizione di
funzioni appropriate è ancora appropriata, dunque la teoria delle funzioni di
misura appropriate include i casi a cui siamo davvero interessati.

\begin{theorem}
  [Teorema di Gerarchia][hierarchy]
  Se $f$ è appropriata allora \begin{enumerate}[(1)]
    \item $\TIME{f(n)}  \subsetneq \TIME*{f(2n+1)^3}$,\footnotemark
    \item $\SPACE{f(n)} \subsetneq \SPACE{f(x) \cdot \log(f(x))}$.
  \end{enumerate}
\end{theorem}

\footnotetext{
Sorprendentemente, questo fatto si dimostra facendo vedere che
il sottoinsieme $\set*{x \given \phi_x(x)\conv 
\text{ in al più } f(\abs x) \text{ passi}}$ di $K$ appartiene 
a $\TIME*{f(2n+3)^3}$ ma non a $\TIME{f(n)}$.
} 

Come conseguenza otteniamo i seguenti due teoremi.

\begin{theorem}
  \[
    \P \subsetneq \EXP \deq \bigunion_{k \in \N} \TIME{2^{n^k}}.
  \]
\end{theorem}
\begin{proof}
  Innanzitutto $\P \subseteq \TIME*{2^n}$ poiché i polinomi sono sempre limitati
  superiormente dalle funzioni esponenziali (definitivamente). Per il \nameref{th:hierarchy} segue che $\TIME*{2^n} \subsetneq \TIME*{2^{(2n + 1)^3}}$, che
  è a sua volta un sottoinsieme di $\EXP$. Segue dunque la tesi.  
\end{proof}

\begin{theorem}
  \[
      \P \subseteq \NP \subseteq \EXP.
  \]
\end{theorem}
\begin{proof}
  Il fatto che $\P \subseteq \NP$ è già stato dimostrato; l'inclusione di $\NP$
  in $\EXP$ deriva dal \Cref{th:NTIME-subset-TIME-exp}.
\end{proof}

Le funzioni di misura appropriate ci consentono finalmente di descrivere la
gerarchia di cui parliamo dall'inizio del capitolo.

\begin{theorem}
  Sia $f$ appropriata, $k$ costante. Allora \begin{enumerate}[(1)]
    \item $\TIME{f} \subseteq \NTIME{f}$,
    \item $\SPACE{f} \subseteq \NSPACE{f}$,
    \item $\NTIME{f} \subseteq \TIME*{k^{\log n + f(n)}}$,
    \item vale la gerarchia \[
        \LOGSPACE \subseteq \P \subseteq \NP \subseteq \PSPACE = \NPSPACE.
    \] Inoltre $\NP \subseteq \EXP$. 
  \end{enumerate}
\end{theorem}

La gerarchia definita sopra tuttavia non è superiormente limitata, ovvero
non esiste una classe che include tutti i problemi. In effetti una tale classe
non può esistere, come garantito dal seguente teorema.

\begin{theorem}
  [Illimitatezza della gerarchia di complessità]
  Per ogni funzione $g : \N \to \N$ calcolabile totale\footnote{non 
  necessariamente appropriata} esiste una funzione $f : \N \to \N$ calc. tot.
  tale che \begin{enumerate}[(1)]
    \item esiste un problema $I$ tale che $I \in \TIME{f}$ ma 
      $I \notin \TIME{g}$,
    \item $f \geq g$ definitivamente, ovvero $f(n) \geq g(n)$ per tutti gli
      $n \in \N$ eccetto un numero finito.   
  \end{enumerate} 
\end{theorem}

Concludiamo la discussione delle funzioni di misura appropriate mostrando
quale è il prezzo da pagare se volessimo considerare tutte le funzioni
calcolabili totali, e non solo quelle appropriate. Lo facciamo enunciando
due teoremi, ma senza dimostrarli.

\begin{theorem}
  [Teorema di Accelerazione, Blum]
  Per ogni funzione $h : \N \to \N$ calcolabile totale, ma non appropriata,
  esiste un problema $I$ tale che, per ogni macchina $M$ che decide $I$ 
  in tempo $f$, esiste una macchina $M'$ che decide $I$ in tempo $f'$ con \[
      f \geq h \circ f'
  \] quasi ovunque.
\end{theorem}

Il Teorema di Accelerazione di Blum ci dice che data una funzione non appropriata
$h$ possiamo costruire un problema $I$ che \emph{non ammette algoritmo ottimo}:
il suo tempo di esecuzione può essere ridotto all'infinito seguendo la
successione di macchine $M, M', M'', \dots$ descritto nell'enunciato.

\begin{theorem}
  [Teorema della Lacuna, Borodin][borodin]
  Esiste una funzione calcolabile totale $f$ (non appropriata) tale che \[
      \TIME{f} = \TIME{2^f}.
  \]
\end{theorem}

Se il Teorema della Lacuna fosse ammissibile la gerarchia delineata verrebbe
distrutta: le classi $\P$ ed $\EXP$ collasserebbero e studiare la complessità
sarebbe inutile.

Un ultimo tentativo è quello di condensare le definizioni di complessità
in spazio e tempo usando il cosiddetto \sstrong{approccio assiomatico alla
complessità}, dovuto ancora una volta a Blum.

\begin{definition}
  [Funzioni che misurano la complessità]
  Una funzione $\phi$ \sstrong{misura la complessità} se è della forma \[
      \phi : \parens[\big]{(\N \to \N) \times \N} \to \N
      \footnote{
        Ovvero prende in input una funzione $\N \to \N$ 
        e un numero naturale e restituisce in output un altro numero naturale.
      }
  \] e soddisfa le seguenti due condizioni:
  \begin{enumerate}[(1)]
    \item per ogni $\psi : \N \to \N$, $x \in \N$ si ha che 
      $\phi(\psi, x)$ converge se e solo se $\psi(x)$ converge,
    \item per ogni $\psi : \N \to \N$, $x, k \in \N$ è possibile decidere 
      se $\phi(\psi, x) = k$.  
  \end{enumerate}
\end{definition}

Il primo assioma ci dice che $\phi$ misura la complessità del calcolo di $\psi$,
il secondo ci assicura che è possibile calcolare la complessità usando $\phi$.
Se $\phi$ contasse il numero di passi per calcolare $\psi(x)$ 
otterremmo la misura di complessità in tempo; se contasse lo spazio otterremmo
la complessità in spazio.

Le funzioni di misura appropriata soddisfano gli assiomi, tuttavia esistono
anche funzioni non appropriate che misurano la complessità e ancora una volta
ci portano a situazioni controintuitive (come il \Cref{th:borodin}).
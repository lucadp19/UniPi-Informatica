\section{Non-Determinismo e Complessità non-deterministica}

Studiamo infine un'ultima estensione delle macchine di Turing.

\begin{definition}
    [Macchine di Turing non deterministiche]
    Una \sstrong{macchina di Turing non deterministica} è una quadrupla 
    $N = (Q, \Sigma, \Delta, q_0)$ definita allo stesso modo di una macchina di Turing
    solita tranne per il fatto che \[
        \Delta  \subseteq 
            \parens[\big]{Q \times \Sigma} \times
            \parens[\big]{(Q \union \set*{h}) \times \Sigma \times \set*{\Left, \Right, \blank}}
    \] è una relazione, detta \sstrong{relazione di transizione}.
\end{definition}

La differenza è quindi che a partire da una configurazione vi sono più configurazioni
raggiungibili in output, e la macchina ogni volta ne sceglie una in modo arbitrario
e quindi \emph{non deterministico}.

Questa definizione può sembrare fuori dal mondo, ma è collegata strettamente a due tipi
di strategie di risoluzione dei problemi molto usate.

\newthought{Forza bruta}
Dato un problema, un possibile modo di risolverlo è sempre quello di generare tutte le
possibili soluzioni e controllarle una ad una: tale metodo viene detto \emph{bruteforce} o
metodo del \emph{forza bruta}, in quanto brutalmente tentiamo ogni possibilità.
Generando lo spazio delle soluzioni esplicitamente, ogni volta che ne tentiamo una stiamo in
realtà percorrendo un cammino su tale albero: la scelta della prossima soluzione da tentare
corrisponde alla \emph{scelta non deterministica} da compiere.

% \newthought{Guess and Try}

\begin{definition}
    [Decisione non deterministica e tempo richiesto]
    Una macchina di Turing non deterministica $N$ \sstrong{decide} un problema $I$ 
    se per ogni $x$ vale che $x \in I$ se e solo se esiste una computazione terminante \[
        N(x) \to^{\ast}_N (\SI, w).
    \] $N$ \sstrong{decide $I$ in tempo non deterministico $f$} se decide $I$ e per ogni
    $x \in I$ esiste una computazione terminante \[
        N(x) \to^t_N (\SI, w)
    \] con $t \leq f(\abs x)$. Infine denotiamo con $\NTIME{f}$ la classe di problemi \[
        \NTIME{f} \deq \set*{I \given \exists N \text{ non det. che decide $I$ in tempo non det. $f$}}.
    \]  
\end{definition}

Avendo definito $\NTIME$, vorremmo sapere in che relazione è con la classe
$\TIME$ definita in precedenza.   

\begin{theorem}[Relazione tra $\NTIME$ e $\TIME$][NTIME-subset-TIME-exp]
    $\NTIME{f} \subseteq \TIME*{c^f}$ per qualche costante $c$.  
\end{theorem}
\begin{proof}
    Sia $I$ un problema in $\NTIME{f}$ e sia $M$ una macchina che lo risolve
    in tempo non det. $f$: vogliamo trovare una macchina $M'$ che lo risolve in
    tempo $c^f$ per qualche costante $c$.
    
    Data la relazione di transizione $\Delta$, consideriamo una coppia 
    $(q, \sigma) \in Q \times \Sigma$ e definiamo \begin{gather*}
        \deg (q, \sigma) \deq \card*{\set*{(q', \sigma', D')} \given
            (q, \sigma, q', \sigma', D') \in \Delta} \\
        \deg \Delta \deq \max \set*{ \deg(q, \sigma) \given (q, \sigma) \in
            Q \times \Sigma }.
    \end{gather*} Intuitivamente, $\deg (q, \sigma)$ è il numero di possibili scelte
    che la macchina non deterministica può fare quando legge $\sigma$ nello stato
    $q$, cioè è il fattore di ramificazione dell'albero delle computazioni
    quando ci troviamo in $(q, \sigma)$, mentre $\deg \Delta$ è il massimo tra
    tutti i fattori di ramificazione.
    
    Per semplicità, chiamiamo $d \deq \deg \Delta$. Ordiniamo lessicograficamente
    le quintuple di $\Delta$: a questo punto ogni computazione consiste in una
    sequenza di scelte $(c_1, \dots, c_t)$ dove ognuna delle scelte è necessariamente
    compresa tra $0$ e $d - 1$.
    
    Per simulare la computazione di $M$ attraverso una macchina deterministica, 
    adottiamo una strategia di visita dell'albero delle computazioni a 
    \emph{profondità limitata}: fissiamo inizialmente $t = 1$, $c_1 = 0$ e
    controlliamo se la computazione codificata da $(c_1)$ termina. Se sì bene,
    altrimenti incrementiamo $c_1$ di $1$ e eseguiamo il controllo.
    Arrivati al punto in cui $c_1 = d - 1$ (e quindi non possiamo incrementarlo
    ulteriormente) aumentiamo $t$ e controlliamo tutte le computazioni lunghe $2$.
    
    Dato che $M$ decide $I$ in tempo non deterministico $f$, per ogni input $x$ 
    avremo che il tempo $t$ richiesto per decidere il caso $x \in I$ è al più
    $f(\abs x)$. In particolare ogni sequenza che rappresenta una computazione
    è al più lunga $f(\abs x)$.    
    Segue che la macchina $M'$ deve fare al più $d^{f(\abs x)}$ scelte, da cui
    $I \in \TIME{d^f}$.
\end{proof}

Introduciamo finalmente la classe dei problemi $\NP$ e la sua analoga versione
per lo spazio, $\NPSPACE$. 

\begin{definition}
    [Classe $\NP$]
    Definiamo la classe $\NP$ come \[
        \NP \deq \bigunion_{k \in \N} \NTIME{n^k}.
    \]
\end{definition}

Osserviamo che ovviamente \[
    \P \subseteq \NP
\] in quanto ogni problema risolvibile in tempo polinomiale deterministico
può anche essere risolto non deterministicamente.
La domanda da \emph{un miliardo di dollari}\footnote{Letteralmente.} è se il
contenimento è stretto, ovvero esiste un problema in $\NP$ che non appartiene a
$P$, o se le due classi sono in realtà le stesse.  

\begin{definition}
    [Classe $\NPSPACE$]
    Un problema $I$ si dice \sstrong{decidibile in spazio non deterministico $f$}
    se esiste una macchina di Turing I/O non deterministica $N$ tale che per ogni
    $x$ \emph{esista} una computazione \[
        N(x, \resp, \dots, \resp) \longrightarrow^{\ast}_N (\SI, w_1, \dots, w_k)
    \] tale che \[
        \sum_{i = 2}^{n-1} \abs*{w_i} \leq f(\abs x).
    \] Chiamiamo $\NSPACE{f}$ la classe \[
        \NSPACE{f} \deq \set*{I \given \exists N \text{ non det. che decide } 
        I \text{ in spazio non det. } f}.
    \] Infine, definiamo la classe $\NPSPACE$ come \[
        \NPSPACE \deq \bigunion_{k \in \N} \NSPACE{n^k}.
    \] 
\end{definition}

Analogamente al caso temporale, $\PSPACE \subseteq \NPSPACE$: potremmo dunque
porci lo stesso problema riguardo al contenimento dei due insiemi.
Tuttavia, al contrario del problema $\P \iseq \NP$, il problema $\PSPACE \iseq
\NPSPACE$ è stato risolto.

\begin{theorem}[Teorema di Savitch]
    \[
        \PSPACE = \NPSPACE.
    \]
\end{theorem}

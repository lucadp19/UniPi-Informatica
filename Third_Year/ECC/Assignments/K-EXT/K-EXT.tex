\documentclass[
    a4paper,
    language=italian,
    oneside,
    10pt,
    article,
    thmstyle=professional
]{notes}
\counterwithout{section}{chapter}
\counterwithout{theorem}{section}

\renewcommand{\compl}[1]{\overline{#1}}
\renewcommand{\K}{\normalfont\textsf{K}}
\NewDocumentCommand{\red}{g}{
    \leq
    \IfValueTF{#1}
        { _{#1} }
        { _{\rec} }
}
\newcommand{\rec}{\normalfont\texttt{rec}}
\newcommand{\EXT}{\normalfont\textsf{EXT}}

\newcommand{\indef}{\text{indef.}}
\newcommand{\altr}{\text{altrimenti}}

\title{Riduzioni di $K$ e $\compl{K}$ a \EXT}
\author{Luca De Paulis}

\begin{document}
    \maketitle

    \section{Risultati preliminari}

    Prima di mostrare le riduzioni $K \red \EXT$ e $\compl{K} \red \EXT$
    introduciamo per comodità la funzione $\nu : \N \to \N$ tale che 
    \begin{itemize}[label={\tiny\raisebox{0.4ex}{\textbullet}}]
        \item se $n \in K$, $\nu(n)$ è il minimo numero intero positivo
            tale che la computazione $\phi_n(n)$ converge 
            in $\nu(n)$ passi,
        \item se $n \notin K$, $\nu(n)$ è indefinito. 
    \end{itemize}

    La funzione $\nu$ è calcolabile: si esegue il calcolo di $\phi_n(n)$;
    se converge in un numero finito di passi si prende come risultato
    il numero minimo di passi per cui la funzione converge,
    altrimenti si continua all'infinito e quindi $\nu(n)$ diverge.

    \begin{lemma}[label=lem:not-ext]
        La funzione $\psi : \N \to \N$ definita da \[
            \psi(n) \deq \begin{cases}
                \nu(n), &n \in K\\
                \indef, &\altr
            \end{cases}
        \] non è estendibile ad una funzione calcolabile totale.
    \end{lemma}
    \begin{proof}
        Supponiamo per assurdo che esista $g : \N \to \N$ calcolabile
        totale tale che $g(n) = \nu(n)$ per ogni $n \in K$.
        Dato che $g$ è totale, per ogni $n \notin K$ 
        il calcolo di $g(n)$ dovrà convergere ad un numero naturale,
        e dunque necessariamente $g \neq \nu$ al di fuori di $K$.   
        
        Possiamo allora costruire la funzione $f : \N \to \set{0,1}$
        definita da \begin{align*}
            f(n) 
            \deq \begin{cases}
                1, &\text{se $\phi_n(n)$ converge in $g(n)$ passi}\\
                0, &\text{se $\phi_n(n)$ non converge in $g(n)$ passi}.
            \end{cases}
        \end{align*}
        Tale funzione è certamente calcolabile: basta eseguire il calcolo
        di $\phi_n(n)$ per $g(n)$ passi e controllare se si è arrivati
        alla convergenza oppure no.
        Notiamo inoltre che la prima condizione equivale a chiedere 
        che $g(n)$ sia uguale a $\nu(n)$, 
        la seconda invece equivale a chiedere che siano diversi.
        
        Tuttavia $g(n) = \nu(n)$ se e solo se $n \in K$, dunque $f$ vale
        $1$ se $n \in K$ e $0$ altrimenti, cioè $f$ è la funzione caratteristica di $K$, che però non è calcolabile. Segue l'assurdo.
    \end{proof}

    \section{Riduzioni}
    Usando il \Cref{lem:not-ext} possiamo mostrare l'esistenza di riduzioni
    di $K$ e $\compl{K}$ a $\EXT$.
    
    \begin{proof}[Riduzione di $K$ a $\EXT$]
        Consideriamo la funzione $\psi : \N^2 \to \N$ definita da \[
            \psi(x, y) \deq \begin{cases}
                \min\set{\nu(x), \nu(y)}, &\text{se $x \in K$ oppure $y \in K$}\\
                \indef &\altr
            \end{cases}
        \] dove $\min\set{\nu(x), \nu(y)}$ è $\nu(x)$ se $\nu(y)$ è indefinito, e viceversa.

        Tale funzione è intuitivamente calcolabile: eseguiamo
        contemporaneamente la computazione di $\phi_x(x)$ e $\phi_y(y)$.
        Se una delle due si ferma prima dell'altra diamo come 
        risultato il numero di passi impiegati per il calcolo della
        computazione terminante; altrimenti il calcolo di $\psi$ non termina.
        
        Per la Tesi di Church-Turing insieme al Teorema del Parametro
        segue che esiste una funzione calcolabile totale $f : \N \to \N$
        tale che \[
            \phi_{f(x)}(y) = \psi(x, y)
        \] per ogni $y \in \N$. Mostriamo che $K \red{f} \EXT$.
        \begin{itemize}[label={\tiny\raisebox{1ex}{\textbullet}}]
            \item Se $x \in K$ allora \[
                \phi_{f(x)}(y) = \min\set{\nu(x), \nu(y)}
            \] e dunque è quantomeno totale. In particolare è banalmente
            estendibile ad una funzione calcolabile totale, 
            e quindi $f(x) \in \EXT$. 
            \item Se $x \notin K$ allora \[
                \phi_{f(x)}(y) = \begin{cases}
                    \nu(y), &\text{se $y \in K$}\\
                    \indef  &\altr.
                \end{cases}
            \] Per il \Cref{lem:not-ext} questa funzione non è estendibile
            ad una funzione calcolabile totale, e dunque $f(x) \notin \EXT$. \qedhere
        \end{itemize}  
    \end{proof}

    \medskip

    \begin{proof}[Riduzione di $\compl{K}$ a $\EXT$]
        Consideriamo la funzione $\psi : \N^2 \to \N$ definita da \[
            \psi(x, y) \deq \begin{cases}
                \nu(y), &\text{se $x \in K$ e $y \in K$}\\
                \indef &\altr.
            \end{cases}
        \]
        Tale funzione è intuitivamente calcolabile: eseguiamo in sequenza
        il calcolo di $\phi_x(x)$ e poi quello di $\phi_y(y)$, misurando
        il numero di passi necessari per la terminazione del calcolo della
        seconda. Se entrambe le computazioni terminano restituiamo come
        risultato $\nu(y)$, altrimenti $\psi$ è indefinita.   
        
        Per la Tesi di Church-Turing insieme al Teorema del Parametro
        segue che esiste una funzione calcolabile totale $f : \N \to \N$
        tale che \[
            \phi_{f(x)}(y) = \psi(x, y)
        \] per ogni $y \in \N$. Mostriamo che $\compl{K} \red{f} \EXT$.
        \begin{itemize}[label={\tiny\raisebox{1ex}{\textbullet}}]
            \item Se $x \in \compl{K}$ allora $\phi_{f(x)}(y)$ è la
            funzione costantemente indefinita, che è estendibile ad esempio
            ad una qualunque funzione costante.
            \item Se $x \notin \compl{K}$ allora \[
                \phi_{f(x)}(y) = \begin{cases}
                    \nu(y), &\text{se $y \in K$}\\
                    \indef  &\altr.
                \end{cases}
            \] Per il \Cref{lem:not-ext} questa funzione non è estendibile
            ad una funzione calcolabile totale, e dunque $f(x) \notin \EXT$. \qedhere
        \end{itemize}  
    \end{proof}
\end{document}

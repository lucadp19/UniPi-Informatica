\documentclass[a4paper]{report}
    \usepackage[utf8]{inputenc}
    \usepackage[italian]{babel}
    \usepackage[T1]{fontenc}
    \usepackage{textcomp, microtype}
    \usepackage{amsmath, amsthm, amssymb, cases, mathtools}
    \usepackage{float}

    \usepackage{hyperref} % ultimo package da caricare!

\restylefloat{table}

\theoremstyle{plain}
\newtheorem{theorem}{Teorema}[section]
\newtheorem{corollary}[theorem]{Corollario}
\newtheorem{proposition}[theorem]{Proposizione}

\theoremstyle{definition}
\newtheorem{example}[theorem]{Esempio}
\newtheorem{definition}[theorem]{Definizione}

\theoremstyle{remark}
\newtheorem*{remark}{Osservazione}
\newtheorem*{solution}{Soluzione}
\newtheorem*{intuition}{Intuizione}

\DeclareMathOperator{\tc}{\text{ tale che }}

\newcommand{\divides}{\mid}
\newcommand{\Mod}[1]{\ \left(#1\right)}
\newcommand{\abs}[1]{\left|#1\right|}
\newcommand{\mcm}[2]{\operatorname{mcm}\left(#1, #2\right)}
\newcommand{\mcd}[2]{\operatorname{mcd}\left(#1, #2\right)}
\newcommand{\N}{\mathbb{N}}
\newcommand{\Z}{\mathbb{Z}}
\newcommand{\Q}{\mathbb{Q}}
\newcommand{\R}{\mathbb{R}}
\newcommand{\C}{\mathbb{C}}

\begin{document}

% \author{Luca De Paulis}
\title{Matematica Discreta}
\maketitle

\tableofcontents

\chapter{Insiemi numerici}

\section{Strutture algebriche fondamentali}

\begin{definition}[Gruppo]
    Si dice \textbf{gruppo} una tripla ($G$, $\cdot$, $e$) formata da \begin{itemize}
        \item un insieme di elementi $G$;
        \item un operazione $\cdot : A \times A \to A$ detta prodotto;
        \item un elemento $e \in G$
    \end{itemize} per cui valgono i seguenti assiomi: 
    \begin{description}
        \item[(Assiomi di gruppo)] Per ogni $a, b, c \in G$ vale che
        \begin{align*}
            &\text{(P1)}      &&(ab) \in G            &\text{(chiusura rispetto a $\cdot$)}\\
            &\text{(P2)}      &&(ab)c = a(bc)         &\text{(associatività di $\cdot$)}\\
            &\text{(P3)}      &&a \cdot e=e \cdot a=a &\text{($e$ el. neutro di $\cdot$)}\\
            &\text{(P4)}     &&\exists a^{-1} \in G. \quad aa^{-1} = e &\text{(inverso per $\cdot$)}
            \intertext{Si dice \textbf{gruppo commutativo} un gruppo per cui vale inoltre il seguente assioma:}
            &\text{(P5)}     &&ab = ba               &\text{(commutatività di $\cdot$)}
        \end{align*}
    \end{description}
\end{definition}

\begin{definition}[Anello]
    Si dice \textbf{anello} una quintupla ($A$, $+$, $\cdot$, $0$, $1$) formata da
    \begin{itemize}
        \item un insieme di elementi $A$;
        \item un operazione $+ : A \times A \to A$ detta somma;
        \item un operazione $\cdot : A \times A \to A$ detta prodotto;
        \item un elemento $0 \in A$;
        \item un elemento $1 \in A$
    \end{itemize} per cui valgono i seguenti assiomi: 
    \begin{description}
        \item[(Assiomi di anello)] Per ogni $a, b, c \in A$ vale che
        \begin{align*}
            &\text{(S1)}      &&(a+b) \in A           &\text{(chiusura rispetto a $+$)}\\
            &\text{(S2)}      &&a+b = b+a             &\text{(commutatività di $+$)}\\
            &\text{(S3)}      &&(a+b)+c = a+(b+c)     &\text{(associatività di $+$)}\\
            &\text{(S4)}      &&a+0=0+a=a             &\text{(0 el. neutro di $+$)}\\
            &\text{(S5)}      &&\exists (-a) \in A. \quad a+(-a) = 0 &\text{(opposto per $+$)}\\
            &\text{(P1)}      &&(ab) \in A            &\text{(chiusura rispetto a $\cdot$)}\\
            &\text{(P2)}      &&(ab)c = a(bc)         &\text{(associatività di $\cdot$)}\\
            &\text{(P3)}      &&a \cdot 1=1 \cdot a=a &\text{(1 el. neutro di $\cdot$)}\\
            &\text{(P4)}      &&(a+b)c = ac + bc      &\text{(distributività 1)} \\
            &\text{(P5)}     &&a(b+c) = ab + ac      &\text{(distributività 2)}
            \intertext{Si dice \textbf{anello commutativo} un anello per cui vale inoltre il seguente assioma:}
            &\text{(P6)}     &&ab = ba               &\text{(commutatività di $\cdot$)}
        \end{align*}
    \end{description} 
\end{definition}

Un tipico esempio di anello commutativo è $\Z$: infatti gli anelli generalizzano le operazioni che possiamo fare sui numeri interi e le loro proprietà fondamentali per estenderle ad altri insiemi con la stessa struttura algebrica.

\begin{definition}[Campo]
    Si dice \textbf{campo} una quintupla ($F$, $+$, $\cdot$, $0$, $1$) formata da
    \begin{itemize}
        \item un insieme di elementi $F$;
        \item un operazione $+ : F \times F \to F$ detta somma;
        \item un operazione $\cdot : F \times F \to F$ detta prodotto;
        \item un elemento $0 \in F$;
        \item un elemento $1 \in F$
    \end{itemize}  per cui valgono i seguenti assiomi: 
    \begin{description}
        \item[(Assiomi di campo)] Per ogni $a, b, c \in F$ vale che
        \begin{align*}
            &\text{(S1)}      &&(a+b) \in F           &\text{(chiusura rispetto a $+$)}\\
            &\text{(S2)}      &&a+b = b+a             &\text{(commutatività di $+$)}\\
            &\text{(S3)}      &&(a+b)+c = a+(b+c)     &\text{(associatività di $+$)}\\
            &\text{(S4)}      &&a+0=0+a=a             &\text{(0 el. neutro di $+$)}\\
            &\text{(S5)}      &&\exists (-a) \in F. \quad a+(-a) = 0 &\text{(opposto per $+$)}\\
            &\text{(P1)}      &&(ab) \in F            &\text{(chiusura rispetto a $\cdot$)}\\        
            &\text{(P2)}      &&ab = ba               &\text{(commutatività di $\cdot$)}\\
            &\text{(P3)}      &&(ab)c = a(bc)         &\text{(associatività di $\cdot$)}\\
            &\text{(P4)}      &&a \cdot 1=1 \cdot a=a &\text{(1 el. neutro di $\cdot$)}\\
            &\text{(P5)}     &&(a+b)c = ac + bc      &\text{(distributività)} \\
            &\text{(P6)}     &&a \neq 0 \implies \exists a^{-1} \in F. \quad aa^{-1} = 1 &\text{(inverso per $\cdot$)}
        \end{align*}
    \end{description} 

    La definizione sopra è equivalente a dire che $F$ è un anello commutativo per cui ogni elemento non nullo ha un inverso moltiplicativo.
\end{definition}

Tra gli insiemi numerici classici, gli insiemi $\Q, \R$ e $\C$ sono tutti esempi di campi: infatti le operazioni di addizione e moltiplicazione sono chiuse rispetto all'insieme, rispettano le proprietà commutativa, associativa e distributiva ed esistono gli inversi per la somma e per il prodotto (per ogni numero diverso da $0$). Il concetto di campo serve quindi a generalizzare la struttura algebrica dei numeri razionali/reali/complessi per altri insiemi numerici.

Nei campi vale la seguente proposizione.
\begin{proposition}
    [Regola di annullamento del prodotto] \label{annullamento_prodotto}
    Sia $\K$ un campo e siano $a, b \in \K$. Allora \[
        ab = 0 \implies a = 0 \lor b = 0.    
    \]
\end{proposition}
\begin{proof}
    Sappiamo che $a = 0 \lor b = 0$ è equivalente a $a \neq 0 \implies b = 0$, dunque supponiamo che $a$ sia diverso da $0$ e dimostriamo che $b$ è zero.

    Dato che $a \neq 0$ allora ammette un inverso. Chiamiamolo $a^{-1}$ e moltiplichiamo entrambi i membri per esso:
    \begin{alignat*}
        {1}
        &a^{-1}(ab) = a^{-1} \cdot 0\\
        \iff &(a^{-1}a)b = 0 \\
        \iff &b = 0
    \end{alignat*}
    che è la tesi.
\end{proof}

\section{Numeri complessi}

\begin{definition}[Unità immaginaria]
    Si dice unità immaginaria il numero $i$ tale che \[
        i^2 = -1.    
    \]
\end{definition}

\begin{definition}[Numeri complessi]
    L'insieme dei numeri complessi $\C$ è l'insieme dei numeri della forma $a+ib$ per qualche $a, b \in \R$, ovvero \[
        \C = \{ a + ib \mid a, b \in \R, i^2 = -1\}.  
    \]
\end{definition}

\begin{definition}[Parte reale e immaginaria]
    Sia $z \in \C$ tale che $z = a + ib$. Allora si dicono rispettivamente \begin{itemize}
        \item parte reale di $z$ il numero $\Re z = a$;
        \item parte immaginaria di $z$ il numero $\Im z = b$.
    \end{itemize}
\end{definition}

\begin{definition}[Somma e prodotto sui complessi]
    Definiamo le seguenti due operazioni su $\C$:
    \begin{itemize}
        \item $+ : \C \times \C \to \C$ tale che $(a + ib) + (c + id) = (a + c) + i(b + d)$;
        \item $\cdot : \C \times \C \to \C$ tale che $(a + ib) \cdot (c + id) = (ac - bd) + i(ad + bc)$.
    \end{itemize}
\end{definition}

\begin{remark}
    Le due operazioni vengono naturalmente dalla somma e dal prodotto tra monomi. Infatti \begin{gather*}
        (a + ib) + (c + id) = a + c + ib + id = (a + c) + i(b + d);\\
        \begin{alignedat}{1}
            (a + ib) \cdot (c + id) &= ac + iad + ibc + i^2bd \\
            &= ac + i(ad + bc) -bd \\
            &= (ac - bd) + i(ad + bc).
        \end{alignedat}
    \end{gather*}
\end{remark}

Notiamo che i numeri complessi della forma $a + i0$ sono numeri reali, dunque $\R \subset \C$. Inoltre possiamo rappresentare i numeri complessi come punti in uno spazio bidimensionale dove la parte reale rappresenta l'ascissa e la parte immaginaria rappresenta l'ordinata: la retta corrispondente all'asse x è il sottoinsieme dei numeri reali.

\begin{definition}[Coniugato complesso]
    Sia $z = a+ib \in \C$. Allora si dice coniugato complesso (o semplicemente coniugato) di $z$ il numero \[
        \conj{z} = a - ib.    
    \]
\end{definition}

\begin{definition}[Norma di un numero complesso]
    Sia $z = a + ib \in \C$. Allora si dice norma di $z$ il numero reale \[
        \abs{z} = \sqrt{a^2 + b^2}.    
    \]
\end{definition}

Notiamo che $\abs{z} = 0$ se e solo se $a = b = 0$, ovvero se $z = 0$.

\begin{proposition}\label{somma_prodotto_tra_coniugati}
    Siano $z, w \in \C$ tali che $z = a+ib$, $w = c + id$. Allora \begin{enumerate}[(i)]
        \item $\conj{z} + \conj{w} = \conj{z + w}$;
        \item $\conj{z}\cdot\conj{w} = \conj{zw}$;
        \item $(\conj{z})^n = \conj{z^n}$.
    \end{enumerate}
\end{proposition}
\begin{proof}
    Dimostriamo i tre fatti.
    \begin{enumerate}[(i)]
        \item Per definizione di somma \begin{alignat*}
            {1}
            \conj{z} + \conj{w} &= (a-ib) + (c - id)\\
            &= (a+c) - i(b+d)\\
            &= \conj{z + w}.
        \end{alignat*}
        \item Per definizione di prodotto \begin{alignat*}
            {1}
            \conj{z}\cdot\conj{w} &= (a-ib)(c - id)\\
            &= (ac - bd) + i(-ad-bc)\\
            &= (ac - bd) - i(ad+bc)\\
            &= \conj{zw}.
        \end{alignat*}
        \item Dimostriamolo per induzione su $n$.
        \begin{description}
            \item[Caso base.] Se $n = 1$ allora banalmente $(\conj{z})^1 = \conj{z} = \conj{z^1}$.
            \item[Passo induttivo.] Supponiamo che la tesi valga per $n$ e dimostriamola per $n+1$. Allora \[
                (\conj{z})^{n+1} = (\conj{z})^{n} \cdot \conj{z} = \conj{z^n} \cdot \conj{z} = \conj{z^{n+1}}
            \] dove l'ultimo passaggio è giustificato dal punto precedente della dimostrazione. \qedhere
        \end{description}
    \end{enumerate}
\end{proof}

\begin{proposition}\label{somma_prodotto_col_coniugato}
    Sia $z = a+ib \in \C$. Allora valgono i seguenti fatti:
    \begin{enumerate}[(i)]
        \item $z + \conj{z} = 2\Re{z}$;
        \item $z\conj{z} = \abs{z}^2$.
    \end{enumerate}
\end{proposition}
\begin{proof}
    Dimostriamo i due fatti.
    \begin{enumerate}[(i)]
        \item Per definizione di somma $z + \conj{z} = (a + ib) + (a - ib) = 2a = 2\Re z$.
        \item Per definizione di prodotto \[
            z\conj{z} = (a + ib)(a - ib) = a^2 - iab + iab - i^2b^2 = a^2 + b^2 = \abs{z}^2.\qedhere
        \] 
    \end{enumerate}
\end{proof}

La proposizione precedente ci consente di trovare l'inverso di qualunque numero non nullo in $\C$.

\begin{proposition}[Inverso tra i complessi]
    Sia $z \in \C, z \neq 0$. Allora \[\frac{1}{z} = \frac{\conj{z}}{\abs{z}^2}.\]
\end{proposition}
\begin{proof}
    Per la proposizione \ref{somma_prodotto_col_coniugato} segue che \[
        z\conj{z} = \abs{z}^2 \iff \frac{1}{z} = \frac{\conj{z}}{\abs{z}^2}. \qedhere   
    \]
\end{proof}

\begin{proposition}[I numeri complessi formano un campo]
    L'insieme $\C$ insieme alle operazioni di somma e prodotto con i rispettivi elementi neutri $0, 1 \in \C$ forma un campo.
\end{proposition}

\subsection{Rappresentazione polare dei numeri complessi}

Dato che possiamo considerare i numeri complessi come punti di un piano bidimensionale possiamo rappresentarli in forma polare, cioè considerando il vettore che congiunge l'origine degli assi con il punto $(a, b)$ che rappresenta il numero complesso $a + ib$. La forma polare di un numero complesso è data dalla coppia $(r, \theta)$, dove $r$ è il raggio del vettore e $\theta$ è l'angolo tra l'asse x e il vettore.

Dunque se $z = a+ib$ è un numero complesso in forma cartesiana, possiamo esprimerlo come $r(\cos\theta + i\sin\theta)$, dove $r = \sqrt{a^2 + b^2} = \abs{z}$ e $\theta = \arctan \frac{a}{b}$.

\begin{definition}[Esponenziale complesso]
    $e^{i\theta} = \cos\theta + i\sin\theta.$
\end{definition}

Sfruttando la definizione precedente possiamo scrivere ogni numero complesso nella forma $re^{i\theta}$ che è la forma polare del numero.

\begin{proposition}
    Siano $e^{i\alpha}, e^{i\beta} \in \C$. Allora vale \[
        e^{i\alpha} e^{i\beta} = e^{i(\alpha + \beta)}.
    \]
\end{proposition}
\begin{proof}
    Per definizione di esponenziale complesso:
    \begin{alignat*}{1}
        e^{i\alpha} e^{i\beta} &= (\cos\alpha + i\sin\alpha)(\cos\beta + i\sin\beta)\\
        &= (\cos\alpha \cos\beta - \sin\alpha \sin\beta) + i(\sin\alpha \cos\beta + \cos\alpha \sin\beta)\\
        &= \cos(\alpha + \beta) + i\sin(\alpha + \beta)\\
        &= e^{i(\alpha + \beta)}. \tag*{\qedhere}
    \end{alignat*}
\end{proof}

\section{Successioni per ricorrenza}

\begin{definition}[Successione]
    Si dice successione a valori in un insieme $A$ una funzione $(a_n) : \N \to A$.
\end{definition}

Solitamente analizzeremo successioni a valori reali, ovvero $(a_n) : \N \to \R$. Inoltre usiamo equivalentemente le notazioni $(a_n)_k$ o $a_k$ per riferirci alla funzione valutata nel punto $k \in \N$.

\begin{definition}[Somma di successioni e prodotto per una costante]
    Sia $S_{\R}$ l'insieme delle successioni a valori reali. Allora definisco una somma tra successioni $+ : S_{\R} \times S_{\R} \to S_{\R}$ tale che \[
        (a_n) + (b_n) = (a_n + b_n)  
    \] e un prodotto per una costante $\cdot : \R \times S_{\R} \to S_{\R}$ tale che \[
        k(a_n) = (ka_n).    
    \]
\end{definition}

\begin{example}
    Sia $a_n = 3^n$ e $b_n = 2n + 1$. Allora $(c_n) = (a_n) + (b_n)$ è la successione definita dalla legge $c_n = 3^n + 2n + 1$, mentre $(d_n) = 3(b_n)$ è la successione definita da $d_n = 6n + 3$.
\end{example}

Queste operazioni rispettano le solite proprietà (associativa, commutativa, distributiva). In particolare vale quindi la seguente proposizione.

\begin{proposition}[L'insieme delle successioni è uno spazio vettoriale]
    L'insieme delle successioni a valori reali $S_{\R}$ insieme alle operazioni di somma e prodotto per costanti e alla successione identicamente nulla $(0_n)$ è uno spazio vettoriale su $\R$.
\end{proposition}

\begin{definition}[Ricorrenza lineare omogenea]
    Si dice ricorrenza lineare omogenea di ordine $k$ un'equazione della forma \begin{equation} \label{ricorrenza}
        a_{n+k} = r_{k-1}a_{n+k-1} + r_{k-2}a_{n+k-2} + \dots + r_{1}a_{n+1} + r_0a_n. 
    \end{equation}
    Una soluzione della ricorrenza lineare \ref{ricorrenza} è una successione $(s_n)$ tale che per ogni $n \in \N$ vale che $s_n, s_{n+1}, \dots, s_{n+k}$ soddisfano la ricorrenza.
\end{definition}

\begin{proposition}
    Sia $A$ l'insieme delle successioni che soddisfano la ricorrenza lineare omogenea \[
        s_{n+k} = r_{k-1}s_{n+k-1} + r_{k-2}s_{n+k-2} + \dots + r_{1}s_{n+1} + r_0s_n.
    \] Allora $A$ è un sottospazio vettoriale di $S_{\R}$.
\end{proposition}
\begin{proof}
    Dobbiamo dimostrare tre fatti:
    \begin{enumerate}[(i)]
        \item $(0_n) \in A$;
        \item se $(a_n), (b_n) \in A$ allora $(c_n) = (a_n) + (b_n) \in A$;
        \item se $h \in \R$, $(a_n) \in A$ allora $(d_n) = h(a_n) \in A$.
    \end{enumerate}

    Sia $n \in \N$ qualsiasi.
    \begin{enumerate}[(i)]
        \item Verifichiamo che $(0_n)$ sia soluzione. La ricorrenza da verificare è \[
            0_{n+k} = r_{k-1}0_{n+k-1} + \dots + r_{1}0_{n+1} + r_00_n.
        \] Ma dato che $(0_n)$ è la successione identicamente nulla, allora questo equivale a dire $0 = 0r_{k-1} + \dots +  + 0r_0 = 0$, che è verificata e quindi $(0_n) \in A$.
        \item Verifichiamo che $(c_n)$ sia soluzione. \begin{align*}
            c_{n+k} &= a_{n+k} + b_{n+k}\\
            &= (r_{k-1}a_{n+k-1} + \dots + r_0a_n) + (r_{k-1}b_{n+k-1} + \dots + r_0b_n) \\
            &= r_{k-1}(a_{n+k-1} + b_{n+k-1}) + \dots + r_0(a_n + b_n) \\
            &= r_{k-1}c_{n+k-1} + \dots + r_0c_0
        \end{align*}
        dunque $(c_n) \in A$.
        \item Verifichiamo che $(d_n)$ sia soluzione. \begin{align*}
            d_{n+k} &= ha_{n+k}\\
            &= h(r_{k-1}a_{n+k-1} + \dots + r_0a_n)\\
            &= r_{k-1}(ha_{n+k-1}) + \dots + r_0(ha_n) \\
            &= r_{k-1}d_{n+k-1} + \dots + r_0d_0
        \end{align*}
        dunque $(d_n) \in A$. \qedhere
    \end{enumerate}
\end{proof}

La proposizione precedente ci permette di trovare una soluzione generale ad una ricorrenza lineare omogenea.

\begin{example}
    Siano $a_n = 3^n$ e $b_n = (-1)^n$ due soluzioni di una ricorrenza lineare omogenea. Allora per la proposizione precedente anche $k_1a_n = k_13^n$ e $k_2b_n = k_2(-1)^n$ saranno soluzioni (per ogni $k_1, k_2 \in \R$), e di conseguenza anche $k_1a_n + k_2b_n = k_13^n + k_2(-1)^n$.
\end{example}

Cerchiamo di risolvere una ricorrenza lineare omogenea.
\begin{example}
    Sia $a_{n+2} = 2a_{n+1} + 3a_n$ una ricorrenza lineare omogenea di ordine $2$. Trovare la soluzione generale. Inoltre trovare una soluzione particolare che soddisfi le condizioni iniziali $a_0 = 0$ e $a_1 = 1$.
\end{example}
\begin{solution}
    Proviamo a risolvere la ricorrenza con una soluzione esponenziale della forma $(\lambda^n)$ al variare di $n \in \N$. Sostituendo otteniamo \begin{alignat*}{1}
        &\lambda^{n+2} = 2\lambda^{n+1} + 3\lambda^{n} \\
        \iff &\lambda^2 = 2\lambda + 3 \\
        \iff &\lambda^2 - 2\lambda - 3.
    \end{alignat*}
    Dunque se $(\lambda^n)$ è una soluzione allora $\lambda$ deve essere radice di quel polinomio di secondo grado, detto polinomio caratteristico della ricorrenza.
    Risolvendolo segue che $\lambda_1 = 3$ e $\lambda_2 = -1$ sono soluzioni, dunque le successioni $(3^n)$ e $((-1)^n)$ sono soluzioni della ricorrenza.

    La soluzione generale della ricorrenza è dunque una successione della forma $(a_n) = k_1(3^n) + k_2((-1)^n)$ al variare di $k_1, k_2 \in \R$.

    Imponiamo ora che $a_0 = 0$ e $a_1 = 1$.
    \begin{equation*}
        \left\{
        \begin{array}{@{}roror }
        3^0k_1 & + & (-1)^0k_2 & = & 0 \\
        3^1k_1 & + & (-1)^1k_2 & = & 1 \\
        \end{array}
        \right. \iff \left\{
        \begin{array}{@{}ror }
        k_1 + k_2 & = & 0 \\
        3k_1 -k_2 & = & 1 \\
        \end{array}
        \right. 
    \end{equation*}
    da cui segue $k_1 = \frac14$, $k_2 = -\frac14$. La successione che soddisfa le condizioni iniziali è dunque $a_n = \frac14(3)^n - \frac14(-1)^n$.
\end{solution}

\begin{definition}
    [Polinomio caratteristico di una ricorrenza]
    Sia $a_{n+k} = r_{k-1}a_{n+k-1} +  \dots +  r_0a_n$ una ricorrenza lineare omogenea di ordine $k$. Allora si dice polinomio caratteristico associato alla ricorrenza il polinomio \[
        p(\lambda) = \lambda^k - r_{k-1}\lambda^{k-1} - \dots - r_0.    
    \]
\end{definition}

Il polinomio caratteristico si ottiene sostituendo alla ricorrenza lineare la successione $(\lambda^n)$, esattamente come abbiamo fatto nell'esempio precedente.

\begin{example}
    Consideriamo la successione di Fibonacci $f_{n+2} = f_{n+1} + f_n$ con $f_0 = 0$, $f_1 = 1$. Trovare una successione che risolva la ricorrenza e soddisfi i casi base.
\end{example}
\begin{solution}
    Il polinomio caratteristico di questa ricorrenza è \[
        p(\lambda) = \lambda^2 - \lambda - 1    
    \] che ha come radici i numeri $\varphi = \frac12(1 + \sqrt5)$ e $\bar{\varphi} = \frac12(1 - \sqrt5)$.

    La soluzione generale della ricorrenza è dunque una successione della forma $(f_n) = k_1(\varphi^n) + k_2(\bar{\varphi}^n)$ al variare di $k_1, k_2 \in \R$.

    Imponiamo ora che $f_0 = 0$ e $f_1 = 1$.
    \begin{equation*}
        \arraycolsep=1.2pt\def\arraystretch{1.3}
        \left\{
        \begin{array}{@{}roror }
        \varphi^0k_1 & + & \bar{\varphi}^0k_2 & = & 0\\
        \varphi^1k_1 & + & \bar{\varphi}^1k_2 & = & 1 \\
        \end{array}
        \right. \iff \left\{
        \begin{array}{@{}roror }
        k_1 & + & k_2 & = & 0\\
        \varphi k_1 & + & \bar{\varphi}k_2 & = & 1 \\
        \end{array}
        \right. 
    \end{equation*}
    da cui segue $k_1 = \frac{1}{\sqrt5}$, $k_2 = -\frac{1}{\sqrt5}$. La successione che soddisfa le condizioni iniziali è dunque \[
        f_n = \frac{1}{\sqrt5}\left(\frac{1 + \sqrt5}{2}\right)^n - \frac{1}{\sqrt5}\left(\frac{1 - \sqrt5}{2}\right)^n.
    \]
\end{solution}

Nel caso che una radice del polinomio caratteristico abbia una molteplicità maggiore di $1$ essa darà luogo a più di una soluzione della ricorrenza, come ci dice la seguente proposizione.
\begin{proposition}
    Sia $p(\lambda)$ il polinomio caratteristico di una ricorrenza lineare omogenea e sia $\lambda_0$ una radice di molteplicità $h$ (ovvero $h$ è il massimo intero per cui $(x - \lambda_0)^h$ compare nella fattorizzazione di $p(\lambda)$) con $h \leq 2$. 
    
    Allora $(\lambda_0^n), (n\lambda_0^n), \dots, (n^{h-1}\lambda_0^n)$ sono tutte soluzioni della ricorrenza lineare omogenea.
\end{proposition}

\begin{example}
    Sia $p(\lambda) = (\lambda - 3)^3(\lambda + 1)^2(\lambda - \sqrt2)^4$. Allora le seguenti sono tutte soluzioni indipendenti della ricorrenza lineare omogenea associata a $p(\lambda)$:
    \begin{multicols}{3}
        \begin{enumerate}[(i)]
        \item $(3^n)$;
        \item $(n3^n)$;
        \item $(n^23^n)$;
        \item $((-1)^n)$;
        \item $(n(-1)^n)$;
        \item $(\sqrt{2}^n)$;
        \item $(n\sqrt{2}^n)$;
        \item $(n^2\sqrt{2}^n)$;
        \item $(n^3\sqrt{2}^n)$.
    \end{enumerate}
    \end{multicols}
    
    La soluzione generale sarà dunque della forma \begin{align*}
        (a_n) = 
            &\ k_1(3^n) + k_2(n3^n) + k_3(n^23^n) + k_4(n(-1)^n) + k_5(n(-1)^n) + \\
            + &\ k_6(\sqrt{2}^n) + k_7(n\sqrt{2}^n) + k_8(n^2\sqrt{2}^n) + k_9(n^3\sqrt{2}^n)
    \end{align*}
    al variare di $k_1, \dots, k_9 \in \R$.
\end{example}
\chapter{Divisori e MCD}

\section{Divisori di un numero}

\subsection{Definizioni e prime conseguenze}

\begin{definition}[Divisore]
    Siano $a, b \in \Z$; allora si dice che $a$ divide $b$ se
    $\exists k \in \Z$ tale che $ak = b$, e si scrive $a \divides b$.
\end{definition}

\begin{definition}[Multiplo]
    Siano $a, b \in \Z$. Allora si dice che $b$ è multiplo di $a$ se $\exists k \in \Z$ tale che $b = ak$.
\end{definition}
\begin{remark}
    La definizione di multiplo è speculare a quella di divisore: se $a$ è divisore di $b$ allora $b$ è multiplo di $a$.
\end{remark}

\begin{proposition} \label{divides_sum_subtr_mult}
    Siano $a, b, n \in \Z$ tali che $n \divides a$ e $n \divides b$. Allora
    \begin{alignat}{2}
        &n \divides a + b \label{divides_sum}\\
        &n \divides a - b \label{divides_subtr} \\
        &n \divides ax \qquad&\forall x \in \Z \label{divides_mult}
    \end{alignat} 
\end{proposition}
\begin{proof}  
    Per ipotesi, dato che $n \divides a$ e $n \divides b$, allora $\exists h, k \in \Z$ tali che
    $nh = a$ e $nk = b$. Dunque:
    \begin{alignat*}{2}
        a + b = nh + nk = n(h + k) &\iff n \divides a + b \\
        a - b = nh - nk = n(h - k) &\iff n \divides a - b \\
        ax = nhx = n(hx) &\iff n \divides ax
    \end{alignat*}
    che è la tesi.
\end{proof}

\begin{definition}[Massimo comun divisore]
    Siano $a, b \in \Z$; allora si dice $\mcd{a}{b}$ il piu' grande intero positivo
    tale che $\mcd{a}{b} \divides a$ e $\mcd{a}{b} \divides b$.
\end{definition}

\begin{definition}[Minimo comune multiplo]
    Siano $a, b \in \Z$. Allora si dice minimo comune multiplo di $a$ e $b$ il numero $d = \mcm{a}{b}$ tale che $d$ è il piu' piccolo multiplo positivo sia di $a$ che di $b$.
\end{definition}

\begin{definition}[Coprimo]
    Siano $a, b \in \Z$. Se $\mcd{a}{b} = 1$ allora $a$ e $b$ si dicono coprimi.
\end{definition}

\begin{remark}
    Siano $a, b \in \Z$. Allora valgono le seguenti proprietà per $\mcd{a}{b}$:
    \begin{align*}
        \mcd{a}{b} &= \mcd{\pm a}{\pm b} \\
        \mcd{a}{1} &= \mcd{1}{a} = 1 \\
        \mcd{a}{0} &= \mcd{0}{a} = 0 \\
        \nexists \mcd{0}{0}
    \end{align*}
\end{remark}

\begin{theorem}[Esistenza e unicità del resto] \label{esistenza_resto}
    Siano $a, b \in \Z$, con $b \neq 0$. Allora esistono e sono unici $q, r \in \Z$ tali che
    \begin{align}
        a = bq + r, \qquad0 \leq r < \abs{b}
    \end{align}
    Tale $r$ si dice resto della divisione di $a$ per $b$, e si indica anche con $r = a\bmod b$.
\end{theorem}

\begin{proof}
    Notiamo che i numeri della forma $a - bq$ formano una progressione aritmetica di passo $b$ al variare di $q \in \Z$. 
    Il resto $r$ definito in questo modo è l'unico elemento di questa progressione compreso tra $0$ e $b - 1$.
\end{proof}

\begin{proposition}\label{mcm|c_iff_a,b|c}
    Siano $a, b, c \in \Z$. Allora \begin{equation}
        \mcm{a}{b} \divides c \iff a \divides c \land b \divides c
    \end{equation}
\end{proposition}
\begin{proof}
    Dimostriamo separatamente i due versi dell'implicazione.
    \begin{itemize}
        \item[($\implies$)] Dato che $\mcm{a}{b}$ è un multiplo di $a$ e di $b$ e per ipotesi $c$ è un multiplo di $\mcm{a}{b}$, allora per transitività segue che $c$ è un multiplo di $a$ e di $b$.
        \item[($\impliedby$)] Supponiamo che $c$ sia un multiplo di $a$ e di $b$. Allora per il teorema \ref{esistenza_resto} esistono $q, r \in \Z$ tali che \[
        c = \mcm{a}{b}q + r
    \]
    con $0 \leq r < \mcm{a}{b}$.
    Dato che $a$, $b$ dividono sia $c$ (per ipotesi) che $\mcm{a}{b}$ (per definizione di mcm), allora segue che essi dividono anche $r$. Ma $0 \leq r < \mcm{a}{b}$, dunque necessariamente $r = 0$, cioè $c = \mcm{a}{b}q$ e quindi $\mcm{a}{b} \divides c$. \qedhere
    \end{itemize}
\end{proof}


\subsection{Algoritmo di Euclide e Teorema di Bezout}

\begin{theorem} \label{mcd_a_a-b}
    Siano $a, b \in \Z$. Allora
    \begin{equation}
        \mcd{a}{b} = \mcd{a}{b - a} = \mcd{a - b}{b}.
    \end{equation} 
\end{theorem}
\begin{proof}  
    Ovviamente $\mcd{a}{b} = \mcd{b}{a}$, dunque se vale la prima uguaglianza varrà anche la seconda,
    in quanto 
    \[
       \mcd{a}{b} = \mcd{b}{a} = \mcd{b}{a - b} = \mcd{a - b}{b}.
    \] 
    Dunque è sufficiente
    dimostrare che $\mcd{a}{b} = \mcd{a}{b - a}$.
    Sia $\mathbb{D}_{x, y}$ l'insieme dei divisori comuni a $x$ e $y$, cioe'
    \[
        \mathbb{D}_{x, y} = \left\{d \tc d \divides x \land d \divides y \right\}  
    \]
    Allora per dimostrare la tesi è sufficiente dimostrare che $\mathbb{D}_{a, b} = \mathbb{D}_{a, b - a}$, in quanto
    se i due insiemi sono uguali necessariamente anche i loro massimi saranno uguali.

    Dimostriamo che $\mathbb{D}_{a, b} \subseteq \mathbb{D}_{a, b - a}$. Sia $d \in \mathbb{D}_{a, b}$, 
    cioè $d \divides a$ e $d \divides b$. Allora per la proposizione \ref{divides_sum_subtr_mult} segue che
    $d \divides b - a$, cioè $d \in \mathbb{D}_{a, b - a}$, 
    cioè $\mathbb{D}_{a, b} \subseteq \mathbb{D}_{a, b - a}$.

    Dimostriamo ora che $\mathbb{D}_{a, b - a} \subseteq \mathbb{D}_{a, b}$. 
    Sia $d \in \mathbb{D}_{a, b - a}$, 
    cioè $d \divides a$ e $d \divides b - a$. Allora per la proposizione \ref{divides_sum_subtr_mult} segue che
    $d \divides a + (b - a)$, cioè $d \divides b$, cioè $d \in \mathbb{D}_{a, b}$, 
    cioè $\mathbb{D}_{a, b - a} \subseteq \mathbb{D}_{a, b}$.

    Dunque dato che valgono sia $\mathbb{D}_{a, b} \subseteq \mathbb{D}_{a, b - a}$ e 
    $\mathbb{D}_{a, b - a} \subseteq \mathbb{D}_{a, b}$, allora vale
     $\mathbb{D}_{a, b} = \mathbb{D}_{a, b - a}$. In particolare il massimo di questi due insiemi
     dovrà essere lo stesso, quindi $\mcd{a}{b} = \mcd{a}{b - a}$, che è la tesi.
\end{proof}

Dunque per calcolare il massimo comun divisore si puo' sfruttare il seguente algoritmo, detto \textbf{algoritmo di Euclide}, che si basa sul teorema \ref{mcd_a_a-b}:
\begin{enumerate}
    \item Se $a = 1$ oppure $b = 1$ allora $\mcd{a}{b} = 1$.
    \item Se $a = 0$ e $b \neq 0$ allora $\mcd{a}{b} = b$.
    \item Se $a \neq 0$ e $b = 0$ allora $\mcd{a}{b} = a$.
    \item Se $a \neq 0$ e $b \neq 0$, allora
        \begin{itemize}
            \item se $a \leq b$ segue che $\mcd{a}{b} = \mcd{a - b}{b}$;
            \item se $a > b$ segue che $\mcd{a}{b} = \mcd{a}{b - a}$
        \end{itemize}
        dove i valori di $\mcd{a - b}{b}$ o $\mcd{a}{b - a}$ vengono calcolati riapplicando l'algoritmo.
\end{enumerate}
Possiamo velocizzare il procedimento usando i resti della divisione invece che la sottrazione:
\begin{enumerate}[4.]
    \item Se $a \neq 0$ e $b \neq 0$, allora
    \begin{itemize}
        \item se $a \leq b$ segue che $\mcd{a}{b} = \mcd{a \bmod b}{b}$;
        \item se $a > b$ segue che $\mcd{a}{b} = \mcd{a}{b \bmod a}$
    \end{itemize}
    dove i valori di $\mcd{a \bmod b}{b}$ o $\mcd{a}{b \bmod a}$ vengono calcolati riapplicando l'algoritmo.
\end{enumerate}

\begin{theorem}
    [Teorema di Bezout] \label{bezout}
    Siano $a, b \in \Z$. Allora esistono $x, y \in \Z$ tali che
    \begin{equation}
        ax + by = \mcd{a}{b}.
    \end{equation}
\end{theorem}
\begin{proof}
    Siano $r_0 = a$, $r_1 = b$. Definisco la successione $(r_n)$ come la sequenza dei resti della divisione dati dall'algoritmo di Euclide per l'mcd:\begin{align*}
        \mcd{a}{b} &= \mcd{r_0}{r_1} &&\text{sia } r_2 = r_0 \bmod{r_1}: \\
        &= \mcd{r_1}{r_2} &&\text{sia } r_3 = r_1 \bmod{r_2}: \\
        &= \dots\\
        &= \mcd{r_n}{r_{n+1}} &&\text{sia } r_{n+2} = r_n \bmod{r_{n+1}}:\\
        &= \dots
    \end{align*}
    Questo processo ha termine quando un resto $r_{m+1}$ è uguale a $0$: in quel caso $\mcd{r_m}{r_{m+1}} = \mcd{r_m}{0}= r_m = \mcd{a}{b}$.

    Dimostriamo che per ogni $n$ possiamo scrivere $r_n$ come $ax_n + by_n$ per qualche $x_n, y_n \in \Z$.
    \begin{description}
        \item[Caso base] Per $r_0$ e $r_1$ è banale: \begin{align*}
            r_0 = 1\cdot a + 0 \cdot b, &&r_1 = 0 \cdot a + 1 \cdot b.
        \end{align*}
        \item[Passo induttivo] Supponiamo di saper scrivere $r_n$ e $r_{n-1}$ come combinazione di $a$ e $b$ e dimostriamo che possiamo farlo anche per $r_{n+2}$.
       
        Per ipotesi induttiva esistono $x_n, y_n, x_{n+1}, y_{n+1} \in \Z$ tali che \begin{align*}
            r_n = ax_n + by_n, && r_{n+1} = ax_{n+1} + by_{n+1}.
        \end{align*}

        Per definizione $r_{n+2} = r_n \bmod{r_{n+1}}$, ovvero per definizione di resto $r_{n+2} = r_n - q_{n+1}r_{n+1}$ per qualche $q_{n+1} \in \Z$. Sostituendo otteniamo \begin{align*}
            r_{n+2} &= r_n - q_{n+1}r_{n+1}\\
            &= ax_n + by_n - q_{n+1}(ax_{n+1} + by_{n+1})\\
            &= a(x_n - q_{n+1}x_{n+1}) + b(y_n - q_{n+1}y_{n+1}).
        \end{align*}
        Dunque $x_{n+2} = x_n - q_{n+1}x_{n+1}$ e $y_{n+2} = y_n - q_{n+1}y_{n+1}$, cioè possiamo esprimere $r_{n+2}$ come combinazione lineare di $a, b$.
    \end{description}

    Dato che per induzione questo risultato vale per tutti gli $n \in \N$, varrà anche per $m$, ovvero esistono $x, y \in \Z$ tali che $r_m = ax + by$. Ma $r_m = \mcd{a}{b}$, dunque la tesi.
\end{proof}

\subsection{Conseguenze del teorema di Bezout}

Elenchiamo in questa sezione alcune conseguenze del teorema di Bezout sulle proprietà dei divisori e sul loro rapporto con il massimo comun divisore di due numeri.

\begin{proposition} \label{n_divides_product}
    Siano $a, b, n \in \Z$. Allora \begin{equation}
        n \divides ab \land \mcd{a}{n} = 1 \implies n \divides b.
    \end{equation}
\end{proposition}
\begin{intuition}
    Se $n$ divide $ab$, allora tutti i fattori primi che dividono $n$ dovranno essere contenuti in $ab$. Dato che $\mcd{n}{a} = 1$, questi fattori non possono essere contenuti in $a$, dunque dovranno essere tutti contenuti in $b$.
\end{intuition}
\begin{proof}
    Per il teorema di Bezout (\ref{bezout}) esistono $x, y \in \Z$ tali che 
    \begin{align*}
        ax + ny &= \mcd{a}{n} = 1 \\
        \intertext{Moltiplicando per $b$ otteniamo}
        abx + nby &= b 
    \end{align*}
    Ma $n \divides abx$ (poiche $n \divides ab$) e $n \divides nby$, dunque $n \divides abx + nby$, 
    cioè $n \divides b$.
\end{proof}

\begin{proposition} \label{greatest_common_divisor}
    Siano $a, b, t \in \Z$ tali che $t \divides a$, $t \divides b$. Allora $t \leq \mcd{a}{b}$.
\end{proposition}
\begin{proof}
    La proposizione deriva direttamente dalla definizione di massimo comun divisore: se $t$ è un divisore comune ad $a$ e $b$, allora $t$ sarà minore o uguale al massimo dei divisori comuni di $a$ e $b$, cioè $t \leq \mcd{a}{b}$.
\end{proof}

\begin{proposition} \label{divisori_dividono_mcd}
    Siano $a, b, t \in \Z$ tali che $t \divides a$, $t \divides b$.  Allora $t \divides \mcd{a}{b}$.
\end{proposition}
\begin{proof}
    Per la proposizione \ref{divides_sum_subtr_mult}, se $t \divides a$ e $t \divides b$ allora $t \divides ax+by$ per ogni $x, y \in \Z$.

    Per il teorema di Bezout (\ref{bezout}) esistono $\bar{x}, \bar{y} \in \Z$ tali che $a\bar{x}+b\bar{y} = \mcd{a}{b}$. Ma quest'espressione è della forma $ax + by$, con $x = \bar{x}$, $y = \bar{y}$, dunque 
    $t \divides a\bar{x} + b\bar{y}$, cioè $t \divides \mcd{a}{b}$.
\end{proof}

\begin{proposition} \label{t_divides_gcd_lincomb}
    Siano $a, b, t \in \Z$. Allora 
    \begin{equation}
        t \divides \mcd{a}{b} \iff (\forall x, y \in \Z. \quad t \divides ax + by).
    \end{equation}
\end{proposition}
\begin{proof}
    Dimostriamo entrambi i versi dell'implicazione.
    \begin{itemize}
        \item[($\implies$)] Se $t \divides \mcd{a}{b}$, allora $t \divides a$ e $t \divides b$, dunque per la proposizione \ref{divides_sum_subtr_mult} segue che $t$ dovrà dividere una qualsiasi combinazione lineare di $a$ e $b$, cioè $t \divides ax+by$ per ogni $x, y \in \Z$.
        \item[($\impliedby$)] Viceversa supponiamo che $t \divides ax+by$ per ogni $x, y \in \Z$. Siano per il teorema di Bezout (\ref{bezout}) $\bar{x}, \bar{y}$ i numeri tali che $a\bar{x} + b\bar{y} = \mcd{a}{b}$. Allora $t$ dovrà dividere anche $a\bar{x} + b\bar{y}$, cioè $t \divides \mcd{a}{b}$. \qedhere
    \end{itemize}
\end{proof}

\begin{proposition} \label{mcd_of_multiples_of_n}
    Siano $a, b, n \in \Z$. Allora \begin{equation}
        \mcd{an}{bn} = n\mcd{a}{b}.
    \end{equation}
\end{proposition}
\begin{intuition}
    Se due numeri hanno $n$ come fattore comune, ovviamente il massimo comun divisore dovrà contenere $n$ e quindi dovrà essere un multiplo di $n$.
\end{intuition}
\begin{proof}
    Osserviamo che se due numeri hanno gli stessi divisori allora sono uguali, a meno del segno.
    Sia $t \in \Z$ tale che $t \divides an$ e $t \divides nb$. Per la proposizione \ref{t_divides_gcd_lincomb} allora 
    \begin{alignat*}
        {2}
        &t \divides \mcd{an}{bn}\\
        \iff &t \divides nax + nby      \qquad& \forall x, y \in \Z \\
        \iff &t \divides n(ax + by)     \qquad& \forall x, y \in \Z \\
        \intertext{dunque scegliendo $x, y$ tali che $ax+by = \mcd{a}{b}$ per Bezout (\ref{bezout})}
        \iff &t \divides n\mcd{a}{b}. \tag*{\qedhere}
    \end{alignat*}
\end{proof}


\begin{corollary} \label{mcd_diviso_mcd}
    Siano $a, b \in \Z$ e sia $d = \mcd{a}{b}$. Allora $\mcd{\frac{a}{d}}{\frac{b}{d}} = 1$.
\end{corollary}
\begin{intuition}
    Se dividiamo due numeri per il loro mcd stiamo eliminando dalla loro fattorizzazione tutti i primi comuni ad entrambi, quindi i due numeri risultanti dall'operazione non potranno avere primi in comune e quindi saranno coprimi.
\end{intuition}
\begin{proof}
    Siano $a^\prime, b^\prime$ tali che $a = a^\prime d, b = b^\prime d$. Allora per la proposizione \ref{mcd_of_multiples_of_n}
    \begin{alignat*}{1}
        \mcd{a}{b} &= \mcd{a^\prime d}{b^\prime d} \\
                   &= d\mcd{a^\prime}{b^\prime} \\
                   &= \mcd{a}{b} \mcd{a^\prime}{b^\prime}.
        \end{alignat*} 
    Dividendo entrambi i membri per $\mcd{a}{b}$ otteniamo \[
        \mcd{a^\prime}{b^\prime} = 1 
    \]
    che, per definizione di $a^\prime, b^\prime$, è equivalente a \[
        \mcd{\frac{a}{d}}{\frac{b}{d}} = 1
    \]
    che è la tesi.
\end{proof}

\section{Numeri primi}

\begin{definition}[Numero primo]
    Sia $p \in \Z$. Si dice che $p$ è primo se se gli unici interi che dividono $p$ sono
    $\pm 1$ e $\pm p$.
\end{definition}

\begin{proposition}\label{primo_divide_prodotto}
    Se $p$ è primo e $p \divides ab$, allora $p \divides a$ oppure $p \divides b$.
\end{proposition}
\begin{proof}
    Supponiamo $p \nmid a$. Dato che $p$ è primo, $\mcd{a}{p} = 1$ oppure $p$.
    Tuttavia se $\mcd{a}{p} = p$ allora $p \divides a$, che va contro l'ipotesi, dunque 
    $\mcd{a}{p} = 1$. Per la proposizione \ref{n_divides_product} allora $p \divides b$, che è la tesi.
\end{proof}

\begin{proposition} \label{prodotto_numeri_coprimi}
    Siano $a, b \in \Z, c \in \Z$ tali che $\mcd{a}{b} = 1$. Allora
    \begin{equation}
        a \divides c \land b \divides c \iff ab \divides c.
    \end{equation}
\end{proposition}
\begin{proof}
    Per il teorema di Bezout (\ref{bezout}) esistono $x, y \in \Z$ tali che $\mcd{a}{b} = 1 = ax+by$, da cui segue $n = nax + nby$.

    Dato che $a \divides n$, $b \divides n$, allora $ab \divides na$ e $ab \divides nb$ per la proposizione \ref{divides_sum_subtr_mult}, quindi per la stessa proposizione $ab$ dividerà una loro qualunque combinazione lineare $nak + nbh$, inclusa quella con $k = x, h = y$.

    Dunque $ab \divides nax + nby$ che è equivalente a dire che $ab \divides n$, cioè la tesi.
\end{proof}


\begin{proposition} \label{prodotto_coprimo_2}
    Siano $a, b, c \in \Z$. Allora
    \begin{equation}
        \mcd{ab}{c} = 1 \iff \mcd{a}{c} = \mcd{b}{c} = 1.
    \end{equation}
\end{proposition}
\begin{intuition}
    Dimostrazione intuitiva: se $a$ e $b$ sono coprimi con $c$ significa che $a$ non ha nessun fattore in comune con $c$, e stessa cosa per $b$. Ma il loro prodotto $ab$ viene diviso dagli stessi primi che dividono $a$ e $b$ separatamente, quindi deve essere anch'esso coprimo con $c$.

    Al contrario, se $ab$ non ha fattori primi in comune con $c$, allora naturalmente $a, b$ (essendo divisori di $ab$) non avranno fattori in comune con $c$.
\end{intuition}

\begin{corollary} \label{prodotto_coprimo_n}
    Siano $a_1, a_2, \dots, a_n \in \Z, c \in \Z$ tali che $a_1, \dots, a_n$ siano coprimi con $c$. Allora anche il loro prodotto $\prod_{i = 1}^{n} a_i$ è coprimo con $c$.
\end{corollary}
\begin{intuition}
    Stessa idea della dimostrazione della proposizione \ref{prodotto_coprimo_2} ma estesa a $n$ numeri.
\end{intuition}

\begin{proposition}
    Siano $a_1, a_2, \dots, a_n \in \Z, c \in \Z$ tali che $a_1, \dots, a_n$ siano coprimi tra loro e che per ogni $i<n$ vale che $a_i \divides c$.
    Allora \begin{equation}
        a_1a_2\dots a_n = \left( \prod_{i = 1}^n a_i \right) \divides c.
    \end{equation}
\end{proposition}
\begin{intuition}
    Quest'ultima proposizione ci dice che se $a_1, \dots, a_n$ non hanno fattori primi in comune e ognuno di loro divide $c$, allora anche il loro prodotto dovrà dividere $c$, perché il loro prodotto è formato esattamente dai fattori primi che dividono $c$.
\end{intuition}
\begin{proof}
    Dimostriamo la proposizione per induzione su $n$.
    \begin{description}
        \item[Caso base.] 
        Sia $n = 0$, cioè $a_1\dots a_n = 1$. Allora banalmente $1 \divides c$.
        \item[Passo induttivo.]         
        Supponiamo che la tesi sia vera per $n-1$ e dimostriamola per $n$. Dunque per ipotesi $ \left(\prod_{i = 1}^{n - 1} a_i\right) \divides c$.
        Ma per il corollario \ref{prodotto_coprimo_n} $a_n$ è coprimo con $\prod_{i = 1}^{n - 1} a_i$, dunque per la proposizione \ref{prodotto_numeri_coprimi} segue che
        \begin{equation*}
            a_n \left(\prod_{i = 1}^{n - 1} a_i\right) = \left( \prod_{i = 1}^{n} a_i \right) \divides c
        \end{equation*}
        che è la tesi per $n$.
    \end{description}
    Dunque la proposizione vale per ogni $n \in N$.
\end{proof}


\subsection{Divisori primi}
\begin{proposition} 
    [Esistenza della scomposizione in primi]
    \label{esistenza_scomposizione_primi}
    Sia $n \in \Z, n > 1$. Allora $n$ puo' essere espresso come prodotto di potenze di numeri primi.
\end{proposition}
\begin{proof}
    Per induzione forte su $n$.
    \begin{description}
        \item[Caso base.]
        Sia $n = 2$. Dato che $2$ è primo, allora è esprimibile come prodotto di numeri primi (in particolare è il prodotto di un solo termine, se stesso).
        \item[Passo induttivo.]
        Supponiamo che la tesi sia vera per $2, 3, \dots, n-1$ (induzione forte) e dimostriamola per $n$.
        Abbiamo due casi:
        \begin{itemize}
            \item se $n$ è primo, allora è un prodotto di primi e quindi la tesi vale;
            \item se $n$ non è primo allora dovranno esistere due numeri $1 < a, b < n$ tali che $n = ab$ (infatti se non esistessero $n$ sarebbe primo). Ma per l'ipotesi induttiva forte sappiamo che tutti i numeri compresi tra $2$ e $n-1$ inclusi sono scomponibili in fattori primi, dunque anche $n = ab$ dovrà esserlo.
        \end{itemize}
    \end{description}
    Dunque dal caso base e dal passo induttivo segue che la tesi vale per ogni $n \geq 2$.
\end{proof}

\begin{theorem}
    [Teorema fondamentale dell'aritmetica]
    Sia $n \in \Z$ e siano $p_1, p_2, \dots, p_k$ i primi che dividono $n$. Inoltre siano $e_1, e_2, \dots, e_k$ i massimi esponenti per cui vale che $p_i^{e_i} \divides n$ per ogni $1 \leq i \leq k$. Allora $n = p_1^{e_1}p_2^{e_2} \dots p_k^{e_k}$.
\end{theorem}
\begin{proof}
    Per la proposizione \ref{esistenza_scomposizione_primi} sappiamo che esistono $p_1, \dots, p_n$. Per la proposizione \ref{prodotto_coprimo_n} segue che \[
        p_1^{e_1}p_2^{e_2} \dots p_k^{e_k} \divides n    
    \]
    in quanto $p_1^{e_1}, p_2^{e_2}, \dots, p_k^{e_k}$ sono coprimi tra loro.

    Dunque $n = m \cdot p_1^{e_1}p_2^{e_2} \dots p_k^{e_k}$ per qualche $m \in \Z$.
    Supponiamo per assurdo che $m \neq 1$. Allora per la proposizione \ref{esistenza_scomposizione_primi} $m$ è scomponibile in numeri primi; ma dato che $m$ è un divisore di $n$ segue che i primi che dividono $m$ devono dividere anche $n$, dunque i primi che dividono $m$ devono essere tra $p_1, \dots, p_k$. 

    Supponiamo senza perdita di generalità che $p_i$ divida $m$. Allora dato che $m \cdot p_1^{e_1}p_2^{e_2} \dots p_k^{e_k} = n$ deve essere $p_i \cdot p_i^{e_i} = p_i^{e_i+1} \divides n$, che è assurdo in quanto abbiamo supposto che $e_i$ fosse il massimo esponente per cui $p_i^{e_i} \divides n$. 
    
    Dunque deve essere $m = 1$, cioè \[
        n = p_1^{e_1}p_2^{e_2} \dots p_k^{e_k}
    \]
    come volevasi dimostrare.
\end{proof}

\begin{proposition}\label{mcd_mcm_in_termini_di_divisori_primi}
    Siano $a, b, k \in \Z$, $p \in \Z$ primo. Allora
    \begin{alignat}{1}
        p^k \divides \mcd{a}{b} &\iff p^k \divides a \land p^k \divides b \\ 
        p^k \divides \mcm{a}{b} &\iff p^k \divides a \lor p^k \divides b.
    \end{alignat}
\end{proposition}
\begin{intuition}
    Il massimo comun divisore di due numeri è un divisore comune ad entrambi, quindi se $p^k$ lo divide deve dividere entrambi i numeri.

    Il minimo comune multiplo invece è formato da tutti i fattori primi comuni e non comuni col massimo esponente, quindi se $p^k$ divide il minimo comune multiplo dovrà dividere almeno uno dei due numeri di partenza.
\end{intuition}

\begin{proposition}\label{mcm_equals_product}
    Siano $a, b \in \Z$. Allora se $\mcd{a}{b} = 1$ segue che $\mcm{a}{b} = \abs{ab}$.
\end{proposition}
\begin{intuition}
    Se i due numeri sono coprimi, allora non hanno fattori primi in comune, dunque il loro minimo comune multiplo sarà formato precisamente da tutti i fattori di entrambi i numeri, cioè dal loro prodotto.
\end{intuition}
\begin{proof}
    Sappiamo per definizione di mcm che $a \divides \mcm{a}{b}$ e $b \divides \mcm{a}{b}$. Dato che $\mcd{a}{b} = 1$ per la proposizione \ref{prodotto_numeri_coprimi} segue che $ab \divides \mcm{a}{b}$, cioè $\abs{ab} \leq \mcm{a}{b}$. Ma $ab$ è un multiplo di $a$ e di $b$, quindi dovrà valere che $\abs{ab} \geq \mcm{a}{b}$ in quanto $\mcm{a}{b}$ è il minimo multiplo comune ad $a$ e $b$. Da cio' segue che $\mcm{a}{b} = \abs{ab}$, cioè la tesi.
\end{proof}

\begin{proposition} \label{mcd_togliere_fattori_non_comuni}
    Siano $a, x, y \in \Z$. Allora 
    \begin{equation}
        \mcd{a}{x} = 1 \implies \mcd{a}{xy} = \mcd{a}{y}.
    \end{equation}
\end{proposition}
\begin{intuition}
    Se stiamo calcolando $\mcd{a}{b}$ dove $b = xy$ e sappiamo che il fattore $x$ non è comune tra $b$ ed $a$, allora possiamo escluderlo dal massimo comun divisore.
\end{intuition}
\begin{proof}
    Dato che $\mcd{a}{x} = 1$, allora se un primo $p$ divide $a$ sicuramente $p$ non divide $x$. Per la proposizione \ref{mcd_mcm_in_termini_di_divisori_primi} allora vale
    \begin{alignat*}
        {1}
        &p^k \divides \mcd{a}{xy} \\ 
        \iff &p^k \divides a \land p^k \divides xy\\
        \intertext{ma $p^k \nmid x$ dunque per la \ref{n_divides_product}}
        \iff &p^k \divides a \land p^k \divides y \\
        \iff &p^k \divides \mcd{a}{y}.
    \end{alignat*}
    Dato che $\mcd{a}{xy}$ e $\mcd{a}{y}$ vengono divisi dagli stessi primi, per il teorema fondamentale devono essere uguali.
\end{proof}

\begin{proposition} \label{distributivita_mcd_su_mcm}
    Siano $a, x, y \in \Z$. Allora 
    \begin{equation}
        \mcd{a}{\mcm{x}{y}} = \mcm{\mcd{a}{x}}{\mcd{a}{y}}.
    \end{equation}
\end{proposition}
\begin{proof}
    Per la proposizione \ref{mcd_mcm_in_termini_di_divisori_primi} allora vale
    \begin{alignat*}
        {1}
        &p^k \divides \mcd{a}{\mcm{x}{y}}\\ 
        \iff &p^k \divides a \land (p^k \divides x \lor p^k \divides y)\\
        \iff &(p^k \divides a \land p^k \divides x) \lor (p^k \divides a \land p^k \divides y) \\
        \iff &p^k \divides \mcm{\mcd{a}{x}}{\mcd{a}{y}}.
    \end{alignat*}
    Dato che $\mcd{a}{\mcm{x}{y}}$ e $\mcm{\mcd{a}{x}}{\mcd{a}{y}}$ vengono divisi dagli stessi primi, per il teorema fondamentale devono essere uguali.
\end{proof}

\begin{proposition}
    Siano $a, x, y \in \Z$. Allora 
    \begin{equation}
        \mcd{x}{y} = 1 \implies \mcd{a}{xy} = \mcd{a}{x}\mcd{a}{y}.
    \end{equation}
\end{proposition}
\begin{intuition}
    Se $x$ e $y$ non hanno fattori in comune, i fattori che $a$ ha in comune con il loro prodotto sono o in $x$ o in $y$, quindi per ottenerli tutti possiamo dividere l'mcd in due e moltiplicare i due risultati.
\end{intuition}
\begin{proof}
    Dato che $\mcd{x}{y} = 1$ allora per la proposizione \ref{mcm_equals_product} vale che $\mcm{x}{y} = \abs{xy}$.
    Dunque $\mcd{a}{xy} = \mcd{a}{\abs{xy}} = \mcd{a}{\mcm{x}{y}} = \mcm{\mcd{a}{x}}{\mcd{a}{y}}$ per la proposizione \ref{distributivita_mcd_su_mcm}.

    Verifichiamo ora che $\mcd{a}{x}$ e $\mcd{a}{y}$ sono coprimi. Per ipotesi sappiamo che $x, y$ sono coprimi; ma dato che $\mcd{a}{x}$ e $\mcd{a}{y}$ sono divisori di $x$ e $y$ rispettivamente, allora dovranno essere anche loro coprimi.

    Dunque per la proposizione \ref{mcm_equals_product} segue che \[
        \mcd{a}{xy} = \mcm{\mcd{a}{x}}{\mcd{a}{y}} = \mcd{a}{x}\mcd{a}{y}
    \] che è la tesi.
\end{proof}

\begin{proposition}\label{a,b|c_iff_(ab/mcd)|c}
    Siano $a, b, c \in \Z$. Allora \begin{equation}
        a \divides c \land b \divides c \iff \frac{ab}{\mcd{a}{b}} \divides c.
    \end{equation}
\end{proposition}
\begin{proof}
    Dimostriamo l'implicazione in entrambi i versi.
    \begin{itemize}
        \item[($\implies$)] Supponiamo che $a \divides c$ e $b \divides c$. Sia $d = \mcd{a}{b}$. Allora dato che $d \divides a$, $d \divides b$ per transitività $d \divides c$, dunque $\frac{a}{d} \divides \frac{c}{d}$ e $\frac{b}{d} \divides \frac{c}{d}$. Ma dato che per il corollario \ref{mcd_diviso_mcd} sappiamo che $\mcd{\frac{a}{d}}{\frac{b}{d}} = 1$, dunque per la \ref{prodotto_numeri_coprimi} segue che il loro prodotto $\frac{ab}{d^2}$ dovrà dividere $\frac{c}{d}$, che è equivalente a dire che $\frac{ab}{d} \divides c$.
        \item[($\impliedby$)] NON SO FARE QUEST'ALTRA DIMOSTRAZIONE \qedhere
    \end{itemize}
\end{proof}

\begin{proposition}
    \label{mcm*mcd=ab}
    Siano $a, b \in \Z$. Allora
    \begin{equation}
        \mcd{a}{b}\mcm{a}{b} = \abs{ab}.
    \end{equation}
\end{proposition}
\begin{proof}
    Sia $c \in \Z$ tale che $a \divides c$, $b \divides c$. Allora per la proposizione \ref{a,b|c_iff_(ab/mcd)|c} segue che $\frac{ab}{\mcd{a}{b}} \divides c$. Inoltre per la proposizione \ref{mcm|c_iff_a,b|c} segue che $\mcm{a}{b} \divides c$.
    Dunque i due numeri $\frac{ab}{\mcd{a}{b}}$ e $\mcm{a}{b}$ hanno gli stessi divisori, dunque devono essere uguali a meno del segno, da cui segue \[
        \mcd{a}{b}\mcm{a}{b} = \abs{ab}. 
    \]
\end{proof}

\section{Equazioni diofantee}

\begin{definition}[Equazione diofantea]
    Siano $a, b, c \in \Z$ noti, $x, y \in \Z$ incognite. Allora un'equazione lineare della forma $ax + by = c$ si dice equazione diofantea.
\end{definition}

\begin{theorem}[Condizione necessaria e sufficiente per le diofantee]
    Siano $a, b, c \in \Z$. Allora l'equazione diofantea $ax + by = c$ ammette soluzioni se e solo se $\mcd{a}{b} \divides c$.
\end{theorem}
\begin{proof}
    Dimostriamo prima che se $\mcd{a}{b} \divides c$ allora esistono soluzioni di $ax + by = c$ e poi dimostriamo che se $\mcd{a}{b} \nmid c$ allora l'equazione $ax + by = c$ non ha soluzioni.
    \begin{itemize}
        \item Supponiamo che $c = k\mcd{a}{b}$ per qualche $k \in \Z$. Allora per il teorema di Bezout \ref{bezout} esistono $x', y^\prime \in \Z$ tali che $ax^\prime + by^\prime = \mcd{a}{b}$. Moltiplicando  entrambi i membri per $k$ otteniamo
        \begin{alignat*}{2} 
            k\mcd{a}{b} &= k(ax^\prime + by')
                        &= akx^\prime + bky'
                        &= a(kx') + b(ky')
        \end{alignat*}
        dunque $x = kx'$ e $y = ky'$ risolvono l'equazione diofantea.
        \item Supponiamo ora che $c$ non sia un multiplo di $\mcd{a}{b}$ e supponiamo per assurdo che l'equazione abbia soluzione, cioè che esistano $x, y \in \Z$ tali che $ax + by = c$. Sia $d = \mcd{a}{b}$.
        
        Per definizione di $\mcd{a}{b}$ e per la proposizione \ref{divides_sum_subtr_mult}, dato che $d \divides a$ e $d \divides b$ segue che $d \divides ax$, $d \divides by$ e dunque $d \divides ax + by$. Ma $ax + by = c$, quindi $d = \mcd{a}{b} \divides c$, che va contro le ipotesi.

        Dunque l'equazione diofantea non ha soluzione, cioè la tesi. \qedhere
    \end{itemize}
\end{proof}

\begin{theorem} [Soluzioni di una diofantea omogenea con coefficienti coprimi]\label{sol_diofantea_omogenea_coprimi}
    Siano $a, b \in \Z$ coprimi. Allora le soluzioni dell'equazione diofantea omogenea $ax + by = 0$ sono tutte e solo della forma $x = -kb, y = ka$ al variare di $k \in \Z$.
\end{theorem}
\begin{proof}
    Dimostriamo innanzitutto che $x = -kb, y = ka$ è una soluzione.
    \begin{alignat*}{1}
        ax + by &= a(-kb) + b(ka)\\
                &= -kab + kab\\
                &= 0.
    \end{alignat*}

    Mostriamo ora che non vi possono essere altre soluzioni. 
    
    Dato che $ax + by = 0$, allora $ax = -by$.
    Dato che $a \divides ax$ allora $a \divides -by$; inoltre per ipotesi $\mcd{a}{-b} = \mcd{a}{b} = 1$.
    Dunque per il teorema \ref{n_divides_product} segue che $a \divides y$, cioè $y = ak$ per qualche $k \in \Z$. Sostituendo ottengo $x = -b\frac{y}{a} = -bk$, che è la tesi.
\end{proof}

\begin{corollary} [Soluzioni di una diofantea omogenea]\label{sol_diofantea_omogenea}
    Se $a, b$ non sono coprimi, allora tutte le soluzioni dell'equazione $ax + by = 0$ saranno della forma $x = -kb^\prime, y = ka^\prime$ dove $a^\prime = \frac{a}{\mcd{a}{b}}$ e $b^\prime = \frac{b^\prime}{\mcd{a}{b}}.$
\end{corollary}
\begin{proof}
    Dato che $a, b$ non sono coprimi, allora possiamo dividere entrambi i membri di $ax + by = 0$ per $\mcd{a}{b}$ ottenendo l'equazione diofantea equivalente $a^\prime x + b^\prime y = 0$. 
    
    Ma per il teorema \ref{mcd_diviso_mcd} $\mcd{a^\prime}{b^\prime} = 1$, dunque per il teorema \ref{sol_diofantea_omogenea_coprimi} le sue soluzioni saranno tutte e solo della forma $x = -kb^\prime, y = ka^\prime$. 
    
    Ma questa equazione è equivalente all'originale, dunque anche le soluzioni di $ax + by = 0$ saranno tutte e solo della forma $x = -kb^\prime, y = ka^\prime$.
\end{proof}

\begin{theorem} [Soluzioni di una diofantea non omogenea] \label{sol_diofantea_non_omog}
    Siano $a, b \in \Z$ e sia $(x, y)$ una soluzione particolare dell'equazione diofantea $ax + by = c$ (se esiste). Allora le soluzioni di quest'equazione sono tutte e solo della forma $(x + x_0, y + y_0)$ al variare di $(x_0, y_0)$ tra le soluzioni dell'equazione omogenea associata $ax + by = 0$.
\end{theorem}
\begin{proof}
    Dimostriamo innanzitutto che se $(x, y)$ è una soluzione della diofantea non omogenea e $(x_0, y_0)$ è una soluzione dell'omogenea, allora $(x+x_0, y+y_0)$ è ancora soluzione della non omogenea.
    \begin{alignat*}{1}
        a(x + x_0) + b(y + y_0) &= ax + ax_0 + by + by_0 \\
                                &= (ax + by) + (ax_0 + by_0) \\
                                &= c + 0\\
                                &= c.
    \end{alignat*}

    Dimostriamo ora che tutte le soluzioni sono di questa forma. Sia $(\bar{x}, \bar{y})$ una soluzione particolare della diofantea non omogenea e $(x, y)$ un'altra soluzione qualsiasi, e mostriamo che la loro differenza è una soluzione dell'omogenea associata.
    \begin{alignat*}
        {1}
        a(x - \bar{x}) + b(y - \bar{y}) &= ax - a\bar{x} + by - b\bar{y} \\
                                        &= (ax + by) - (a\bar{x} + b\bar{y}) \\
                                        &= c - c\\
                                        &= 0
    \end{alignat*}
    che è la tesi.
\end{proof}

\chapter{Congruenze}

\section{Relazione di congruenza}

\begin{definition} \label{def_congr}
    Siano $a, b, m \in \Z$, $m > 0$. Allora si dice che $a$ e' congruo a $b$ modulo $m$ se e solo se $a - b$ e' un multiplo di $m$, e si scrive
    \begin{align*}
        &a \equiv b \pmod m
    \end{align*}
\end{definition}

\begin{theorem}
    Siano $a, b, m \in \Z$, $m > 0$. Allora la relazione di congruenza $\equiv \pmod(m)$ e' una relazione di equivalenza, e dunque soddisfa le proprieta':
    \begin{align}
        &\text{Riflessiva: } &&a \equiv a \Mod{m} \label{congr_rifl}\\
        &\text{Simmetrica: } &&a \equiv b \Mod{m} \implies b \equiv a \Mod{m} \label{congr_simm}\\
        &\text{Riflessiva: } &&a \equiv b \Mod{m} \land b \equiv c \Mod{m} \implies a \equiv c \Mod{m}  \label{congr_trans}
    \end{align}
\end{theorem}
\begin{proof} Dimostriamo le tre proprieta' della congruenza come relazione di equivalenza.
    \begin{enumerate}
        \item $a - a = 0 = 0m$, dunque $a \equiv a \Mod{m}$.
        \item Se $a - b = km$ allora $b - a = -(a - b) = -km = (-k)m$, cioe' $b \equiv a \Mod{m}$.
        \item Se $a - b = km$ e $b - c = hm$ allora $a - c = (a - b) + (b - c) = km + hm = (k + h)m$, 
            cioe' $a \equiv c \Mod{m}$.
    \end{enumerate}
\end{proof}

\begin{theorem} \label{equiv_congr_resto}
    Siano $a, b, m \in \Z$, $m > 0$. Allora
    \begin{equation}
        a \equiv b \Mod{m} \iff a \bmod m = b \bmod m
    \end{equation}
    cioe' $a$ e' congruo a $b$ se e solo se $a$ e $b$ hanno lo stesso resto quando divisi per $m$.
\end{theorem}
\begin{proof}
    Dimostriamo l'implicazione nei due versi.

    Siano $r = a \bmod m$, $r' = b \bmod m$ i resti di $a$ e $b$ modulo $m$, 
    cioe' $a = cq + r$ e $b = cq' + r'$ per qualche $q, q' \in \Z$. Supponiamo 
    che $r = a \bmod m = b \bmod m = b$. Allora
    \begin{alignat*}
        {1}
        a - b &= cq + r - cq' - r' \\
              &= c(q - q')
    \end{alignat*}
    cioe' $a \equiv b \Mod{m}$.

    Ora supponiamo che $a \equiv b \Mod{m}$ e dimostriamo che i resti di $a$ e $b$ modulo $m$ siano uguali.
    Per la proposizione \ref{esistenza_resto} esistono $q, r \in \Z$ tale che $b = mq + r$ e $0 \leq r < m$.
    Allora per definizione di congruenza per qualche $k \in \Z$ avremo
    \begin{alignat*}
        {1}
        a &= b + mk \\
          &= mq + r + mk \\
          &= m(q + k) + r
    \end{alignat*}
    cioe' $r$ e' il resto di $a$ modulo $m$.
\end{proof}

\begin{proposition}
    Siano $a, b, a', b', m \in \Z$, $m > 0$. Allora valgono le seguenti
    \begin{align}
        &a \equiv b \Mod{m}\land a' \equiv b' \Mod{m}\implies a+a' \equiv b+b'\Mod{m} \label{somma_congrui}\\
        &a \equiv b \Mod{m}\land a' \equiv b' \Mod{m}\implies a-a' \equiv b-b'\Mod{m} \label{differenza_congrui}\\
        &a \equiv b \Mod{m}\land a' \equiv b' \Mod{m}\implies aa' \equiv bb'\Mod{m} \label{prodotto_congrui}
    \end{align}
\end{proposition}
\begin{proof}
    \begin{enumerate}
        \item Per definizione di congruenza $m \divides a - b$ e $m \divides a' - b'$. Per la proposizione \ref{divides_sum_subtr_mult} segue che $m \divides (a - b) + (a' - b')$, cioe' $m \divides (a + a') - (b + b')$, che e' equivalente a $a+a' \equiv b+b'\Mod{m}$.
        \item Per definizione di congruenza $m \divides a - b$ e $m \divides a' - b'$. Per la proposizione \ref{divides_sum_subtr_mult} segue che $m \divides (a - b) - (a' - b')$, cioe' $m \divides (a - a') - (b - b')$, che e' equivalente a $a-a' \equiv b-b'\Mod{m}$.
        \item Per definizione di congruenza, scriviamo $a - b = km$ e $a' - b' = hm$, che e' equivalente a $b = a - km$ e $b' = a' - hm$. Dunque
        \begin{alignat*}
            {1}
            bb' &= (a - km)(a' - hm) \\
                &= aa' - ahm - a'km + khm \\
                &= aa' - (ah + a'k - kh)m \\
        \end{alignat*}    
        che e' equivalente a
        \begin{alignat*} {1}
            aa' - bb' &= (ah + a'k - kh)m \\
            \iff aa' &\equiv bb' \Mod{m}.
        \end{alignat*}
        
    \end{enumerate}
    
\end{proof}

\section{Equazioni con congruenze lineari}

\begin{proposition}
    Siano $a, b, c \in \Z$; sia $ax + by = c$ un'equazione diofantea. Allora tutte le soluzioni della diofantea sono soluzioni delle equazioni $ax \equiv c \Mod{b}$ e $by \equiv c \Mod{a}$.
\end{proposition}
\begin{proof}
    Dimostriamo entrambi i versi dell'implicazione.
    \begin{enumerate}
        \item Siano $x, y \in \Z$ tali che $ax + by = c$. Dato che $ax + by$ e' uguale a $c$ segue che $ax + by \equiv c \Mod{b}$. Ma $b \equiv 0 \Mod{b}$, dunque $x$ sara' anche soluzione di $ax \equiv c \Mod{b}$. Analogo ragionamento considerando $ax + by \equiv c \Mod{a}$.
        \item Sia $x \in Z$ tale che $ax \equiv c \Mod{b}$. Allora per definizione di congruenza esiste $k \in \Z$ per cui $ax - c = bk$. Sia $y = -k$; l'equazione e' quindi equivalente a $ax + by = c$, cioe' la coppia $(x, y)$ e' una soluzione dell'equazione diofantea. Analogo ragionamento se partiamo da $by \equiv c \Mod{a}$.
    \end{enumerate}
\end{proof}

Tramite questa proposizione possiamo risolvere ogni equazione contenente congruenze risolvendo l'equazione diofantea associata, o viceversa.

\begin{definition}
    Siano $a \in \Z$; allora si dice che $a$ e' invertibile modulo $m$ se esiste  $x \in \Z$ tale che \[
        ax \equiv 1 \Mod{m}
    .\]
    In particolare tra tutti gli $x$ che soddisfano la relazione precedente, il numero $x$ tale che $0 \leq x < m$ si dice inverso di $a$ modulo $m$.
\end{definition}
Per calcolare gli inversi modulo $m$ basta fare una tabella $m \times m$ in cui le righe e le colonne contengono i numeri tra $0$ e $m-1$, e nella casella $ij$ c'e' il prodotto tra i numeri $i$ e $j$ modulo $m$.

Notiamo che non sempre i numeri diversi da $0$ ammettono inverso modulo $m$.

\begin{theorem}\label{invertibilita_mod_m}
    Siano $a, m \in \Z$. Allora $a$ e' invertibile modulo $m$ se e solo se $\mcd{a}{m} = 1$. 
\end{theorem}
\begin{proof}
    Supponiamo $\mcd{a}{m} = 1$. Allora per il teorema di Bezout \ref{bezout} $\exists x, y \in \Z$ tali che
    \begin{alignat*}{1}
        &ax + my = 1 \\
        \iff &ax - 1 = m(-y) \\
        \iff &ax \equiv 1 \Mod{m}
    \end{alignat*}
    dunque $x$ e' l'inverso di $a$ modulo $m$.

    Supponiamo che $a$ sia invertibile modulo $m$, cioe' che $\exists x \in \Z$ tale che $ax \equiv= 1 \Mod{m}$. Ma sappiamo che $ax + my$ e' un multiplo di $\mcd{a}{m}$, quindi anche $1$ dovra' essere un multiplo di $\mcd{a}{m}$, cioe' $\mcd{a}{m} = 1$, che e' la tesi.
\end{proof}
\begin{corollary}
    Se $p$ e' primo e $a \not\equiv 0 \Mod{p}$, allora $a$ e' invertibile modulo $p$.
\end{corollary}
\begin{proof}
    Se $p$ e' primo, allora necessariamente $p$ e' coprimo con tutti i numeri che non sono suoi multipli, cioe' con tutti gli $a$ tali che $a \equiv_p 0$. Dunque se $a \equiv_p 0$ allora $\mcd{a}{p} = 1$, cioe' per il teorema precedente $a$ e' invertibile modulo $p$.
\end{proof}

\begin{proposition} \label{se_invertibile_allora_soluzione}
    Siano $a, b, m \in \Z$; allora se $a$ e' invertibile modulo $m$ segue che $\exists x \in \Z$ tale che $ax \equiv b \Mod{m}$.
\end{proposition}
\begin{proof}
    Dato che $a$ e' invertibile modulo $m$ esistera' $x' \in \Z$ tale che $ax' \equiv 1 \Mod{m}$. Moltiplicando entrambi i membri per $b$ otteniamo $ax'b \equiv b \Mod{m}$, dunque la $x \equiv x'b \Mod{b}$ soddisfa $ax \equiv b \Mod{m}$, cioe' la tesi.
\end{proof}

\begin{proposition} \label{cong_ha_soluzione_sse_mcd_div_b}
    Siano $a, b, m, x \in \Z$; allora l'equazione $ax \equiv b \Mod{m}$ ha soluzione se e solo se $\mcd{a}{m} \divides b$.
\end{proposition}
\begin{proof}
    Dimostriamo l'implicazione nei due versi.
    \begin{itemize}
        \item Supponiamo che $ax \equiv b \Mod{m}$ ammetta soluzione. Allora esiste $y \in \Z$ tale che $ax - my = b$. Dato che $a$ e $m$ sono multipli di $\mcd{a}{m}$, allora lo sara' anche la combinazione lineare $ax - my$ che e' uguale a $b$, cioe' $\mcd{a}{m} \divides b$.
        \item Supponiamo che $d = \mcd{a}{m}$ divida $b$. Allora $d\divides a$, $d \divides b$, $d \divides m$. Siano $a' = \frac{a}{d}, b' = \frac{b}{d}, m' = \frac{m}{d}$. Allora 
        \begin{align*}
            &ax \equiv b \Mod{m}\\
            \iff{} &ax - b = mk   &\text{per qualche $k \in \Z$} \\
            \iff{} &a'dx - b'd = m'dk &\text{per qualche $k \in \Z$} \\
            \iff{} &a'x - b' = m'k &\text{per qualche $k \in \Z$} \\
            \iff{} &a'x \equiv b'\Mod{m'}.
        \end{align*}
        Ma per il corollario \ref{mcd_diviso_mcd} $\mcd{a'}{m'} = 1$, dunque $a'$ e' invertibile modulo $m'$, dunque per la proposizione \ref{se_invertibile_allora_soluzione} segue che $a'x \equiv b' \Mod{m'}$ ha soluzione. Tuttavia $a'x \equiv b' \Mod{m'}$ e' equivalente a $ax \equiv b \Mod{m}$, dunque anche $ax \equiv b \Mod{m}$ ha soluzione e in particolare ha le stesse soluzioni di $a'x \equiv b' \Mod{m'}$.
    \end{itemize}
\end{proof}

\begin{proposition}
    Se vogliamo semplificare una congruenza possiamo sfruttare le seguenti regole:
    \begin{alignat}{3}
        A &\equiv B \Mod{m} \quad &\iff      \quad &A + c \equiv B + c \Mod{m} \\
        A &\equiv B \Mod{m} \quad &\implies  \quad &cA \equiv cB \Mod{m} \\
        A &\equiv B \Mod{m} \quad &\iff      \quad &(A \bmod m) \equiv (B \bmod m) \Mod{m} \\
        Ad &\equiv Bd \Mod{m} \quad &\implies\quad &A \equiv B \Mod{m} \qquad \text{se }\mcd{d}{m} = 1\\
        Ad &\equiv Bd \Mod{md} \quad &\iff   \quad &A \equiv B \Mod{m}
    \end{alignat}
\end{proposition}
\begin{proof}
    Dimostriamo le 5 proposizioni.
    \begin{enumerate}
        \item Dato che $c \equiv c \Mod{m}$, si tratta di un caso particolare della \ref{somma_congrui}. Inoltre l'implicazione inversa si ricava dalla \ref{differenza_congrui}, dunque si tratta di un'equivalenza.
        \item Dato che $c \equiv c \Mod{m}$, si tratta di un caso particolare della \ref{prodotto_congrui}.
        \item Dato che $A \equiv (A \bmod m) \Mod{m}$ e $B \equiv (B \bmod m) \Mod{m}$, per transitivita' otteniamo che $A \equiv B \Mod{m}$ e' equivalente a $(A \bmod m) \equiv (B \bmod m) \Mod{m}$.
        \item Se $\mcd{d}{m} = 1$ allora esiste l'inverso di $d$ modulo $m$. Chiamiamo $x$ questo inverso e moltiplichiamo entrambi i membri della congruenza per $x$, ottenendo
        \begin{alignat*}
            {1}
            Ad &\equiv Bd \Mod{m}  \\
            \iff Adx &\equiv Bdx \Mod{m} \\
            \iff A \cdot 1 &\equiv B \cdot 1 \Mod{m} \\
            \iff A &\equiv B \Mod{m}.
        \end{alignat*}
        \item Per definizione di congruenza esiste $y \in \Z$ tale che
        \begin{alignat*}
            {1}
            Ad &= Bd + mdy \\
            \iff A &= B + my \\
            \iff A &\equiv B \Mod{m}.
        \end{alignat*}
    \end{enumerate}
\end{proof}

\begin{proposition}
    Siano $a, b, m \in \Z$ noti, $x \in \Z$ non noto. Allora per risolvere l'equazione $ax \equiv b \Mod{m}$ possiamo ricondurci ad uno dei seguenti tre casi:
    \begin{enumerate}
        \item se $\mcd{a}{m} = 1$, allora l'equazione ha soluzione $x \equiv by \Mod{m}$, dove $y$ e' l'inverso di $a$ modulo $m$;
        \item se $\mcd{a}{m} \neq 1$, $d = \mcd{a}{m} \divides b$, allora l'equazione e' equivalente all'equazione $a'x \equiv b' \Mod{m'}$, con $a' = \frac{a}{d}$, $b' = \frac{b}{d}$, $m' = \frac{m}{d}$, che ha soluzione;
        \item se $\mcd{a}{m} \neq 1$, $\mcd{a}{m} \nmid b$, allora l'equazione non ha soluzione.
    \end{enumerate}
\end{proposition}
\begin{proof}
    I tre casi sono conseguenza diretta della proposizione \ref{cong_ha_soluzione_sse_mcd_div_b}. Infatti
    \begin{enumerate}
        \item Per la \ref{cong_ha_soluzione_sse_mcd_div_b} l'equazione ha soluzione. Se $y$ e' l'inverso di $a$, moltiplicando entrambi i membri per $y$ otteniamo la soluzione $x \equiv by \Mod{m}$.
        \item Per la \ref{cong_ha_soluzione_sse_mcd_div_b} l'equazione ha soluzione. 
        Sia $d = \mcd{a}{m}$. Allora la congruenza e' equivalente a $ax - b = mk$ per qualche $k \in \Z$. Dato che $a, b, m$ sono divisibili per $d$, dividendo per $d$ otteniamo l'equazione equivalente
        \begin{alignat*}
            {1}
            &\frac{a}{d}x - \frac{b}{d} = \frac{m}{d}k \\
            \iff &\frac{a}{d}x \equiv \frac{b}{d} \Mod{\frac{m}{d}}
        \end{alignat*}
        Ma per il corollario \ref{mcd_diviso_mcd} $\mcd{\frac{a}{d}}{\frac{m}{d}} = 1$, dunque possiamo trovare la soluzione sfruttando il primo caso.
        \item Per la \ref{cong_ha_soluzione_sse_mcd_div_b} l'equazione non ha soluzione. 
        % Scriviamo l'equazione nella forma equivalente $ax - mk = b$ per qualche $k \in \Z$. Dato che $\mcd{a}{m}$ divide sia $ax$ che $mk$, allora dovra' dividere anche $b$, ma per ipotesi $\mcd{a}{m} \nmid b$, dunque l'equazione non puo' avere soluzione.
    \end{enumerate}
\end{proof}

\section{Sistemi di congruenze}

\begin{theorem}
    [Teorema Cinese del Resto]
    Dato un sistema di congruenze in forma normale 
    \begin{equation*}
        \left\{
        \begin{alignedat}{1}
            x&\equiv a_1 \Mod{m_1}\\
            x&\equiv a_2 \Mod{m_2}\\
            &\vdotswithin{\equiv} \\
            x&\equiv a_n \Mod{m_n}
        \end{alignedat}      
        \right . 
    \end{equation*}
    se i moduli $m_1, m_2, \dots, m_n$ sono a due a due coprimi (cioe' se per ogni $i \neq j$ vale che $\mcd{m_i}{m_j} = 1$) allora il sistema ha soluzione, ed e' equivalente ad una singola congruenza del tipo
    \begin{equation}
        x \equiv x_0 \pmod{m_1 m_2 \dots m_n}.
    \end{equation} 
\end{theorem}

\begin{proposition}
    Dato un sistema di congruenze 
    \begin{equation*}
        \left\{
        \begin{alignedat}{1}
            a_1x &\equiv b_1 \Mod{m_1}\\
            a_2x &\equiv b_2 \Mod{m_2}\\
            &\vdotswithin{\equiv} \\
            a_nx &\equiv b_n \Mod{m_n}
        \end{alignedat}      
        \right . 
    \end{equation*}
    se $x_0$ e' una soluzione particolare, allora tutte le soluzioni del sistema si ottengono sommando a $x_0$ un multiplo di $\operatorname{mcm}(m_1, m_2, \dots, m_n)$; o equivalentemente la soluzione del sistema e' una singola congruenza della forma
    \begin{equation}
        x \equiv x_0 \pmod{\operatorname{mcm}(m_1, m_2, \dots, m_n)}
    \end{equation}
\end{proposition}

\section{Struttura algebrica degli interi modulo m}

\begin{definition}
    Siano $a, n \in \Z$; allora si dice classe di resto $[a]_n$ l'insieme 
    \begin{equation}
        [a]_n = \left\{x \in \Z \mid x \equiv a \Mod{n}\right\}.
    \end{equation}
    Il numero $a$ si dice rappresentante della classe $[a]_n$.
\end{definition}

Due classi di resto si dicono uguali se contengono gli stessi elementi.
Il rappresentante di una classe non e' unico, anzi per ogni classe ci sono infinite scelte che corrispondono a tutti i numeri appartenenti alla classe. Vale quindi la seguente osservazione:
\begin{remark}
    $a \equiv b \Mod{m} \iff [a]_n = [b]_n$.
\end{remark}

Notiamo che per ogni numero $n$ ci sono esattamente $n$ classi di resto modulo $n$: infatti ce n'e' una esattamente per ogni possibile resto della divisione per $n$, cioe' per ogni numero tra $0$ e $n-1$ inclusi.

\begin{definition}
    Si dice insieme degli interi modulo $n$ l'insieme
    \begin{equation}
        \Z/(n) = \left\{ [0]_n, [1]_n, \dots, [n-1]_n\right\}.
    \end{equation}
\end{definition}

Possiamo definire due operazioni in $\Z/(n)$ che sono le operazioni di somma ($+ : \Z/(n) \times \Z/(n) \to \Z/(n)$) e prodotto ($\cdot : \Z/(n) \times \Z/(n) \to \Z/(n)$) tali che:
\begin{align}
    &[a]_n + [b]_n = [a+b]_n    &\forall [a]_n, [b]_n \in \Z/(n)\\
    &[a]_n \cdot [b]_n = [ab]_n &\forall [a]_n, [b]_n \in \Z/(n)
\end{align}

\begin{remark}
    Le operazioni di somma e prodotto sono ben definite: il loro risultato non cambia a seconda dei rappresentanti scelti per le classi di congruenza.
\end{remark}

\begin{proposition}\label{Z(n)_anello}
    Per ogni $n \geq 2$ l'insieme $\Z/(n)$ con le operazioni di somma e prodotto tra classi e con gli elementi $[0]_n, [1]_n$ che svolgono il ruolo di $0$ e $1$ e' un anello commutativo.
\end{proposition}
\begin{proof}
    E' facile verificare che valgono gli assiomi degli anelli.
\end{proof}

\begin{proposition}
    Per ogni $p \geq 2$, $p$ primo, l'insieme $\Z/(p)$ con le operazioni di somma e prodotto tra classi e con gli elementi $[0]_n, [1]_n$ che svolgono il ruolo di $0$ e $1$ e' un campo.
\end{proposition}
\begin{proof}
    Per la proposizione \ref{Z(n)_anello} sappiamo che $\Z/(p)$ e' un anello commutativo. Per la proposizione \ref{invertibilita_mod_m} un numero e' invertibile modulo $p$ se e solo se e' coprimo con $pm$; ma tutti i numeri che non sono multipli di $p$ sono coprimi con $p$, dunque tutte le classi tranne $[0]_p$ sono invertibili, dunque esiste l'inverso per la moltiplicazione per ogni elemento non nullo, cioe' $\Z/(p)$ e' un campo.
\end{proof}

\section{Binomiale e Triangolo di Tartaglia}

\begin{definition}
    Si dice \textbf{coefficiente binomiale} $\binom{n}{k}$ il numero intero tale che \begin{equation}
        \binom{n}{k} = \frac{n!}{k!(n-k)!}
    \end{equation}    
\end{definition}

\begin{proposition}\label{simmetria_binomiale}
    Sia $n \in \Z$, $k \in \Z$ tale che $0 \leq k \leq n$. Allora \begin{equation}
        \binom{n}{k} = \binom{n}{n-k}
    \end{equation}
\end{proposition}
\begin{proof}
    \[\binom{n}{n - k} = \frac{n!}{(n-k)!(n-(n-k))!} = \frac{n!}{k!(n-k)!} = \binom{n}{k} \qedhere\]
\end{proof}

\begin{proposition} \label{binomiale_ricorsivo}
    Sia $n \in \Z$, $k \in \Z$ tale che $0 \leq k \leq n$. Allora \begin{equation}
        \binom{n}{k} = \begin{cases}
            1 &\text{se } k = 0 \text{ oppure } k = n \\
            \binom{n - 1}{k - 1} + \binom{n - 1}{k} &\text{altrimenti}.
        \end{cases}
    \end{equation}
\end{proposition}
\begin{proof}
    Se $k = 0$ allora \[\binom{n}{0} = \frac{n!}{0!(n-0)!} = \frac{n!}{n!} = 1.\] 
    Inoltre per la proposizione \ref{simmetria_binomiale} segue che \[\binom{n}{n} = \binom{n}{n - n} = \binom{n}{0} = 1.\]

    Se $0 < k < n$ allora \begin{alignat*}
        {1}
        \binom{n - 1}{k - 1} + \binom{n - 1}{k} &= \frac{(n-1)!}{(k-1)!(n-1-(k-1))!} + \frac{(n-1)!}{(k)!(n-1-k)!} \\[1em]
        &= \frac{(n-1)!}{(k-1)!(n-1-k)!(n-k)} + \frac{(n-1)!}{k(k-1)!(n-1-k)!} \\[1em]
        &= \frac{(n-1)!k + (n-k)(n-1)!}{k(k-1)!(n-1-k)!(n-k)} \\[1em]
        &= \frac{(n-1)!k + n(n-1)! - k(n-1)!}{k!(n-k)!} \\
        &= \frac{n!}{k!(n-k)!} \\[1em]
        &= \binom{n}{k}
    \end{alignat*}
    che e' la tesi.
\end{proof}

\begin{theorem}[del binomiale] \label{binomiale}
    Siano $x, y, n \in \Z$. Allora vale che
    \begin{equation}
        (x+y)^n = \binom{n}{0}x^0y^n + \binom{n}{1}x^1y^{n-1} + \dots + \binom{n}{n}x^ny^0 = \sum_{k=0}^n \binom{n}{k}x^{n-k}y^k
    \end{equation}
\end{theorem}

\begin{definition}
    Si dice triangolo di Tartaglia un triangolo che ha le seguenti proprieta':
    \begin{enumerate}
        \item le righe sono numerate a partire da $0$;
        \item ogni riga ha $n + 1$ elementi, che vengono numerati da $0$ a $n$;
        \item l'elemento in riga $n$ e posizione $k$ si indica con $T_{n, k}$;
        \item $T_{n, 0} = T_{n, n} = 1$;
        \item per ogni $n \geq 0$, $0 < k \leq n$, $T_{n + 1, k} = T_{n, k - 1} + T_{n, k}$.
    \end{enumerate}
\end{definition}


\begin{proposition}
    Sia $n \in \Z$. Allora per ogni $k \in \Z$ tale che $0 \leq k \leq n$ segue che \begin{equation}
        T_{n,k} = \binom{n}{k}
    \end{equation}
\end{proposition}
\begin{proof}
    Per induzione su $n$.
    \begin{itemize}
        \item[\textbf{Caso base.}]

        Sia $n = 0$, allora dato che $0 \leq k \leq n$ segue che $k = 0$. Dunque
        \[
            T_{0, 0} = 1 = \binom{0}{0}.    
        \]
        \item[\textbf{Passo induttivo.}]
        
        Supponiamo che la tesi sia vera per $n$ e dimostriamola per $n+1$. 
        \begin{itemize}
            \item Se $k = 0$ oppure $k = n + 1$ allora per definizione del triangolo di Tartaglia $T_{n+1, 0} = T_{n+1, n+1} = 1$ che e' esattamente $\binom{n+1}{0} = \binom{n+1}{n+1}$ (per la proposizione \ref{binomiale_ricorsivo}),
            \item Se $0 < k < n+1$ allora per definizione del triangolo di Tartaglia segue che \[
                T_{n+1, k} = T_{n, k-1} + T_{n, k} = \binom{n}{k-1} + \binom{n}{k} = \binom{n+1}{k}    
            \] dove l'ultimo passaggio viene dalla proposizione \ref{binomiale_ricorsivo}.
        \end{itemize}
    \end{itemize}
    Dunque la tesi e' vera per ogni $n \in \Z$.
\end{proof}

\begin{proposition}
    Il triangolo di Tartaglia gode delle seguenti proprieta':
    \begin{enumerate}
        \item la somma degli elementi della riga $n$ e' $2^n$;
        \item la somma a segni alterni degli elementi di ogni riga e' $0$;
        \item nella riga $n$, l'elemento al posto $k$ e l'elemento al posto $n-k$ hanno lo stesso valore.
    \end{enumerate}
\end{proposition}
\begin{proof}
    Dimostriamo le tre proposizioni.
    \begin{enumerate}
        \item Dimostriamo che $2^n = \sum_{k=0}^n T_{n, k} = \sum_{k=0}^n \binom{n}{k}$.
        \[2^n = (1+1)^n = \sum_k^0 \binom{n}{k}1^{n-k}1^k = \sum_{k=0}^n \binom{n}{k}\]
        \item La somma a segni alterni della riga $n$-esima e' \[\sum_{k=0}^n (-1)^kT_{n, k} = \sum_{k=0}^n (-1)^k\binom{n}{k} = \sum_{k=0}^n (-1)^k1^{n-k}\binom{n}{k} = (1-1)^k = 0^k = 0.\]
        \item Dobbiamo dimostrare che $T_{n, k} = T_{n, n-k}$. Ma dato che $T_{n, k} = \binom{n}{k}$ e $T_{n, n-k} = \binom{n}{n-k}$, allora questo e' equivalente a dimostrare che $\binom{n}{k} = \binom{n}{n-k}$, che e' vero per la proposizione \ref{simmetria_binomiale}. \qedhere
    \end{enumerate}
\end{proof}

\begin{proposition}\label{binomio_pk_divisibile_p}
    Se $p$ e' primo, allora per ogni $k$ tale che $0 < k < p$ vale che
    \begin{equation}
        \binom{p}{k} \equiv 0 \Mod{p}
    \end{equation}
\end{proposition}
\begin{proof}
    Consideriamo un $k$ generico tale che $0 < k < p$.
    Allora \[
        \binom{p}{k} = \frac{p!}{k!(p-k)!} \iff p! = \binom{p}{k}(p-k)!k!    
    \]
    Ma $p \divides p!$, dunque $p \divides \binom{p}{k}(p-k)!k!$, dunque per la proposizione \ref{primo_divide_prodotto} segue che \[
        p \divides \binom{p}{k} \text{ oppure } p \divides (p-k)! \text{ oppure } p \divides k!
    .\]

    Notiamo che sia $k$ che $p-k$ sono numeri minori di $p$, dunque $k!$ e $(p-k)!$ sono un prodotto di numeri minori di $p$. Ma $p$ e' primo, dunque e' coprimo con tutti i numeri che non siano un multiplo di $p$ (e quindi e' coprimo con tutti i numeri compresi tra $0$ e $p$ esclusi), dunque per la proposizione \ref{prodotto_coprimo_n} $p$ deve essere coprimo anche con $k!$ e con $(p-k)!$. 
    
    Da cio' segue che $p$ non puo' dividere $k!$ e $(p-k)!$.
    L'ultima possibilita' e' che $p \divides \binom{p}{k}$, che e' equivalente a dire che $\binom{p}{k} \equiv 0 \Mod{p}$.
\end{proof}

\begin{proposition}\label{(x+y)^p_congr_x^p+y^p}
    Siano $x, y, p \in \Z$, $p$ primo. Allora
    \begin{equation}
        (x+y)^p \equiv x^p + y^p \Mod{p}
    \end{equation}
\end{proposition}
\begin{proof}
    Per il teorema del Binomiale (\ref{binomiale}) sappiamo che
    \begin{alignat*}{1}
        (x+y)^p &= \binom{p}{0}x^p + \binom{p}{1}x^{p-1}y^1 + \dots + \binom{p}{i}x^{p-i}y^i + \dots + \binom{p}{p}y^p \\
        \intertext{Ma per la proposizione \ref{binomio_pk_divisibile_p} tutti i termini intermedi di questa somma sono congrui a $0$ modulo $p$, dunque:}
        &\equiv \binom{p}{0}x^p + \binom{p}{p}y^p \Mod{p}\\
        &\equiv x^p + y^p \Mod{p}
    \end{alignat*}
    come volevasi dimostrare.
\end{proof}

\begin{corollary}\label{(x_1+x_n)^p_congr_x_1^p+x_n^p}
    Siano $x_1, x_2, \dots, x_n, p \in \Z$, $p$ primo. Allora
    \begin{equation}
        (x_1+x_2+\dots+x_n)^p \equiv x_1^p + x_2^p + \dots + x_n^p \Mod{p}
    \end{equation}
\end{corollary}
\begin{proof}
    Per induzione su n.
    \begin{itemize}
        \item[\textbf{Caso base.}]

        Sia $n = 1$. Allora $x_1^p \equiv x_1^p \Mod{p}$ ovviamente.
        \item[\textbf{Passo induttivo.}]
        
        Supponiamo che la tesi sia vera per $n-1$ e dimostriamola per $n$.
        \begin{alignat*}{1}
            (x_1+x_2+\dots+x_n)^p &\equiv ((x_1+x_2+\dots+x_{n-1})+x_n)^p \Mod{p}\\
            \intertext{(per la proposizione \ref{(x+y)^p_congr_x^p+y^p})}
            &\equiv (x_1+x_2+\dots+x_{n-1})^p+x_n^p\Mod{p}\\
            \intertext{(per ipotesi induttiva)}
            &\equiv x_1^p + x_2^p + \dots + x_{n-1}^p + x_n^p\Mod{p}
        \end{alignat*}
        che e' la tesi per $n$.
    \end{itemize}
    Dunque dal caso base e dal passo induttivo segue che la tesi vale per ogni $n$.
\end{proof}

\begin{theorem}
    [Piccolo Teorema di Fermat] \label{th_fermat}
    Se $p$ e' primo, allora $x^p \equiv x \Mod{p}$.
\end{theorem}
\begin{proof}
    \begin{alignat*}{1}
        x^p &\equiv (\overbrace{1 +\dots+ 1}^{x \text{ volte}})^p \Mod{p}\\
        \intertext{(per il corollario \ref{(x_1+x_n)^p_congr_x_1^p+x_n^p})}
        &\equiv \overbrace{1^p +\dots+ 1^p}^{x \text{ volte}} \Mod{p}\\
        &\equiv \overbrace{1 +\dots+ 1}^{x \text{ volte}} \Mod{p}\\
        &\equiv x \Mod{p}
    \end{alignat*}
    che e' la tesi.
\end{proof}

\begin{corollary} \label{corollario_fermat}
    Se $p$ e' primo e $x \nequiv 0 \Mod{p}$ allora $x^{p-1} \equiv 1 \Mod{p}$.
\end{corollary}
\begin{proof}
    Per il piccolo teorema di Fermat (\ref{th_fermat}) vale che $x^p \equiv x \Mod{p}$. Dato che $x \nequiv 0 \Mod{p}$ allora segue che $p$ e $x$ sono coprimi, dunque $x$ e' invertibile modulo $p$. Moltiplicando entrambi i membri per l'inverso $x^{-1}$ otteniamo \begin{alignat*}
        {1}
        &x^px^{-1} \equiv x\cdot x^{-1} \Mod{p}\\
        \iff &x^{p-1} \equiv 1 \Mod{p}
    \end{alignat*}
    che e' la tesi.    
\end{proof}

\section{Congruenze esponenziali}

Iniziamo con un esempio di congruenza esponenziale.
\begin{example}
    Trovare tutte le soluzioni di $3^x \equiv 5 \Mod{7}$.
\end{example}
\begin{solution}
    Proviamo per tentativi:
    \begin{alignat*}
        {2}
        &x = 0 \implies &&3^0 \equiv 1 \nequiv 5 \Mod{7}\\
        &x = 1 \implies &&3^1 \equiv 3 \nequiv 5 \Mod{7}\\
        &x = 2 \implies &&3^2 \equiv 9 \equiv 2 \nequiv 5 \Mod{7}\\
        &x = 3 \implies &&3^3 \equiv 3^2 \cdot 3 \equiv 2 \cdot 3 \equiv 6 \nequiv 5 \Mod{7}\\
        &x = 4 \implies &&3^4 \equiv 3^2 \cdot 3^2 \equiv 2 \cdot 2 \equiv 4 \nequiv 5 \Mod{7}\\
        &x = 5 \implies &&3^5 \equiv 3^2 \cdot 3^3 \equiv 2 \cdot 6 \equiv 12 \equiv 5 \Mod{7}\\
        &x = 6 \implies &&3^6 \equiv 3^3 \cdot 3^3 \equiv 6 \cdot 6 \equiv 36 \equiv 1 \nequiv 5 \Mod{7}
    \end{alignat*}
    Dunque $x = 5$ e' una soluzione. Non possiamo dire pero' che le soluzioni sono tutti i numeri della forma $x = 5 + 7k$, perche' possiamo notiare che i numeri sembrano ripetersi con periodo $6$ e non $7$ (infatti $3^0 \equiv 3^6 \equiv 1 \Mod{7}$). 

    Dimostriamo che se $x = 5$ e' soluzione, allora anche $x = 5 +6k$ lo e'. Infatti \[
        3^{5 + 6k} \equiv 3^5 \cdot 3^{6k} \equiv 3^5 \cdot 1^k \equiv 5 \Mod{7}.
    \]
    Dunque le soluzioni sono tutte le $x$ tali che $x \equiv 5 \Mod{6}$. Questo vale anche per $x$ negativi, ma dobbiamo definire $x^{-1}$ non come $\frac{1}{x}$ ma come l'inverso di $x$ modulo $m$.
\end{solution}

\begin{definition}
    Siano $a, m \in \Z$, $a \nmid m$. Allora si dice ordine di $a$ modulo $m$ il piu' piccolo intero positivo $\ord{a}{m}$ tale che \begin{equation}
        a^{\ord{a}{m}} \equiv 1 \Mod{m}.
    \end{equation}
\end{definition}

\begin{remark}
    Notiamo che $\ord{a}{m}$ deve essere positivo, e dunque in particolare maggiore di $0$. Inoltre la condizione $a \nmid m$, che equivale a $a \nequiv 0 \Mod{m}$ serve ad evitare la congruenza banale $0^x \equiv b \Mod{m}$, che ha soluzione se e solo se $b \equiv 0 \Mod{m}$.
\end{remark}

\begin{proposition}\label{multipli_ord_equiv_1}
    Siano $a, m \in \Z$, $a \nmid m$. Allora per ogni $k \in \Z$ vale che \begin{equation}
        a^{k\ord{a}{m}} \equiv 1 \Mod{m}.
    \end{equation}
\end{proposition}
\begin{proof}
    \[
        a^{k\ord{a}{m}} \equiv (a^{\ord{a}{m}})^k \equiv 1^k \equiv 1 \Mod{m}. \qedhere
    \]
\end{proof}

\begin{proposition}\label{solo_multipli_ord_equiv_1}
    Siano $a, m \in \Z$, $a \nmid m$. Allora \begin{equation}
        a^x \equiv 1 \Mod{m} \iff x \equiv 0 \Mod{\ord{a}{m}}
    \end{equation}
\end{proposition}
\begin{proof}
    Per definizione di congruenza 
    \[x \equiv 0 \Mod{\ord{a}{m}} \iff x \mid \ord{a}{m} \iff x = \ord{a}{m}\cdot k\] 
    per qualche $k \in \Z$.

    Per l'unicita' del resto della divisione euclidea (\ref{esistenza_resto}) possiamo scrivere che $x = q\ord{a}{m} + r$ per qualche $q, r \in \Z$ con $0 \leq r < \ord{a}{m}$. Questo e' equivalente a dire \begin{alignat*}
        {1}
        &\begin{alignedat}
            {1}
            a^x &= a^{q\ord{a}{m} + r}\\
            &= a^{q\ord{a}{m}}\cdot a^r
        \end{alignedat} \\
        \intertext{che equivale a}
        &\begin{alignedat}
            {1}
            a^x &\equiv a^{q\ord{a}{m}}\cdot a^r \Mod{m} \\
            &\equiv 1 \cdot a^r \Mod{m}\\
            &\equiv a^r \Mod{m}
        \end{alignedat}
    \end{alignat*}
    dove abbiamo sfruttato la proposizione \ref{multipli_ord_equiv_1} per dire che $a^{q\ord{a}{m}}\equiv 1 \Mod{m}$.

    Dunque dato che $a^x \equiv a^r \Mod{m}$ segue che $a^x \equiv 1 \Mod{m}$ se e solo se $a^r \equiv 1 \Mod{m}$. Ma $r < \ord{a}{m}$, dunque se $r$ fosse maggiore di $0$ avremmo trovato un numero minore di $\ord{a}{m}$ per cui $a^r \equiv 1 \Mod{m}$, che e' assurdo poiche' va contro la minimalita' di $\ord{a}{m}$.

    Segue che $r = 0$, cioe' $x = q\ord{a}{m}$, cioe' equivalentemente $x \equiv 0 \Mod{\ord{a}{m}}$, come volevasi dimostrare.
\end{proof}

\begin{proposition}
    Siano $a, b, m \in \Z$, $a \nmid m$. Se $x_0 \in \Z$ e' una soluzione di $a^x \equiv b \Mod{m}$ allora le soluzioni sono tutte e solo della forma \begin{equation}
        x \equiv x_0 \Mod{\ord{a}{m}}.
    \end{equation}
\end{proposition}
\begin{proof}
    Dimostriamo che se $x = x_0 + k\ord{a}{m}$ allora $x$ e' soluzione.
    \begin{alignat*}
        {1}
        a^{x_0 + k\ord{a}{m}} &\equiv a^{x_0}a^{k\ord{a}{m}} \Mod{m} \\
        &\equiv b \cdot 1 \Mod{m} \\
        &\equiv b \Mod{m}.
    \end{alignat*}
    Dimostriamo ora che se $x$ e' soluzione, allora $x \equiv x_0 \Mod{\ord{a}{m}}$, cioe' equivalentemente $x - x_0 = k\ord{a}{m}$.
    \begin{alignat*}
        {1}
        a^{x - x_0} &\equiv a^{x}a^{-x_0} \Mod{m} \\
        &\equiv b \cdot b^{-1} \Mod{m} \\
        &\equiv 1 \Mod{m}.
    \end{alignat*}
    Ma per la proposizione \ref{solo_multipli_ord_equiv_1} $a^{x - x_0} \equiv 1 \Mod{m}$ se e solo se $x - x_0 \equiv 0 \Mod{\ord{a}{m}}$, cioe' se e solo se $x \equiv x_0 \Mod{\ord{a}{m}}$, che e' la tesi.
\end{proof}

\begin{proposition}
    Siano $a, p \in \Z$, $a \nmid p$, $p$ primo. Allora vale che $\ord{a}{p} \divides p-1$.
\end{proposition}
\begin{proof}
    Per il corollario al piccolo teorema di Fermat (\ref{corollario_fermat}) sappiamo che $a^{p-1} \equiv 1 \Mod{p}$, cioe' $p-1$ e' una soluzione dell'equazione $a^x \equiv 1 \Mod{p}$. 
    
    Per la proposizione \ref{solo_multipli_ord_equiv_1} segue che $p-1 \equiv 0 \Mod{\ord{a}{p}}$, cioe' $\ord{a}{p} \mid p-1$, che e' la tesi.
\end{proof}

Dunque se dobbiamo trovare l'ordine di un numero $a$ modulo un primo $p$ ci basta provare tutti i divisori di $p - 1$ fino a quando non troviamo il minimo divisore che soddisfa la proprieta'.


\end{document}
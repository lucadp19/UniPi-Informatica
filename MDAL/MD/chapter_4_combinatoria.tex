\chapter{Calcolo combinatorio}

\section{Insiemi e stringhe}

\begin{definition}
    Sia $A$ un insieme di elementi. Allora si dice cardinalita' di $A$ il numero di elementi contenuti in $A$, e si indica con $\abs{A}$.
\end{definition}
Ad esempio se $A = \left\{ 4, 1, 5, 7, 9\right\}$ allora $\abs{A} = 5$.

\begin{definition}
    Siano $A$, $B$ due insiemi. Allora si dice prodotto cartesiano di $A$ e di $B$ l'insieme $A \times B$ tale che \[
        A \times B = \left\{ (a, b) \mid a \in A, b \in B\right\}   
    .\]
\end{definition}
Il prodotto cartesiano di due insiemi e' quindi l'insieme delle coppie ordinate $(a, b)$ dove il primo elemento appartiene al primo insieme e il secondo appartiene al secondo insieme. 

Notiamo che le coppie con gli stessi elementi ma diverso ordine sono diverse: prendiamo $A = \left\{ 4, 5, 6 \right\}, B = \left\{ 3, 4, 5 \right\}$; allora \[
    A \times B = \left\{ (4, 3), (4, 4), (4, 5), (5, 3), (5, 4), (5, 5), (6, 3), (6, 4), (6, 5)\right\}   
.\] Quindi $(4, 5) \neq (5, 4)$ ad esempio.

Possiamo definire anche un prodotto cartesiano di tre o piu' insiemi nello stesso modo: gli elementi del prodotto cartesiano di $n$ insiemi si dicono $n$-uple oppure stringhe.

\begin{proposition}
    Siano $A_1, \dots, A_n$ insiemi. Allora \begin{equation}
        \abs{A_1 \cdots A_n} = \abs{A_1} \cdots \abs{A_n}.
    \end{equation}
\end{proposition}
\begin{proof}
    Possiamo scegliere il primo elemento tra uno qualunque degli elementi del primo insieme: abbiamo dunque $\abs{A_1}$ possibilita'; per ognuna di queste abbiamo $\abs{A_2}$ possibilita' di scegliere il secondo elemento, dunque abbiamo $\abs{A_1} \cdot \abs{A_2}$ possibilita' per i primi due; e cosi' via.
\end{proof}

\section{Conteggi particolari}

\subsubsection{Disposizioni con ripetizione}
Supponiamo di avere un insieme $A$ di cardinalita' $\abs{A} = n$ e voler costruire una stringa di $k$ elementi che appartengono ad $n$, potendoli anche ripetere.

Queste stringhe sono tutte e sole le stringhe formate scegliendo uno degli elementi di $A$ come primo carattere, uno degli elementi di $A$ come secondo, eccetera... Dunque sono tutte e solo le stringhe che appartengono a $\overbrace{A \times \dots \times A}^{k \text{ volte}} = A^k$ che ha cardinalita' $\abs{A^k} = \abs{A}^k = n^k$.

Questo numero si dice \textbf{numero delle disposizioni con ripetizioni di $\bm{n}$ elementi in $\bm{k}$ posizioni}, e si indica con $D'_{n, k} = n^k$.

\subsubsection{Disposizioni senza ripetizione}
Supponiamo di avere un insieme $A$ di cardinalita' $\abs{A} = n$ e voler costruire una stringa di $k$ elementi che appartengono ad $n$, senza poterli ripetere.

Seguendo il ragionamento di prima abbiamo $n$ possibilita' per il primo elemento, $n-1$ per il secondo, in quanto dobbiamo escludere il primo per evitare di ripeterlo, $n-2$ per il terzo, eccetera, fino ad arrivare a $n-(k-1) = n - k + 1$ per il $k$-esimo. Dunque il numero totale di possibilita' e' \[
    n(n-1)\cdots (n-k+1) = \frac{n(n-1)\cdots (n-k+1)(n-k)\cdots 2\cdot 1}{(n-k)\cdots 2\cdot 1} = \frac{n!}{(n-k)!}
\]

Questo numero si dice \textbf{numero delle disposizioni senza ripetizioni di $\bm{n}$ elementi in $\bm{k}$ posizioni}, e si indica con $D_{n, k} = \frac{n!}{(n-k)!}$.

\subsubsection{Permutazioni senza ripetizione}

Consideriamo una stringa di $n$ elementi distinti e cerchiamo di contare i modi in cui possiamo ordinarli per ottenere stringhe diverse. Notiamo che questo e' equivalente a chiederci in quanti modi possiamo creare una stringa di $n$ elementi a partire da un insieme di $n$ elementi, cioe' a quante sono le disposizioni senza ripetizione di $n$ elementi in $n$ posizioni: \[
    D_{n, n} = \frac{n!}{(n - n)!} = \frac{n!}{0!} = n!    
\]

Questo numero si dice \textbf{numero delle permutazioni senza ripetizioni di $\bm{n}$ elementi}, e si indica con $P_{n} = n!$.

\subsubsection{Anagrammi}

Consideriamo una stringa di $n$ elementi, non necessariamente distinti, e cerchiamo di contare i modi in cui possiamo scambiarli di posto ottenendo stringhe diverse.

Dato che alcuni elementi si ripetono, scambiandoli di posto otterremmo una stringa uguale all'originale, dunque il numero degli anagrammi di una parola di $n$ lettere non necessariamente distinte e' minore al numero delle permutazioni senza ripetizione.

Per capire come ottenere il numero di anagrammi, consideriamo un esempio. Prendiamo la parola ASSASSINI e rendiamo le lettere distinte tra loro aggiungendo dei pedici, ottenendo A$_1$S$_1$S$_2$A$_2$S$_3$S$_4$I$_1$NI$_2$. A questo punto il numero di anagrammi di questa parola e' esattamente il numero di permutazioni senza ripetizioni, dunque $9!$ stringhe. 

Se torniamo alla parola originale pero' alcune possibilita' sono contate piu' volte, in quanto si ottengono scambiando due lettere uguali (ad esempio A$_1$S$_1$S$_2$A$_2$S$_3$S$_4$I$_1$NI$_2$ e A$_1$S$_3$S$_2$A$_2$S$_1$S$_4$I$_1$NI$_2$ si ottengono l'una dall'altra scambiando una S, dunque contano come la stessa stringa). 

Possiamo scambiare le $4$ S tra loro in $4!$ modi, le A in $2!$ modi e le I in $2!$ modi; il risultato sara' quindi ottenuto dividendo il numero totale delle permutazioni per ciascuno dei modi di permutare le lettere uguali (che sono il numero di possibilita' che contiamo piu' volte), ottenendo in questo caso \[ 
    \frac{9!}{4!2!2!}
.\]

\subsubsection{Combinazioni senza ripetizioni}

Consideriamo un insieme di $n$ elementi distinti e cerchiamo di contare i modi di estrarre da questo insieme un sottoinsieme di $k$ elementi; dato che siamo interessati ai sottoinsiemi l'ordine degli elementi non e' rilevante, dunque il numero che cerchiamo deve essere minore di $D_{n, k}$.

Chiamiamo il numero di sottoinsiemi distinti \textbf{numero delle combinazioni senza ripetizioni} e indichiamolo con $C_{n, k}$. Possiamo notare che ogni sottoinsieme ottenuto corrisponde ad una stringa di lunghezza $k$ che puo' essere permutata in $P_k$ modi per ottenere tutte le disposizioni possibili. Dunque \begin{alignat*}
    {1}
    C_{n, k}P_k   &= D_{n, k}  \\
    \iff C_{n, k} &= \frac{D_{n,k}}{P_k}\\
                  &= \frac{n!}{k!(n-k)!} \\
                  &= \binom{n}{k}.
\end{alignat*}

\begin{example}
    Contare i sottoinsiemi di $\left\{ 1, 2, 3, 4, 5\right\}$ di cardinalita' $3$.
\end{example}
\begin{solution}
    Dato che stiamo cercando sottoinsiemi l'ordine non conta, dunque dobbiamo usare le combinazioni, ottenendo che la soluzione e' \[
        C_{5, 3} = \binom{5}{3} =  \frac{5!}{3!2!} = \frac{5*4}{2} = 10
    .\]
\end{solution}

\begin{example}
    Quante sono le stringhe binarie di lunghezza $n$? Quante sono le stringhe binarie di lunghezza $10$ con $3$ cifre '$1$' e $7$ cifre '$0$'?
\end{example}
\begin{solution}
    Ogni posizione di una stringa binaria puo' essere riempita con $2$ valori ('0' oppure '1') e ci sono $n$ posizioni, dunque il numero di stringhe possibili e' il numero di disposizioni con ripetizione di $2$ elementi in $n$ posizioni, cioe' $2^n$.

    Per risolvere il secondo punto possiamo scegliere due strade equivalenti.
    \begin{enumerate}[(i)]
        \item Prendiamo una qualunque stringa che rispetta le condizioni, come 1110000000; allora il numero di stringhe con 3 uni e 7 zeri e' il numero di anagrammi di questa stringa, che e' \[
            \frac{10!}{3!7!} = \frac{10 \cdot 9 \cdot 8}{3 \cdot 2} = 120.   
        \]
        \item Notiamo che il problema e' equivalente al seguente problema: ho 10 posizioni possibili e devo sceglierne 3 in cui mettere le cifre '1'; infatti a quel punto le altre 7 sono automaticamente riempite con zeri. Dunque dato che dobbiamo scegliere un sottoinsieme (non ordinato) di posizioni tra 10 posizioni, il risultato sara' \[
            \binom{10}{3} = \frac{10!}{3!7!} = \frac{10 \cdot 9 \cdot 8}{3 \cdot 2} = 120.   
        \]
        Scegliere le 7 posizioni per gli zeri e' ovviamente equivalente.
    \end{enumerate}
\end{solution}

\begin{definition}
    Sia $A$ un insieme. Allora si dice insieme delle parti di $A$ l'insieme $\mathcal{P}(A)$ tale che \[
        \mathcal{P}(A) = \left\{ X \mid X \subseteq A\right\}    
    .\]
\end{definition}

Ad esempio se $A = \{1, 2, 3\}$ allora \[
    \mathcal{P}(A) = \left\{ \varnothing, \{1\}, \{2\}, \{3\}, \{1, 2\}, \{1, 3\}, \{2, 3\}, \{1, 2, 3\} \right\}.
\]

\begin{proposition}
    Sia $A$ un insieme tale che $\abs{A} = n$. Allora $\abs{\mathcal{P}(A)} = 2^n$.
\end{proposition}
\begin{proof}
    Passiamo ad una rappresentazione alternativa dei sottoinsiemi di $A$: fissiamo un ordine per l'insieme iniziale, poi associamo ad ogni sottoinsieme una stringa binaria che ha in posizione $i$ il numero 1 se e solo se l'elemento in posizione $i$ nell'insieme iniziale e' contenuto nel sottoinsieme.

    Ad esempio, se $A = \{1, 2, 3, 4\}$ e il sottoinsieme e' $\{3, 4\}$ allora la stringa binaria ad esso associato e' 0011, in quanto gli elementi in prima e seconda posizione dell'insieme $A$ (1 e 2) non sono nel sottoinsieme, mentre gli elementi in terza e quarta sono nel sottoinsieme. 

    Questa associazione e' biunivoca, poiche' a ogni sottoinsieme corrisponde una e una sola stringa e ad ogni stringa corrispone uno e un solo sottoinsieme, dunque il numero di stringhe e' uguale al numero di sottoinsiemi.

    Dato che ci sono $2^n$ stringhe binarie di lunghezza $n$, allora ci saranno $2^n$ sottoinsiemi, dunque la cardinalita' dell'insieme delle parti $\abs{\mathcal{P}(A)}$ sara' $2^n$.
\end{proof}

\begin{example}
    Consideriamo l'insieme dei numeri da 1 a 100. In quanti modi posso formare un sottoinsieme di cardinalita' 3 in modo che la somma dei numeri sia pari?
\end{example}
\begin{solution}
    Posso farlo in due modi diversi:
    \begin{enumerate}[(i)]
        \item Scelgo tre numeri pari. Allora il problema diventa equivalente a scegliere tre numeri tra i numeri pari da 1 a 100, che sono 50; abbiamo quindi \[
            \binom{50}{3} = \frac{50!}{3!47!} = \frac{50 * 49 * 48}{3 \cdot 2}   
        \] modi per scegliere tre numeri pari.
        \item Scelgo due numeri dispari e un pari. I due numeri dispari vanno scelti in $\binom{50}{2}$ modi, in quanto non conta l'ordine; il pari puo' essere scelto tra uno qualsiasi dei 50 numeri pari, dunque abbiamo \[
            50\binom{50}{2} = 50\frac{50!}{2!48!} = 50\frac{50 * 49}{2}
        \] modi.
    \end{enumerate}
    In totale i modi per formare un sottoinsieme che rispetti le condizioni del testo sono \[
        \binom{50}{3} + 50\binom{50}{2} = 50\frac{50 * 49 * 48}{3 \cdot 2}\frac{50 * 49}{2}.           
    \]
\end{solution}

\begin{example}
    Quante sono le triple ordinate $(x, y , z)$ di numeri naturali tali che $x + y + z = 4$?
\end{example}
\begin{solution}
    Possiamo rappresentare una soluzione nel seguente modo:
    \[
        \overbrace{1 \dots 1}^{x \text{ volte}} \mid   \overbrace{1 \dots 1}^{y \text{ volte}} \mid\overbrace{1 \dots 1}^{z \text{ volte}}
    \] dove le linee verticali fanno da separatori tra i gruppi di uni che vanno contate nelle $x$, nelle $y$ e nelle $z$. 
    
    Dato che la somma deve fare $4$ devono esserci 4 uni e due separatori, cioe' in totale 6 oggetti; allora il numero di soluzioni puo' essere ottenuto trovando il numero di permutazioni di questi oggetti, che e' \[
        \frac{6!}{2!4!} = \frac{6 \cdot 5}{2} = 15.
    \]

    Alternativamente dopo aver rappresentato la soluzione come stringa di uni e separatori, potevamo equivalentemente scegliere 2 posizioni su 6 per mettere i separatori, e cio' si puo' fare in \[
        \binom{6}{2} = \frac{6!}{2!4!} = 15    
    \] modi. A quel punto tutte le altre posizioni vengono occupate da uni, dunque la soluzione al problema non cambia.
\end{solution}

\begin{example}
    Abbiamo 4 colori: giallo, rosso, verde e blu. \begin{enumerate}
        [(i)]
        \item Quanti colori posso formare usando 5 gocce di questi colori?
        \item Quanti colori posso formare usando 5 gocce di questi colori avendo a disposizione solo 3 gocce di ogni colore?
    \end{enumerate}
\end{example}
\begin{solution} Indichiamo con $g$ il numero di gocce di colore giallo che usiamo, $r$ la gocce di rosso, $v$ le gocce di verde e con $b$ le gocce di blu.
    \begin{enumerate}[(i)]
        \item Il problema e' equivalente a chiederci quante quadruple $(g, r, v, b)$ di naturali soddisfano l'equazione $g + r + v + b = 5$. Rappresentiamo come nell'esercizio precedente una soluzione come uni e separatori: \[
            \overbrace{1 \dots 1}^{g \text{ volte}} \mid \overbrace{1 \dots 1}^{r \text{ volte}} \mid \overbrace{1 \dots 1}^{v \text{ volte}} \mid \overbrace{1 \dots 1}^{b \text{ volte}}.
        \] Dunque abbiamo $8$ oggetti in totale, di cui $5$ sono uni e $3$ sono separatori. Segue quindi che il numero totale di modi per permutare questi oggetti e' \[
            \binom{8}{3} = \frac{8!}{5!3!} = \frac{8\cdot 7 \cdot 6}{6} = 56.
        \]
        \item Dal conto del punto (i) sappiamo che senza restrizioni abbiamo $56$ possibilita'. Da queste dobbiamo togliere tutte le possibilita' in cui usiamo 4 o 5 gocce di un colore.
     
        Ho esattamente 4 possibilita' di usare 5 gocce dello stesso colore: \[
            (5, 0, 0, 0); \quad (0, 5, 0, 0); \quad (0, 0, 5, 0); \quad (0, 0, 0, 5).
        \] 

        Contiamo ora in quanti modi posso avere quattro gocce dello stesso colore. Innanzitutto scelgo il colore di cui uso 4 gocce, e posso farlo in 4 modi; dopo scelgo il colore che uso per l'ultima goccia, ed ho 3 possibilita'; in tutto ho quindi $12 = 4 \cdot 3$ possibilita'. (Notiamo che in questo caso non sto scegliendo due colori tra 4 senza considerare l'ordine, in quanto il primo colore e' quello di cui uso 4 gocce, dunque l'ordine conta.)

        In totale avro' quindi $56 - 4 - 12 = 40$ possibilita'.
    \end{enumerate}
\end{solution}

Forniamo ora una dimostrazione del teorema del Binomiale (\ref{binomiale}) usando il calcolo combinatorio.
\begin{proof}[Dimostrazione del Teorema del Binomiale (\ref{binomiale})]
    Per definizione \[
        (x + y)^n = \overbrace{(x+y)\cdots (x+y)}^{n \text{ volte}}.
    \] Dunque il risultato dell'elevamento a potenza sara' la somma di tutti i monomi formati da una sequenza di $x$ e $y$ di lunghezza $n$, in quanto ogni addendo sara' dato dalla scelta di $x$ o di $y$ in ogni fattore di $(x+y)\cdots (x+y)$.

    Ad esempio \begin{alignat*}{1}
        (x+y)^3 &= (x+y)(x+y)(x+y) \\
                &= xxx + xxy + xyx + xyy + yxx + yxy + yyx + yyy\\
                &= xxx + (xxy + xyx + yxx) + (xyy + yxy + yyx) + yyy\\
                &= x^3 + 3x^2y + 3xy^2 + y^3.
    \end{alignat*}

    Ognuno di questi monomi puo' essere semplificato alla forma $x^ky^h$, ma dato che il numero di $x$ e $y$ deve essere $n$ segue che $k + h = n$, cioe' $h = n - k$, cioe' ogni monomio puo' essere scritto nella forma $x^ky^{n-k}$. Dato che ogni monomio ha al minimo 0 fattori uguali a $x$ e al massimo $n$ fattori uguali a $x$, sommando i monomi uguali insieme otterremo:
    \begin{alignat*}
        {1}
        (x+y)^n &= c_0x^0y^n + c_1x^1y^{n-1} + \dots + c_{n-1}x_{n-1}y_1 + c_nx^ny^0\\
        &= \sum_{k = 0}^n c_kx^ky^{n-k}
    \end{alignat*}
    dove i coefficienti $c_k$ rappresentano il numero di monomi uguali sommati insieme.

    Troviamo quindi un'espressione per $c_k$ per un $k$ generico. Dato che $c_k$ rappresenta il numero di termini nell'espressione originale che possono essere semplificati a $x^ky^{n-k}$, allora $c_k$ indica il numero di termini con $k$ fattori uguali a $x$ (edi conseguenza i restanti $n-k$ uguali a $y$). 
    
    Il numero di questi termini e' uguale al numero di modi in cui possiamo posizionare $k$ '$x$' in una stringa di $n$ elementi, che e' il numero di modi in cui possiamo scegliere $k$ posizioni da un insieme di $n$ elementi. Inoltre una volta scelte le posizioni per le $x$ le $y$ andranno in tutte le restanti $n-k$ posizioni, dunque \[
        c_k = \binom{n}{k}.
    \]

    Sostituendolo nell'espressione per $(x+y)^n$ otteniamo \[
        (x+y)^n = \sum_{k = 0}^n \binom{n}{k}x^ky^{n-k}
    \] che e' la tesi.
\end{proof}

\section{Poker}

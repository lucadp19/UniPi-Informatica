\chapter{Calcolo combinatorio}

\section{Insiemi e stringhe}

\begin{definition}
    Sia $A$ un insieme di elementi. Allora si dice cardinalita' di $A$ il numero di elementi contenuti in $A$, e si indica con $\abs{A}$.
\end{definition}
Ad esempio se $A = \left\{ 4, 1, 5, 7, 9\right\}$ allora $\abs{A} = 5$.

\begin{definition}
    Siano $A$, $B$ due insiemi. Allora si dice prodotto cartesiano di $A$ e di $B$ l'insieme $A \times B$ tale che \[
        A \times B = \left\{ (a, b) \mid a \in A, b \in B\right\}   
    .\]
\end{definition}
Il prodotto cartesiano di due insiemi e' quindi l'insieme delle coppie ordinate $(a, b)$ dove il primo elemento appartiene al primo insieme e il secondo appartiene al secondo insieme. 

Notiamo che le coppie con gli stessi elementi ma diverso ordine sono diverse: prendiamo $A = \left\{ 4, 5, 6 \right\}, B = \left\{ 3, 4, 5 \right\}$; allora \[
    A \times B = \left\{ (4, 3), (4, 4), (4, 5), (5, 3), (5, 4), (5, 5), (6, 3), (6, 4), (6, 5)\right\}   
.\] Quindi $(4, 5) \neq (5, 4)$ ad esempio.

Possiamo definire anche un prodotto cartesiano di tre o piu' insiemi nello stesso modo: gli elementi del prodotto cartesiano di $n$ insiemi si dicono $n$-uple oppure stringhe.

\begin{proposition}
    Siano $A_1, \dots, A_n$ insiemi. Allora \begin{equation}
        \abs{A_1 \cdots A_n} = \abs{A_1} \cdots \abs{A_n}.
    \end{equation}
\end{proposition}
\begin{proof}
    Possiamo scegliere il primo elemento tra uno qualunque degli elementi del primo insieme: abbiamo dunque $\abs{A_1}$ possibilita'; per ognuna di queste abbiamo $\abs{A_2}$ possibilita' di scegliere il secondo elemento, dunque abbiamo $\abs{A_1} \cdot \abs{A_2}$ possibilita' per i primi due; e cosi' via.
\end{proof}

\section{Conteggi particolari}

\subsubsection{Disposizioni con ripetizione}
Supponiamo di avere un insieme $A$ di cardinalita' $\abs{A} = n$ e voler costruire una stringa di $k$ elementi che appartengono ad $n$, potendoli anche ripetere.

Queste stringhe sono tutte e sole le stringhe formate scegliendo uno degli elementi di $A$ come primo carattere, uno degli elementi di $A$ come secondo, eccetera... Dunque sono tutte e solo le stringhe che appartengono a $\overbrace{A \times \dots \times A}^{k \text{ volte}} = A^k$ che ha cardinalita' $\abs{A^k} = \abs{A}^k = n^k$.

Questo numero si dice \textbf{numero delle disposizioni con ripetizioni di $\bm{n}$ elementi in $\bm{k}$ posizioni}, e si indica con $D'_{n, k} = n^k$.

\subsubsection{Disposizioni senza ripetizione}
Supponiamo di avere un insieme $A$ di cardinalita' $\abs{A} = n$ e voler costruire una stringa di $k$ elementi che appartengono ad $n$, senza poterli ripetere.

Seguendo il ragionamento di prima abbiamo $n$ possibilita' per il primo elemento, $n-1$ per il secondo, in quanto dobbiamo escludere il primo per evitare di ripeterlo, $n-2$ per il terzo, eccetera, fino ad arrivare a $n-(k-1) = n - k + 1$ per il $k$-esimo. Dunque il numero totale di possibilita' e' \[
    n(n-1)\cdots (n-k+1) = \frac{n(n-1)\cdots (n-k+1)(n-k)\cdots 2\cdot 1}{(n-k)\cdots 2\cdot 1} = \frac{n!}{(n-k)!}
\]

Questo numero si dice \textbf{numero delle disposizioni senza ripetizioni di $\bm{n}$ elementi in $\bm{k}$ posizioni}, e si indica con $D_{n, k} = \frac{n!}{(n-k)!}$.

\subsubsection{Permutazioni senza ripetizione}

Consideriamo una stringa di $n$ elementi distinti e cerchiamo di contare i modi in cui possiamo ordinarli per ottenere stringhe diverse. Notiamo che questo e' equivalente a chiederci in quanti modi possiamo creare una stringa di $n$ elementi a partire da un insieme di $n$ elementi, cioe' a quante sono le disposizioni senza ripetizione di $n$ elementi in $n$ posizioni: \[
    D_{n, n} = \frac{n!}{(n - n)!} = \frac{n!}{0!} = n!    
\]

Questo numero si dice \textbf{numero delle permutazioni senza ripetizioni di $\bm{n}$ elementi}, e si indica con $P_{n} = n!$.

\subsubsection{Anagrammi}

Consideriamo una stringa di $n$ elementi, non necessariamente distinti, e cerchiamo di contare i modi in cui possiamo scambiarli di posto ottenendo stringhe diverse.

Dato che alcuni elementi si ripetono, scambiandoli di posto otterremmo una stringa uguale all'originale, dunque il numero degli anagrammi di una parola di $n$ lettere non necessariamente distinte e' minore al numero delle permutazioni senza ripetizione.

Per capire come ottenere il numero di anagrammi, consideriamo un esempio. Prendiamo la parola ASSASSINI e rendiamo le lettere distinte tra loro aggiungendo dei pedici, ottenendo A$_1$S$_1$S$_2$A$_2$S$_3$S$_4$I$_1$NI$_2$. A questo punto il numero di anagrammi di questa parola e' esattamente il numero di permutazioni senza ripetizioni, dunque $9!$ stringhe. 

Se torniamo alla parola originale pero' alcune possibilita' sono contate piu' volte, in quanto si ottengono scambiando due lettere uguali (ad esempio A$_1$S$_1$S$_2$A$_2$S$_3$S$_4$I$_1$NI$_2$ e A$_1$S$_3$S$_2$A$_2$S$_1$S$_4$I$_1$NI$_2$ si ottengono l'una dall'altra scambiando una S, dunque contano come la stessa stringa). 

Possiamo scambiare le $4$ S tra loro in $4!$ modi, le A in $2!$ modi e le I in $2!$ modi; il risultato sara' quindi ottenuto dividendo il numero totale delle permutazioni per ciascuno dei modi di permutare le lettere uguali (che sono il numero di possibilita' che contiamo piu' volte), ottenendo in questo caso \[ 
    \frac{9!}{4!2!2!}
.\]

\subsubsection{Combinazioni senza ripetizioni}

Consideriamo un insieme di $n$ elementi distinti e cerchiamo di contare i modi di estrarre da questo insieme un sottoinsieme di $k$ elementi; dato che siamo interessati ai sottoinsiemi l'ordine degli elementi non e' rilevante, dunque il numero che cerchiamo deve essere minore di $D_{n, k}$.

Chiamiamo il numero di sottoinsiemi distinti \textbf{numero delle combinazioni senza ripetizioni} e indichiamolo con $C_{n, k}$. Possiamo notare che ogni sottoinsieme ottenuto corrisponde ad una stringa di lunghezza $k$ che puo' essere permutata in $P_k$ modi per ottenere tutte le disposizioni possibili. Dunque \begin{alignat*}
    {1}
    C_{n, k}P_k   &= D_{n, k}  \\
    \iff C_{n, k} &= \frac{D_{n,k}}{P_k}\\
                  &= \frac{n!}{k!(n-k)!} \\
                  &= \binom{n}{k}.
\end{alignat*}

\subsection{Esempi}

\begin{example}
    Contare i sottoinsiemi di $\left\{ 1, 2, 3, 4, 5\right\}$ di cardinalita' $3$.
\end{example}
\begin{solution}
    Dato che stiamo cercando sottoinsiemi l'ordine non conta, dunque dobbiamo usare le combinazioni, ottenendo che la soluzione e' \[
        C_{5, 3} = \binom{5}{3} =  \frac{5!}{3!2!} = \frac{5*4}{2} = 10
    .\]
\end{solution}

\begin{example}
    Quante sono le stringhe binarie di lunghezza $n$? Quante sono le stringhe binarie di lunghezza $10$ con $3$ cifre '$1$' e $7$ cifre '$0$'?
\end{example}
\begin{solution}
    Ogni posizione di una stringa binaria puo' essere riempita con $2$ valori ('0' oppure '1') e ci sono $n$ posizioni, dunque il numero di stringhe possibili e' il numero di disposizioni con ripetizione di $2$ elementi in $n$ posizioni, cioe' $2^n$.

    Per risolvere il secondo punto possiamo scegliere due strade equivalenti.
    \begin{enumerate}[(i)]
        \item Prendiamo una qualunque stringa che rispetta le condizioni, come 1110000000; allora il numero di stringhe con 3 uni e 7 zeri e' il numero di anagrammi di questa stringa, che e' \[
            \frac{10!}{3!7!} = \frac{10 \cdot 9 \cdot 8}{3 \cdot 2} = 120.   
        \]
        \item Notiamo che il problema e' equivalente al seguente problema: ho 10 posizioni possibili e devo sceglierne 3 in cui mettere le cifre '1'; infatti a quel punto le altre 7 sono automaticamente riempite con zeri. Dunque dato che dobbiamo scegliere un sottoinsieme (non ordinato) di posizioni tra 10 posizioni, il risultato sara' \[
            \binom{10}{3} = \frac{10!}{3!7!} = \frac{10 \cdot 9 \cdot 8}{3 \cdot 2} = 120.   
        \]
        Scegliere le 7 posizioni per gli zeri e' ovviamente equivalente.
    \end{enumerate}
\end{solution}

\begin{definition}
    Sia $A$ un insieme. Allora si dice insieme delle parti di $A$ l'insieme $\mathcal{P}(A)$ tale che \[
        \mathcal{P}(A) = \left\{ X \mid X \subseteq A\right\}    
    .\]
\end{definition}

Ad esempio se $A = \{1, 2, 3\}$ allora \[
    \mathcal{P}(A) = \left\{ \varnothing, \{1\}, \{2\}, \{3\}, \{1, 2\}, \{1, 3\}, \{2, 3\}, \{1, 2, 3\} \right\}.
\]

\begin{proposition}
    Sia $A$ un insieme tale che $\abs{A} = n$. Allora $\abs{\mathcal{P}(A)} = 2^n$.
\end{proposition}
\begin{proof}
    Passiamo ad una rappresentazione alternativa dei sottoinsiemi di $A$: fissiamo un ordine per l'insieme iniziale, poi associamo ad ogni sottoinsieme una stringa binaria che ha in posizione $i$ il numero 1 se e solo se l'elemento in posizione $i$ nell'insieme iniziale e' contenuto nel sottoinsieme.

    Ad esempio, se $A = \{1, 2, 3, 4\}$ e il sottoinsieme e' $\{3, 4\}$ allora la stringa binaria ad esso associato e' 0011, in quanto gli elementi in prima e seconda posizione dell'insieme $A$ (1 e 2) non sono nel sottoinsieme, mentre gli elementi in terza e quarta sono nel sottoinsieme. 

    Questa associazione e' biunivoca, poiche' a ogni sottoinsieme corrisponde una e una sola stringa e ad ogni stringa corrispone uno e un solo sottoinsieme, dunque il numero di stringhe e' uguale al numero di sottoinsiemi.

    Dato che ci sono $2^n$ stringhe binarie di lunghezza $n$, allora ci saranno $2^n$ sottoinsiemi, dunque la cardinalita' dell'insieme delle parti $\abs{\mathcal{P}(A)}$ sara' $2^n$.
\end{proof}

\begin{example}
    Consideriamo l'insieme dei numeri da 1 a 100. In quanti modi posso formare un sottoinsieme di cardinalita' 3 in modo che la somma dei numeri sia pari?
\end{example}
\begin{solution}
    Posso farlo in due modi diversi:
    \begin{enumerate}[(i)]
        \item Scelgo tre numeri pari. Allora il problema diventa equivalente a scegliere tre numeri tra i numeri pari da 1 a 100, che sono 50; abbiamo quindi \[
            \binom{50}{3} = \frac{50!}{3!47!} = \frac{50 * 49 * 48}{3 \cdot 2}   
        \] modi per scegliere tre numeri pari.
        \item Scelgo due numeri dispari e un pari. I due numeri dispari vanno scelti in $\binom{50}{2}$ modi, in quanto non conta l'ordine; il pari puo' essere scelto tra uno qualsiasi dei 50 numeri pari, dunque abbiamo \[
            50\binom{50}{2} = 50\frac{50!}{2!48!} = 50\frac{50 * 49}{2}
        \] modi.
    \end{enumerate}
    In totale i modi per formare un sottoinsieme che rispetti le condizioni del testo sono \[
        \binom{50}{3} + 50\binom{50}{2} = 50\frac{50 * 49 * 48}{3 \cdot 2}\frac{50 * 49}{2}.           
    \]
\end{solution}

\begin{example}
    Quante sono le triple ordinate $(x, y , z)$ di numeri naturali tali che $x + y + z = 4$?
\end{example}
\begin{solution}
    Possiamo rappresentare una soluzione nel seguente modo:
    \[
        \overbrace{1 \dots 1}^{x \text{ volte}} \mid   \overbrace{1 \dots 1}^{y \text{ volte}} \mid\overbrace{1 \dots 1}^{z \text{ volte}}
    \] dove le linee verticali fanno da separatori tra i gruppi di uni che vanno contate nelle $x$, nelle $y$ e nelle $z$. 
    
    Dato che la somma deve fare $4$ devono esserci 4 uni e due separatori, cioe' in totale 6 oggetti; allora il numero di soluzioni puo' essere ottenuto trovando il numero di permutazioni di questi oggetti, che e' \[
        \frac{6!}{2!4!} = \frac{6 \cdot 5}{2} = 15.
    \]

    Alternativamente dopo aver rappresentato la soluzione come stringa di uni e separatori, potevamo equivalentemente scegliere 2 posizioni su 6 per mettere i separatori, e cio' si puo' fare in \[
        \binom{6}{2} = \frac{6!}{2!4!} = 15    
    \] modi. A quel punto tutte le altre posizioni vengono occupate da uni, dunque la soluzione al problema non cambia.
\end{solution}

\begin{example}
    Abbiamo 4 colori: giallo, rosso, verde e blu. \begin{enumerate}
        [(i)]
        \item Quanti colori posso formare usando 5 gocce di questi colori?
        \item Quanti colori posso formare usando 5 gocce di questi colori avendo a disposizione solo 3 gocce di ogni colore?
    \end{enumerate}
\end{example}
\begin{solution} Indichiamo con $g$ il numero di gocce di colore giallo che usiamo, $r$ la gocce di rosso, $v$ le gocce di verde e con $b$ le gocce di blu.
    \begin{enumerate}[(i)]
        \item Il problema e' equivalente a chiederci quante quadruple $(g, r, v, b)$ di naturali soddisfano l'equazione $g + r + v + b = 5$. Rappresentiamo come nell'esercizio precedente una soluzione come uni e separatori: \[
            \overbrace{1 \dots 1}^{g \text{ volte}} \mid \overbrace{1 \dots 1}^{r \text{ volte}} \mid \overbrace{1 \dots 1}^{v \text{ volte}} \mid \overbrace{1 \dots 1}^{b \text{ volte}}.
        \] Dunque abbiamo $8$ oggetti in totale, di cui $5$ sono uni e $3$ sono separatori. Segue quindi che il numero totale di modi per permutare questi oggetti e' \[
            \binom{8}{3} = \frac{8!}{5!3!} = \frac{8\cdot 7 \cdot 6}{6} = 56.
        \]
        \item Dal conto del punto (i) sappiamo che senza restrizioni abbiamo $56$ possibilita'. Da queste dobbiamo togliere tutte le possibilita' in cui usiamo 4 o 5 gocce di un colore.
     
        Ho esattamente 4 possibilita' di usare 5 gocce dello stesso colore: \[
            (5, 0, 0, 0); \quad (0, 5, 0, 0); \quad (0, 0, 5, 0); \quad (0, 0, 0, 5).
        \] 

        Contiamo ora in quanti modi posso avere quattro gocce dello stesso colore. Innanzitutto scelgo il colore di cui uso 4 gocce, e posso farlo in 4 modi; dopo scelgo il colore che uso per l'ultima goccia, ed ho 3 possibilita'; in tutto ho quindi $12 = 4 \cdot 3$ possibilita'. (Notiamo che in questo caso non sto scegliendo due colori tra 4 senza considerare l'ordine, in quanto il primo colore e' quello di cui uso 4 gocce, dunque l'ordine conta.)

        In totale avro' quindi $56 - 4 - 12 = 40$ possibilita'.
    \end{enumerate}
\end{solution}

\begin{example}
    Consideriamo una tabella $3 \times 3$. In quanti modi possiamo colorare ogni casella di bianco o di nero in modo che:
    \begin{enumerate}[(i)]
        \item ogni riga e' colorata in modo diverso;
        \item esiste una e una sola riga bianca;
        \item esiste almeno una riga monocromatica.
    \end{enumerate}
\end{example}
\begin{solution}
    Innanzitutto consideriamo una singola riga: dato che e' formata da $3$ caselle che possono essere colorate in 2 modi, in totale la riga potra' essere colorata in $2^3$ modi distinti.
    \begin{enumerate}
        [(i)]
        \item Coloro la prima riga: dato che non ho ulteriori restrizioni lo posso fare in $2^3 = 8$ modi. Ora ho solo 7 modi per colorare la seconda riga, in quanto non posso ripetere la stessa colorazione della prima; e infine ho 6 modi per colorare l'ultima, in quanto non posso ripetere la prima o la seconda. In totale ho quindi $8\cdot 7\cdot 6$ modi di colorare la tabella.
        \item Ho $3$ possibilita' per scegliere la riga bianca. Ora ho $7$ modi per colorare la prima riga non bianca e sempre 7 modi per colorare la seconda riga non bianca (devo escludere la colorazione formata da 3 quadratini bianchi). In totale ho dunque $3 \cdot 7\cdot 7$ modi di colorare la tabella.
        \item Potrei essere tentato da ragionare in questo modo: scelgo una riga monocromatica tra 3, posso colorarla in due modi (tutta bianca o tutta nera), ora posso colorare le due righe rimanenti in 8 modi ciascuna. In realta' questo ragionamento non funziona, perche' conterei piu' volte combinazioni in cui ci sono piu' righe monocromatiche.
        Abbiamo quindi due modi alternativi a disposizione. \begin{enumerate}[1.]
            \item Conto separatamente i casi in cui ci sono una, due o tre righe monocromatiche.
            
            Supponiamo che ci sia una singola riga monocromatica. Dunque dobbiamo scegliere quale riga e' (3 possibilita') e se la riga e' bianca o nera (2 possibilita'); infine le altre due righe non possono essere monocromatiche dunque avremo $8-2 = 6$ possibilita' per ciascuna di esse. Dunque in questo caso ci sono $3\cdot 2 \cdot 6^2 = 6^3$ possibilita'.

            Supponiamo che ci siano due righe monocromatiche. Allora possiamo sceglierle tra le tre in $\binom{3}{2} = 3$ modi, e possiamo scegliere il loro colore in $2\cdot2 = 4$ modi. L'ultima riga deve essere non monocromatica, dunque possiamo sceglierla in $6$ modi diversi, ottenendo in totale $3 \cdot 4 \cdot 6 = 72$ possibilita'.

            Supponiamo infine che tutte le righe siano monocromatiche. Per ognuna di esse abbiamo due possibilita', dunque abbiamo in totale $2^3 = 8$ scelte possibili.

            Facendo la somma delle varie possibilita' otteniamo che ci sono $6^3 + 72 + 8 = 296$ tabelle con almeno una riga monocromatica.
            \item Passo al complementare: conto i casi in cui non c'e' nessuna riga monocromatica e lo sottraggo dal totale. Se nessuna riga e' monocromatica significa che abbiamo $8 - 2 = 6$ possibilita' per riga; il numero totale di possibilita' senza restrizioni e' $2^9$ dunque la soluzione al terzo punto e' $2^9 - 6^3 = 296$.
        \end{enumerate}
    \end{enumerate}
\end{solution}

\subsection{Teorema del binomiale}
Forniamo ora una dimostrazione del teorema del Binomiale (\ref{binomiale}) usando il calcolo combinatorio.
\begin{proof}[Dimostrazione del Teorema del Binomiale (\ref{binomiale})]
    Per definizione \[
        (x + y)^n = \overbrace{(x+y)\cdots (x+y)}^{n \text{ volte}}.
    \] Dunque il risultato dell'elevamento a potenza sara' la somma di tutti i monomi formati da una sequenza di $x$ e $y$ di lunghezza $n$, in quanto ogni addendo sara' dato dalla scelta di $x$ o di $y$ in ogni fattore di $(x+y)\cdots (x+y)$.

    Ad esempio \begin{alignat*}{1}
        (x+y)^3 &= (x+y)(x+y)(x+y) \\
                &= xxx + xxy + xyx + xyy + yxx + yxy + yyx + yyy\\
                &= xxx + (xxy + xyx + yxx) + (xyy + yxy + yyx) + yyy\\
                &= x^3 + 3x^2y + 3xy^2 + y^3.
    \end{alignat*}

    Ognuno di questi monomi puo' essere semplificato alla forma $x^ky^h$, ma dato che il numero di $x$ e $y$ deve essere $n$ segue che $k + h = n$, cioe' $h = n - k$, cioe' ogni monomio puo' essere scritto nella forma $x^ky^{n-k}$. Dato che ogni monomio ha al minimo 0 fattori uguali a $x$ e al massimo $n$ fattori uguali a $x$, sommando i monomi uguali insieme otterremo:
    \begin{alignat*}
        {1}
        (x+y)^n &= c_0x^0y^n + c_1x^1y^{n-1} + \dots + c_{n-1}x_{n-1}y_1 + c_nx^ny^0\\
        &= \sum_{k = 0}^n c_kx^ky^{n-k}
    \end{alignat*}
    dove i coefficienti $c_k$ rappresentano il numero di monomi uguali sommati insieme.

    Troviamo quindi un'espressione per $c_k$ per un $k$ generico. Dato che $c_k$ rappresenta il numero di termini nell'espressione originale che possono essere semplificati a $x^ky^{n-k}$, allora $c_k$ indica il numero di termini con $k$ fattori uguali a $x$ (edi conseguenza i restanti $n-k$ uguali a $y$). 
    
    Il numero di questi termini e' uguale al numero di modi in cui possiamo posizionare $k$ '$x$' in una stringa di $n$ elementi, che e' il numero di modi in cui possiamo scegliere $k$ posizioni da un insieme di $n$ elementi. Inoltre una volta scelte le posizioni per le $x$ le $y$ andranno in tutte le restanti $n-k$ posizioni, dunque \[
        c_k = \binom{n}{k}.
    \]

    Sostituendolo nell'espressione per $(x+y)^n$ otteniamo \[
        (x+y)^n = \sum_{k = 0}^n \binom{n}{k}x^ky^{n-k}
    \] che e' la tesi.
\end{proof}

\subsection{Poker}

Cerchiamo di calcolare il numero di modi in cui possiamo fare punteggi nel Poker tradizionale. 

Innanzitutto descriviamo il gioco: abbiamo un mazzo di 52 carte divise in 4 semi, con valori 1, 2, 3, 4, 5, 6, 7, 8, 9, 10, J, Q, K.
Si pescano 5 carte e viene assegnato un punteggio alla mano se e' in una delle seguenti combinazioni: \begin{itemize}
    \item[\textbf{Scala reale massima.}] una scala formata da 10, J, Q, K, 1 dello stesso seme (es. 10-J-Q-K-1 di cuori);
    \item[\textbf{Scala reale.}]  una scala formata da cinque carte consecutive dello stesso seme (es. 8-9-10-J-Q di fiori);
    \item[\textbf{Colore.}]  cinque carte dello stesso seme (es. 1-3-5-6-J di fiori);    
    \item[\textbf{Scala.}] una scala formata da cinque carte consecutive, non tutte dello stesso seme;
    \item[\textbf{Poker.}]  quattro carte dal valore uguale e una diversa (es. 3-3-3-3-5);
    \item[\textbf{Full.}] una coppia di carte dal valore uguale e un tris di carte dal valore uguale (es. 2-2-4-4-4);
    \item[\textbf{Tris.}] tre carte dal valore uguale e due diverse (es. 3-3-3-1-2);
    \item[\textbf{Doppia coppia.}] due coppie di carte dal valore uguale e una diversa (es. J-J-5-5-8);
    \item[\textbf{Coppia.}]una coppia di carte dal valore uguale e tre diverse (es. K-K-1-9-10).
\end{itemize}

Calcoliamo il numero di mani possibili. Dato che dobbiamo scegliere 5 carte da 52 e non abbiamo altre restrizioni, il numero di mani possibili e' \[\binom{52}{5}.\]

Ora calcoliamo il numero di possibilita' per ogni combinazione che da' punteggio.
\begin{itemize}
    \item[\textbf{Scala reale massima.}] Ho una scala reale massima per seme, dunque in tutto devo avere 4 scale reali massime.
    \item[\textbf{Scala reale.}] Ho 10 scale reali per seme: infatti ne posso costruire una che parte da 1, una che parte da 2,..., una che parte da 10. Dato che ho 4 semi, il numero totale di scale reali e' $4 \cdot 10 = 40$. 
    \item[\textbf{Colore.}] Scegliamo un seme (4 possibilita'). Allora il numero di colori che possiamo fare in un dato seme equivale al numero di modi in cui possiamo scegliere 5 carte di quel seme, che e' $\binom{13}{5}$. Dunque in totale il numero di colori che non siano scale reali e' $4\binom{13}{5}-40$.
    \item[\textbf{Scala.}] Ci sono 10 sequenze di valori possibili che formano una scala: infatti ne posso costruire una che parte da 1, una che parte da 2,..., una che parte da 10. Dato che posso scegliere liberamente il seme di ogni carta della scala, il numero totale di scale e' $4^5 \cdot 10$. Sottraendo ad esse il numero di scale reali otteniamo il numero di scale non reali, che sono $4^5 \cdot 10 - 40$.
    \item[\textbf{Poker.}] Scelgo innanzitutto la carta che si deve ripetere 4 volte, e dato che ho 13 valori diversi avro' 13 possibilita' di sceglierla. Dopo devo scegliere l'ultima carta, che puo' essere una qualunque tra le $52 - 4 = 48$ rimanenti. Dunque in totale ho $13\cdot 48$ modi per fare poker.
    \item[\textbf{Full.}] Scelgo il valore della carta che formera' il tris, e posso sceglierlo in 13 modi; poi scelgo il valore della carta che formera' la coppia, che posso scegliere in $13-1 = 12$ modi. 
    
    Ora devo scegliere i valori dei semi: per quanto riguarda il tris devo scegliere 3 semi su 4 (l'ordine non conta), e posso farlo in $\binom{4}{3} = 4$ modi; per quanto riguarda la coppia devo scegliere 2 semi su 4 e posso farlo in $\binom{4}{2} = 6$ modi. 
    
    In totale ho dunque $4\cdot 6\cdot 12 \cdot 13$ modi di fare full. 
    \item[\textbf{Tris.}] Scelgo il valore della carta del tris in $13$ modi e i semi in $\binom{4}{3} = 4$ modi. Non posso tuttavia scegliere le altre due carte liberamente, in quanto altrimenti potremmo ricadere nel poker o nel full. 
    
    Dunque scelgo il valore delle due carte rimanenti tra 12 valori (tutti tranne quello del tris), poi scelgo un seme tra 4 per la prima e un seme tra 4 per la seconda. In totale ho quindi $13 \cdot 4 \cdot \binom{12}{2} \cdot 4^2 = 4^3 \cdot 13 \binom{12}{2}$ modi di fare tris.
    \item[\textbf{Doppia coppia.}] Scelgo il valore delle due coppie tra 13 valori possibili in $\binom{13}{2}$ modi; ora scelgo i semi delle due coppie, e posso farlo in $\binom{4}{2}$ per la prima e $\binom{4}{2}$ per la seconda. 
    
    Scelgo la quinta carta in modo che abbia un valore diverso dalle due coppie per non contare anche i full: dunque posso sceglierla in $13 - 2 = 11$ modi per 4 semi diversi. In totale ho quindi $4 \cdot 11\binom{13}{2}\binom{4}{2}^2$ modi di ottenere una doppia coppia.
    \item[\textbf{Coppia.}] Scelgo il valore della coppia tra 13 valori possibili e i semi in $\binom{4}{2}$ modi. Scelgo le tre carte rimanenti diverse tra di loro e diverse dal valore della coppia per evitare doppie coppie, full e poker: dunque ho $\binom{12}{3}$ modi di sceglierle e, dato che ognuna di esse puo' avere un seme qualunque, ho $4$ possibilita' per ciascuna. In totale ho quindi $4^3 \cdot 13 \binom{4}{2} \binom{12}{3}$ modi di fare coppia.
\end{itemize}
\chapter{Polinomi}

\section{Definizioni base}

\begin{definition}[Polinomio]
    Sia $A$ un anello. Allora si dice polinomio a coefficienti in $A$ un'espressione del tipo \begin{equation}
        p(x) = a_0 + a_1x^1 + a_2x^2+ \dots + a_nx^n
    \end{equation}
    dove $a_0, \dots, a_n \in A$.

    L'insieme di tutti i polinomi a coefficienti in $A$ si indica con $A[x]$.
\end{definition}

Nel seguito indicheremo con $A$ un anello generico (come $\Z$, $\Z/(n)$ con $n$ non primo) e con $\K$ un campo generico (come $\Q$, $\R$, $\C$, $\Z/(p)$ con $p$ primo). Notiamo inoltre che, dato che un campo è anche un anello, tutte le definizioni e le proposizioni che valgono per gli anelli valgono anche per i campi (ma non viceversa).

\begin{definition}[Funzione associata ad un polinomio]
    Sia $p(x) \in A[x]$. Allora la funzione associata al polinomio $p$ e' la funzione $p : A \to A$ tale che per ogni $x \in A$ vale che \[
        p(x) = a_0 + a_1x^1 + a_2x^2+ \dots + a_nx^n. 
    \]
\end{definition}

\begin{definition}[Grado di un polinomio]
    Sia $p(x) \in A[x]$ (ovvero, sia $p(x)$ un polinomio a coefficienti nell'anello $A$). Si dice grado di $p$ il massimo $n$ tale che $a_n \neq 0$, e si indica con $\deg p$.
\end{definition}

\begin{definition}[Radice di un polinomio]
    Sia $p(x) \in A[x]$. Allora si dice che $r \in A$ è radice (o zero) di $p$ se \begin{equation}
        p(r) = 0.
    \end{equation}
\end{definition}

\begin{proposition}[Radici di un polinomio di grado 0]
    Sia $p(x) \in A[x]$ tale che $\deg p = 0$. Allora $p$ non ha radici oppure ne ha infinite.
\end{proposition}
\begin{proof}
    Dato che $\deg p = 0$, allora sarà $p(x) = a_0$ per qualche $a_0 \in A$.

    Abbiamo due casi: \begin{itemize}
        \item se $a_0 = 0$, allora $p(x) = 0$ per ogni $x \in A$, dunque ogni elemento di $A$ è radice del polinomio;
        \item se $a_0 \neq 0$ allora $p(x) \neq 0$ per ogni $x \in A$, dunque nessun elemento di $A$ è radice del polinomio. \qedhere
    \end{itemize}
\end{proof}


\begin{proposition}[Radici di un polinomio di primo grado]
    Sia $p(x) \in \K[x]$ tale che $\deg p = 1$. Allora esiste almeno una radice di $p$.
\end{proposition}
\begin{proof}
    Dato che $\deg p = 1$, allora sarà $p(x) = a_0 + a_1x$ per qualche $a_0, a_1 \in \K$.

    Sia $r$ una radice di $p$, allora deve valere che \begin{alignat*}
        {1}
        &p(r) = 0 \\
        \iff &a_0 + a_1r = 0 \\
        \iff &a_1r = -a_0 \\
        \iff &r = -a_0a_1^{-1}
    \end{alignat*}
    ovvero esiste $r \in \K$ e vale $-a_0a_1^{-1}$.
\end{proof}

\begin{remark}
    I polinomi di grado $1$ potrebbero non avere radici in un anello $A$. Ad esempio sia $p(x) \in \Z/(8)$, $p(x) = -3 + 4x$. Allora \begin{alignat*}
        {1}
        &4x - 3\equiv 0 \Mod{8} \\
        \iff &4x \equiv 3 \Mod{8}
    \end{alignat*}
    che non ha soluzioni in quanto $\mcd{4}{8} \nmid 3$.
\end{remark}

\begin{remark}
    I polinomi di secondo grado possono avere radici o possono non averne.
    \begin{itemize}
        \item $p(x) \in \Q[x]$, $p(x) = x^2 - 4$ ha come radici $x = \pm 2$;
        \item $p(x) \in \Q[x]$, $p(x) = x^2 - 2$ non ha radici;
        \item $p(x) \in \R[x]$, $p(x) = x^2 - 2$ ha come radici $x = \pm \sqrt{2}$;
        \item $p(x) \in \R[x]$, $p(x) = x^2 + 1$ non ha radici;        \item $p(x) \in \C[x]$, $p(x) = x^2 + 1$ ha come radici $x = \pm i$.
    \end{itemize}
\end{remark}

\begin{definition}[Polinomio monico]
    Sia $p(x) \in A[x]$ un polinomio. Allora $p$ si dice monico se il coefficiente del termine di grado massimo è 1, ovvero se \[
        p(x) = a_0 + a_1x + \dots + a_{n-1}x^{n-1} + x^n    
    \] per qualche $a_0, \dots, a_{n-1} \in A$.
\end{definition}

\begin{proposition}[Il grado del prodotto è la somma dei gradi]\label{grado_prodotto_somma_gradi}
    Siano $p(x), q(x) \in A[x]$. Allora $\deg (p \cdot q) = \deg p + \deg q$.
\end{proposition}

\section{Divisione e fattorizzazioni}

\begin{theorem}[Esistenza e unicità della divisione polinomiale nei campi]
    Siano $a(x), b(x) \in \K[x]$. Allora esistono e sono unici $q(x), r(x) \in \K[x]$ tali che \begin{equation}
        a(x) = b(x)q(x) + r(x), \quad \text{con } \deg r < \deg b.
    \end{equation} 
\end{theorem}
\begin{proof}
    Siano $n = \deg a$ e $m = \deg b$, ovvero \begin{align*}
        a(x) = a_nx^n + a_{n-1}x^{n-1} + \dots + a_0, &&b(x) = b_mx^m + b_{m-1}x^{m-1} + \dots + b_0.
    \end{align*}
    Dimostriamolo per induzione su $n$.
    \begin{description}
        \item[Caso base] Se $n < m$ allora possiamo scrivere la divisione polinomiale come \[
            a(x) = 0\cdot b(x) + a(x)    
        \] ovvero $q(x) = 0$ e $r(x) = a(x)$, che soddisfa automaticamente il requisito $\deg r < \deg b$.
        \item[Passo induttivo] Supponiamo di poter eseguire la divisione polinomiale per ogni $n^\prime < n$ e dimostriamolo per $n$. Supponiamo inoltre che $n > m$, altrimenti ricadiamo nel caso base.
        
        Scriviamo i polinomi $a(x)$ e $b(x)$ come $a(x) = a_nx^n + a^\prime(x)$, $b(x) = b_mx^m + b^\prime(x)$, dove $a^\prime(x)$ e $b^\prime(x)$ sono due polinomi di grado $n-1$ e $m-1$ rispettivamente che contengono tutti i termini di grado minore del massimo.

        Eseguiamo un passo della divisione polinomiale tra $a$ e $b$: scegliamo come quoziente il polinomio $q_0(x) = \frac{a_n}{b_m} x^{n-m}$. In questo modo infatti otteniamo \begin{align*}
            r_0(x) &= a(x) - q_0(x)b(x) \\
            &= a_nx^n + a^\prime(x) - \frac{a_n}{b_m}x^{n-m}(b_mx^m + b^\prime(x)) \\
            &= a_nx^n + a^\prime(x) - a_nx^n - \frac{a_n}{b_m}x^{n-m}b^\prime(x) \\
            &= a^\prime(x) - \frac{a_n}{b_m}x^{n-m}b^\prime(x).
        \end{align*}
        Il grado del resto $r_0$ è minore del grado di $a$ poiché i due addendi hanno grado $n-1$ e $(m-1)-(n-m) < n$. Dunque possiamo eseguire la divisione polinomiale tra $r_0$ e $b$ per ipotesi induttiva, ottenendo che $r_0(x) = q^\prime(x)b(x) + r^\prime(x)$ con $\deg r^\prime < \deg b$. 
        
        Sostituendo nell'equazione per $a$ otteniamo che \begin{align*}
            a(x) &= q_0(x)b(x) + r_0(x)\\
            &= q_0(x)b(x) + q^\prime(x)b(x) + r^\prime(x) \\
            &= (q_0(x) + q^\prime(x))b(x) + r^\prime(x).
        \end{align*} 
        Dunque possiamo eseguire la divisione polinomiale tra $a(x)$ e $b(x)$, ottenendo come unici quoziente e resto i polinomi $q(x) = q_0(x) + q^\prime(x)$ e $r(x) = r^\prime(x)$.
    \end{description}
\end{proof}

\begin{proposition}[Esistenza e unicità della divisione polinomiale negli anelli]\label{divisione_polinomi}
    Siano $p(x), g(x) \in A[x]$ con $g$ monico. Allora esistono e sono unici $q(x), r(x) \in A[x]$ tali che \begin{equation}
        p(x) = g(x)q(x) + r(x), \quad \text{con } \deg r < \deg g.
    \end{equation} 
\end{proposition}

\begin{definition}[Divisibilità tra polinomi]\label{divisione_polinomi_anello}
    Siano $p(x), g(x) \in A[x]$. Allora si dice che $g(x) \divides p(x)$ se e solo se esiste $q(x) \in A[x]$ tale che \[
        p(x) = g(x)q(x).    
    \]
\end{definition}

Per questa definizione di divisione euclidea possiamo definire un massimo comun divisore e un minimo comune multiplo tra polinomi.

\begin{definition}
    [Massimo comun divisore tra polinomi]
    Siano $p(x), g(x) \in K[x]$. Allora si dice massimo comun divisore di $p(x), g(x)$ il polinomio $h(x)$ di grado massimo tale che \[
        h(x) \divides p(x) \land h(x) \divides g(x)    
    \] e si indica con $\mcd{f(x)}{g(x)}$.
\end{definition}

\begin{definition}
    [Minimo comune multiplo tra polinomi]
    Siano $p(x), g(x) \in K[x]$. Allora si dice minimo comune multiplo di $p(x), g(x)$ il polinomio $h(x)$ di grado minimo tale che \[
        p(x) \divides h(x) \land g(x) \divides h(x)  
    \] ovvero tale che $h(x)$ è multiplo sia di $p(x)$ che di $g(x)$ e si indica con $\mcm{f(x)}{g(x)}$.
\end{definition}

Molti teoremi sul massimo comun divisore e massimo comune multiplo continuano a valere negli insiemi dei polinomi, come ad esempio il seguente.

\begin{proposition}
    Siano $p(x), g(x) \in \K[x]$. Allora $\mcd{p(x)}{g(x)} = \mcd{p(x) - g(x)h(x)}{g(x)}$, con $h(x) \in \K[x]$.
\end{proposition}

\begin{proposition}\label{resto_uguale_valutazione_nel_punto}
    Sia $p(x) \in A[x]$ e sia $a \in A$. Sia $r(x) \in A[x]$ il resto della divisione di $p(x)$ per $(x-a)$. Allora $r(x) = r_0$ per qualche $r_0 \in A$ e $p(a) = r_0$.
\end{proposition}
\begin{proof}
    Per il teorema sulla divisione tra polinomi (\ref{divisione_polinomi_anello}), dato che $x-a$ è un polinomio monico, sappiamo che esistono $q(x), r(x) \in A[x]$ tali che \[
        p(x) = (x-a)q(x) + r(x).
    \]

    Dato che $\deg (x - a) = 1$ e $\deg r < \deg a$ segue che $\deg r = 0$, cioè per ogni $x \in A$ vale che $r(x) = r_0$ per qualche $r_0 \in A$.

    Ora valutiamo il polinomio in $a$, ottenendo \begin{alignat*}
        {1}
        p(a) &= (a-a)q(a) + r(a)\\
        &= 0q(a) + r_0\\
        &= r_0
    \end{alignat*}
    che è la tesi.
\end{proof}

\begin{theorem}
    [Teorema di Ruffini] \label{th_Ruffini}
    Sia $p(x) \in A[x]$ e sia $a \in A$. Allora \[
        (x - a) \divides p(x) \iff p(a) = 0    
    \] ovvero $x - a$ divide $p(x)$ se e solo se $a$ è una radice di $p$.
\end{theorem}
\begin{proof}
    Per il teorema sulla divisione tra polinomi (\ref{divisione_polinomi_anello}), dato che $x-a$ è un polinomio monico, sappiamo che esistono $q(x), r(x) \in A[x]$ tali che \[
        p(x) = (x-a)q(x) + r(x).
    \]

    Dato che $\deg (x - a) = 1$ e $\deg r < \deg a$ segue che $\deg r = 0$, cioè per ogni $x \in A$ vale che $r(x) = r_0$ per qualche $r_0 \in A$.

    Per la proposizione \ref{resto_uguale_valutazione_nel_punto} segue che $p(a) = r_0$, dunque $a$ è radice se e solo se $r_0 = 0$, cioè se e solo se il polinomio $p(x)$ è divisibile per $(x - a)$.
\end{proof}

\begin{definition}[Polinomio irriducibile]
    Sia $p(x) \in A[x]$. Allora $p$ si dice irriducibile se non esistono $a(x), b(x) \in A[x]$ tali che valgano le seguenti tre condizioni: \begin{itemize}
        \item $\deg a < \deg p$;
        \item $\deg b < \deg p$;
        \item $p(x) = a(x)b(x)$.
    \end{itemize}
\end{definition}

\begin{remark}
    Tutti i polinomi $p$ tali che $\deg p \leq 1$ sono irriducibili.
\end{remark}

\begin{definition}[Fattorizzazione di un polinomio]
    Sia $p \in A[x]$. Fattorizzare $p$ significa trovare $a_1(x), \dots, a_n(x) \in A[x]$ tali che: \begin{itemize}
        \item $a_1(x), \dots, a_n(x)$ sono tutti irriducibili;
        \item $p(x) = a_1(x) \cdots a_n(x)$.
    \end{itemize}
\end{definition}

\begin{proposition}
    Sia $p(x) \in \K[x]$. Allora la fattorizzazione di $p$ è unica (a meno di fattori costanti).
\end{proposition}

\begin{remark}
    La fattorizzazione di $p$ non è unica negli anelli!
\end{remark}

\begin{proposition}\label{radice_di_p_sse_radice_fattore}
    Sia $p(x) \in \K[x]$ e siano $g(x), h(x) \in \K[x]$ tali che $p(x) = g(x)h(x)$. Allora $r \in \K$ è una radice di $p$ se e solo se è radice di $g$ o è radice di $h$.
\end{proposition}
\begin{proof}
    Dimostriamo entrambi i versi dell'implicazione.
    \begin{description}
        \item[($\implies$)] Supponiamo che $p(r) = 0$. Allora $p(r) = g(r)h(r) = 0$, dunque per la regola di annullamento del prodotto (\ref{annullamento_prodotto}) segue che $g(r) = 0 \lor h(r) = 0$.
        \item[($\impliedby$)] Supponiamo senza perdita di generalità che $g(r) = 0$. Allora $p(r) = g(r)h(r) = 0h(r) = 0$. \qedhere 
    \end{description}
\end{proof}
\begin{corollary}
    Sia $p(x) \in \K[x]$ e sia $p_1(x), \dots, p_n(x) \in \K[x]$ una fattorizzazione di $p$. Allora $r \in \K$ è una radice di $p$ se e solo se è radice di uno tra $p_1, \dots, p_n$.
\end{corollary}
\begin{proof}
    Dimostriamo entrambi i versi dell'implicazione.
    \begin{description}
        \item[($\implies$)] Infatti $p(r) = p_1(r) \cdots p_n(r) = 0$, dunque per la regola di annullamento del prodotto (\ref{annullamento_prodotto}) segue che \[
            p_1(r) = 0 \lor p_2(r) = 0 \lor \dots \lor p_n(r) = 0    
        \] ovvero la tesi.
        \item[($\impliedby$)] Supponiamo senza perdita di generalità che $p_1(r) = 0$. Allora $p(r) = p_1(r)p_2(r)\cdots p_n(r) = 0p_2(r)\cdots p_n(r) = 0$. \qedhere
    \end{description} 
\end{proof}

\begin{remark}
    La proposizione \ref{radice_di_p_sse_radice_fattore} vale soltanto nei campi, mentre negli anelli vale solo una delle due implicazioni (in particolare quella che dice che se $r$ è radice di un fattore, allora è radice anche del polinomio) in quanto non vale la regola di annullamento del prodotto. La conseguenza di questo fatto è che in un anello $A$ un polinomio ha piu' radici di quante ne abbiano i suoi fattori, cioè la fattorizzazione di un polinomio non è unica.
\end{remark}

\begin{proposition}[Un polinomio di grado n ha al massimo n radici in un campo]
    Sia $p(x) \in \K[x]$ tale che $n = \deg p$. Allora $p$ ha al massimo $n$ radici.
\end{proposition}
\begin{proof}
    Dimostriamolo per induzione su $n$.
    \begin{description}
        \item[Caso base.] Se $n = 1$ allora $p(x) = a_0 + a_1x$, dunque l'unica radice è $-a_0a_1^{-1}$.
        \item[Passo induttivo.] Supponiamo che la tesi valga per $n$ e dimostriamola per $n+1$. Consideriamo due casi: \begin{itemize}
            \item Se $p$ non ha radici allora $p$ ha meno radici di $n+1$, che è la tesi.
            \item Se $p$ ha una radice $r$ allora $p$ è divisibile per $x-r$, ovvero esiste $q(x) \in \K[x]$ tale che $p(x) = (x-r)q(x)$. Per la proposizione \ref{grado_prodotto_somma_gradi} segue che $\deg q = n$, dunque per ipotesi induttiva il numero di radici di $q$ è minore o uguale a $n$. Aggiungendo la radice $r$ segue infine che $p$ ha al massimo $n + 1$ radici.
        \end{itemize}  
    \end{description}
    Dunque per induzione la tesi vale per ogni $n \geq 1$.
\end{proof}

\section{Fattorizzazione in insiemi specifici}

\subsection{Fattorizzazione in $\C$}

\begin{theorem}
    [Teorema Fondamentale dell'Algebra] \label{th_fondamentale_algebra}
    Sia $p(x) \in \C[x]$ tale che $\deg p \geq 1$. Allora esiste almeno una radice di $p$ in $\C$, ovvero esiste almeno un $\lambda \in \C$ tale che $p(\lambda) = 0$.
\end{theorem}

\begin{corollary}[Ogni polinomio ha $n$ radici complesse]\label{conseguenza_th_fondamentale}
    Sia $p(x) \in \C$ e sia $n = \deg p \geq 1$. Allora $p$ ha esattamente $n$ radici complesse, ovvero $p$ è fattorizzabile in esattamente $n$ fattori lineari (non necessariamente distinti).
\end{corollary}
\begin{proof}
    Dimostriamo per induzione su $n$.

    \begin{description}
        \item[Caso base] Sia $n = 1$. Allora $p(x) = a_0 + a_1x$ che ammette la radice $-a_0a_1^{-1}$. Inoltre $p(x)$ è fattorizzabile in fattori lineari in quanto esso stesso è lineare.
        \item[Passo induttivo] Supponiamo che la tesi valga per $n$ e dimostriamo che vale anche per $n + 1$.
        
        Sia $p(x)$ di grado $n+1$. Allora per il Teorema Fondamentale dell'Algebra (\ref{th_fondamentale_algebra}) $p$ ammette almeno una radice $\lambda \in \C$.

        Per il teorema di Ruffini (\ref{th_Ruffini}) segue che $(x - \lambda)$ è un divisore di $p(x)$, ovvero esiste $q(x) \in \C[x]$ tale che \[
            p(x) = (x - \lambda)q(x).    
        \] Il grado di $q$ dovrà essere $\deg p - 1 = n+1 - 1 = n$, dunque per ipotesi induttiva $q$ ha $n$ radici ed è fattorizzabile in $n$ fattori lineari.

        Di conseguenza $p$ ha $n+1$ radici ed è fattorizzabile in $n+1$ fattori lineari.
    \end{description}
    Per induzione allora la tesi vale per ogni $n \geq 1$.
\end{proof}

\begin{proposition}[Un polinomio reale ha radici complesse coniugate]\label{radici_coniugate_a_due_a_due}
    Sia $p(x) \in \R[x]$ un polinomio a coefficienti reali. Allora $\lambda \in \C$ è una radice di $p$ se e solo se $\conj{\lambda}$ è una radice di $p$.
\end{proposition}
\begin{proof}
    Sia $p(x) = a_0 + a_1x + \dots + a_nx^n$ con $a_0, \dots, a_n \in \R$. Supponiamo che $p(\lambda) = 0$. Allora \begin{align*}
        p(\conj{\lambda}) &= a_0 + a_1\conj{\lambda} + \dots + a_n\conj{\lambda^n} &&\text{(se $a \in \R$ allora $\conj{a} = a$)}\\
        &= \conj{a_0} + \conj{a_1}\cdot\conj{\lambda} + \dots + \conj{a_n}\cdot\conj{\lambda^n} &&\text{(per la \ref{somma_prodotto_tra_coniugati})}\\
        &= \conj{a_0} + \conj{a_1\lambda} + \dots + \conj{a_n\lambda^n} &&\text{(per la \ref{somma_prodotto_tra_coniugati})}\\
        &= \conj{a_0 + a_1\lambda + \dots + a_n\lambda^n} \\
        &= \conj{p(\lambda)}.
    \end{align*}

    Dunque $p(\lambda) = 0 \iff \conj{p(\lambda)} = \conj{0} \iff p(\conj{\lambda}) = 0$, come volevasi dimostrare.
\end{proof}

\subsection{Fattorizzazione in $\R$}

\begin{lemma}\label{(x-l)(x-conj(l))_in_R}
    Sia $\lambda \in \C$. Allora il polinomio \[
        p(x) = (x-\lambda)(x-\conj\lambda)    
    \] è un polinomio a coefficienti reali.
\end{lemma}
\begin{proof}
    \begin{align*}
        p(x) &= (x-\lambda)(x-\conj\lambda)\\
        &= x^2 - (\lambda + \conj\lambda)x + \lambda\conj{\lambda} &&\text{(per \ref{somma_prodotto_col_coniugato})}\\
        &= x^2 - 2(\Re \lambda)x + \abs{\lambda}^2.
    \end{align*}
    Dato che $2\Re \lambda, \abs{\lambda}^2 \in \R$ segue la tesi.
\end{proof}

\begin{proposition}
    Sia $p(x) \in \R[x]$. Allora $p$ è fattorizzabile su $\R$ come prodotto di fattori di grado minore o uguale a 2.
\end{proposition}
\begin{proof}
    Dimostriamolo per induzione forte su $n = \deg p$.

    \begin{description}
        \item[Caso base] Sia $n = 1$ oppure $n = 2$. Allora $p$ è banalmente già fattorizzato in fattori di grado $1$ o $2$.
        \item[Passo induttivo] Sia $n > 2$ e supponiamo che la tesi sia vera per ogni $n^\prime$ tale che $1 \leq n^\prime < n$.
        
        Per il teorema fondamentale dell'algebra (\ref{th_fondamentale_algebra}) allora esiste sicuramente $\lambda \in \C$ tale che $p(\lambda) = 0$. Distinguiamo due casi.
        \begin{itemize}
            \item Se $\lambda \in \R$ allora per il teorema di Ruffini (\ref{th_Ruffini}) segue che $(x-\lambda) \divides p(x)$, ovvero esiste $g(x) \in \R[x]$ tale che \[
                p(x) = (x-\lambda)g(x).    
            \] Dato che $\deg g = \deg p - 1 = n - 1$ segue che per ipotesi induttiva $g$ è fattorizzabile come prodotto di fattori di grado $1$ o $2$, dunque anche $f$ lo è.
            \item Se $\lambda \notin \R$ allora per la proposizione \ref{radici_coniugate_a_due_a_due} anche $\conj{\lambda}$ è radice di $p$. Per il teorema di Ruffini (\ref{th_Ruffini}) segue che \[
                (x-\lambda) \divides p(x),\quad (x - \conj{\lambda}) \divides p(x)
            \] dunque \[
                (x-\lambda)(x-\conj\lambda) \divides p(x).
            \]

            Per il lemma \ref{(x-l)(x-conj(l))_in_R} il polinomio $h(x) = (x-\lambda)(x-\conj\lambda)$ è un polinomio a coefficienti reali, dunque dato che $h(x) \divides p(x)$ esiste $g(x) \in \R[x]$ tale che \[
                p(x) = h(x)g(x) = (x-\lambda)(x-\conj\lambda) \cdot g(x).  
            \] Dato che $\deg g = \deg p - 2 = n - 2$ segue che per ipotesi induttiva $g$ è fattorizzabile come prodotto di fattori di grado $1$ o $2$, dunque anche $p$ lo è.
        \end{itemize}
    \end{description}

    Per induzione ogni polinomio in $\R[x]$ è esprimibile come prodotto di fattori di grado minore o uguale a $2$.
\end{proof}

Un modo equivalente di esprimere la proposizione sopra è dire che ogni polinomio a coefficienti reali di grado maggiore di due è riducibile in $\R$.

\subsection{Fattorizzazione in $\Z$ o in $\Q$}

\begin{proposition}
    [Radici razionali di un polinomio a coefficienti interi]
    Sia $p(x) \in \Z[x]$ un polinomio a coefficienti interi tale che \[
        p(x) = a_0 + a_1x + a_2x^2 + \dots + a_nx^n.    
    \] Sia $\frac{c}{d} \in \Q$ ridotta ai minimi termini (ovvero $\mcd{c}{d} = 1$). 
    
    Allora se $\frac{c}{d}$ è una radice di $p$ segue che $c \divides a_0$ e $d \divides a_n$.
\end{proposition}
\begin{proof}
    Per definizione di radice di un polinomio \[
        p\left(\frac{c}{d}\right) = a_0 + a_1\frac{c}{d} + \dots + a_{n-1}\left(\frac{c}{d}\right)^{n-1} + a_n\left(\frac{c}{d}\right)^n = 0.
    \]
    Moltiplicando entrambi i membri per $d^n$ otteniamo \[
        \iff a_0d^n + a_1cd^{n-1} + \dots + a_{n-1}c^{n-1}d + a_nc^n = 0.
    \]
    Se vale l'uguaglianza, allora i due membri saranno anche congrui modulo $d$: \[
        a_0d^n + a_1cd^{n-1} + \dots + a_{n-1}c^{n-1}d + a_nc^n \equiv 0 \Mod{d}.
    \]
    Tutti i termini tranne l'ultimo contengono una potenza di $d$, dunque:
    \begin{align*}
        \iff &a_nc^n \equiv 0 \Mod{d} \\
        \intertext{Dato che $\mcd{c}{d} = 1$, allora $c^n$ è invertibile modulo $d$)}
        \iff &a_n \equiv 0 \Mod{d}\\
        \iff &d \divides a_n.
    \end{align*}

    Consideriamo ora la congruenza modulo $c$: \[
        a_0d^n + a_1cd^{n-1} + \dots + a_{n-1}c^{n-1}d + a_nc^n \equiv 0 \Mod{c}.
    \]
    Per lo stesso ragionamento segue che:
    \begin{align*}
        \iff &a_0d^n \equiv 0 \Mod{c}\\
        \iff &a_0 \equiv 0 \Mod{c} \\
        \iff &c \divides a_0. \qedhere
    \end{align*}
\end{proof}

\begin{theorem}
    [Lemma di Gauss] \label{lemma_di_Gauss}
    Sia $p(x) \in \Z[x]$ un polinomio a coefficienti interi. Allora se $p$ è riducibile in $\Q[x]$ segue che $p$ è riducibile in $\Z[x]$.
\end{theorem}

Per il Lemma di Gauss dunque se vogliamo cercare la fattorizzazione di un polinomio in $\Q[x]$ ci basta cercare una fattorizzazione in $\Z[x]$ (cioè tra i polinomi a coefficienti interi). Viceversa, se non esiste una fattorizzazione in $\Z[x]$ allora il polinomio è irriducibile anche in $\Q[x]$.
\begin{proposition}
    [Criterio della riduzione modulo un primo] \label{criterio_riduzione}
    Sia $f(x) \in \Z[x]$ tale che $f(x) = a_0 + \dots + a_nx^n$. 
    
    Allora se esiste un primo $p \in \Z$ tale che $f_p(x) \in \Z/(p)[x]$, $f_p(x) = [a_0]_p + \dots + [a_n]_px^n $ (ovvero il polinomio $f$ con coefficienti in $\Z/(p)$) è irriducibile segue che $f(x)$ è irriducibile in $\Z[x]$.
\end{proposition}
\begin{proof}
    Sia $p$ un primo tale che $f_p(x)$ è irriducibile in $\Z/(p)[x]$. Supponiamo per assurdo che $f(x)$ sia riducibile in $\Z[x]$. Per definizione esisteranno $g(x), h(x) \in \Z[x]$ tali che $f(x) = g(x)h(x)$.

    Siano $g_p(x), h_p(x) \in \Z/(p)[x]$ le riduzioni modulo $p$ di $g(x), h(x)$. Per definizione di riduzione modulo $p$ dovrà essere \begin{align*}
        f(x) \equiv f_p(x) \Mod{p}, &&g(x) \equiv g_p(x) \Mod{p}, &&h(x) \equiv h_p(x) \Mod{p}
    \end{align*} da cui segue \[
        f_p(x) \equiv f(x) \equiv g(x)h(x) \equiv g_p(x)h_p(x) \Mod{p}    
    \] ovvero che $f_p(x) = g_p(x) \cdot h_p(x)$ in $\Z/(p)[x]$. Ma questo è assurdo in quanto per ipotesi $f_p(x)$ è irriducibile in $\Z/(p)[x]$, dunque $f(x)$ dovrà essere irriducibile in $\Z[x]$.
\end{proof}

\begin{proposition}
    [Criterio di Eisenstein] \label{criterio_eisenstein}
    Sia $f(x) \in \Z[x]$ tale che $f(x) = a_0 + \dots + a_nx^n$. Allora se esiste un primo $p \in \Z$ tale che \begin{itemize}
        \item $a_0 \equiv a_1 \equiv \dots \equiv a_{n-1} \Mod{p}$;
        \item $p \nmid a_n$;
        \item $p^2 \nmid a_0$
    \end{itemize}
    allora $f$ è irriducibile in $\Z[x]$.
\end{proposition}
\begin{proof}
    Sia $f_p(x) \in \Z/(p)[x]$ il polinomio ottenuto considerando $f(x)$ nel campo $\Z/(p)$. Per le prime due ipotesi segue che \begin{align*}
        f_p(x) &= [a_n]_px^n + [a_{n-1}]_px^{n-1} + \dots + [a_0]_p \\
        &= [a_n]_px^n
    \end{align*} con $[a_n]_p \neq 0$ per la seconda ipotesi.

    Supponiamo per assurdo che $f(x)$ si fattorizzi in $\Z[x]$, ovvero che esistano $g(x), h(x) \in \Z[x]$ tali che \[
        f(x) = g(x)h(x).
    \] Notiamo che $a_0 = f(0) = g(0)h(0)$. 
    
    Consideriamo ore le rispettive proiezioni $g_p(x), h_p(x) \in \Z/(p)[x]$. Dato che $g(x) \divides f(x)$, $h(x) \divides f(x)$ segue che $g_p(x) \divides f_p(x)$ e $h_p(x) \divides f_p(x)$. Ma $f_p(x)$ è un monomio, quindi anche $g_p(x)$ e $h_p(x)$ devono essere monomi:
    \[
        g_p(x) = kx^r, h_p(x) = hx^{n-r}    
    \] con $k, r \in \Z/(p)$, $k, r$ entrambi diversi da $0$.

    Notiamo che $g_p(0) = h_p(0) = 0$, che in $\Z/(p)$ significa che $g(0) \equiv h(0) \equiv 0 \Mod{p}$, cioè $p \divides g(0)$, $p \divides h(0)$. Ma questo significa che $p^2 \divides g(0)h(0)$, ovvero $p^2 \divides a_0$, che è assurdo poiché abbiamo assunto che $p^2 \nmid a_0$.

    Dunque segue che $f(x)$ non è fattorizzabile in $\Z[x]$, che è la tesi.
\end{proof}

\begin{proposition}
    [Criterio della sostituzione] \label{criterio_sost}
    Sia $p(x) \in \Z[x]$ e sia $n \in \Z$. Allora $p(x)$ è riducibile in $\Z[x]$ se e solo se $p(x+n)$ è riducibile in $\Z[x]$.
\end{proposition}

\subsection{Polinomi ciclotomici}

\begin{definition}
    [Polinomio ciclotomico]
    Sia $p \in \Z$ primo. Si dice polinomio ciclotomico il polinomio \[
        f(x) = 1 + x + \dots + x^{p-1}.    
    \]
\end{definition}

\begin{proposition}
    Sia $p \in \Z$ primo e sia $f(x) \in \Z[x]$ il polinomio ciclotomico di grado $p-1$. 

    $f(x)$ non è fattorizzabile in $\Z[x]$.
\end{proposition}
\begin{proof}
    Notiamo che $f(x)$ rappresenta una serie geometrica, dunque \[
        f(x) = 1 + x + \dots + x^{p-1} = \frac{x^p - 1}{x - 1}.    
    \] Per il criterio di sostituzione (\ref{criterio_sost}) vale che $f(x)$ è riducibile in $\Z[x]$ se e solo se $f(x+1)$ lo è. Dunque
    \begin{align*}
        f(x+1) &= \frac{(x+1)^p - 1}{x+1-1} &&\text{(per il teorema del Binomiale (\ref{binomiale}))}\\
        &= \frac1x \left( x^p + \binom{p}{1}x^{n-1} + \dots + \binom{p}{p-1}x + 1 - 1 \right)\\
        &= \frac1x \left( x^p + \binom{p}{1}x^{n-1} + \dots + \binom{p}{p-1}x \right)\\
        &= x^{p-1} + \binom{p}{1}x^{n-2} + \dots + \binom{p}{p-1}.
    \end{align*}

    Applichiamo il criterio di Eisenstein col primo $p$:
    \begin{itemize}
        \item per la proposizione $\ref{binomio_pk_divisibile_p}$ segue che tutti i coefficienti tranne il coefficiente direttore sono congrui a $0$ modulo $p$;
        \item dato che il coefficiente direttore è $1$, segue che $p \nmid 1$;
        \item il termine noto è $\binom{p}{p-1} = p$, dunque $p^2 \nmid p$.
    \end{itemize}
    Per il criterio di Eisenstein (\ref{criterio_eisenstein}) dunque il polinomio $f(x+1)$ non è fattorizzabile in $\Z[x]$, dunque per il criterio di sostituzione neanche $f(x)$ lo è. 
\end{proof}
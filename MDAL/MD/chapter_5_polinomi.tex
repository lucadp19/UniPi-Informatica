\chapter{Polinomi}

\section{Definizioni base}

\begin{definition}
    Sia $A$ un anello. Allora si dice polinomio a coefficienti in $A$ una funzione $p : A \to A$ tale che \begin{equation}
        p(x) = a_0 + a_1x^1 + a_2x^2+ \dots + a_nx^n
    \end{equation}
    dove $a_0, \dots, a_n \in A$.

    L'insieme di tutti i polinomi a coefficienti in $A$ si indica con $A[x]$.
\end{definition}

Nel seguito indicheremo con $A$ un anello generico (come $\Z$, $\Z/(n)$ con $n$ non primo) e con $\K$ un campo generico (come $\Q$, $\R$, $\C$, $\Z/(p)$ con $p$ primo). Notiamo inoltre che, dato che un campo e' anche un anello, tutte le definizioni e le proposizioni che valgono per gli anelli valgono anche per i campi (ma non viceversa).

\begin{definition}
    Sia $p \in A[x]$ (ovvero, sia $p(x)$ un polinomio a coefficienti nell'anello $A$). Si dice grado di $p$ il massimo $n$ tale che $a_n \neq 0$, e si indica con $\deg p$.
\end{definition}

\begin{definition}
    Sia $p \in A[x]$. Allora si dice che $r \in A$ e' radice (o zero) di $p$ se \begin{equation}
        p(r) = 0.
    \end{equation}
\end{definition}

\begin{proposition}
    Sia $p \in A[x]$ tale che $\deg p = 0$. Allora $p$ non ha radici oppure ne ha infinite.
\end{proposition}
\begin{proof}
    Dato che $\deg p = 0$, allora sara' $p(x) = $ per qualche $a_0 \in A$.

    Abbiamo due casi: \begin{itemize}
        \item se $a_0 = 0$, allora $p(x) = 0$ per ogni $x \in A$, dunque ogni elemento di $A$ e' radice del polinomio;
        \item se $a_0 \neq 0$ allora $p(x) \neq 0$ per ogni $x \in A$, dunque nessun elemento di $A$ e' radice del polinomio. \qedhere
    \end{itemize}
\end{proof}


\begin{proposition}
    Sia $p \in \K[x]$ tale che $\deg p = 1$. Allora esiste almeno una radice di $p$.
\end{proposition}
\begin{proof}
    Dato che $\deg p = 1$, allora sara' $p(x) = a_0 + a_1x$ per qualche $a_0, a_1 \in \K$.

    Sia $r$ una radice di $p$, allora deve valere che \begin{alignat*}
        {1}
        &p(r) = 0 \\
        \iff &a_0 + a_1r = 0 \\
        \iff &a_1r = -a_0 \\
        \iff &r = -a_0a_1^{-1}
    \end{alignat*}
    ovvero esiste $r \in \K$ e vale $-a_0a_1^{-1}$.
\end{proof}

\begin{remark}
    I polinomi di grado $1$ potrebbero non avere radici in un anello $A$. Ad esempio sia $p \in \Z/(8)$, $p(x) = -3 + 4x$. Allora \begin{alignat*}
        {1}
        &4x - 3\equiv 0 \Mod{8} \\
        \iff &4x \equiv 3 \Mod{8}
    \end{alignat*}
    che non ha soluzioni in quanto $\mcd{4}{8} \nmid 3$.
\end{remark}

\begin{remark}
    I polinomi di secondo grado possono avere radici o possono non averne.
    \begin{itemize}
        \item $p \in \Q[x]$, $p(x) = x^2 - 4$ ha come radici $x = \pm 2$;
        \item $p \in \Q[x]$, $p(x) = x^2 - 2$ non ha radici;
        \item $p \in \R[x]$, $p(x) = x^2 - 2$ ha come radici $x = \pm \sqrt{2}$;
        \item $p \in \R[x]$, $p(x) = x^2 + 1$ non ha radici;        \item $p \in \C[x]$, $p(x) = x^2 + 1$ ha come radici $x = \pm i$.
    \end{itemize}
\end{remark}

\begin{definition}
    Sia $p \in A[x]$ un polinomio. Allora $p$ si dice monico se il coefficiente del termine di grado massimo e' 1, ovvero se \[
        p(x) = a_0 + a_1x + \dots + a_{n-1}x^{n-1} + x^n    
    \] per qualche $a_0, \dots, a_n \in A$.
\end{definition}

\section{Divisione e fattorizzazioni}

\begin{theorem}[Esistenza e unicita' della divisione polinomiale nei campi]
    Siano $p, q \in \K[x]$. Allora esistono e sono unici $q, r \in \K[x]$ tali che \begin{equation}
        p(x) = g(x)q(x) + r(x), \quad \text{con } \deg r < \deg g.
    \end{equation} 
\end{theorem}

\begin{proposition}[Esistenza e unicita' della divisione polinomiale negli anelli]\label{divisione_polinomi}
    Siano $p, g \in A[x]$ con $g$ monico. Allora esistono e sono unici $q, r \in A[x]$ tali che \begin{equation}
        p(x) = g(x)q(x) + r(x), \quad \text{con } \deg r < \deg g.
    \end{equation} 
\end{proposition}

\begin{definition}\label{divisione_polinomi_anello}
    Siano $p, g \in A[x]$. Allora si dice che $g \divides p$ se e solo se esiste $q \in A[x]$ tale che \[
        p(x) = g(x)q(x).    
    \]
\end{definition}

\begin{proposition}\label{resto_uguale_valutazione_nel_punto}
    Sia $p$ in $A[x]$ e sia $a \in A$. Sia $r \in A[x]$ il resto della divisione di $p$ per $(x-a)$. Allora $r(x) = r_0$ per qualche $r_0 \in A$ e $p(a) = r_0$.
\end{proposition}
\begin{proof}
    Per il teorema sulla divisione tra polinomi (\ref{divisione_polinomi_anello}), dato che $x-a$ e' un polinomio monico, sappiamo che esistono $q, r \in A[x]$ tali che \[
        p(x) = (x-a)q(x) + r(x).
    \]

    Dato che $\deg (x - a) = 1$ e $\deg r < \deg a$ segue che $\deg r = 0$, cioe' $r(x) = r_0$ per qualche $r_0 \in A$.

    Ora valutiamo il polinomio in $a$, ottenendo \begin{alignat*}
        {1}
        p(a) &= (a-a)q(a) + r(a)\\
        &= 0q(a) + r_0\\
        &= r_0
    \end{alignat*}
    che e' la tesi.
\end{proof}

\begin{theorem}
    [di Ruffini] \label{th_Ruffini}
    Sia $p$ in $A[x]$ e sia $a \in A$. Allora \[
        (x - a) \divides p \iff p(a) = 0    
    \] ovvero $x - a$ divide $p$ se e solo se $a$ e' una radice di $p$.
\end{theorem}
\begin{proof}
    Per il teorema sulla divisione tra polinomi (\ref{divisione_polinomi_anello}), dato che $x-a$ e' un polinomio monico, sappiamo che esistono $q, r \in A[x]$ tali che \[
        p(x) = (x-a)q(x) + r(x).
    \]

    Dato che $\deg (x - a) = 1$ e $\deg r < \deg a$ segue che $\deg r = 0$, cioe' $r(x) = r_0$ per qualche $r_0 \in A$.

    Per la proposizione \ref{resto_uguale_valutazione_nel_punto} segue che $p(a) = r_0$, dunque $a$ e' radice se e solo se $r_0 = 0$, cioe' se e solo se il polinomio $p$ e' divisibile per $(x - a)$.
\end{proof}

\begin{definition}[Irriducibile]
    Sia $p \in A[x]$. Allora $p$ si dice irriducibile se non esistono $a, b \in A[x]$ tali che valgano le seguenti tre condizioni: \begin{itemize}
        \item $\deg a < \deg p$;
        \item $\deg b < \deg p$;
        \item $p(x) = a(x)b(x)$.
    \end{itemize}
\end{definition}

\begin{remark}
    Tutti i polinomi $p$ tali che $\deg p \leq 1$ sono irriducibili.
\end{remark}

\begin{definition}[Fattorizzazione di un polinomio]
    Sia $p \in A[x]$. Fattorizzare $p$ significa trovare $a_1, \dots, a_n \in A[x]$ tali che: \begin{itemize}
        \item $a_1, \dots, a_n$ sono tutti irriducibili;
        \item $p(x) = a_1(x) \cdots a_n(x)$.
    \end{itemize}
\end{definition}
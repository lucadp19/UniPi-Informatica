\chapter{Polinomi}

\section{Definizioni base}

\begin{definition}
    Sia $A$ un anello. Allora si dice polinomio a coefficienti in $A$ una funzione $p : A \to A$ tale che \begin{equation}
        p(x) = a_0 + a_1x^1 + a_2x^2+ \dots + a_nx^n
    \end{equation}
    dove $a_0, \dots, a_n \in A$.

    L'insieme di tutti i polinomi a coefficienti in $A$ si indica con $A[x]$.
\end{definition}

Nel seguito indicheremo con $A$ un anello generico (come $\Z$, $\Z/(n)$ con $n$ non primo) e con $\K$ un campo generico (come $\Q$, $\R$, $\C$, $\Z/(p)$ con $p$ primo). Notiamo inoltre che, dato che un campo e' anche un anello, tutte le definizioni e le proposizioni che valgono per gli anelli valgono anche per i campi (ma non viceversa).

\begin{definition}
    Sia $p(x) \in A[x]$ (ovvero, sia $p(x)$ un polinomio a coefficienti nell'anello $A$). Si dice grado di $p$ il massimo $n$ tale che $a_n \neq 0$, e si indica con $\deg p$.
\end{definition}

\begin{definition}
    Sia $p(x) \in A[x]$. Allora si dice che $r \in A$ e' radice (o zero) di $p$ se \begin{equation}
        p(r) = 0.
    \end{equation}
\end{definition}

\begin{proposition}
    Sia $p(x) \in A[x]$ tale che $\deg p = 0$. Allora $p$ non ha radici oppure ne ha infinite.
\end{proposition}
\begin{proof}
    Dato che $\deg p = 0$, allora sara' $p(x) = a_0$ per qualche $a_0 \in A$.

    Abbiamo due casi: \begin{itemize}
        \item se $a_0 = 0$, allora $p(x) = 0$ per ogni $x \in A$, dunque ogni elemento di $A$ e' radice del polinomio;
        \item se $a_0 \neq 0$ allora $p(x) \neq 0$ per ogni $x \in A$, dunque nessun elemento di $A$ e' radice del polinomio. \qedhere
    \end{itemize}
\end{proof}


\begin{proposition}
    Sia $p(x) \in \K[x]$ tale che $\deg p = 1$. Allora esiste almeno una radice di $p$.
\end{proposition}
\begin{proof}
    Dato che $\deg p = 1$, allora sara' $p(x) = a_0 + a_1x$ per qualche $a_0, a_1 \in \K$.

    Sia $r$ una radice di $p$, allora deve valere che \begin{alignat*}
        {1}
        &p(r) = 0 \\
        \iff &a_0 + a_1r = 0 \\
        \iff &a_1r = -a_0 \\
        \iff &r = -a_0a_1^{-1}
    \end{alignat*}
    ovvero esiste $r \in \K$ e vale $-a_0a_1^{-1}$.
\end{proof}

\begin{remark}
    I polinomi di grado $1$ potrebbero non avere radici in un anello $A$. Ad esempio sia $p(x) \in \Z/(8)$, $p(x) = -3 + 4x$. Allora \begin{alignat*}
        {1}
        &4x - 3\equiv 0 \Mod{8} \\
        \iff &4x \equiv 3 \Mod{8}
    \end{alignat*}
    che non ha soluzioni in quanto $\mcd{4}{8} \nmid 3$.
\end{remark}

\begin{remark}
    I polinomi di secondo grado possono avere radici o possono non averne.
    \begin{itemize}
        \item $p(x) \in \Q[x]$, $p(x) = x^2 - 4$ ha come radici $x = \pm 2$;
        \item $p(x) \in \Q[x]$, $p(x) = x^2 - 2$ non ha radici;
        \item $p(x) \in \R[x]$, $p(x) = x^2 - 2$ ha come radici $x = \pm \sqrt{2}$;
        \item $p(x) \in \R[x]$, $p(x) = x^2 + 1$ non ha radici;        \item $p(x) \in \C[x]$, $p(x) = x^2 + 1$ ha come radici $x = \pm i$.
    \end{itemize}
\end{remark}

\begin{definition}
    Sia $p(x) \in A[x]$ un polinomio. Allora $p$ si dice monico se il coefficiente del termine di grado massimo e' 1, ovvero se \[
        p(x) = a_0 + a_1x + \dots + a_{n-1}x^{n-1} + x^n    
    \] per qualche $a_0, \dots, a_{n-1} \in A$.
\end{definition}

\begin{proposition}\label{grado_prodotto_somma_gradi}
    Siano $p(x), q(x) \in A[x]$. Allora $\deg (p \cdot q) = \deg p + \deg q$.
\end{proposition}

\section{Divisione e fattorizzazioni}

\begin{theorem}[Esistenza e unicita' della divisione polinomiale nei campi]
    Siano $p(x), q(x) \in \K[x]$. Allora esistono e sono unici $q(x), r(x) \in \K[x]$ tali che \begin{equation}
        p(x) = g(x)q(x) + r(x), \quad \text{con } \deg r < \deg g.
    \end{equation} 
\end{theorem}

\begin{proposition}[Esistenza e unicita' della divisione polinomiale negli anelli]\label{divisione_polinomi}
    Siano $p(x), g(x) \in A[x]$ con $g$ monico. Allora esistono e sono unici $q(x), r(x) \in A[x]$ tali che \begin{equation}
        p(x) = g(x)q(x) + r(x), \quad \text{con } \deg r < \deg g.
    \end{equation} 
\end{proposition}

\begin{definition}[Divisibilita' tra polinomi]\label{divisione_polinomi_anello}
    Siano $p(x), g(x) \in A[x]$. Allora si dice che $g(x) \divides p(x)$ se e solo se esiste $q(x) \in A[x]$ tale che \[
        p(x) = g(x)q(x).    
    \]
\end{definition}

\begin{proposition}\label{resto_uguale_valutazione_nel_punto}
    Sia $p(x) \in A[x]$ e sia $a \in A$. Sia $r(x) \in A[x]$ il resto della divisione di $p(x)$ per $(x-a)$. Allora $r(x) = r_0$ per qualche $r_0 \in A$ e $p(a) = r_0$.
\end{proposition}
\begin{proof}
    Per il teorema sulla divisione tra polinomi (\ref{divisione_polinomi_anello}), dato che $x-a$ e' un polinomio monico, sappiamo che esistono $q(x), r(x) \in A[x]$ tali che \[
        p(x) = (x-a)q(x) + r(x).
    \]

    Dato che $\deg (x - a) = 1$ e $\deg r < \deg a$ segue che $\deg r = 0$, cioe' per ogni $x \in A$ vale che $r(x) = r_0$ per qualche $r_0 \in A$.

    Ora valutiamo il polinomio in $a$, ottenendo \begin{alignat*}
        {1}
        p(a) &= (a-a)q(a) + r(a)\\
        &= 0q(a) + r_0\\
        &= r_0
    \end{alignat*}
    che e' la tesi.
\end{proof}

\begin{theorem}
    [di Ruffini] \label{th_Ruffini}
    Sia $p(x) \in A[x]$ e sia $a \in A$. Allora \[
        (x - a) \divides p(x) \iff p(a) = 0    
    \] ovvero $x - a$ divide $p(x)$ se e solo se $a$ e' una radice di $p$.
\end{theorem}
\begin{proof}
    Per il teorema sulla divisione tra polinomi (\ref{divisione_polinomi_anello}), dato che $x-a$ e' un polinomio monico, sappiamo che esistono $q(x), r(x) \in A[x]$ tali che \[
        p(x) = (x-a)q(x) + r(x).
    \]

    Dato che $\deg (x - a) = 1$ e $\deg r < \deg a$ segue che $\deg r = 0$, cioe' per ogni $x \in A$ vale che $r(x) = r_0$ per qualche $r_0 \in A$.

    Per la proposizione \ref{resto_uguale_valutazione_nel_punto} segue che $p(a) = r_0$, dunque $a$ e' radice se e solo se $r_0 = 0$, cioe' se e solo se il polinomio $p(x)$ e' divisibile per $(x - a)$.
\end{proof}

\begin{definition}[Irriducibile]
    Sia $p(x) \in A[x]$. Allora $p$ si dice irriducibile se non esistono $a(x), b(x) \in A[x]$ tali che valgano le seguenti tre condizioni: \begin{itemize}
        \item $\deg a < \deg p$;
        \item $\deg b < \deg p$;
        \item $p(x) = a(x)b(x)$.
    \end{itemize}
\end{definition}

\begin{remark}
    Tutti i polinomi $p$ tali che $\deg p \leq 1$ sono irriducibili.
\end{remark}

\begin{definition}[Fattorizzazione di un polinomio]
    Sia $p \in A[x]$. Fattorizzare $p$ significa trovare $a_1(x), \dots, a_n(x) \in A[x]$ tali che: \begin{itemize}
        \item $a_1(x), \dots, a_n(x)$ sono tutti irriducibili;
        \item $p(x) = a_1(x) \cdots a_n(x)$.
    \end{itemize}
\end{definition}

\begin{proposition}
    Sia $p(x) \in \K[x]$. Allora la fattorizzazione di $p$ e' unica (a meno di fattori costanti).
\end{proposition}

\begin{remark}
    La fattorizzazione di $p$ non e' unica negli anelli!
\end{remark}

\begin{proposition}\label{radice_di_p_sse_radice_fattore}
    Sia $p(x) \in \K[x]$ e siano $g(x), h(x) \in \K[x]$ tali che $p(x) = g(x)h(x)$. Allora $r \in \K$ e' una radice di $p$ se e solo se e' radice di $g$ o e' radice di $h$.
\end{proposition}
\begin{proof}
    Dimostriamo entrambi i versi dell'implicazione.
    \begin{description}
        \item[($\implies$)] Infatti $p(r) = g(r)h(r) = 0$, dunque per la regola di annullamento del prodotto (\ref{annullamento_prodotto}) segue che $h(r) = 0 \lor h(r) = 0$.
        \item[($\impliedby$)] Supponiamo senza perdita di generalita' che $g(r) = 0$. Allora $p(r) = g(r)h(r) = 0h(r) = 0$. \qedhere 
    \end{description}
\end{proof}
\begin{corollary}
    Sia $p(x) \in \K[x]$ e sia $p_1(x), \dots, p_n(x) \in \K[x]$ una fattorizzazione di $p$. Allora $r \in \K$ e' una radice di $p$ se e solo se e' radice di uno tra $p_1, \dots, p_n$.
\end{corollary}
\begin{proof}
    Dimostriamo entrambi i versi dell'implicazione.
    \begin{description}
        \item[($\implies$)] Infatti $p(r) = p_1(r) \cdots p_n(r) = 0$, dunque per la regola di annullamento del prodotto (\ref{annullamento_prodotto}) segue che \[
            p_1(r) = 0 \lor p_2(r) = 0 \lor \dots \lor p_n(r) = 0    
        \] ovvero la tesi.
        \item[($\impliedby$)] Supponiamo senza perdita di generalita' che $p_1(r) = 0$. Allora $p(r) = p_1(r)p_2(r)\cdots p_n(r) = 0p_2(r)\cdots p_n(r) = 0$. \qedhere
    \end{description} 
\end{proof}

\begin{remark}
    La proposizione \ref{radice_di_p_sse_radice_fattore} vale soltanto nei campi, mentre negli anelli vale solo una delle due implicazioni (in particolare quella che dice che se $r$ e' radice di un fattore, allora e' radice anche del polinomio) in quanto non vale la regola di annullamento del prodotto. La conseguenza di questo fatto e' che in un anello $A$ un polinomio ha piu' radici di quante ne abbiano i suoi fattori, cioe' la fattorizzazione di un polinomio non e' unica.
\end{remark}

\begin{proposition}
    Sia $p(x) \in \K[x]$ tale che $n = \deg p$. Allora $p$ ha al massimo $n$ radici.
\end{proposition}
\begin{proof}
    Dimostriamolo per induzione su $n$.
    \begin{description}
        \item[Caso base.] Se $n = 1$ allora $p(x) = a_0 + a_1x$, dunque l'unica radice e' $-a_0a_1^{-1}$.
        \item[Passo induttivo.] Supponiamo che la tesi valga per $n$ e dimostriamola per $n+1$. Consideriamo due casi: \begin{itemize}
            \item Se $p$ non ha radici allora $p$ ha meno radici di $n+1$, che e' la tesi.
            \item Se $p$ ha una radice $r$ allora $p$ e' divisibile per $x-r$, ovvero esiste $q(x) \in \K[x]$ tale che $p(x) = (x-r)q(x)$. Per la proposizione \ref{grado_prodotto_somma_gradi} segue che $\deg q = n$, dunque per ipotesi induttiva il numero di radici di $q$ e' minore o uguale a $n$. Aggiungendo la radice $r$ segue infine che $p$ ha al massimo $n + 1$ radici.
        \end{itemize}  
    \end{description}
    Dunque per induzione la tesi vale per ogni $n \geq 1$.
\end{proof}
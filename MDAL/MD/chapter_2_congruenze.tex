\chapter{Congruenze}

\section{Relazione di congruenza}

\begin{definition} \label{def_congr}
    Siano $a, b, m \in \Z$, $m > 0$. Allora si dice che $a$ e' congruo a $b$ modulo $m$ se e solo se $a - b$ e' un multiplo di $m$, e si scrive
    \begin{align*}
        &a \equiv b \pmod m
    \end{align*}
\end{definition}

\begin{theorem}
    Siano $a, b, m \in \Z$, $m > 0$. Allora la relazione di congruenza $\equiv \pmod(m)$ e' una relazione di equivalenza, e dunque soddisfa le proprieta':
    \begin{align}
        &\text{Riflessiva: } &&a \equiv a \Mod{m} \label{congr_rifl}\\
        &\text{Simmetrica: } &&a \equiv b \Mod{m} \implies b \equiv a \Mod{m} \label{congr_simm}\\
        &\text{Riflessiva: } &&a \equiv b \Mod{m} \land b \equiv c \Mod{m} \implies a \equiv c \Mod{m}  \label{congr_trans}
    \end{align}
\end{theorem}
\begin{proof} Dimostriamo le tre proprieta' della congruenza come relazione di equivalenza.
    \begin{enumerate}
        \item $a - a = 0 = 0m$, dunque $a \equiv a \Mod{m}$.
        \item Se $a - b = km$ allora $b - a = -(a - b) = -km = (-k)m$, cioe' $b \equiv a \Mod{m}$.
        \item Se $a - b = km$ e $b - c = hm$ allora $a - c = (a - b) + (b - c) = km + hm = (k + h)m$, 
            cioe' $a \equiv c \Mod{m}$.
    \end{enumerate}
\end{proof}

\begin{theorem} \label{equiv_congr_resto}
    Siano $a, b, m \in \Z$, $m > 0$. Allora
    \begin{equation}
        a \equiv b \Mod{m} \iff a \bmod m = b \bmod m
    \end{equation}
    cioe' $a$ e' congruo a $b$ se e solo se $a$ e $b$ hanno lo stesso resto quando divisi per $m$.
\end{theorem}
\begin{proof}
    Dimostriamo l'implicazione nei due versi.

    Siano $r = a \bmod m$, $r' = b \bmod m$ i resti di $a$ e $b$ modulo $m$, 
    cioe' $a = cq + r$ e $b = cq' + r'$ per qualche $q, q' \in \Z$. Supponiamo 
    che $r = a \bmod m = b \bmod m = b$. Allora
    \begin{alignat*}
        {1}
        a - b &= cq + r - cq' - r' \\
              &= c(q - q')
    \end{alignat*}
    cioe' $a \equiv b \Mod{m}$.

    Ora supponiamo che $a \equiv b \Mod{m}$ e dimostriamo che i resti di $a$ e $b$ modulo $m$ siano uguali.
    Per la proposizione \ref{esistenza_resto} esistono $q, r \in \Z$ tale che $b = mq + r$ e $0 \leq r < m$.
    Allora per definizione di congruenza per qualche $k \in \Z$ avremo
    \begin{alignat*}
        {1}
        a &= b + mk \\
          &= mq + r + mk \\
          &= m(q + k) + r
    \end{alignat*}
    cioe' $r$ e' il resto di $a$ modulo $m$.
\end{proof}

\begin{proposition}
    Siano $a, b, a', b', m \in \Z$, $m > 0$. Allora valgono le seguenti
    \begin{align}
        &a \equiv b \Mod{m}\land a' \equiv b' \Mod{m}\implies a+a' \equiv b+b'\Mod{m} \label{somma_congrui}\\
        &a \equiv b \Mod{m}\land a' \equiv b' \Mod{m}\implies a-a' \equiv b-b'\Mod{m} \label{differenza_congrui}\\
        &a \equiv b \Mod{m}\land a' \equiv b' \Mod{m}\implies aa' \equiv bb'\Mod{m} \label{prodotto_congrui}
    \end{align}
\end{proposition}
\begin{proof}
    \begin{enumerate}
        \item Per definizione di congruenza $m \divides a - b$ e $m \divides a' - b'$. Per la proposizione \ref{divides_sum_subtr_mult} segue che $m \divides (a - b) + (a' - b')$, cioe' $m \divides (a + a') - (b + b')$, che e' equivalente a $a+a' \equiv b+b'\Mod{m}$.
        \item Per definizione di congruenza $m \divides a - b$ e $m \divides a' - b'$. Per la proposizione \ref{divides_sum_subtr_mult} segue che $m \divides (a - b) - (a' - b')$, cioe' $m \divides (a - a') - (b - b')$, che e' equivalente a $a-a' \equiv b-b'\Mod{m}$.
        \item Per definizione di congruenza, scriviamo $a - b = km$ e $a' - b' = hm$, che e' equivalente a $b = a - km$ e $b' = a' - hm$. Dunque
        \begin{alignat*}
            {1}
            bb' &= (a - km)(a' - hm) \\
                &= aa' - ahm - a'km + khm \\
                &= aa' - (ah + a'k - kh)m \\
        \end{alignat*}    
        che e' equivalente a
        \begin{alignat*} {1}
            aa' - bb' &= (ah + a'k - kh)m \\
            \iff aa' &\equiv bb' \Mod{m}.
        \end{alignat*}
        
    \end{enumerate}
    
\end{proof}

\section{Equazioni con congruenze lineari}

\begin{proposition}
    Siano $a, b, c \in \Z$; sia $ax + by = c$ un'equazione diofantea. Allora tutte le soluzioni della diofantea sono soluzioni delle equazioni $ax \equiv c \Mod{b}$ e $by \equiv c \Mod{a}$.
\end{proposition}
\begin{proof}
    Dimostriamo entrambi i versi dell'implicazione.
    \begin{enumerate}
        \item Siano $x, y \in \Z$ tali che $ax + by = c$. Dato che $ax + by$ e' uguale a $c$ segue che $ax + by \equiv c \Mod{b}$. Ma $b \equiv 0 \Mod{b}$, dunque $x$ sara' anche soluzione di $ax \equiv c \Mod{b}$. Analogo ragionamento considerando $ax + by \equiv c \Mod{a}$.
        \item Sia $x \in Z$ tale che $ax \equiv c \Mod{b}$. Allora per definizione di congruenza esiste $k \in \Z$ per cui $ax - c = bk$. Sia $y = -k$; l'equazione e' quindi equivalente a $ax + by = c$, cioe' la coppia $(x, y)$ e' una soluzione dell'equazione diofantea. Analogo ragionamento se partiamo da $by \equiv c \Mod{a}$.
    \end{enumerate}
\end{proof}

Tramite questa proposizione possiamo risolvere ogni equazione contenente congruenze risolvendo l'equazione diofantea associata, o viceversa.

\begin{definition}
    Siano $a \in \Z$; allora si dice che $a$ e' invertibile modulo $m$ se esiste  $x \in \Z$ tale che \[
        ax \equiv 1 \Mod{m}
    .\]
    In particolare tra tutti gli $x$ che soddisfano la relazione precedente, il numero $x$ tale che $0 \leq x < m$ si dice inverso di $a$ modulo $m$.
\end{definition}
Per calcolare gli inversi modulo $m$ basta fare una tabella $m \times m$ in cui le righe e le colonne contengono i numeri tra $0$ e $m-1$, e nella casella $ij$ c'e' il prodotto tra i numeri $i$ e $j$ modulo $m$.

Notiamo che non sempre i numeri diversi da $0$ ammettono inverso modulo $m$.

\begin{theorem}
    Siano $a, m \in \Z$. Allora $a$ e' invertibile modulo $m$ se e solo se $\mcd{a}{m} = 1$. 
\end{theorem}
\begin{proof}
    Supponiamo $\mcd{a}{m} = 1$. Allora per il teorema di Bezout \ref{bezout} $\exists x, y \in \Z$ tali che
    \begin{alignat*}{1}
        &ax + my = 1 \\
        \iff &ax - 1 = m(-y) \\
        \iff &ax \equiv 1 \Mod{m}
    \end{alignat*}
    dunque $x$ e' l'inverso di $a$ modulo $m$.

    Supponiamo che $a$ sia invertibile modulo $m$, cioe' che $\exists x \in \Z$ tale che $ax \equiv= 1 \Mod{m}$. Ma sappiamo che $ax + my$ e' un multiplo di $\mcd{a}{m}$, quindi anche $1$ dovra' essere un multiplo di $\mcd{a}{m}$, cioe' $\mcd{a}{m} = 1$, che e' la tesi.
\end{proof}
\begin{corollary}
    Se $p$ e' primo e $a \not\equiv 0 \Mod{p}$, allora $a$ e' invertibile modulo $p$.
\end{corollary}
\begin{proof}
    Se $p$ e' primo, allora necessariamente $p$ e' coprimo con tutti i numeri che non sono suoi multipli, cioe' con tutti gli $a$ tali che $a \equiv_p 0$. Dunque se $a \equiv_p 0$ allora $\mcd{a}{p} = 1$, cioe' per il teorema precedente $a$ e' invertibile modulo $p$.
\end{proof}

\begin{proposition} \label{se_invertibile_allora_soluzione}
    Siano $a, b, m \in \Z$; allora se $a$ e' invertibile modulo $m$ segue che $\exists x \in \Z$ tale che $ax \equiv b \Mod{m}$.
\end{proposition}
\begin{proof}
    Dato che $a$ e' invertibile modulo $m$ esistera' $x' \in \Z$ tale che $ax' \equiv 1 \Mod{m}$. Moltiplicando entrambi i membri per $b$ otteniamo $ax'b \equiv b \Mod{m}$, dunque la $x \equiv x'b \Mod{b}$ soddisfa $ax \equiv b \Mod{m}$, cioe' la tesi.
\end{proof}

\begin{proposition} \label{cong_ha_soluzione_sse_mcd_div_b}
    Siano $a, b, m, x \in \Z$; allora l'equazione $ax \equiv b \Mod{m}$ ha soluzione se e solo se $\mcd{a}{m} \divides b$.
\end{proposition}
\begin{proof}
    Dimostriamo l'implicazione nei due versi.
    \begin{itemize}
        \item Supponiamo che $ax \equiv b \Mod{m}$ ammetta soluzione. Allora esiste $y \in \Z$ tale che $ax - my = b$. Dato che $a$ e $m$ sono multipli di $\mcd{a}{m}$, allora lo sara' anche la combinazione lineare $ax - my$ che e' uguale a $b$, cioe' $\mcd{a}{m} \divides b$.
        \item Supponiamo che $d = \mcd{a}{m}$ divida $b$. Allora $d\divides a$, $d \divides b$, $d \divides m$. Siano $a' = \frac{a}{d}, b' = \frac{b}{d}, m' = \frac{m}{d}$. Allora 
        \begin{align*}
            &ax \equiv b \Mod{m}\\
            \iff{} &ax - b = mk   &\text{per qualche $k \in \Z$} \\
            \iff{} &a'dx - b'd = m'dk &\text{per qualche $k \in \Z$} \\
            \iff{} &a'x - b' = m'k &\text{per qualche $k \in \Z$} \\
            \iff{} &a'x \equiv b'\Mod{m'}.
        \end{align*}
        Ma per il corollario \ref{mcd_diviso_mcd} $\mcd{a'}{m'} = 1$, dunque $a'$ e' invertibile modulo $m'$, dunque per la proposizione \ref{se_invertibile_allora_soluzione} segue che $a'x \equiv b' \Mod{m'}$ ha soluzione. Tuttavia $a'x \equiv b' \Mod{m'}$ e' equivalente a $ax \equiv b \Mod{m}$, dunque anche $ax \equiv b \Mod{m}$ ha soluzione e in particolare ha le stesse soluzioni di $a'x \equiv b' \Mod{m'}$.
    \end{itemize}
\end{proof}

\begin{proposition}
    Se vogliamo semplificare una congruenza possiamo sfruttare le seguenti regole:
    \begin{alignat}{3}
        A &\equiv B \Mod{m} \quad &\iff      \quad &A + c \equiv B + c \Mod{m} \\
        A &\equiv B \Mod{m} \quad &\implies  \quad &cA \equiv cB \Mod{m} \\
        A &\equiv B \Mod{m} \quad &\iff      \quad &(A \bmod m) \equiv (B \bmod m) \Mod{m} \\
        Ad &\equiv Bd \Mod{m} \quad &\implies\quad &A \equiv B \Mod{m} \qquad \text{se }\mcd{d}{m} = 1\\
        Ad &\equiv Bd \Mod{md} \quad &\iff   \quad &A \equiv B \Mod{m}
    \end{alignat}
\end{proposition}
\begin{proof}
    Dimostriamo le 5 proposizioni.
    \begin{enumerate}
        \item Dato che $c \equiv c \Mod{m}$, si tratta di un caso particolare della \ref{somma_congrui}. Inoltre l'implicazione inversa si ricava dalla \ref{differenza_congrui}, dunque si tratta di un'equivalenza.
        \item Dato che $c \equiv c \Mod{m}$, si tratta di un caso particolare della \ref{prodotto_congrui}.
        \item Dato che $A \equiv (A \bmod m) \Mod{m}$ e $B \equiv (B \bmod m) \Mod{m}$, per transitivita' otteniamo che $A \equiv B \Mod{m}$ e' equivalente a $(A \bmod m) \equiv (B \bmod m) \Mod{m}$.
        \item Se $\mcd{d}{m} = 1$ allora esiste l'inverso di $d$ modulo $m$. Chiamiamo $x$ questo inverso e moltiplichiamo entrambi i membri della congruenza per $x$, ottenendo
        \begin{alignat*}
            {1}
            Ad &\equiv Bd \Mod{m}  \\
            \iff Adx &\equiv Bdx \Mod{m} \\
            \iff A \cdot 1 &\equiv B \cdot 1 \Mod{m} \\
            \iff A &\equiv B \Mod{m}.
        \end{alignat*}
        \item Per definizione di congruenza esiste $y \in \Z$ tale che
        \begin{alignat*}
            {1}
            Ad &= Bd + mdy \\
            \iff A &= B + my \\
            \iff A &\equiv B \Mod{m}.
        \end{alignat*}
    \end{enumerate}
\end{proof}

\begin{proposition}
    Siano $a, b, m \in \Z$ noti, $x \in \Z$ non noto. Allora per risolvere l'equazione $ax \equiv b \Mod{m}$ possiamo ricondurci ad uno dei seguenti tre casi:
    \begin{enumerate}
        \item se $\mcd{a}{m} = 1$, allora l'equazione ha soluzione $x \equiv by \Mod{m}$, dove $y$ e' l'inverso di $a$ modulo $m$;
        \item se $\mcd{a}{m} \neq 1$, $d = \mcd{a}{m} \divides b$, allora l'equazione e' equivalente all'equazione $a'x \equiv b' \Mod{m'}$, con $a' = \frac{a}{d}$, $b' = \frac{b}{d}$, $m' = \frac{m}{d}$, che ha soluzione;
        \item se $\mcd{a}{m} \neq 1$, $\mcd{a}{m} \nmid b$, allora l'equazione non ha soluzione.
    \end{enumerate}
\end{proposition}
\begin{proof}
    I tre casi sono conseguenza diretta della proposizione \ref{cong_ha_soluzione_sse_mcd_div_b}. Infatti
    \begin{enumerate}
        \item Per la \ref{cong_ha_soluzione_sse_mcd_div_b} l'equazione ha soluzione. Se $y$ e' l'inverso di $a$, moltiplicando entrambi i membri per $y$ otteniamo la soluzione $x \equiv by \Mod{m}$.
        \item Per la \ref{cong_ha_soluzione_sse_mcd_div_b} l'equazione ha soluzione. 
        Sia $d = \mcd{a}{m}$. Allora la congruenza e' equivalente a $ax - b = mk$ per qualche $k \in \Z$. Dato che $a, b, m$ sono divisibili per $d$, dividendo per $d$ otteniamo l'equazione equivalente
        \begin{alignat*}
            {1}
            &\frac{a}{d}x - \frac{b}{d} = \frac{m}{d}k \\
            \iff &\frac{a}{d}x \equiv \frac{b}{d} \Mod{\frac{m}{d}}
        \end{alignat*}
        Ma per il corollario \ref{mcd_diviso_mcd} $\mcd{\frac{a}{d}}{\frac{m}{d}} = 1$, dunque possiamo trovare la soluzione sfruttando il primo caso.
        \item Per la \ref{cong_ha_soluzione_sse_mcd_div_b} l'equazione non ha soluzione. 
        % Scriviamo l'equazione nella forma equivalente $ax - mk = b$ per qualche $k \in \Z$. Dato che $\mcd{a}{m}$ divide sia $ax$ che $mk$, allora dovra' dividere anche $b$, ma per ipotesi $\mcd{a}{m} \nmid b$, dunque l'equazione non puo' avere soluzione.
    \end{enumerate}
\end{proof}

\section{Sistemi di congruenze}

\begin{theorem}
    [Teorema Cinese del Resto]
    Dato un sistema di congruenze in forma normale 
    \begin{equation*}
        \left\{
        \begin{alignedat}{1}
            x&\equiv a_1 \Mod{m_1}\\
            x&\equiv a_2 \Mod{m_2}\\
            &\vdotswithin{\equiv} \\
            x&\equiv a_n \Mod{m_n}
        \end{alignedat}      
        \right . 
    \end{equation*}
    se i moduli $m_1, m_2, \dots, m_n$ sono a due a due coprimi (cioe' se per ogni $i \neq j$ vale che $\mcd{m_i}{m_j} = 1$) allora il sistema ha soluzione, ed e' equivalente ad una singola congruenza del tipo
    \begin{equation}
        x \equiv x_0 \pmod{m_1 m_2 \dots m_n}.
    \end{equation} 
\end{theorem}

\begin{proposition}
    Dato un sistema di congruenze 
    \begin{equation*}
        \left\{
        \begin{alignedat}{1}
            a_1x &\equiv b_1 \Mod{m_1}\\
            a_2x &\equiv b_2 \Mod{m_2}\\
            &\vdotswithin{\equiv} \\
            a_nx &\equiv b_n \Mod{m_n}
        \end{alignedat}      
        \right . 
    \end{equation*}
    se $x_0$ e' una soluzione particolare, allora tutte le soluzioni del sistema si ottengono sommando a $x_0$ un multiplo di $\operatorname{mcm}(m_1, m_2, \dots, m_n)$; o equivalentemente la soluzione del sistema e' una singola congruenza della forma
    \begin{equation}
        x \equiv x_0 \pmod{\operatorname{mcm}(m_1, m_2, \dots, m_n)}
    \end{equation}
\end{proposition}

\section{Struttura algebrica degli interi modulo m}

\begin{definition}
    Siano $a, n \in \Z$; allora si dice classe di resto $[a]_n$ l'insieme 
    \begin{equation}
        [a]_n = \left\{x \in \Z \mid x \equiv a \Mod{n}\right\}.
    \end{equation}
    Il numero $a$ si dice rappresentante della classe $[a]_n$.
\end{definition}

Due classi di resto si dicono uguali se contengono gli stessi elementi.
Il rappresentante di una classe non e' unico, anzi per ogni classe ci sono infinite scelte che corrispondono a tutti i numeri appartenenti alla classe. Vale quindi la seguente osservazione:
\begin{remark}
    $a \equiv b \Mod{m} \iff [a]_n = [b]_n$.
\end{remark}

Notiamo che per ogni numero $n$ ci sono esattamente $n$ classi di resto modulo $n$: infatti ce n'e' una esattamente per ogni possibile resto della divisione per $n$, cioe' per ogni numero tra $0$ e $n-1$ inclusi.

\begin{definition}
    Si dice insieme degli interi modulo $n$ l'insieme
    \begin{equation}
        \Z/(n) = \left\{ [0]_n, [1]_n, \dots, [n-1]_n\right\}.
    \end{equation}
\end{definition}

Possiamo definire due operazioni in $\Z/(n)$ che sono le operazioni di somma ($+$) e prodotto ($\cdot :$) tali che:
\begin{align}
    &[a]_n + [b]_n = [a+b]_n    &\forall [a]_n, [b]_n \in \Z/(n)\\
    &[a]_n \cdot [b]_n = [ab]_n &\forall [a]_n, [b]_n \in \Z/(n)
\end{align}

\begin{proposition}
    Per ogni $n \geq 2$ l'insieme $\Z/(n)$ e' un anello.
\end{proposition}

\section{Binomiale e Triangolo di Tartaglia}

\begin{definition}
    Si dice \textbf{coefficiente binomiale} $\binom{n}{k}$ il numero intero definito come \begin{equation}
        \binom{n}{k} = \frac{n!}{k!(n-k)!}
    \end{equation}    
\end{definition}

\begin{theorem}[del binomiale] \label{binomiale}
    Siano $x, y, n \in \Z$. Allora vale che
    \begin{equation}
        (x+y)^n = \binom{n}{0}x^0y^n + \binom{n}{1}x^1y^{n-1} + \dots + \binom{n}{n}x^ny^0 = \sum_{k=0}^n \binom{n}{k}x^{n-k}y^k
    \end{equation}
\end{theorem}

\begin{proposition}\label{binomio_pk_divisibile_p}
    Se $p$ e' primo, allora per ogni $k$ tale che $0 < k < p$ vale che
    \begin{equation}
        \binom{p}{k} \equiv 0 \Mod{p}
    \end{equation}
\end{proposition}
\begin{proof}
    Consideriamo un $k$ generico tale che $0 < k < p$.
    Allora \[
        \binom{p}{k} = \frac{p!}{k!(p-k)!} \iff p! = \binom{p}{k}(p-k)!k!    
    \]
    Ma $p \divides p!$, dunque $p \divides \binom{p}{k}(p-k)!k!$, dunque per la proposizione \ref{primo_divide_prodotto} segue che \[
        p \divides \binom{p}{k} \text{ oppure } p \divides (p-k)! \text{ oppure } p \divides k!
    .\]

    Notiamo che sia $k$ che $p-k$ sono numeri minori di $p$, dunque $k!$ e $(p-k)!$ sono un prodotto di numeri minori di $p$. Ma $p$ e' primo, dunque e' coprimo con tutti i numeri che non siano un multiplo di $p$ (e quindi e' coprimo con tutti i numeri compresi tra $0$ e $p$ esclusi), dunque per la proposizione \ref{prodotto_coprimo_n} $p$ deve essere coprimo anche con $k!$ e con $(p-k)!$. 
    
    Da cio' segue che $p$ non puo' dividere $k!$ e $(p-k)!$.
    L'ultima possibilita' e' che $p \divides \binom{p}{k}$, che e' equivalente a dire che $\binom{p}{k} \equiv 0 \Mod{p}$.
\end{proof}

\begin{proposition}\label{(x+y)^p_congr_x^p+y^p}
    Siano $x, y, p \in \Z$, $p$ primo. Allora
    \begin{equation}
        (x+y)^p \equiv x^p + y^p \Mod{p}
    \end{equation}
\end{proposition}
\begin{proof}
    Per il teorema del Binomiale (\ref{binomiale}) sappiamo che
    \begin{alignat*}{1}
        (x+y)^p &= \binom{p}{0}x^p + \binom{p}{1}x^{p-1}y^1 + \dots + \binom{p}{i}x^{p-i}y^i + \dots + \binom{p}{p}y^p \\
        \intertext{Ma per la proposizione \ref{binomio_pk_divisibile_p} tutti i termini intermedi di questa somma sono congrui a $0$ modulo $p$, dunque:}
        &\equiv \binom{p}{0}x^p + \binom{p}{p}y^p \Mod{p}\\
        &\equiv x^p + y^p \Mod{p}
    \end{alignat*}
    come volevasi dimostrare.
\end{proof}

\begin{corollary}\label{(x_1+x_n)^p_congr_x_1^p+x_n^p}
    Siano $x_1, x_2, \dots, x_n, p \in \Z$, $p$ primo. Allora
    \begin{equation}
        (x_1+x_2+\dots+x_n)^p \equiv x_1^p + x_2^p + \dots + x_n^p \Mod{p}
    \end{equation}
\end{corollary}
\begin{proof}
    Per induzione su n.
    \begin{itemize}
        \item \textbf{Caso base.}

        Sia $n = 1$. Allora $x_1^p \equiv x_1^p \Mod{p}$ ovviamente.
        \item \textbf{Passo induttivo.}
        
        Supponiamo che la tesi sia vera per $n-1$ e dimostriamola per $n$.
        \begin{alignat*}{1}
            (x_1+x_2+\dots+x_n)^p &\equiv ((x_1+x_2+\dots+x_{n-1})+x_n)^p \Mod{p}\\
            \intertext{(per la proposizione \ref{(x+y)^p_congr_x^p+y^p})}
            &\equiv (x_1+x_2+\dots+x_{n-1})^p+x_n^p\Mod{p}\\
            \intertext{(per ipotesi induttiva)}
            &\equiv x_1^p + x_2^p + \dots + x_{n-1}^p + x_n^p\Mod{p}
        \end{alignat*}
        che e' la tesi per $n$.
    \end{itemize}
    Dunque dal caso base e dal passo induttivo segue che la tesi vale per ogni $n$.
\end{proof}

\begin{theorem}
    [Piccolo Teorema di Fermat]
    Se $p$ e' primo, allora $x^p \equiv x \Mod{p}$.
\end{theorem}
\begin{proof}
    \begin{alignat*}{1}
        x^p &\equiv (\overbrace{1 +\dots+ 1}^{x \text{ volte}})^p \Mod{p}\\
        \intertext{(per il corollario \ref{(x_1+x_n)^p_congr_x_1^p+x_n^p})}
        &\equiv \overbrace{1^p +\dots+ 1^p}^{x \text{ volte}} \Mod{p}\\
        &\equiv \overbrace{1 +\dots+ 1}^{x \text{ volte}} \Mod{p}\\
        &\equiv x \Mod{p}
    \end{alignat*}
    che e' la tesi.
\end{proof}
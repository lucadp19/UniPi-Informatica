\section{Algoritmo di Euclide}

Il metodo più efficiente per calcolare il massimo comun divisore tra due numeri è dato dall'algoritmo di Euclide, che si basa sul seguente teorema.

\begin{theorem} \label{gcd_a_a-b}
    Siano $a, b \in \Z$. Allora
    \begin{equation}
        \gcd{a, b} = \gcd{a, b - a} = \gcd{a - b, b}.
    \end{equation} 
\end{theorem}
\begin{proof}  
    Ovviamente $\gcd{a, b} = \gcd{b, a}$, dunque se vale la prima uguaglianza varrà anche la seconda,
    in quanto 
    \[
       \gcd{a, b} = \gcd{b, a} = \gcd{b, a - b} = \gcd{a - b, b}.
    \] 
    Dunque è sufficiente
    dimostrare che $\gcd{a, b} = \gcd{a, b - a}$.

    Sia $\mathbb{D}_{x, y}$ l'insieme dei divisori comuni a $x$ e $y$, cioè
    \[
        \mathbb{D}_{x, y} = \set*{d \given d \divides x \;\land\; d \divides y }.
    \]
    Allora per dimostrare la tesi è sufficiente dimostrare che $\mathbb{D}_{a, b} = \mathbb{D}_{a, b - a}$, in quanto
    se i due insiemi sono uguali necessariamente anche i loro massimi saranno uguali.

    Dimostriamo che $\mathbb{D}_{a, b} \subseteq \mathbb{D}_{a, b - a}$. Sia $d \in \mathbb{D}_{a, b}$, 
    cioè $d \divides a$ e $d \divides b$. Allora per la \autoref{divides_sum_subtr_mult} segue che
    $d \divides b - a$, cioè $d \in \mathbb{D}_{a, b - a}$, 
    cioè $\mathbb{D}_{a, b} \subseteq \mathbb{D}_{a, b - a}$.

    Dimostriamo ora che $\mathbb{D}_{a, b - a} \subseteq \mathbb{D}_{a, b}$. 
    Sia $d \in \mathbb{D}_{a, b - a}$, 
    cioè $d \divides a$ e $d \divides b - a$. Allora per la \autoref{divides_sum_subtr_mult} segue che
    $d \divides a + (b - a)$, cioè $d \divides b$, cioè $d \in \mathbb{D}_{a, b}$, 
    cioè $\mathbb{D}_{a, b - a} \subseteq \mathbb{D}_{a, b}$.

    Dunque dato che valgono sia $\mathbb{D}_{a, b} \subseteq \mathbb{D}_{a, b - a}$ e 
    $\mathbb{D}_{a, b - a} \subseteq \mathbb{D}_{a, b}$, allora vale
    $\mathbb{D}_{a, b} = \mathbb{D}_{a, b - a}$. 
    
    In particolare il massimo di questi due insiemi
    dovrà essere lo stesso, quindi $\gcd{a, b} = \gcd{a, b - a}$, che è la tesi.
\end{proof}

Dunque per calcolare il massimo comun divisore si può sfruttare il seguente algoritmo, detto \strong{algoritmo di Euclide}, che si basa sul \autoref{gcd_a_a-b}:
\begin{enumerate}
    \item Se $a = 1$ oppure $b = 1$ allora $\gcd{a, b} = 1$.
    \item Se $a = 0$ e $b \neq 0$ allora $\gcd{a, b} = b$.
    \item Se $a \neq 0$ e $b = 0$ allora $\gcd{a, b} = a$.
    \item Se $a \neq 0$ e $b \neq 0$, allora
        \begin{itemize}
            \item se $a \leq b$ segue che $\gcd{a, b} = \gcd{a - b, b}$;
            \item se $a > b$ segue che $\gcd{a, b} = \gcd{a, b - a}$
        \end{itemize}
        dove i valori di $\gcd{a - b}{b}$ o $\gcd{a, b - a}$ vengono calcolati riapplicando l'algoritmo.
\end{enumerate}
Possiamo velocizzare il procedimento usando i resti della divisione invece che la sottrazione:
\begin{enumerate}[start=4]
    \item Se $a \neq 0$ e $b \neq 0$, allora
    \begin{itemize}
        \item se $a \leq b$ segue che $\gcd{a, b} = \gcd{a \bmod b, b}$;
        \item se $a > b$ segue che $\gcd{a, b} = \gcd{a, b \bmod a}$
    \end{itemize}
    dove i valori di $\gcd{a \bmod b, b}$ o $\gcd{a, b \bmod a}$ vengono calcolati riapplicando l'algoritmo.
\end{enumerate}

I passaggi dell'algoritmo di Euclide ci consentono inoltre di calcolare un'identità molto importante, chiamata \strong{identità di Bézout}.

\begin{theorem}
    [Teorema di Bezout] \label{bezout}
    Siano $a, b \in \Z$. Allora esistono $x, y \in \Z$ tali che
    \begin{equation}
        ax + by = \gcd{a, b}.
    \end{equation}
\end{theorem}
\begin{proof}
    Siano $r_0 = a$, $r_1 = b$. Definisco la successione $\seqn*(r_n)$ come la sequenza dei resti della divisione dati dall'algoritmo di Euclide per l'gcd:\begin{align*}
        \gcd{a, b} &= \gcd{r_0, r_1} &&\text{sia } r_2 = r_0 \bmod{r_1}: \\
        &= \gcd{r_1, r_2} &&\text{sia } r_3 = r_1 \bmod{r_2}: \\
        &= \dots\\
        &= \gcd{r_n, r_{n+1}} &&\text{sia } r_{n+2} = r_n \bmod{r_{n+1}}:\\
        &= \dots
    \end{align*}
    Questo processo ha termine quando un resto $r_{m+1}$ è uguale a $0$: in quel caso $\gcd{r_m, r_{m+1}} = \gcd{r_m, 0}= r_m = \gcd{a, b}$.

    Dimostriamo che per ogni $n$ possiamo scrivere $r_n$ come $ax_n + by_n$ per qualche $x_n, y_n \in \Z$.
    \begin{description}
        \item[Caso base] Per $r_0$ e $r_1$ è banale: \begin{align*}
            r_0 = 1\cdot a + 0 \cdot b, &&r_1 = 0 \cdot a + 1 \cdot b.
        \end{align*}
        \item[Passo induttivo] Supponiamo di saper scrivere $r_n$ e $r_{n-1}$ come combinazione di $a$ e $b$ e dimostriamo che possiamo farlo anche per $r_{n+2}$.
               
        Per ipotesi induttiva esistono $x_n, y_n, x_{n+1}, y_{n+1} \in \Z$ tali che \begin{align*}
            r_n = ax_n + by_n, && r_{n+1} = ax_{n+1} + by_{n+1}.
        \end{align*}

        Per definizione sappiamo che \[
            r_{n+2} = r_n \bmod{r_{n+1}}, 
        \] ovvero per definizione di resto \[
            r_{n+2} = r_n - q_{n+1}r_{n+1}
        \] per qualche $q_{n+1} \in \Z$. Sostituendo otteniamo \begin{align*}
            r_{n+2} &= r_n - q_{n+1}r_{n+1}\\
            &= ax_n + by_n - q_{n+1}(ax_{n+1} + by_{n+1})\\
            &= a(x_n - q_{n+1}x_{n+1}) + b(y_n - q_{n+1}y_{n+1}).
        \end{align*}
        Dunque $x_{n+2} = x_n - q_{n+1}x_{n+1}$ e $y_{n+2} = y_n - q_{n+1}y_{n+1}$, cioè possiamo esprimere $r_{n+2}$ come combinazione lineare di $a, b$.
    \end{description}

    Dato che per induzione questo risultato vale per tutti gli $n \in \N$, varrà anche per $m$, ovvero esistono $x, y \in \Z$ tali che $r_m = ax + by$. Ma $r_m = \gcd{a, b}$, dunque la tesi.
\end{proof}

\begin{example}
    Cerchiamo il massimo comun divisore tra $252$ e $198$ e i coefficienti del Teorema di Bezout con l'Algoritmo di Euclide.

    I passi dell'algoritmo ci dicono di sottrarre dal più grande dei due numeri l'altro ripetutamente: il modo più veloce di far questa cosa è sottrarre un multiplo del numero più piccolo in modo da ottenere il più piccolo resto positivo. Scriviamo esattamente il multiplo usato a destra dei passaggi dell'algoritmo in quanto sarà utile più tardi.
    \begin{alignat*}{5}
            &\gcd{252, 198}   \qquad\qquad &54  &= {}&252  {}&- \boxed{1} \cdot 198\\
        = {}&\gcd{54, 198}    \qquad\qquad &36  &= {}&198  {}&- \boxed{3} \cdot 54\\
        = {}&\gcd{54, 36}     \qquad\qquad &18  &= {}&54   {}&- \boxed{1} \cdot 36\\
        = {}&\gcd{18, 36}     \qquad\qquad &0   &= {}&36   {}&- \boxed{2} \cdot 18\\
        = {}&\gcd{18, 0}      \\
        = {}&18.
    \end{alignat*}

    Per il \nameref{bezout} dovranno quindi esistere $x_0, y_0 \in \Z$ tali che \[
        252x_0 + 198y_0 = 18.    
    \] Per trovarli seguiamo la catena di equazioni scritta sopra, ricavando i coefficienti della combinazione lineare per ognuno dei numeri che compare nello svolgimento dell'algoritmo.
    
    Innanzitutto ovviamente vale che \[
        252 = 252 \cdot \boxed{1} + 198 \cdot \boxed{0}, \quad
        198 = 252 \cdot \boxed{0} + 198 \cdot \boxed{1}.
    \] La prima uguaglianza nell'algoritmo di Euclide ci dice inoltre che \[
        54 = 252 \cdot \boxed{1} + 198 \cdot \boxed{-1}.
    \] Per ricavare le decomposizioni successive sfruttiamo le precedenti: \begin{align*}
        36  &= 198 - \boxed{3} \cdot 54\\
            &= (252 \cdot \boxed{0} + 198 \cdot \boxed{1}) - \boxed{3} \cdot (252 \cdot \boxed{1} + 198 \cdot \boxed{-1}) \\
            &= 252 \cdot \boxed{0} + 198 \cdot \boxed{1} + 252 \cdot \boxed{-3} + 198 \cdot \boxed{3}\\
            &= 252 \cdot \boxed{-3} + 198 \cdot \boxed{4}.\\
        18  &= 54 - \boxed{1} \cdot 36\\
            &= (252 \cdot \boxed{1} + 198 \cdot \boxed{-1}) - \boxed{1} \cdot (252 \cdot \boxed{-3} + 198 \cdot \boxed{4}) \\
            &= 252 \cdot \boxed{1} + 198 \cdot \boxed{-1} + 252 \cdot \boxed{3} + 198 \cdot \boxed{-4}\\
            &= 252 \cdot \boxed{4} + 198 \cdot \boxed{-5}.
    \end{align*}
    Abbiamo quindi trovato i coefficienti del teorema di Bezout: scegliendo $x_0 = 4$, $y_0 = -5$ riusciamo a scrivere $18$ come combinazione lineare a coefficienti interi di $252$ e $198$.

    Osserviamo che ogni volta svolgiamo i calcoli solo sui \emph{coefficienti} di $252$ e $198$: possiamo abbreviare il procedimento usando una semplice tabella con questa forma:
    \begin{figure}[H]
        % \begin{table}[]
            \centering
            \begin{tabular}{l|ll}
                       & $252$  & $198$ \\ \hline
                $252$  & $1$    & $0$   \\
                $198$  & $0$    & $1$   \\
                $54$   & $1$    & $-1$  \\
                $36$   & $-3$   & $4$   \\
                $18$   & $4$    & $-5$
            \end{tabular}
        % \end{table}
    \end{figure}
    Ogni riga ci dice i coefficienti della combinazione lineare, esattamente come prima. Le prime due righe sono quindi immediate, mentre le successive possono essere trovate semplicemente con questo procedimento:
    \begin{itemize}
        \item cerchiamo l'equazione dell'algoritmo di Euclide che definisce il numero che stiamo considerando (ad esempio la quarta riga ha come numero $36$, definito da $36 = 198 - \boxed 3 \cdot 54$);
        \item ricaviamo il coefficiente della prima colonna prendendo i coefficienti della prima colonna dei numeri che compaiono nell'equazione (in questo caso $198$ e $54$) e sottraendoli come ci dice l'equazione (otteniamo quindi $0 - \boxed 3 \cdot 1 = -3$, che è il coefficiente della prima colonna della quarta riga);
        \item ricaviamo il coefficiente della seconda colonna esattamente allo stesso modo.
    \end{itemize}
\end{example}

\subsection{Conseguenze del teorema di Bezout}

Elenchiamo in questa sezione alcune conseguenze del teorema di Bezout sulle proprietà dei divisori e sul loro rapporto con il massimo comun divisore di due numeri.

\begin{proposition} \label{n_divides_product}
    Siano $a, b, n \in \Z$. Allora \begin{equation}
        n \divides ab \;\land\; \gcd{a, n} = 1 \implies n \divides b.
    \end{equation}
\end{proposition}
\begin{intuition}
    Se $n$ divide $ab$, allora tutti i fattori primi che dividono $n$ dovranno essere contenuti in $ab$. Dato che $\gcd{n, a} = 1$, questi fattori non possono essere contenuti in $a$, dunque dovranno essere tutti contenuti in $b$.
\end{intuition}
\begin{proof}
    Per il teorema di Bezout (\ref{bezout}) esistono $x, y \in \Z$ tali che 
    \begin{align*}
        ax + ny &= \gcd{a, n} = 1 \\
        \intertext{Moltiplicando per $b$ otteniamo}
        abx + nby &= b 
    \end{align*}
    Ma $n \divides abx$ (poiché $n \divides ab$) e $n \divides nby$, dunque $n$ divide la loro somma, ovvero \[
        n \divides abx + nby = b(ax + ny) = b
    \]
    da cui la tesi.
\end{proof}

Cerchiamo ora di studiare come si relaziona il massimo comune divisore con gli altri divisori di due numeri dati. La prima proposizione ci dice che un divisore comune è sempre minore o uguale del mcd, mentre la seconda ci dice che un divisore comune è sempre \emph{un divisore} del mcd.

\begin{proposition} \label{greatest_common_divisor}
    Siano $a, b, t \in \Z$ tali che $t \divides a$, $t \divides b$. Allora $t \leq \gcd{a, b}$.
\end{proposition}
\begin{proof}
    La proposizione deriva direttamente dalla definizione di massimo comun divisore: se $t$ è un divisore comune ad $a$ e $b$, allora $t$ sarà minore o uguale al massimo dei divisori comuni di $a$ e $b$, cioè $t \leq \gcd{a, b}$.
\end{proof}

\begin{proposition} \label{divisori_dividono_gcd}
    Siano $a, b, t \in \Z$ tali che $t \divides a$, $t \divides b$. Allora $t \divides \gcd{a, b}$.
\end{proposition}
\begin{proof}
    Per la proposizione \ref{divides_sum_subtr_mult}, se $t \divides a$ e $t \divides b$ allora $t \divides ax+by$ per ogni $x, y \in \Z$.

    Per il teorema di Bezout (\ref{bezout}) esistono $\bar{x}, \bar{y} \in \Z$ tali che $a\bar{x}+b\bar{y} = \gcd{a, b}$. Ma quest'espressione è della forma $ax + by$, con $x = \bar{x}$, $y = \bar{y}$, dunque 
    $t \divides a\bar{x} + b\bar{y}$, cioè $t \divides \gcd{a, b}$.
\end{proof}

Possiamo anche dare una condizione necessaria e sufficiente per cui un numero $t$ divide il massimo comun divisore di altri due numeri.

\begin{proposition} \label{t_divides_gcd_lincomb}
    Siano $a, b, t \in \Z$. Allora 
    \begin{equation}
        t \divides \gcd{a, b} \iff (\forall x, y \in \Z. \quad t \divides ax + by).
    \end{equation}
\end{proposition}
\begin{proof}
    Dimostriamo entrambi i versi dell'implicazione.
    \begin{itemize}
        \item[($\implies$)] Se $t \divides \gcd{a, b}$, allora $t \divides a$ e $t \divides b$, dunque per la proposizione \ref{divides_sum_subtr_mult} segue che $t$ dovrà dividere una qualsiasi combinazione lineare di $a$ e $b$, cioè $t \divides ax+by$ per ogni $x, y \in \Z$.
        \item[($\impliedby$)] Viceversa supponiamo che $t \divides ax+by$ per ogni $x, y \in \Z$. Siano per il \hyperref[bezout]{Teorema di Bezout} $\bar{x}, \bar{y}$ i numeri tali che $a\bar{x} + b\bar{y} = \gcd{a, b}$. Allora $t$ dovrà dividere anche $a\bar{x} + b\bar{y}$, cioè $t \divides \gcd{a, b}$. \qedhere
    \end{itemize}
\end{proof}

\begin{proposition} \label{gcd_of_multiples_of_n}
    Siano $a, b, n \in \Z$. Allora \begin{equation}
        \gcd{an, bn} = n\gcd{a, b}.
    \end{equation}
\end{proposition}
\begin{intuition}
    Se due numeri hanno $n$ come fattore comune, ovviamente il massimo comun divisore dovrà contenere $n$ e quindi dovrà essere un multiplo di $n$.
\end{intuition}
\begin{proof}
    Osserviamo che se due numeri hanno gli stessi divisori allora sono uguali, a meno del segno.
    Sia $t \in \Z$ tale che $t \divides an$ e $t \divides nb$. Per la \autoref{t_divides_gcd_lincomb} allora 
    \begin{alignat*}
        {2}
        &t \divides \gcd{an, bn}\\
        \iff &t \divides nax + nby      \qquad& \forall x, y \in \Z \\
        \iff &t \divides n(ax + by)     \qquad& \forall x, y \in \Z \\
        \intertext{dunque scegliendo $x, y$ tali che $ax+by = \gcd{a, b}$ per Bezout (\ref{bezout})}
        \iff &t \divides n\gcd{a, b}. \tag*{\qedhere}
    \end{alignat*}
\end{proof}


\begin{corollary} \label{gcd_diviso_gcd}
    Siano $a, b \in \Z$ e sia $d = \gcd{a, b}$. Allora $\gcd{\nicefrac{a}{d}, \nicefrac{b}{d}} = 1$.
\end{corollary}
\begin{intuition}
    Se dividiamo due numeri per il loro gcd stiamo eliminando dalla loro fattorizzazione tutti i primi comuni ad entrambi, quindi i due numeri risultanti dall'operazione non potranno avere primi in comune e quindi saranno coprimi.
\end{intuition}
\begin{proof}
    Siano $a^\prime, b^\prime$ tali che $a = a^\prime d, b = b^\prime d$. Allora per la \autoref{gcd_of_multiples_of_n}
    \begin{alignat*}{1}
        \gcd{a, b} &= \gcd{a^\prime d, b^\prime d} \\
                   &= d\gcd{a^\prime, b^\prime} \\
                   &= \gcd{a, b} \gcd{a^\prime, b^\prime}.
        \end{alignat*} 
    Dividendo entrambi i membri per $\gcd{a, b}$ otteniamo \[
        \gcd*{a^\prime, b^\prime} = 1 
    \]
    che, per definizione di $a^\prime, b^\prime$, è equivalente a \[
        \gcd*{\frac{a}{d}, \frac{b}{d}} = 1
    \]
    che è la tesi.
\end{proof}

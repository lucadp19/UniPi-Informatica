\section{Forma polare}

Nella sezione precedente abbiamo visto come ad ogni numero complesso può essere associato un vettore nel piano complesso le cui coordinate corrispondono alla parte reale e alla parte immaginaria del numero complesso in esame.

I vettori nel piano possono però essere rappresentati anche da un altro punto di vista: ad ogni vettore può essere associata la sua lunghezza e l'angolo che il vettore forma con ĺ'asse delle ascisse: 
\begin{center}
    \begin{tikzpicture}
        \coordinate (a) at (1, 0);
        \draw[step=1cm,gray!25!,very thin] (-4,-3) grid (4,3);
        \draw[thick,->] (-3,0) -- (3,0) node[anchor=north west] {Re};
        \draw[thick,->] (0,-2) -- (0,2) node[anchor=south east] {Im};
        \draw[red,thick,->] (0,0) coordinate (O) -- (2, 1) coordinate (P)
        node[midway,above] {$\rho$};

        \draw pic[draw,angle radius=1cm,"$\theta$" shift={(6mm,1mm)}] {angle=a--O--P};
    \end{tikzpicture}
\end{center}

Per formalizzare questa associazione, consideriamo innanzitutto l'insieme dei numeri complessi con modulo uguale ad $1$. Per definizione di modulo, un numero complesso $z = a+ib$ ha modulo $1$ se e solo se \[
    \sqrt{a^2 + b^2} = 1 \iff a^2 + b^2 = 1.
\] I numeri di modulo unitario formano quindi una circonferenza di raggio $1$ con centro nell'origine degli assi:
\begin{center}
    \begin{tikzpicture}
        \coordinate (a) at (1, 0);
        \draw[step=1.5cm,gray!25!,very thin] (-4,-3) grid (4,3);
        \draw[thick,->] (-3,0) -- (3,0) node[anchor=north west] {Re};
        \draw[thick,->] (0,-3) -- (0,3) node[anchor=south east] {Im};
        % \draw[red,thick,->] (0,0) coordinate (O) -- (60:1) coordinate (P)
        % node[midway,above] {$1$};

        \draw[blue,thick,->] (0,0) coordinate (O) -- (135:1.5) coordinate (Q)
        node[midway,above] {$1$};

        \draw[color=black, very thick](0,0) circle (1.5);

        % \draw pic[draw,color=red,angle radius=0.5cm,"$\nicefrac{\pi}{3}$" shift={(6mm,1mm)}] {angle=a--O--P};
        \draw pic[draw,color=blue,angle radius=0.75cm,"$\frac{3\pi}{4}$" shift={(6mm,1mm)}] {angle=a--O--Q};
    \end{tikzpicture}
\end{center}

Ogni vettore di questa circonferenza è univocamente determinato dall'angolo che forma con l'asse delle ascisse: dato un angolo $\theta$, il vettore che corrisponde a $\theta$ avrà come coordinate $(\cos \theta, \sin \theta)$, dunque il corrispondente numero complesso sarà \[
    z = \cos \theta + i\sin\theta.    
\]

La prossima proposizione ci mostra come moltiplicare tra di loro numeri complessi di modulo unitario.
\begin{proposition}\label{prop:product_unitary}
    Siano $z, w \in \C$ tali che \[
        z = \cos \theta + i\sin\theta, \quad w = \cos \phi + i\sin\phi.
    \]
    Allora vale che \begin{equation}
        zw = \cos (\theta + \phi) + i\sin (\theta + \phi).
    \end{equation}
\end{proposition}
\begin{proof}
    \begin{align*}
        zw &= (\cos \theta + i\sin\theta)(\cos \phi + i\sin\phi)\\
        &= (\cos \theta\cos \phi - \sin\theta\sin\phi) + i(\cos\theta\sin\phi + \sin\theta\cos\phi)\\
        &= \cos(\theta + \phi) + i\sin(\theta+\phi). \qedhere
    \end{align*}
\end{proof}

Dunque molitplicare tra di loro due numeri complessi di angoli $\theta$ e $\phi$ e di modulo unitario ci restituisce un numero complesso di modulo unitario e di angolo $\theta + \phi$: equivale quindi a ruotare uno dei due vettori per l'angolo associato all'altro.

Consideriamo ora un vettore con modulo $\rho \geq 0$ qualunque. Tramite la trigonometria possiamo ricavare le sue coordinate:
\begin{center}
    \begin{tikzpicture}
        % \coordinate (a) at (1, 0);
        \draw[step=1cm,gray!25!,very thin] (-2,-2) grid (4,3);
        \draw[thick,->] (-1,0) -- (3,0) node[anchor=north west] {Re};
        \draw[thick,->] (0,-1) -- (0,2) node[anchor=south east] {Im};
        \draw[red,thick,->] (0,0) coordinate (O) -- (2, 1) coordinate (P)
        node[midway,above] {$\rho$};

        \draw[black,<->] (2,0) coordinate (A) -- (P)
        node[midway,right] {$\rho\sin\theta$};
        \draw[black,<->] (O) -- (A)
        node[midway,below] {$\rho\cos\theta$};

        \draw pic[draw,angle radius=1cm,"$\theta$" shift={(6mm,1mm)}] {angle=A--O--P};
    \end{tikzpicture}
\end{center}

Dunque un vettore di modulo $\rho$ e angolo $\theta$ ha come coordinate \[
    (\rho\cos\theta, \rho\sin\theta),
\] da cui segue che il corrispondente numero complesso è della forma \[
    z = \rho\cos\theta + \rho\sin\theta = \rho(\cos\theta + \sin\theta).
\]

Anche in questo caso moltiplicare due numeri complessi è particolarmente facile:
\begin{proposition} \label{prop:product_polar}
    Siano $z, w \in \C$ tali che \[
        z = r_1(\cos \theta + i\sin\theta), \quad w = r_2(\cos \phi + i\sin\phi).
    \]
    Allora vale che \begin{equation}
        zw = r_1r_2(\cos (\theta + \phi) + i\sin (\theta + \phi)).
    \end{equation}
\end{proposition}
\begin{proof}
    \begin{align*}
        zw &= r_1(\cos \theta + i\sin\theta) \cdot r_2(\cos \phi + i\sin\phi)\\
        &= r_1r_2 \cdot ((\cos \theta + i\sin\theta)(\cos \phi + i\sin\phi)) \tag{per la \autoref{prop:product_unitary}}\\
        &= r_1r_2(\cos (\theta + \phi) + i\sin (\theta + \phi)). \qedhere
    \end{align*}
\end{proof}

In questo caso il prodotto tra due numeri complessi corrisponde al vettore con \begin{itemize}
    \item modulo uguale al prodotto dei moduli,
    \item angolo dato dalla rotazione di uno dei due vettori per l'angolo definito dal secondo.
\end{itemize}

Possiamo quindi introdurre la \emph{forma polare} di un numero complesso.
\begin{definition}
    [Forma polare]
    Sia $z \in \C$ un numero complesso con modulo $\rho$ e angolo associato $\theta$. Si dice forma polare di $z$ la forma \begin{equation}
        z = \rho(\cos\theta +i\sin\theta) = \rho e^{i\theta}.
    \end{equation}
    L'angolo $\theta$ viene detto \emph{argomento} del numero complesso $z$ e lo si indica con $\arg z$.
\end{definition}

Per trasformare un numero complesso da una forma all'altra basta sfruttare un po' di trigonometria:
\paragraph{Dalla forma cartesiana alla polare} Consideriamo un numero $z = a+ib \in \C$ espresso in forma cartesiana. Per portarlo in forma polare dobbiamo trovare $\rho = \abs*{z}$ e $\arg z$.

Per definizione di modulo, $\abs*{z} = \sqrt{a^2 + b^2}$. Per trovare l'argomento basta fare l'arcotangente del rapporto tra i cateti, facendo attenzione al quadrante in cui ci troviamo:
\begin{equation}\label{eq:argument}
    \arg z = \begin{cases}
        \arctan \frac{b}{a}, &\text{se } a > 0 \\
        \arctan \frac{b}{a} + \pi, &\text{se } a < 0 \\
        \nicefrac{\pi}{2} &\text{se } a = 0, b > 0\\
        \nicefrac{3\pi}{2} &\text{se } a = 0, b < 0.
    \end{cases}    
\end{equation}
\paragraph{Dalla forma polare alla cartesiana} Se $z = \rho(\cos\theta +i\sin\theta)$ è un numero complesso in forma polare, per portarlo in forma cartesiana basta calcolare le due funzioni trigonometriche: \begin{align*}
    z = \rho\cos\theta + i\rho\sin\theta\\
    \implies a = \rho\cos\theta, b = \rho\sin\theta.
\end{align*}

\begin{example}
    Il numero $i = 0 + 1i$ ha come forma polare $e^{i\frac{\pi}{2}}$. Infatti: \begin{itemize}
        \item $\abs*{i} = \abs*{0 + 1i} = \sqrt{0^2 + 1^2} = 1$.
        \item $\arg i = \nicefrac{\pi}{2}$ poiché ci troviamo nel terzo caso della \eqref{eq:argument}.
    \end{itemize}

    Ciò è evidente anche disegnando il numero $i$ nel piano complesso:
    \begin{center}
        \begin{tikzpicture}
            \coordinate (a) at (1, 0);
            \draw[step=1cm,gray!25!,very thin] (-3,-3) grid (3,3);
            \draw[thick,->] (-2,0) -- (2,0) node[anchor=north west] {Re};
            \draw[thick,->] (0,-2) -- (0,2) node[anchor=south east] {Im};
            \draw[red,thick,->] (0,0) coordinate (O) -- (0, 1) coordinate (i)
            node[midway,left] {$\abs*{i} = 1$};
    
            \draw pic[draw,angle radius=0.5cm,"$\frac{\pi}{2}$" shift={(3mm,3mm)}] {angle=a--O--i};

            \filldraw [black] (i) circle (1pt)
            node[above right] {$i = 0 + 1i$};
        \end{tikzpicture}
    \end{center}
\end{example}

La forma "esponenziale", data da $z = \rho e^{i\theta}$ è comoda poiché più sintetica della forma con le funzioni trigonometriche. Inoltre, essa continua a rispettare la \autoref{prop:product_polar}: \[
    r_1e^{i\theta} \cdot r_2e^{i\phi} = r_1r_2 e^{i(\theta + \phi)}.    
\]

Prima di studiare le potenze e le radici $n$-esime nei complessi,facciamo alcune osservazioni finali.
\begin{remark}
    I numeri complessi di modulo unitario sono tutti e soli della forma $z = e^{i\theta}$, in quanto il loro modulo è uguale ad $1$.
\end{remark}
\begin{remark}
    Il coniugato in forma polare di $\rho e^{i\theta}$ è il numero $\rho e^{-i\theta}$, dunque la forma polare è comoda anche per calcolare i coniugati di numeri complessi.
\end{remark} 
\begin{remark}
    I numeri reali, essendo tutti sull'asse delle ascisse, hanno argomento $0$ (se sono positivi) oppure $\pi$ (se sono negativi): dunque i numeri reali sono tutti e solo delle forme $\rho e^{i0} = rho$ oppure $\rho e^{i\pi} = -\rho$.
\end{remark}
\begin{remark}
    Due numeri complessi in forma polare sono uguali se e solo se \begin{itemize}
        \item i loro moduli sono uguali,
        \item i loro argomenti sono uguali, a meno di un multiplo intero di $2\pi$.
    \end{itemize}
    Infatti gli angoli $\theta$ e $\theta + 2k\pi$ sono uguali per ogni $k \in \Z$, dunque è necessario considerare che gli argomenti non sono necessariamente in $[0, 2\pi)$.
\end{remark}
\documentclass[a4paper]{report}
    \usepackage[utf8]{inputenc}
    \usepackage[italian]{babel}
    \usepackage[T1]{fontenc}
    \usepackage{textcomp, microtype}
    \usepackage{amsmath, amsthm, amssymb, cases, mathtools, bm, enumerate}
    \usepackage{array}
    \usepackage{float}

    \usepackage{hyperref} % ultimo package da caricare!

\restylefloat{table}

\theoremstyle{plain}
\newtheorem{theorem}{Teorema}[section]
\newtheorem{corollary}[theorem]{Corollario}
\newtheorem{proposition}[theorem]{Proposizione}

\theoremstyle{definition}
\newtheorem{example}[theorem]{Esempio}
\newtheorem{definition}[theorem]{Definizione}

\theoremstyle{remark}
\newtheorem*{remark}{Osservazione}
\newtheorem*{solution}{Soluzione}
\newtheorem*{intuition}{Intuizione}

\DeclareMathOperator{\tc}{\ tale che\ }

\newcolumntype{z}{r<{{}}}
\newcolumntype{o}{@{}>{{}}c<{{}}@{}}

% matrix with lines
\makeatletter
\renewcommand*\env@matrix[1][*\c@MaxMatrixCols c]{%
  \hskip -\arraycolsep
  \let\@ifnextchar\new@ifnextchar
  \array{#1}}
\makeatother

\newcommand{\divides}{\mid}
\newcommand{\Mod}[1]{\ \left(#1\right)}
\newcommand{\abs}[1]{\left|#1\right|}
\newcommand{\mcm}[2]{\operatorname{mcm}\left(#1, #2\right)}
\newcommand{\mcd}[2]{\operatorname{mcd}\left(#1, #2\right)}
\newcommand{\Span}[1]{\operatorname{span}\left\{#1\right\}}
\newcommand{\ang}[1]{\left\langle #1 \right\rangle}
\newcommand{\rk}[1]{\operatorname{rango}\left( #1 \right)}
\newcommand{\Imm}[1]{\operatorname{Im}#1}
\newcommand{\N}{\mathbb{N}}
\newcommand{\Z}{\mathbb{Z}}
\newcommand{\Q}{\mathbb{Q}}
\newcommand{\R}{\mathbb{R}}
\newcommand{\C}{\mathbb{C}}
\newcommand{\K}{\mathbb{K}}
\newcommand{\M}{\mathbb{M}}

\begin{document}

% \author{Luca De Paulis}
\title{Algebra Lineare}
\maketitle

\tableofcontents

\chapter{Insiemi numerici}

\section{Strutture algebriche fondamentali}

\begin{definition}[Gruppo]
    Si dice \textbf{gruppo} una tripla ($G$, $\cdot$, $e$) formata da \begin{itemize}
        \item un insieme di elementi $G$;
        \item un operazione $\cdot : A \times A \to A$ detta prodotto;
        \item un elemento $e \in G$
    \end{itemize} per cui valgono i seguenti assiomi: 
    \begin{description}
        \item[(Assiomi di gruppo)] Per ogni $a, b, c \in G$ vale che
        \begin{align*}
            &\text{(P1)}      &&(ab) \in G            &\text{(chiusura rispetto a $\cdot$)}\\
            &\text{(P2)}      &&(ab)c = a(bc)         &\text{(associatività di $\cdot$)}\\
            &\text{(P3)}      &&a \cdot e=e \cdot a=a &\text{($e$ el. neutro di $\cdot$)}\\
            &\text{(P4)}     &&\exists a^{-1} \in G. \quad aa^{-1} = e &\text{(inverso per $\cdot$)}
            \intertext{Si dice \textbf{gruppo commutativo} un gruppo per cui vale inoltre il seguente assioma:}
            &\text{(P5)}     &&ab = ba               &\text{(commutatività di $\cdot$)}
        \end{align*}
    \end{description}
\end{definition}

\begin{definition}[Anello]
    Si dice \textbf{anello} una quintupla ($A$, $+$, $\cdot$, $0$, $1$) formata da
    \begin{itemize}
        \item un insieme di elementi $A$;
        \item un operazione $+ : A \times A \to A$ detta somma;
        \item un operazione $\cdot : A \times A \to A$ detta prodotto;
        \item un elemento $0 \in A$;
        \item un elemento $1 \in A$
    \end{itemize} per cui valgono i seguenti assiomi: 
    \begin{description}
        \item[(Assiomi di anello)] Per ogni $a, b, c \in A$ vale che
        \begin{align*}
            &\text{(S1)}      &&(a+b) \in A           &\text{(chiusura rispetto a $+$)}\\
            &\text{(S2)}      &&a+b = b+a             &\text{(commutatività di $+$)}\\
            &\text{(S3)}      &&(a+b)+c = a+(b+c)     &\text{(associatività di $+$)}\\
            &\text{(S4)}      &&a+0=0+a=a             &\text{(0 el. neutro di $+$)}\\
            &\text{(S5)}      &&\exists (-a) \in A. \quad a+(-a) = 0 &\text{(opposto per $+$)}\\
            &\text{(P1)}      &&(ab) \in A            &\text{(chiusura rispetto a $\cdot$)}\\
            &\text{(P2)}      &&(ab)c = a(bc)         &\text{(associatività di $\cdot$)}\\
            &\text{(P3)}      &&a \cdot 1=1 \cdot a=a &\text{(1 el. neutro di $\cdot$)}\\
            &\text{(P4)}      &&(a+b)c = ac + bc      &\text{(distributività 1)} \\
            &\text{(P5)}     &&a(b+c) = ab + ac      &\text{(distributività 2)}
            \intertext{Si dice \textbf{anello commutativo} un anello per cui vale inoltre il seguente assioma:}
            &\text{(P6)}     &&ab = ba               &\text{(commutatività di $\cdot$)}
        \end{align*}
    \end{description} 
\end{definition}

Un tipico esempio di anello commutativo è $\Z$: infatti gli anelli generalizzano le operazioni che possiamo fare sui numeri interi e le loro proprietà fondamentali per estenderle ad altri insiemi con la stessa struttura algebrica.

\begin{definition}[Campo]
    Si dice \textbf{campo} una quintupla ($F$, $+$, $\cdot$, $0$, $1$) formata da
    \begin{itemize}
        \item un insieme di elementi $F$;
        \item un operazione $+ : F \times F \to F$ detta somma;
        \item un operazione $\cdot : F \times F \to F$ detta prodotto;
        \item un elemento $0 \in F$;
        \item un elemento $1 \in F$
    \end{itemize}  per cui valgono i seguenti assiomi: 
    \begin{description}
        \item[(Assiomi di campo)] Per ogni $a, b, c \in F$ vale che
        \begin{align*}
            &\text{(S1)}      &&(a+b) \in F           &\text{(chiusura rispetto a $+$)}\\
            &\text{(S2)}      &&a+b = b+a             &\text{(commutatività di $+$)}\\
            &\text{(S3)}      &&(a+b)+c = a+(b+c)     &\text{(associatività di $+$)}\\
            &\text{(S4)}      &&a+0=0+a=a             &\text{(0 el. neutro di $+$)}\\
            &\text{(S5)}      &&\exists (-a) \in F. \quad a+(-a) = 0 &\text{(opposto per $+$)}\\
            &\text{(P1)}      &&(ab) \in F            &\text{(chiusura rispetto a $\cdot$)}\\        
            &\text{(P2)}      &&ab = ba               &\text{(commutatività di $\cdot$)}\\
            &\text{(P3)}      &&(ab)c = a(bc)         &\text{(associatività di $\cdot$)}\\
            &\text{(P4)}      &&a \cdot 1=1 \cdot a=a &\text{(1 el. neutro di $\cdot$)}\\
            &\text{(P5)}     &&(a+b)c = ac + bc      &\text{(distributività)} \\
            &\text{(P6)}     &&a \neq 0 \implies \exists a^{-1} \in F. \quad aa^{-1} = 1 &\text{(inverso per $\cdot$)}
        \end{align*}
    \end{description} 

    La definizione sopra è equivalente a dire che $F$ è un anello commutativo per cui ogni elemento non nullo ha un inverso moltiplicativo.
\end{definition}

Tra gli insiemi numerici classici, gli insiemi $\Q, \R$ e $\C$ sono tutti esempi di campi: infatti le operazioni di addizione e moltiplicazione sono chiuse rispetto all'insieme, rispettano le proprietà commutativa, associativa e distributiva ed esistono gli inversi per la somma e per il prodotto (per ogni numero diverso da $0$). Il concetto di campo serve quindi a generalizzare la struttura algebrica dei numeri razionali/reali/complessi per altri insiemi numerici.

Nei campi vale la seguente proposizione.
\begin{proposition}
    [Regola di annullamento del prodotto] \label{annullamento_prodotto}
    Sia $\K$ un campo e siano $a, b \in \K$. Allora \[
        ab = 0 \implies a = 0 \lor b = 0.    
    \]
\end{proposition}
\begin{proof}
    Sappiamo che $a = 0 \lor b = 0$ è equivalente a $a \neq 0 \implies b = 0$, dunque supponiamo che $a$ sia diverso da $0$ e dimostriamo che $b$ è zero.

    Dato che $a \neq 0$ allora ammette un inverso. Chiamiamolo $a^{-1}$ e moltiplichiamo entrambi i membri per esso:
    \begin{alignat*}
        {1}
        &a^{-1}(ab) = a^{-1} \cdot 0\\
        \iff &(a^{-1}a)b = 0 \\
        \iff &b = 0
    \end{alignat*}
    che è la tesi.
\end{proof}

\section{Numeri complessi}

\begin{definition}[Unità immaginaria]
    Si dice unità immaginaria il numero $i$ tale che \[
        i^2 = -1.    
    \]
\end{definition}

\begin{definition}[Numeri complessi]
    L'insieme dei numeri complessi $\C$ è l'insieme dei numeri della forma $a+ib$ per qualche $a, b \in \R$, ovvero \[
        \C = \{ a + ib \mid a, b \in \R, i^2 = -1\}.  
    \]
\end{definition}

\begin{definition}[Parte reale e immaginaria]
    Sia $z \in \C$ tale che $z = a + ib$. Allora si dicono rispettivamente \begin{itemize}
        \item parte reale di $z$ il numero $\Re z = a$;
        \item parte immaginaria di $z$ il numero $\Im z = b$.
    \end{itemize}
\end{definition}

\begin{definition}[Somma e prodotto sui complessi]
    Definiamo le seguenti due operazioni su $\C$:
    \begin{itemize}
        \item $+ : \C \times \C \to \C$ tale che $(a + ib) + (c + id) = (a + c) + i(b + d)$;
        \item $\cdot : \C \times \C \to \C$ tale che $(a + ib) \cdot (c + id) = (ac - bd) + i(ad + bc)$.
    \end{itemize}
\end{definition}

\begin{remark}
    Le due operazioni vengono naturalmente dalla somma e dal prodotto tra monomi. Infatti \begin{gather*}
        (a + ib) + (c + id) = a + c + ib + id = (a + c) + i(b + d);\\
        \begin{alignedat}{1}
            (a + ib) \cdot (c + id) &= ac + iad + ibc + i^2bd \\
            &= ac + i(ad + bc) -bd \\
            &= (ac - bd) + i(ad + bc).
        \end{alignedat}
    \end{gather*}
\end{remark}

Notiamo che i numeri complessi della forma $a + i0$ sono numeri reali, dunque $\R \subset \C$. Inoltre possiamo rappresentare i numeri complessi come punti in uno spazio bidimensionale dove la parte reale rappresenta l'ascissa e la parte immaginaria rappresenta l'ordinata: la retta corrispondente all'asse x è il sottoinsieme dei numeri reali.

\begin{definition}[Coniugato complesso]
    Sia $z = a+ib \in \C$. Allora si dice coniugato complesso (o semplicemente coniugato) di $z$ il numero \[
        \conj{z} = a - ib.    
    \]
\end{definition}

\begin{definition}[Norma di un numero complesso]
    Sia $z = a + ib \in \C$. Allora si dice norma di $z$ il numero reale \[
        \abs{z} = \sqrt{a^2 + b^2}.    
    \]
\end{definition}

Notiamo che $\abs{z} = 0$ se e solo se $a = b = 0$, ovvero se $z = 0$.

\begin{proposition}\label{somma_prodotto_tra_coniugati}
    Siano $z, w \in \C$ tali che $z = a+ib$, $w = c + id$. Allora \begin{enumerate}[(i)]
        \item $\conj{z} + \conj{w} = \conj{z + w}$;
        \item $\conj{z}\cdot\conj{w} = \conj{zw}$;
        \item $(\conj{z})^n = \conj{z^n}$.
    \end{enumerate}
\end{proposition}
\begin{proof}
    Dimostriamo i tre fatti.
    \begin{enumerate}[(i)]
        \item Per definizione di somma \begin{alignat*}
            {1}
            \conj{z} + \conj{w} &= (a-ib) + (c - id)\\
            &= (a+c) - i(b+d)\\
            &= \conj{z + w}.
        \end{alignat*}
        \item Per definizione di prodotto \begin{alignat*}
            {1}
            \conj{z}\cdot\conj{w} &= (a-ib)(c - id)\\
            &= (ac - bd) + i(-ad-bc)\\
            &= (ac - bd) - i(ad+bc)\\
            &= \conj{zw}.
        \end{alignat*}
        \item Dimostriamolo per induzione su $n$.
        \begin{description}
            \item[Caso base.] Se $n = 1$ allora banalmente $(\conj{z})^1 = \conj{z} = \conj{z^1}$.
            \item[Passo induttivo.] Supponiamo che la tesi valga per $n$ e dimostriamola per $n+1$. Allora \[
                (\conj{z})^{n+1} = (\conj{z})^{n} \cdot \conj{z} = \conj{z^n} \cdot \conj{z} = \conj{z^{n+1}}
            \] dove l'ultimo passaggio è giustificato dal punto precedente della dimostrazione. \qedhere
        \end{description}
    \end{enumerate}
\end{proof}

\begin{proposition}\label{somma_prodotto_col_coniugato}
    Sia $z = a+ib \in \C$. Allora valgono i seguenti fatti:
    \begin{enumerate}[(i)]
        \item $z + \conj{z} = 2\Re{z}$;
        \item $z\conj{z} = \abs{z}^2$.
    \end{enumerate}
\end{proposition}
\begin{proof}
    Dimostriamo i due fatti.
    \begin{enumerate}[(i)]
        \item Per definizione di somma $z + \conj{z} = (a + ib) + (a - ib) = 2a = 2\Re z$.
        \item Per definizione di prodotto \[
            z\conj{z} = (a + ib)(a - ib) = a^2 - iab + iab - i^2b^2 = a^2 + b^2 = \abs{z}^2.\qedhere
        \] 
    \end{enumerate}
\end{proof}

La proposizione precedente ci consente di trovare l'inverso di qualunque numero non nullo in $\C$.

\begin{proposition}[Inverso tra i complessi]
    Sia $z \in \C, z \neq 0$. Allora \[\frac{1}{z} = \frac{\conj{z}}{\abs{z}^2}.\]
\end{proposition}
\begin{proof}
    Per la proposizione \ref{somma_prodotto_col_coniugato} segue che \[
        z\conj{z} = \abs{z}^2 \iff \frac{1}{z} = \frac{\conj{z}}{\abs{z}^2}. \qedhere   
    \]
\end{proof}

\begin{proposition}[I numeri complessi formano un campo]
    L'insieme $\C$ insieme alle operazioni di somma e prodotto con i rispettivi elementi neutri $0, 1 \in \C$ forma un campo.
\end{proposition}

\subsection{Rappresentazione polare dei numeri complessi}

Dato che possiamo considerare i numeri complessi come punti di un piano bidimensionale possiamo rappresentarli in forma polare, cioè considerando il vettore che congiunge l'origine degli assi con il punto $(a, b)$ che rappresenta il numero complesso $a + ib$. La forma polare di un numero complesso è data dalla coppia $(r, \theta)$, dove $r$ è il raggio del vettore e $\theta$ è l'angolo tra l'asse x e il vettore.

Dunque se $z = a+ib$ è un numero complesso in forma cartesiana, possiamo esprimerlo come $r(\cos\theta + i\sin\theta)$, dove $r = \sqrt{a^2 + b^2} = \abs{z}$ e $\theta = \arctan \frac{a}{b}$.

\begin{definition}[Esponenziale complesso]
    $e^{i\theta} = \cos\theta + i\sin\theta.$
\end{definition}

Sfruttando la definizione precedente possiamo scrivere ogni numero complesso nella forma $re^{i\theta}$ che è la forma polare del numero.

\begin{proposition}
    Siano $e^{i\alpha}, e^{i\beta} \in \C$. Allora vale \[
        e^{i\alpha} e^{i\beta} = e^{i(\alpha + \beta)}.
    \]
\end{proposition}
\begin{proof}
    Per definizione di esponenziale complesso:
    \begin{alignat*}{1}
        e^{i\alpha} e^{i\beta} &= (\cos\alpha + i\sin\alpha)(\cos\beta + i\sin\beta)\\
        &= (\cos\alpha \cos\beta - \sin\alpha \sin\beta) + i(\sin\alpha \cos\beta + \cos\alpha \sin\beta)\\
        &= \cos(\alpha + \beta) + i\sin(\alpha + \beta)\\
        &= e^{i(\alpha + \beta)}. \tag*{\qedhere}
    \end{alignat*}
\end{proof}

\section{Successioni per ricorrenza}

\begin{definition}[Successione]
    Si dice successione a valori in un insieme $A$ una funzione $(a_n) : \N \to A$.
\end{definition}

Solitamente analizzeremo successioni a valori reali, ovvero $(a_n) : \N \to \R$. Inoltre usiamo equivalentemente le notazioni $(a_n)_k$ o $a_k$ per riferirci alla funzione valutata nel punto $k \in \N$.

\begin{definition}[Somma di successioni e prodotto per una costante]
    Sia $S_{\R}$ l'insieme delle successioni a valori reali. Allora definisco una somma tra successioni $+ : S_{\R} \times S_{\R} \to S_{\R}$ tale che \[
        (a_n) + (b_n) = (a_n + b_n)  
    \] e un prodotto per una costante $\cdot : \R \times S_{\R} \to S_{\R}$ tale che \[
        k(a_n) = (ka_n).    
    \]
\end{definition}

\begin{example}
    Sia $a_n = 3^n$ e $b_n = 2n + 1$. Allora $(c_n) = (a_n) + (b_n)$ è la successione definita dalla legge $c_n = 3^n + 2n + 1$, mentre $(d_n) = 3(b_n)$ è la successione definita da $d_n = 6n + 3$.
\end{example}

Queste operazioni rispettano le solite proprietà (associativa, commutativa, distributiva). In particolare vale quindi la seguente proposizione.

\begin{proposition}[L'insieme delle successioni è uno spazio vettoriale]
    L'insieme delle successioni a valori reali $S_{\R}$ insieme alle operazioni di somma e prodotto per costanti e alla successione identicamente nulla $(0_n)$ è uno spazio vettoriale su $\R$.
\end{proposition}

\begin{definition}[Ricorrenza lineare omogenea]
    Si dice ricorrenza lineare omogenea di ordine $k$ un'equazione della forma \begin{equation} \label{ricorrenza}
        a_{n+k} = r_{k-1}a_{n+k-1} + r_{k-2}a_{n+k-2} + \dots + r_{1}a_{n+1} + r_0a_n. 
    \end{equation}
    Una soluzione della ricorrenza lineare \ref{ricorrenza} è una successione $(s_n)$ tale che per ogni $n \in \N$ vale che $s_n, s_{n+1}, \dots, s_{n+k}$ soddisfano la ricorrenza.
\end{definition}

\begin{proposition}
    Sia $A$ l'insieme delle successioni che soddisfano la ricorrenza lineare omogenea \[
        s_{n+k} = r_{k-1}s_{n+k-1} + r_{k-2}s_{n+k-2} + \dots + r_{1}s_{n+1} + r_0s_n.
    \] Allora $A$ è un sottospazio vettoriale di $S_{\R}$.
\end{proposition}
\begin{proof}
    Dobbiamo dimostrare tre fatti:
    \begin{enumerate}[(i)]
        \item $(0_n) \in A$;
        \item se $(a_n), (b_n) \in A$ allora $(c_n) = (a_n) + (b_n) \in A$;
        \item se $h \in \R$, $(a_n) \in A$ allora $(d_n) = h(a_n) \in A$.
    \end{enumerate}

    Sia $n \in \N$ qualsiasi.
    \begin{enumerate}[(i)]
        \item Verifichiamo che $(0_n)$ sia soluzione. La ricorrenza da verificare è \[
            0_{n+k} = r_{k-1}0_{n+k-1} + \dots + r_{1}0_{n+1} + r_00_n.
        \] Ma dato che $(0_n)$ è la successione identicamente nulla, allora questo equivale a dire $0 = 0r_{k-1} + \dots +  + 0r_0 = 0$, che è verificata e quindi $(0_n) \in A$.
        \item Verifichiamo che $(c_n)$ sia soluzione. \begin{align*}
            c_{n+k} &= a_{n+k} + b_{n+k}\\
            &= (r_{k-1}a_{n+k-1} + \dots + r_0a_n) + (r_{k-1}b_{n+k-1} + \dots + r_0b_n) \\
            &= r_{k-1}(a_{n+k-1} + b_{n+k-1}) + \dots + r_0(a_n + b_n) \\
            &= r_{k-1}c_{n+k-1} + \dots + r_0c_0
        \end{align*}
        dunque $(c_n) \in A$.
        \item Verifichiamo che $(d_n)$ sia soluzione. \begin{align*}
            d_{n+k} &= ha_{n+k}\\
            &= h(r_{k-1}a_{n+k-1} + \dots + r_0a_n)\\
            &= r_{k-1}(ha_{n+k-1}) + \dots + r_0(ha_n) \\
            &= r_{k-1}d_{n+k-1} + \dots + r_0d_0
        \end{align*}
        dunque $(d_n) \in A$. \qedhere
    \end{enumerate}
\end{proof}

La proposizione precedente ci permette di trovare una soluzione generale ad una ricorrenza lineare omogenea.

\begin{example}
    Siano $a_n = 3^n$ e $b_n = (-1)^n$ due soluzioni di una ricorrenza lineare omogenea. Allora per la proposizione precedente anche $k_1a_n = k_13^n$ e $k_2b_n = k_2(-1)^n$ saranno soluzioni (per ogni $k_1, k_2 \in \R$), e di conseguenza anche $k_1a_n + k_2b_n = k_13^n + k_2(-1)^n$.
\end{example}

Cerchiamo di risolvere una ricorrenza lineare omogenea.
\begin{example}
    Sia $a_{n+2} = 2a_{n+1} + 3a_n$ una ricorrenza lineare omogenea di ordine $2$. Trovare la soluzione generale. Inoltre trovare una soluzione particolare che soddisfi le condizioni iniziali $a_0 = 0$ e $a_1 = 1$.
\end{example}
\begin{solution}
    Proviamo a risolvere la ricorrenza con una soluzione esponenziale della forma $(\lambda^n)$ al variare di $n \in \N$. Sostituendo otteniamo \begin{alignat*}{1}
        &\lambda^{n+2} = 2\lambda^{n+1} + 3\lambda^{n} \\
        \iff &\lambda^2 = 2\lambda + 3 \\
        \iff &\lambda^2 - 2\lambda - 3.
    \end{alignat*}
    Dunque se $(\lambda^n)$ è una soluzione allora $\lambda$ deve essere radice di quel polinomio di secondo grado, detto polinomio caratteristico della ricorrenza.
    Risolvendolo segue che $\lambda_1 = 3$ e $\lambda_2 = -1$ sono soluzioni, dunque le successioni $(3^n)$ e $((-1)^n)$ sono soluzioni della ricorrenza.

    La soluzione generale della ricorrenza è dunque una successione della forma $(a_n) = k_1(3^n) + k_2((-1)^n)$ al variare di $k_1, k_2 \in \R$.

    Imponiamo ora che $a_0 = 0$ e $a_1 = 1$.
    \begin{equation*}
        \left\{
        \begin{array}{@{}roror }
        3^0k_1 & + & (-1)^0k_2 & = & 0 \\
        3^1k_1 & + & (-1)^1k_2 & = & 1 \\
        \end{array}
        \right. \iff \left\{
        \begin{array}{@{}ror }
        k_1 + k_2 & = & 0 \\
        3k_1 -k_2 & = & 1 \\
        \end{array}
        \right. 
    \end{equation*}
    da cui segue $k_1 = \frac14$, $k_2 = -\frac14$. La successione che soddisfa le condizioni iniziali è dunque $a_n = \frac14(3)^n - \frac14(-1)^n$.
\end{solution}

\begin{definition}
    [Polinomio caratteristico di una ricorrenza]
    Sia $a_{n+k} = r_{k-1}a_{n+k-1} +  \dots +  r_0a_n$ una ricorrenza lineare omogenea di ordine $k$. Allora si dice polinomio caratteristico associato alla ricorrenza il polinomio \[
        p(\lambda) = \lambda^k - r_{k-1}\lambda^{k-1} - \dots - r_0.    
    \]
\end{definition}

Il polinomio caratteristico si ottiene sostituendo alla ricorrenza lineare la successione $(\lambda^n)$, esattamente come abbiamo fatto nell'esempio precedente.

\begin{example}
    Consideriamo la successione di Fibonacci $f_{n+2} = f_{n+1} + f_n$ con $f_0 = 0$, $f_1 = 1$. Trovare una successione che risolva la ricorrenza e soddisfi i casi base.
\end{example}
\begin{solution}
    Il polinomio caratteristico di questa ricorrenza è \[
        p(\lambda) = \lambda^2 - \lambda - 1    
    \] che ha come radici i numeri $\varphi = \frac12(1 + \sqrt5)$ e $\bar{\varphi} = \frac12(1 - \sqrt5)$.

    La soluzione generale della ricorrenza è dunque una successione della forma $(f_n) = k_1(\varphi^n) + k_2(\bar{\varphi}^n)$ al variare di $k_1, k_2 \in \R$.

    Imponiamo ora che $f_0 = 0$ e $f_1 = 1$.
    \begin{equation*}
        \arraycolsep=1.2pt\def\arraystretch{1.3}
        \left\{
        \begin{array}{@{}roror }
        \varphi^0k_1 & + & \bar{\varphi}^0k_2 & = & 0\\
        \varphi^1k_1 & + & \bar{\varphi}^1k_2 & = & 1 \\
        \end{array}
        \right. \iff \left\{
        \begin{array}{@{}roror }
        k_1 & + & k_2 & = & 0\\
        \varphi k_1 & + & \bar{\varphi}k_2 & = & 1 \\
        \end{array}
        \right. 
    \end{equation*}
    da cui segue $k_1 = \frac{1}{\sqrt5}$, $k_2 = -\frac{1}{\sqrt5}$. La successione che soddisfa le condizioni iniziali è dunque \[
        f_n = \frac{1}{\sqrt5}\left(\frac{1 + \sqrt5}{2}\right)^n - \frac{1}{\sqrt5}\left(\frac{1 - \sqrt5}{2}\right)^n.
    \]
\end{solution}

Nel caso che una radice del polinomio caratteristico abbia una molteplicità maggiore di $1$ essa darà luogo a più di una soluzione della ricorrenza, come ci dice la seguente proposizione.
\begin{proposition}
    Sia $p(\lambda)$ il polinomio caratteristico di una ricorrenza lineare omogenea e sia $\lambda_0$ una radice di molteplicità $h$ (ovvero $h$ è il massimo intero per cui $(x - \lambda_0)^h$ compare nella fattorizzazione di $p(\lambda)$) con $h \leq 2$. 
    
    Allora $(\lambda_0^n), (n\lambda_0^n), \dots, (n^{h-1}\lambda_0^n)$ sono tutte soluzioni della ricorrenza lineare omogenea.
\end{proposition}

\begin{example}
    Sia $p(\lambda) = (\lambda - 3)^3(\lambda + 1)^2(\lambda - \sqrt2)^4$. Allora le seguenti sono tutte soluzioni indipendenti della ricorrenza lineare omogenea associata a $p(\lambda)$:
    \begin{multicols}{3}
        \begin{enumerate}[(i)]
        \item $(3^n)$;
        \item $(n3^n)$;
        \item $(n^23^n)$;
        \item $((-1)^n)$;
        \item $(n(-1)^n)$;
        \item $(\sqrt{2}^n)$;
        \item $(n\sqrt{2}^n)$;
        \item $(n^2\sqrt{2}^n)$;
        \item $(n^3\sqrt{2}^n)$.
    \end{enumerate}
    \end{multicols}
    
    La soluzione generale sarà dunque della forma \begin{align*}
        (a_n) = 
            &\ k_1(3^n) + k_2(n3^n) + k_3(n^23^n) + k_4(n(-1)^n) + k_5(n(-1)^n) + \\
            + &\ k_6(\sqrt{2}^n) + k_7(n\sqrt{2}^n) + k_8(n^2\sqrt{2}^n) + k_9(n^3\sqrt{2}^n)
    \end{align*}
    al variare di $k_1, \dots, k_9 \in \R$.
\end{example}
\chapter{Spazi vettoriali}

\section{Spazi vettoriali}
\begin{definition}
    Si dice \textbf{spazio vettoriale su un campo $\K$} un insieme $V$ di elementi, detti \textbf{vettori}, insieme con due operazioni $+ : V \times V \to V$ e $\cdot : \K \times V \to V$ e un elemento $\bm{0_V} \in V$ che soddisfano i seguenti assiomi:
    \[\forall \bm{v}, \bm{w}, \bm{u} \in V, \quad\forall h, k \in \K \]
    \begin{align}
        &\text{1.} &&(\bm{v} + \bm{w}) \in V                                        &\text{(chiusura di V rispetto a $+$)} \\      
        &\text{2.} &&\bm{v} + \bm{w} = \bm{w} + \bm{v}                              &\text{(commutativita' di $+$)} \\
        &\text{3.} &&(\bm{v} + \bm{w}) + \bm{u} = \bm{w} + (\bm{v} + \bm{u})        &\text{(associativita' di $+$)} \\
        &\text{4.} &&\bm{0_V} + \bm{v} = \bm{v} + \bm{0_V} = \bm{v}                 &\text{($\bm{0_V}$ el. neutro di $+$)} \\
        &\text{5.} &&\exists (-\bm{v}) \in V. \quad\bm{v} + (\bm{-v}) = \bm{0_V}    &\text{(opposto per $+$)} \\
        &\text{6.} &&k\bm{v} \in V                                                  &\text{(chiusura di V rispetto a $\cdot$)} \\
        &\text{7.} &&k(\bm{v} + \bm{w}) = k\bm{v} + k\bm{w}                         &\text{(distributivita' 1)} \\
        &\text{8.} &&(k + h)\bm{v}= k\bm{v} + h\bm{v}                               &\text{(distributivita' 2)} \\
        &\text{9.} &&(kh)\bm{v}= k(h\bm{v})                                         &\text{(associativita' di $\cdot$)} \\
        &\text{10.}&&1\bm{v}= \bm{v}                                                &\text{(1 el. neutro di $\cdot$)}
    \end{align}
\end{definition}
 
Spesso il campo $\K$ su cui e' definito uno spazio vettoriale $V$ e' il campo dei numeri reali $\R$ o il campo dei numeri complessi $\C$. Supporremo che gli spazi vettoriali siano definiti su $\R$ a meno di diverse indicazioni. Le definizioni valgono comunque in generale anche su campi $\K$ diversi da $\R$ o $\C$.

\begin{example}
    Possiamo fare diversi esempi di spazi vettoriali. Ad esempio sono spazi vettoriali:
    \begin{enumerate}
        \item i vettori geometrici dove:
        \begin{itemize}
            \item l'elemento neutro e' il vettore nullo;
            \item la somma e' definita tramite la regola del parallelogramma;
            \item il prodotto per scalare e' definito nel modo usuale;
        \end{itemize}
        \item i vettori colonna $n \times 1$ o i vettori riga $1 \times n$ dove:
        \begin{itemize}
            \item l'elemento neutro e' il vettore composto da $n$ elementi $0$;
            \item la somma e' definita come somma tra componenti;
            \item il prodotto per scalare e' definito come prodotto tra lo scalare e ciascuna componente;
        \end{itemize}
        \item le matrici $n \times m$, indicate con $\M_{n \times m}(\K)$;
        \item i polinomi di grado minore o uguale a $n$, indicati con $\K[x]^{\leq n}$;
        \item tutti i polinomi, indicati con $\K[x]$.
    \end{enumerate}
\end{example}

\begin{definition}
    Sia $V$ uno spazio vettoriale, $A \subseteq V$. Allora si dice che $A$ e' un sottospazio vettoriale di $V$ (o semplicemente sottospazio) se
    \begin{align}
        &\bm{0_V} \in A \\
        &(\bm{v} + \bm{w}) \in A    &&\forall \bm{v}, \bm{w} \in A \\
        &(k\bm{v}) \in A            &&\forall k \in \R, \bm{v} \in A
    \end{align}
\end{definition}

\begin{proposition}
    Le soluzioni di un sistema omogeneo $A\bm{x} = \bm{0}$ con $n$ variabili formano un sottospazio di $\R^n$.
\end{proposition}
\begin{proof}
    Chiamiamo $S$ l'insieme delle soluzioni. Dato che le soluzioni sono vettori colonna di $n$ elementi, $S \subseteq \R^n$. Verifichiamo ora le condizioni per cui $S$ e' un sottospazio di $\R^n$:
    \begin{enumerate}
        \item $\bm{0}$ appartiene a $S$, poiche' $A\bm{0} = \bm{0}$;
        \item Se $\bm{x}, \bm{y}$ appartengono ad $S$, allora $A(\bm{x} + \bm{y}) = A\bm{x} + A\bm{y} = \bm{0} + \bm{0} = \bm{0}$, dunque $\bm{x} + \bm{y} \in S$;
        \item Se $\bm{x}$ appartiene ad $S$, allora $A(k\bm{x}) = kA\bm{x} = k\bm{0} = \bm{0}$, dunque $k\bm{x} \in S$.
    \end{enumerate}
    Dunque $S$ e' un sottospazio di $\R^n$.
\end{proof}

\section{Combinazioni lineari e span}
\begin{definition}
    Sia $V$ uno spazio vettoriale e $\bm{v_1}, \bm{v_2}, \dots, \bm{v_n} \in V$. Allora il vettore $\bm{v} \in V$ si dice combinazione lineare di $\bm{v_1}, \bm{v_2}, \dots, \bm{v_n}$ se 
    \begin{equation}
        \bm{v}= a_1\bm{v_1} + a_2\bm{v_2} + \dots + a_n\bm{v_n} 
    \end{equation}
    per qualche $a_1, a_2, \dots, a_n \in \R$.
\end{definition}

\begin{definition}
    Sia $V$ uno spazio vettoriale e $\bm{v_1}, \dots, \bm{v_n} \in V$. Si indica con $\Span{\bm{v_1}, \dots, \bm{v_n}}$ l'insieme dei vettori che si possono ottenere come combinazione lineare di $\bm{v_1}, \dots, \bm{v_n}$:
    \begin{equation}
        \Span{\bm{v_1}, \dots, \bm{v_n}} = \left\{a_1\bm{v_1} + \dots + a_n\bm{v_n} \mid a_1, \dots, a_n \in \R\right\}
    \end{equation}
\end{definition}

\begin{proposition}
    Sia $A \in \M_{n \times m}(\R)$ e siano $\bm{a_1}, \bm{a_2}, \dots, \bm{a_m} \in \R^n$ le sue colonne. Allora l'immagine della matrice e' uguale allo span delle sue colonne.
\end{proposition}
\begin{proof}
    L'immagine della matrice e' l'insieme di tutti i vettori del tipo $A \cdot \begin{pmatrix}
        x_1 & \dots & x_m
    \end{pmatrix}^T$ al variare di $x_1, \dots, x_m \in \R$. 
    \begin{alignat*}
        {1}
        A\begin{pmatrix} x_1 \\ \vdots \\ x_m \end{pmatrix}
            &= A\left(\begin{pmatrix} x_1 \\ \vdots \\ 0 \end{pmatrix} + \dots + \begin{pmatrix} 0 \\ \vdots \\ x_m \end{pmatrix}\right)\\
            &= A\left(x_1\begin{pmatrix} 1 \\ \vdots \\ 0 \end{pmatrix} + \dots + x_m\begin{pmatrix} 0 \\ \vdots \\ 1 \end{pmatrix}\right)\\
            &= x_1A\begin{pmatrix} 1 \\ \vdots \\ 0 \end{pmatrix} + \dots + x_mA\begin{pmatrix} 0 \\ \vdots \\ 1 \end{pmatrix}\\
        \intertext{Ma sappiamo per la proposizione \ref{j-esima_colonna} che moltiplicare una matrice per un vettore che contiene tutti $0$ tranne un $1$ in posizione $j$ ci da' come risultato la $j$-esima colonna della matrice, dunque:}
            &= x_1\bm{a_1} + \dots + x_m\bm{a_m} 
    \end{alignat*}
    Ma i vettori che appartengono allo span delle colonne di $A$ sono tutti e solo del tipo $x_1\bm{a_1} + \dots + x_m\bm{a_m}$, dunque $\Imm{A} = \Span{\bm{a_1}, \bm{a_2}, \dots, \bm{a_m}}$, come volevasi dimostrare.
\end{proof}

\begin{proposition}
    Sia $V$ uno spazio vettoriale, $\bm{v_1}, \dots, \bm{v_n} \in V$. Allora $A = \Span{v_1, \dots, v_n} \subseteq V$ e' un sottospazio di $V$.
\end{proposition}
\begin{proof}
    Dimostriamo che valgono le tre condizioni per cui $A$ e' un sottospazio di $V$:
    \begin{enumerate}
        \item $\bm{0_V}$ appartiene ad $A$, in quanto basta scegliere $a_1 = \dots = a_n = 0$;
        \item Siano $\bm{v}, \bm{w} \in A$. Allora per qualche $a_1, \dots, a_n, b_1, \dots, b_n \in \R$ vale che \begin{alignat*}{1}
            \bm{v} + \bm{w} &= (a_1\bm{v_1} + \dots + a_n\bm{v_n}) + (b_1\bm{v_1} + \dots + b_n\bm{v_n}) \\
            &= (a_1 + b_1)\bm{v_1} + \dots + (a_n + b_n)\bm{v_n} \in A
        \end{alignat*}
        \item Siano $\bm{v} \in A, k \in \R$. Allora per qualche $a_1, \dots, a_n \in \R$ vale che \begin{alignat*}{1}
            k\bm{v} &= k(a_1\bm{v_1} + \dots + a_n\bm{v_n})  \\
            &= (ka_1)\bm{v_1} + \dots + (ka_n)\bm{v_n} \in A
        \end{alignat*}
    \end{enumerate}
    cioe' $A$ e' un sottospazio di $V$.
\end{proof}

Vale anche l'implicazione inversa: ogni sottospazio di $V$ puo' essere descritto come span di alcuni suoi vettori.

\subsubsection{Forma parametrica e cartesiana}

\begin{proposition}
    Ogni sottospazio vettoriale di $R^n$ puo' essere descritto in due forme:
    \begin{itemize}
        \item forma parametrica: come span di alcuni vettori, cioe' come immagine di una matrice;
        \item forma cartesiana: come insieme delle soluzioni di un sistema lineare omogeneo, cioe' come kernel di una matrice.
    \end{itemize}
\end{proposition}
\begin{remark}
    Per essere piu' precisi dovremmo parlare di immagine e di kernel dell'applicazione lineare associata alla matrice. 
\end{remark}
\begin{example}
    Consideriamo il sottospazio di $R^3$ generato dall'insieme delle soluzioni dell'equazione $3x + 4y + 5z = 0$ (forma cartesiana) e chiamiamolo $W$.
    Cerchiamo di esprimere $W$ in forma parametrica: \begin{alignat*}{1}
        W &= \left\{ \begin{pmatrix} x \\ y \\ z \end{pmatrix} \in \R^3 \mid 3x + 4y + 5z = 0\right\} \\
        \intertext{Scegliamo $y, z$ libere, da cui segue $x = -\frac{4}{3}y -\frac{5}{3}z$. Sostituendolo otteniamo: }
        &= \left\{ \begin{pmatrix} -\frac{4}{3}y -\frac{5}{3}z \\ y \\ z \end{pmatrix} \mid y, z \in \R \right\}\\
        &= \left\{ y\begin{pmatrix} -\frac{4}{3} \\ 1 \\ 0 \end{pmatrix} + z\begin{pmatrix} -\frac{5}{3} \\ 0 \\ 1 \end{pmatrix} \mid y, z \in \R \right\}\\
        &= \left\{ y\begin{pmatrix} -\frac{4}{3} \\ 1 \\ 0 \end{pmatrix} + z\begin{pmatrix} -\frac{5}{3} \\ 0 \\ 1 \end{pmatrix} \mid y, z \in \R \right\}\\
        &= \Span{\begin{pmatrix} -\frac{4}{3} \\ 1 \\ 0 \end{pmatrix}; \begin{pmatrix} -\frac{5}{3} \\ 0 \\ 1 \end{pmatrix}}
    \end{alignat*}

    Se torniamo indietro notiamo che \begin{alignat*}{1}
        W &= \left\{ \begin{pmatrix} -\frac{4}{3}y -\frac{5}{3}z \\ y \\ z \end{pmatrix} \mid y, z \in \R \right\}\\
        &= \left\{ \begin{pmatrix} -\frac{4}{3}y -\frac{5}{3}z \\ y + 0z \\ 0y+z \end{pmatrix} \mid y, z \in \R \right\}\\
        &= \left\{ \begin{pmatrix} -\frac{4}{3} & -\frac{5}{3} \\ 1 & 0 \\ 0 & 1 \end{pmatrix}\begin{pmatrix} y \\ z \end{pmatrix} \mid y, z \in \R \right\}
    \end{alignat*}
    che e' la definizione di immagine della matrice $\begin{psmallmatrix}
        -\frac{4}{3} & -\frac{5}{3} \\ 1 & 0 \\ 0 & 1 
    \end{psmallmatrix}$.

    Dunque $W = \Imm{\begin{psmallmatrix}
        -\frac{4}{3} & -\frac{5}{3} \\ 1 & 0 \\ 0 & 1 
    \end{psmallmatrix}}$.
\end{example}
\begin{example}
    Consideriamo il sottospazio di $R^3$ generato dallo span dei vettori $\begin{psmallmatrix} 1 \\ 2 \\ 3 \end{psmallmatrix}$, $\begin{psmallmatrix} 4 \\ 5 \\ 6 \end{psmallmatrix}$ (forma parametrica) e chiamiamolo $W$.
    Cerchiamo di esprimere $W$ in forma cartesiana: \begin{alignat*}{1}
        W &= \Span{\begin{pmatrix} 1 \\ 2 \\ 3 \end{pmatrix}, \begin{pmatrix} 4 \\ 5 \\ 6 \end{pmatrix}} \\
            &= \left\{ \begin{pmatrix} x \\ y \\ z \end{pmatrix} \in \R^3 \mid \exists a, b \in \R. \begin{pmatrix} x \\ y \\ z \end{pmatrix} = a\begin{pmatrix} 1 \\ 2 \\ 3 \end{pmatrix} + b\begin{pmatrix} 4 \\ 5 \\ 6 \end{pmatrix} \right\}\\
            &= \left\{ \begin{pmatrix} x \\ y \\ z \end{pmatrix} \in \R^3 \mid \exists a, b \in \R. \begin{pmatrix} x \\ y \\ z \end{pmatrix} = \begin{pmatrix} a + 4b \\ 2a+5b \\ 3a+6b \end{pmatrix}\right\}\\
            &= \left\{ \begin{pmatrix} x \\ y \\ z \end{pmatrix} \in \R^3 \mid \exists a, b \in \R. \begin{pmatrix} x \\ y \\ z \end{pmatrix} = \begin{pmatrix} 1 & 4 \\ 2&5 \\ 3&6 \end{pmatrix} \begin{pmatrix}a \\ b\end{pmatrix}\right\}
    \end{alignat*}
    dunque e' sufficiente capire in che casi il sistema ha soluzione.
    Risolviamo il sistema e imponiamo che non ci siano equazioni impossibili:
    \begin{gather*}
        \begin{pmatrix}[cc|c]
            1&4&x \\ 2&5&y \\ 3&6&z 
        \end{pmatrix} \xrightarrow[R_2 - 2R_1]{R_3 - 3R_1}
        \begin{pmatrix}[cc|c]
            1&4&x \\ 0&-3&y-2x \\ 0&-6&z-3x 
        \end{pmatrix} \xrightarrow[R_3 - 2R_2]{}
        \begin{pmatrix}[cc|c]
            1&4&x \\ 0&-3&y-2x \\ 0&0&x-2y+z 
        \end{pmatrix}
    \end{gather*}
    Dato che non devono esserci equazioni impossibili, segue che tutti i vettori di $W$ sono della forma $\begin{psmallmatrix}x\\y\\z\end{psmallmatrix}$ con $x - 2y + z = 0$. Dunque \[
        W = \left\{ \begin{pmatrix}
            x\\y\\z
        \end{pmatrix}\in \R^3 \mid x-2y+z = 0\right\}    
    \] e' la forma cartesiana di $W$.

    Notiamo che dire che $x, y, z \in \R$ sono tali che $x-2y+z = 0$ e' equivalente a dire che \[
        \begin{pmatrix}
            1 &-2 &1
        \end{pmatrix} \cdot \begin{pmatrix}
            x \\ y \\ z
        \end{pmatrix} = 0
    \]
    cioe' $W$ e' formato da tutti e solo i vettori che fanno parte del kernel della matrice $A = \begin{pmatrix} 1 &-2 &1 \end{pmatrix}$, cioe' $W = \ker \begin{pmatrix} 1 &-2 &1 \end{pmatrix}$.
\end{example}

\subsubsection{Indipendenza e dipendenza lineare}

\begin{definition}
    Sia $V$ uno spazio vettoriale, $\bm{v_1}, \dots, \bm{v_n} \in V$. Allora l'insieme $\left\{ \bm{v_1}, \dots, \bm{v_n} \right\}$ si dice insieme di vettori linearmente indipendenti se
    \begin{equation}
        a_1\bm{v_1} + \dots + a_n\bm{v_n} = \bm{0_V} \iff a_1 = \dots = a_n = 0
    \end{equation}
    cioe' se l'unica combinazione lineare di $\bm{v_1}, \dots, \bm{v_n}$ che da' come risultato il vettore nullo e' quella con $a_1 = \dots = a_n = 0$.
\end{definition}

Possiamo usare una definizione alternativa di dipendenza lineare, equivalente alla precedente, tramite questa proposizione:
\begin{proposition}\label{dip_se_e'_comb_lin}
    Sia $V$ uno spazio vettoriale, $\bm{v_1}, \dots, \bm{v_n} \in V$. Allora l'insieme dei vettori $\left\{ \bm{v_1}, \dots, \bm{v_n} \right\}$ e' linearmente dipendente se e solo se almeno uno di essi e' esprimibile come combinazione lineare degli altri. 
\end{proposition}
\begin{proof}
    Dimostriamo entrambi i versi dell'implicazione.
    \begin{itemize}
        \item Supponiamo che $\left\{ \bm{v_1}, \dots, \bm{v_n} \right\}$ sia linearemente dipendente, cioe' che esistano $a_1, \dots, a_n$ non tutti nulli tali che \[
            a_1\bm{v_1} + a_2\bm{v_2} + \dots + a_n\bm{v_n} = \bm{0_V}   
        .\]
        Supponiamo senza perdita di generalita' $a_1 \neq 0$, allora segue che \[
            \bm{v_1} = -\frac{a_2}{a_1}\bm{v_1} - \dots - \frac{a_n}{a_1}\bm{v_n}
        \]
        dunque $\bm{v_1}$ puo' essere espresso come combinazione lineare degli altri vettori.
        \item Supponiamo che il vettore $\bm{v_1}$ sia esprimibile come combinazione lineare degli altri (senza perdita di generalita'), cioe' che esistano $k_2, \dots, k_n \in \R$ tali che \[
            \bm{v_1} = k_2\bm{v_2} + \dots + k_n\bm{v_n}
        .\]
        Consideriamo una generica combinazione lineare di $v_1, v_2, \dots, v_n$:
        \begin{alignat*}
            {1}
            & a_1\bm{v_1} + a_2\bm{v_2} + \dots + a_n\bm{v_n} \\
            = & a_1(k_2\bm{v_2} + \dots + k_n\bm{v_n}) + a_2\bm{v_2} + \dots + a_n\bm{v_n} \\
            = & (a_1k_2 + a_2)\bm{v_2} + \dots + (a_1k_n + a_n)\bm{v_n}
        \end{alignat*}
        Se scegliamo $a_1 \in \R$ libero, $a_i = -a_1k_i$ per ogni $2 \leq i \leq n$, otterremo
        \begin{alignat*}{1}
            & (a_1k_2 + a_2)\bm{v_2} + \dots + (a_1k_n + a_n)\bm{v_n} \\
            = & (a_1k_2 - a_1k_2)\bm{v_2} + \dots + (a_1k_n - a_1k_n)\bm{v_n} \\
            = & 0\bm{v_2} + \dots + 0\bm{v_n} \\
            = & \bm{0_V}
        \end{alignat*}
        dunque esiste una scelta dei coefficienti $a_1, a_2, \dots, a_n$ diversa da $a_1 = \dots = a_n = 0$ per cui la combinazione lineare da' come risultato il vettore nullo, cioe' l'insieme dei vettori non e' linearmente indipendente. \qedhere
    \end{itemize}
\end{proof}

Inoltre per comodita' spesso si dice che i vettori $\bm{v_1}, \dots, \bm{v_n}$ sono indipendenti, invece di dire che l'insieme formato da quei vettori e' un insieme linearmente indipendente.

\begin{proposition}\label{aggiunto_vettore_indipendente}
    Sia $V$ uno spazio vettoriale, $\bm v \in V$ e $\bm{v_1}, \dots, \bm{v_n} \in V$ indipendenti. Allora i due fatti seguenti sono equivalenti:
    \begin{enumerate}
        \item $\bm v \notin \Span{\bm{v_1}, \dots, \bm{v_n}}$;
        \item $\bm{v_1}, \dots, \bm{v_n}, \bm v$ e' ancora un insieme di vettori linearmente indipendenti.
    \end{enumerate}
\end{proposition}
\begin{proof}
    Dimostriamo entrambi i versi dell'implicazione.
    \begin{itemize}
        \item[($\implies$)] Supponiamo che $\bm v \notin \Span{\bm{v_1}, \dots, \bm{v_n}}$.
        
        Se $\bm{v_1}, \dots, \bm{v_n}, \bm v$ e' un insieme di vettori linearmente indipendenti per definizione l'unica combinazione lineare $a_1\bm{v_1} + \dots + a_n\bm{v_n} + b\bm{v}$ che da' come risultato il vettore nullo $\bm{0}$ deve essere quella con coefficienti tutti nulli.

        Supponiamo per assurdo $b \neq 0$,. Allora \begin{alignat*}{1}
            &\bm 0 = a_1\bm{v_1} + \dots + a_n\bm{v_n} + b\bm{v}\\
            \iff &-b\bm{v} = a_1\bm{v_1} + \dots + a_n\bm{v_n}\\
            \iff &\bm{v} = -\frac{a_1}{b}\bm{v_1} - \dots - \frac{a_n}{b}\bm{v_n}
        \end{alignat*}
        cioe' $v \in \Span{\bm{v_1}, \dots, \bm{v_n}}$, che pero' e' assurdo perche' per ipotesi $\bm v \notin \Span{\bm{v_1}, \dots, \bm{v_n}}$.

        Dunque $b = 0$, cioe' \begin{alignat*}{1}
            &\bm 0 = a_1\bm{v_1} + \dots + a_n\bm{v_n} + b\bm{v}\\
            \iff &\bm 0 = a_1\bm{v_1} + \dots + a_n\bm{v_n}
            \intertext{Tuttavia $\bm{v_1}, \dots, \bm{v_n}$ sono linearmente indipendenti, dunque l'unica scelta dei coefficienti che annulla la combinazione lineare e' quella con tutti i coefficienti nulli:}
            \iff &a_1 = \dots = a_n = b = 0
        \end{alignat*}
        cioe' $\bm{v_1}, \dots, \bm{v_n}, \bm v$ e' ancora un insieme di vettori linearmente indipendenti.
        \item[($\impliedby$)] Supponiamo che $\bm{v_1}, \dots, \bm{v_n}, \bm v$ sia un insieme di vettori linearmente indipendenti. 
        
        Per la proposizione \ref{dip_se_e'_comb_lin} sappiamo che un insieme di vettori e' linearmente dipendente se e solo se almeno uno di essi puo' essere scritto come combinazione lineare degli altri, cioe' se e solo se almeno uno di essi e' nello span degli altri.
        Ma questo e' equivalente a dire che un insieme di vettori e' linearmente indipendente se e solo se nessuno di essi e' nello span degli altri, dunque dato che $\bm{v_1}, \dots, \bm{v_n}, \bm v$ e' un insieme di vettori linearmente indipendenti segue che $\bm{v}$ non puo' appartenere a $\Span{\bm{v_1}, \dots, \bm{v_n}}$. \qedhere
    \end{itemize}
\end{proof}

\begin{proposition} \label{span_Gauss}
    Sia $V$ uno spazio vettoriale e $\bm{v_1}, \dots, \bm{v_n} \in V$. Allora per ogni $k \in \R$ e per ogni $i, j \leq n$.
    \begin{equation}
        \Span{\bm{v_1}, \dots, \bm{v_i}, \bm{v_j}, \dots, \bm{v_n}} = \Span{\bm{v_1}, \dots, \bm{v_i} + k\bm{v_j}, \bm{v_j}, \dots, \bm{v_n}}.
    \end{equation}
\end{proposition}
\begin{proof}
    Supponiamo che $v \in \Span{\bm{v_1}, \dots, \bm{v_i}, \bm{v_j}, \dots, \bm{v_n}}$. Allora per definizione esisteranno $a_1, \dots, a_n \in \R$ tali che
    \begin{alignat*}{1}
        v &= a_1\bm{v_1} + \dots + a_i\bm{v_i} + a_j\bm{v_j} + \dots + a_n\bm{v_n} \\
        \intertext{Aggiungiamo e sottraiamo $a_ik\bm{v_j}$ al secondo membro.}
        &= a_1\bm{v_1} + \dots + a_i\bm{v_i} + a_j\bm{v_j} + \dots + a_n\bm{v_n} + a_ik\bm{v_j} - a_ik\bm{v_j}\\
        &= a_1\bm{v_1} + \dots + a_i\bm{v_i} + a_ik\bm{v_j} + a_j\bm{v_j} - a_ik\bm{v_j} + \dots + a_n\bm{v_n}\\
        &= a_1\bm{v_1} + \dots + a_i(\bm{v_i} + k\bm{v_j}) + (a_j - a_ik)\bm{v_j} + \dots + a_n\bm{v_n}\\
        \implies v &\in \Span{\bm{v_1}, \dots, \bm{v_i} + k\bm{v_j}, \bm{v_j}, \dots, \bm{v_n}}. 
    \end{alignat*}

    Si dimostra l'altro verso nello stesso modo.

    Dunque in entrambi gli insiemi ci sono gli stessi elementi, cioe' i due span sono uguali.
\end{proof}

Notiamo inoltre che se scambiamo due vettori o se moltiplichiamo un vettore per uno scalare otteniamo uno span equivalente a quello di partenza. Quindi possiamo "semplificare" uno span di vettori tramite mosse di Gauss per colonna, come suggerisce la prossima proposizione.

\begin{proposition} \label{span_colonne_indipendenti}
    Siano $\bm{v_1}, \dots, \bm{v_n} \in \R^m$ dei vettori colonna. Allora per stabilire quali di questi vettori sono indipendenti consideriamo la matrice $A$ che contiene come colonna $i$-esima il vettore colonna $v_i$ e riduciamola a scalini per colonna. Lo span delle colonne non nulle della matrice ridotta a scalini e' uguale allo span di $\bm{v_1}, \dots, \bm{v_n}$.
\end{proposition}
\begin{proof} 
    Consideriamo la matrice $\bar{A}$ ridotta a scalini. Allora per la proposizione \ref{span_Gauss} lo span delle sue colonne e' uguale allo span dei vettori iniziali. 

    Tutte le colonne nulle possono essere eliminate da questo insieme, in quanto il vettore nullo e' sempre linearmente dipendente.

    Le colonne rimanenti sono sicuramente linearmente indipendenti: infatti dato che la matrice e' a scalini per colonna per annullare il primo pivot dobbiamo annullare il primo vettore, per annullare il secondo dobbiamo annullare il secondo e cosi' via. Dunque lo span dei vettori colonna non nulli rimanenti e' uguale allo span dei vettori iniziali.
\end{proof}

Notiamo che alla fine di questo procedimento otteniamo vettori colonna che sono diversi dai vettori iniziali, ma questi vettori hanno pivot ad "altezze diverse".

\begin{example}
    Siano $\bm{v_1}, \bm{v_2}, \bm{v_3}, \bm{v_4} \in \R^3$ tali che \[
        \bm{v_1} = \begin{pmatrix}
            1 \\ 2 \\ 3
        \end{pmatrix}, \bm{v_2} = \begin{pmatrix}
            3 \\ 7 \\ 4
        \end{pmatrix}, \bm{v_3} = \begin{pmatrix}
            2 \\ 4 \\ 6
        \end{pmatrix}, \bm{v_4} = \begin{pmatrix}
            -1 \\ 7 \\ 2
        \end{pmatrix}.
    \] Si trovi un insieme di vettori di $\R^3$ indipendenti con lo stesso span di $\bm{v_1}, \bm{v_2}, \bm{v_3}, \bm{v_4}$.
\end{example}
\begin{solution}
    Per la proposizione precedente mettiamo i vettori come colonne di una matrice e semplifichiamola tramite mosse di colonna:
    \begin{gather*}
        \begin{pmatrix}[c|c|c|c]
            1 & 3 & 2 & -1 \\ 2 & 7 & 4 & 7 \\ 3 & 4 & 6 & 2
        \end{pmatrix} \xrightarrow[C_4 + C_1]{C_2 - 3C_1, C_3 - 2C_1} \begin{pmatrix}
            [c|c|c|c]
            1 & 0 & 0 & 0 \\ 2 & 1 & -2 & 1 \\ 3 & -5 & -3 & -7
        \end{pmatrix} \\ 
        \xrightarrow[C_4 - C_2]{C_3 + 2C_2} \begin{pmatrix}
            [c|c|c|c]
            1 & 0 & 0 & 0 \\ 2 & 1 & 0 & 0 \\ 3 & -5 & -13 & -2
        \end{pmatrix} \xrightarrow[]{C_4 - \frac{2}{13}C_3} \begin{pmatrix}
            [c|c|c|c]
            1 & 0 & 0 & 0 \\ 2 & 1 & 0 & 0 \\ 3 & -5 & -13 & 0
        \end{pmatrix}
    \end{gather*}
    Dunque i vettori $\bm{w_1} = \begin{psmallmatrix} 1 \\ 2 \\ 3 \end{psmallmatrix}, \bm{w_2} = \begin{psmallmatrix} 0 \\ 1 \\ -5 \end{psmallmatrix}, \bm{w_3} = \begin{psmallmatrix} 0 \\ 0 \\ -13 \end{psmallmatrix}$ sono indipendenti e per la proposizione precedente vale che \[
        \Span{\bm{v_1}, \bm{v_2}, \bm{v_3}, \bm{v_4}} = \Span{\bm{w_1}, \bm{w_2}, \bm{w_3}}.
    \]
\end{solution}

\section{Generatori e basi}
\begin{definition}
    Sia $V$ uno spazio vettoriale, $\bm{v_1}, \dots, \bm{v_n} \in V$. Allora si dice che ${\bm{v_1}, \dots, \bm{v_n}}$ e' un insieme di generatori di $V$, oppure che l'insieme ${\bm{v_1}, \dots, \bm{v_n}}$ genera $V$, se
    \begin{equation}
        \Span{\bm{v_1}, \dots, \bm{v_n}} = V.
    \end{equation}
\end{definition}

Per comodita' spesso si dice che i vettori $\bm{v_1}, \dots, \bm{v_n}$ sono generatori di $V$, invece di dire che l'insieme formato da quei vettori e' un insieme di generatori.

\begin{definition}
    Sia $V$ uno spazio vettoriale, $\bm{v_1}, \dots, \bm{v_n} \in V$. Allora si dice che $\mathcal{B} = \ang{\bm{v_1}, \dots, \bm{v_n}}$ e' una base di $V$ se
    \begin{itemize}
        \item i vettori $\bm{v_1}, \dots, \bm{v_n}$ generano $V$;
        \item i vettori $\bm{v_1}, \dots, \bm{v_n}$ sono linearmente indipendenti.
    \end{itemize}
\end{definition}

\begin{definition}
    Sia $V$ uno spazio vettoriale. Allora il numero di vettori in una sua base si dice dimensione dello spazio vettoriale $V$, e si indica con $\dim V$.
\end{definition}

Sapendo che un insieme di vettori genera un sottospazio di $\R^n$ (o $\R^n$ stesso) si puo' trovare una base del sottospazio (o di $\R^n$) disponendo i vettori come colonne di una matrice e semplificandoli, come abbiamo visto in precedenza. Tuttavia se vogliamo \textbf{estrarre una base} dal nostro insieme di vettori allora possiamo procedere in un modo leggermente diverso, che utilizza le mosse di Gauss per riga.

\begin{proposition}\label{estrarre_una_base}
    Siano $\bm{v_1}, \dots, \bm{v_m} \in \R^n$ dei vettori che generano $V \subseteq \R^n$ sottospazio di $\R^n$. Allora possiamo porre i vettori come colonne di una matrice e ridurla a scalini per riga. Alla fine del procedimento i vettori che originariamente erano nelle colonne con i pivot sono indipendenti e generano $V$, dunque formano una base di $V$.
\end{proposition}
\begin{proof}
    Consideriamo i $k$ vettori indipendenti che sono nell'insieme $\bm{v_1}, \dots, \bm{v_m}$ e chiamiamoli $\bm{w_1}, \dots, \bm{w_k}$.
    Consideriamo una loro combinazione lineare qualunque $x_1\bm{w_1} + \dots + x_k\bm{w_k}$ e la poniamo uguale a $\bm{0}$; questo e' equivalente a dire \[
        A\begin{pmatrix}
            x_1 \\ \vdots \\ x_k
        \end{pmatrix} = \bm{0}
    \] dove $A$ e' la matrice le cui colonne sono i vettori $\bm{w_1}, \dots, \bm{w_k}$. 
    
    Dato che i $k$ vettori sono indipendenti l'unica soluzione di questo sistema e' il vettore nullo, dunque il sistema ha una sola soluzione e quindi deve avere $0$ variabili libere, cioe' il numero di pivot della matrice ridotta a scalini deve essere uguale al numero di colonne.

    Se aggiungessimo vettori non indipendenti a questo insieme per definizione di dipendenza lineare allora non avremmo piu' una singola soluzione, dunque le colonne che abbiamo aggiunto non possono contenere pivot.
\end{proof}

Notiamo che alla fine del procedimento non otteniamo dei vettori colonna che generano il nostro sottospazio, ma dobbiamo andarli a scegliere dall'insieme iniziale: in questo senso possiamo estrarre una base da un insieme di generatori.

\begin{example}
    Sia $V \subseteq \R^4$ tale che $V = \Span{\bm{c_1}, \bm{c_2}, \bm{c_3}, \bm{c_4}}$ dove \[
        \bm{c_1} = \begin{pmatrix}
            2 \\ 0 \\ 1 \\ 1
        \end{pmatrix}, \bm{c_2} = \begin{pmatrix}
            3 \\ -2 \\ -2 \\ 0
        \end{pmatrix}, \bm{c_3} = \begin{pmatrix}
            1 \\ 0 \\ -1 \\ 1
        \end{pmatrix}, \bm{c_4} = \begin{pmatrix}
            0 \\ 1 \\ -2 \\ \frac{1}{3}
        \end{pmatrix}.
    \] Si estragga una base di $V$ da questi quattro vettori.
\end{example}
\begin{solution}
    Utilizziamo il metodo proposto dalla proposizione precedente. 
    \begin{gather*}
        \begin{pmatrix}
            2&3&1&0\\0&-2&0&1\\1&-2&-1&-2\\1&0&\frac{1}{3}&\frac13
        \end{pmatrix} \xrightarrow[R_4 - \frac{1}{2}R_2]{R_3 - \frac{1}{2}R_2}
        \begin{pmatrix}
            2&3&1&0\\0&-2&0&1\\0&-\frac72&-\frac32&-2\\0&-\frac{3}{2}&\frac{1}{6}&\frac13
        \end{pmatrix} \xrightarrow[R_4 \times 6]{R_2\times \frac12, R_3\times 2} \\
        \begin{pmatrix}
            2&3&1&0\\0&-1&0&\frac12\\0&-7&-3&-2\\0&-9&1&2
        \end{pmatrix} \xrightarrow[R_4-9R_2]{R_3 -7R_2} 
        \begin{pmatrix}
            2&3&1&0\\0&-1&0&\frac12\\0&0&-3&-\frac{15}{2}\\0&0&-1&-\frac{5}{2}
        \end{pmatrix}\\ \xrightarrow[]{R_4 -\frac13R_3} 
        \begin{pmatrix}
            2&3&1&0\\0&-1&0&\frac12\\0&0&-3&-\frac{15}{2}\\0&0&0&0
        \end{pmatrix}.
    \end{gather*}
    Notiamo dunque che i pivot sono nelle colonne $1$, $2$ e $3$, che corrispondono ai vettori $\bm{c_1}, \bm{c_2}, \bm{c_3}$ che per la proposizione precedente sono indipendenti e generano $V$, dunque $\ang{\bm{c_1}, \bm{c_2}, \bm{c_3}}$ e' una base di $V$.
\end{solution}

\begin{proposition}\label{base=dim_gener_indip}
    Sia $V$ uno spazio vettoriale e sia $\left\{\bm{v_1}, \dots, \bm{v_n} \right\}$ un insieme di $n$ vettori di $V$. Se valgono due dei seguenti fatti
    \begin{itemize}
        \item $n = \dim V$;
        \item $\left\{\bm{v_1}, \dots, \bm{v_n} \right\}$ e' un insieme di generatori di $V$;
        \item $\left\{\bm{v_1}, \dots, \bm{v_n} \right\}$ sono linearmente indipendenti;
    \end{itemize}
    allora vale anche il terzo e $\ang{\bm{v_1}, \dots, \bm{v_n}}$ e' una base di $V$.
\end{proposition}

\begin{example}
    Consideriamo lo spazio dei polinomi di grado minore o uguale a due $\R[x]^{\leq 2}$. Mostrare che $\alpha = \ang{1, (x-1), (x-1)^2}$ e' una base di $\R[x]^{\leq 2}$.
\end{example}
\begin{solution}
    Sappiamo che la base standard di $\R[x]^{\leq 2}$ e' la base $\ang{1, x, x^2}$, dunque $\dim \left( \R[x]^{\leq 2} \right) = 3$. Dato che la base $\alpha$ ha esattamente $3$ vettori, per la proposizione \ref{base=dim_gener_indip} ci basta dimostrare una tra:
    \begin{itemize}
        \item i tre vettori sono indipendenti;
        \item i tre vettori generano $\R[x]^{\leq 2}$.
    \end{itemize}
    Per esercizio, verifichiamole entrambe.
    \begin{itemize}
        \item Verifichiamo che sono linearmente indipendenti: consideriamo una generica combinazione lineare dei tre vettori e poniamola uguale al vettore $\bm{0} = 0 + 0x + 0x^2$. \begin{alignat*}{1}
            a \cdot 1 + b \cdot (x - 1) + c \cdot (x - 1)^2 &= 0+0x+0x^2 \\
            \iff a + bx - b + cx^2 -2cx + c &= 0+0x+0x^2 \\
            \iff (a-b+c) +(b-2c)x + cx^2 &= 0+0x+0x^2
        \end{alignat*} Dunque $a, b, c$ devono soddisfare il seguente sistema:
        \begin{equation*}
            \left\{
            \begin{array}{@{}rororor }
            a & - & b & + & c  & = & 0 \\
              &   & b & - & 2c & = & 0 \\
              &   &   &   & c  & = & 0 \\
            \end{array}  
            \right.
        \end{equation*}
        che ha soluzione solo per $a = b = c = 0$. Dunque i tre vettori sono indipendenti e, sapendo che $\dim \left( \R[x]^{\leq 2} \right) = 3$, sono una base di $\R[x]^{\leq 2}$.
        \item Verifichiamo che i tre vettori generano $\R[x]^{\leq 2}$. Un modo per farlo e' verificare che i vettori che compongono la base canonica di $\R[x]^{\leq 2}$ sono nello span di $\{1, (x-1), (x-1)^2\}$: infatti, dato che la base canonica genera tutto lo spazio, se essa e' nello span anche tutto il resto dello spazio sara' nello span dei nostri tre vettori.
        \begin{itemize}
            \item $1 = 1\cdot 1 + 0\cdot (x-1) + 0 \cdot (x-1)^2$, dunque $1 \in \Span{1, (x-1), (x-1)^2}$
            \item $x = 1\cdot 1 + 1\cdot (x-1) + 0 \cdot (x-1)^2$, dunque $x \in \Span{1, (x-1), (x-1)^2}$
            \item Dato che non e' immediato vedere come scrivere $x^2$ in termini di $1, (x-1), (x-1)^2$ cerchiamo di trovare i coefficienti algebricamente:
            \begin{alignat*}{1}
                x^2 &= a \cdot 1 + b\cdot (x-1) + c \cdot (x-1)^2 \\
                    &= (a-b+c) +(b-2c)x + cx^2
            \end{alignat*}
            dunque uguagliando i coefficienti dei termini dello stesso grado otteniamo
            \begin{equation*}
                \left\{\begin{array}{@{}rororor }
                    a & - & b & + & c  & = & 0 \\
                      &   & b & - & 2c & = & 0 \\
                      &   &   &   & c  & = & 1 \\
                \end{array} \right. \implies 
                \left\{\begin{array}{@{}rororor }
                    a & = & 1 \\
                    b & = & 2 \\
                    c & = & 1 \\
                \end{array} \right.
            \end{equation*}
            Quindi $x^2$ e' esprimibile come combinazione lineare di $1, (x-1), (x-1)^2$ (in particolare $x^2 = 1 + 2(x-1) + (x-1)^2$), dunque \[x^2 \in \Span{1, (x-1), (x-1)^2}.\]
        \end{itemize}
        Abbiamo quindi verificato che i vettori che formano la base canonica di $\R[x]^{\leq 2}$ fanno parte dello span dei nostri tre vettori, dunque se la base canonica genera tutto lo spazio anche $\{1, (x-1), (x-1)^2\}$ sono generatori. Inoltre, dato che $\dim (\R[x]^{\leq 2}) = 3$ segue che $\ang{1, (x-1), (x-1)^2}$ e' una base di $\R[x]^{\leq 2}$.
    \end{itemize}
\end{solution}

\begin{definition}
    Sia $V$ uno spazio vettoriale, $\bm{v} \in V$ e $\mathcal{B} = \ang{\bm{v_1}, \dots, \bm{v_n}}$ una base di $V$. Allora si dice vettore delle coordinate di $\bm{v}$ rispetto a $\mathcal{B}$ il vettore colonna
    \begin{equation}
        [\bm{v}]_{\mathcal{B}} = \begin{bmatrix}
                                    a_1 \\
                                    a_2 \\
                                    \vdots \\
                                    a_n
                                 \end{bmatrix} \in \R^n
    \end{equation}
    tale che \begin{equation}
        \bm{v} = a_1\bm{v_1} + \dots + a_n\bm{v_n}
    \end{equation}
\end{definition}

\begin{proposition}
    Sia $V$ uno spazio vettoriale, $\bm{v} \in V$ e $\mathcal{B} = \ang{\bm{v_1}, \dots, \bm{v_n}}$ una base di $V$. Allora le coordinate di $\bm{v}$ rispetto a $\mathcal{B}$ sono uniche.
\end{proposition}
\begin{proof}
    Supponiamo per assurdo che esistano due vettori colonna distinti $\bm{a}$, $\bm{b}$ che rappresentino le coordinate di $\bm{v}$ rispetto a $\mathcal{B}$. Allora
    \begin{alignat*}
        {1}
        \bm{0_V}  &= \bm{v} - \bm{v} \\
                &= (a_1\bm{v_1} + \dots + a_n\bm{v_n}) - (b_1\bm{v_1} + \dots + b_n\bm{v_n}) \\
                &= (a_1 - b_1)\bm{v_1} + \dots + (a_n - b_n)\bm{v_n}
    \end{alignat*}
    Ma per definizione di base $\bm{v_1}, \dots, \bm{v_n}$ sono linearmente indipendenti, dunque l'unica combinazione lineare che da' come risultato il vettore $\bm{0_V}$ e' quella in cui tutti i coefficienti sono $0$. Da cio' segue che
    \begin{gather*}
        a_1 - b_1 = a_2 - b_2 = \dots = a_n - b_n = 0 \\
        \implies \bm{a} = \begin{bmatrix}
            a_1 \\
            a_2 \\
            \vdots \\
            a_n
        \end{bmatrix}
        = 
        \begin{bmatrix}
            b_1 \\
            b_2 \\
            \vdots \\
            b_n
        \end{bmatrix} = \bm{b}
    \end{gather*}
    cioe' i due vettori sono uguali. Ma cio' e' assurdo poiche' abbiamo supposto $\bm{a} \neq \bm{b}$, dunque le coordinate di $\bm{v}$ rispetto a $\mathcal{B}$ devono essere uniche.
\end{proof}

\begin{example}
    Sia $V \subseteq \M_{2 \times 2}(\R)$ tale che $V$ e' il sottospazio delle matrici simmetriche. Trovare una base di $V$ e trovare le coordinate di $\bm{u} = \begin{psmallmatrix}
        3 & 4 \\ 4 & 6
    \end{psmallmatrix} \in V$ rispetto alla base trovata.
\end{example}
\begin{solution}
    Cerco di esprimere un generico vettore $\bm{v} \in V$ in termini della condizione che definisce il sottospazio.
    \begin{alignat*}
        {1}
        V &= \left\{ \begin{pmatrix} a&b\\b&c \end{pmatrix}\mid a, b, c \in \R\right\} \\
        \intertext{Isolando i contributi di $a$, $b$ e $c$ ottengo}
        &= \left\{ \bm{v} \in \M_{2 \times 2}(\R) \mid \exists a, b, c \in \R. \bm{v} = \begin{pmatrix} a&0\\0&0 \end{pmatrix} + \begin{pmatrix} 0&b\\b&0 \end{pmatrix} + \begin{pmatrix} 0&0\\0&c \end{pmatrix}\right\} \\
        &= \left\{ \bm{v} \in \M_{2 \times 2}(\R) \mid \exists a, b, c \in \R. \bm{v} = a\begin{pmatrix} 1&0\\0&0 \end{pmatrix} + b\begin{pmatrix} 0&1\\1&0 \end{pmatrix} + c\begin{pmatrix} 0&0\\0&1 \end{pmatrix}\right\} \\
        &= \Span{\begin{pmatrix} 1&0\\0&0 \end{pmatrix}, \begin{pmatrix} 0&1\\1&0 \end{pmatrix}, \begin{pmatrix} 0&0\\0&1 \end{pmatrix}} \\
    \end{alignat*}

    Ora dobbiamo mostrare che $\begin{psmallmatrix} 1&0\\0&0 \end{psmallmatrix}, \begin{psmallmatrix} 0&1\\1&0 \end{psmallmatrix}, \begin{psmallmatrix} 0&0\\0&1 \end{psmallmatrix}$ sono linearmente indipendenti. Consideriamo una loro generica combiazione lineare e imponiamola uguale a $0$:
    \begin{gather*}
        x\begin{pmatrix} 1&0\\0&0 \end{pmatrix} + y\begin{pmatrix} 0&1\\1&0 \end{pmatrix} + z\begin{pmatrix} 0&0\\0&1 \end{pmatrix} = \begin{pmatrix} 0&0\\0&0 \end{pmatrix} \\
        \iff \begin{pmatrix} x&y\\y&z \end{pmatrix} = \begin{pmatrix} 0&0\\0&0 \end{pmatrix} \\
        \iff x = y = z = 0.
    \end{gather*}
    Dunque $\mathcal{B} = \ang{\begin{psmallmatrix} 1&0\\0&0 \end{psmallmatrix}, \begin{psmallmatrix} 0&1\\1&0 \end{psmallmatrix}, \begin{psmallmatrix} 0&0\\0&1 \end{psmallmatrix}}$ e' una base di $V$.

    Per trovare le coordinate di $\bm{u}$ esprimiamo in termini della base:
    \begin{equation*}
        \begin{pmatrix}
            3 & 4 \\ 4 & 6
        \end{pmatrix} = 3\begin{pmatrix} 1&0\\0&0 \end{pmatrix} + 4\begin{pmatrix} 0&1\\1&0 \end{pmatrix} + 6\begin{pmatrix} 0&0\\0&1 \end{pmatrix}
    \end{equation*}
    dunque $[\bm{u}]_{\mathcal{B}} = \begin{pmatrix}
        3 \\ 4 \\ 6
    \end{pmatrix}$.
\end{solution}

Notiamo che sembra esserci una relazione biunivoca tra un vettore di $V$ e le sue coordinate in $\R^n$ rispetto ad una base. Infatti (come vedremo nella prossima parte) la relazione tra vettore di $V$ e vettore colonna delle sue coordinate e' un isomorfismo: essi rappresentano lo stesso oggetto sotto forme diverse. Quindi spesso per fare calcoli (ad esempio semplificare un insieme di vettori per trovare una base) possiamo passare allo spazio isomorfo $\R^n$, sfruttare i vettori colonna e le matrici (ad esempio facendo mosse di Gauss per riga o per colonna) e infine passare di nuovo allo spazio originale.

Abbiamo mostrato come estrarre una base di un sottospazio a partire da un insieme di generatori. Ora vogliamo \textbf{completare una base} di un sottospazio ad una base dello spazio vettoriale che lo contiene.

\begin{theorem}
    [del completamento ad una base] \label{th_completamento}
    Sia $V$ uno spazio vettoriale di dimensione finita $n = \dim V$ e sia $\mathcal{B} = \ang{\bm{v_1}, \dots, \bm{v_k}}$ un insieme di $k$ vettori linearmente indipendenti. Allora vale che $k \leq n$ ed esistono $n - k$ vettori di $V$, diciamo ${\bm{w_1}, \dots, \bm{w_{n-k}}}$ tali che $\mathcal{B}' = \ang{\bm{v_1}, \dots, \bm{v_k}, \bm{w_1}, \dots, \bm{w_{n-k}}}$ e' una base di $V$.
\end{theorem}
\begin{proof}
    Non possono esserci piu' di $n$ vettori indipendenti in uno spazio di dimensione $n$, dunque $k \leq n$. Ora dimostriamo che possiamo completare $\mathcal{B}$ ad una base di $V$.

    Se $k = n$ allora per la proposizione \ref{base=dim_gener_indip} gli $n$ vettori indipendenti sono gia' una base, dunque abbiamo finito.

    Se $k < n$ allora esistera' sicuramente $\bm{w_1} \notin \Span{\bm{v_1}, \dots, \bm{v_k}}$ (altrimenti i vettori genererebbero l'intero spazio vettoriale e sarebbero quindi una base), dunque $\ang{\bm{v_1}, \dots, \bm{v_k}, \bm{w_1}}$ sono ancora indipendenti.
    
    Continuiamo a ripetere questo processo fino a quando l'insieme di vettori non genera l'intero spazio vettoriale $V$. Sia $\mathcal{B}'$ l'insieme creato tramite questo processo. Allora $\mathcal{B}'$ e' un insieme di vettori indipendenti che generano $V$, dunque e' una base di $V$, dunque dovra' contenere $n$ vettori. Ma dato che inizialmente avevamo $k$ vettori, per completare ad una base di $V$ abbiamo dovuto aggiungere $n-k$ vettori di $V$.
\end{proof}

\subsubsection{Procedimento per completare ad una base di $\R^n$}

Sia $V = \R^n$ e siano $\left\{ \bm{v_1}, \dots, \bm{v_k} \right\}$ indipendenti.

Allora formo la matrice $M$ che ha come colonne i vettori $\bm{v_1}, \dots, \bm{v_k}$ e la riduco a scalini per colonna tramite mosse di Gauss di colonna, ottenendo una matrice $M'$ che ha come colonne i vettori $\bm{v'_1}, \dots, \bm{v'_k}$.

Questi vettori sono indipendenti (poiche' le mosse di colonna non modificano lo span) e sono a scalini, dunque dovranno avere pivot su righe diverse, e dovranno averne esattamente $k \leq n$. Allora aggiungo $n - k$, ognuno con un pivot su una riga diversa da quelle gia' occupate: la matrice finale sara' una matrice quadrata con $n$ pivot, dunque sara' formata da colonne indipendenti che formano una base di $\R^n$.

\begin{example}
    Sia $V = \R^4$, $A \subseteq V$ tale che \[
        A = \Span{\begin{pmatrix}2\\0\\1\\1\end{pmatrix}, \begin{pmatrix}3\\-2\\-2\\0\end{pmatrix}, \begin{pmatrix}1\\0\\-1\\\frac13\end{pmatrix}, \begin{pmatrix}0\\1\\-2\\\frac13\end{pmatrix}}.    
    \] Trovare una base di $A$ e completarla ad una base di $\R^4$.
\end{example}
\begin{solution}
    Troviamo una base di $A$ tramite mosse di colonna:
    \begin{gather*}
        \begin{pmatrix}[c|c|c|c]
            2&3&1&0\\0&-2&0&1\\1&-2&-1&-2\\1&0&\frac13&\frac13
        \end{pmatrix} \xrightarrow[]{\text{scambio}}
        \begin{pmatrix}[c|c|c|c]
            1&0&3&2\\0&1&-2&0\\-1&-2&-2&1\\\frac13&\frac13&0&1
        \end{pmatrix} \xrightarrow[C_4 - 2C_1]{C_3 - 3C_1} \\
        \xrightarrow[C_4 - 2C_1]{C_3 - 3C_1} \begin{pmatrix}[c|c|c|c]
            1&0&0&0\\0&1&-2&0\\-1&-2&1&3\\\frac13&\frac13&-1&\frac13
        \end{pmatrix} \xrightarrow[]{C_3 + 2C_1}
        \begin{pmatrix}[c|c|c|c]
            1&0&0&0\\0&1&0&0\\-1&-2&-3&3\\\frac13&\frac13&-\frac13&\frac13
        \end{pmatrix} \xrightarrow[]{C_4 + C_3} \\ \xrightarrow[]{C_4 + C_3}
        \begin{pmatrix}[c|c|c|c]
            1&0&0&0\\0&1&0&0\\-1&-2&-3&0\\\frac13&\frac13&-\frac13&0
        \end{pmatrix}
    \end{gather*}
    Dunque una base di $A$ e' formata dai vettori $\ang{\begin{psmallmatrix}1\\0\\-1\\\frac13 \end{psmallmatrix}, \begin{psmallmatrix}0\\1\\-2\\\frac13 \end{psmallmatrix}, \begin{psmallmatrix}0\\0\\-3\\-\frac13 \end{psmallmatrix} }$.

    Notiamo che i pivot di questi vettori sono ad altezza $1$, $2$ e $3$, dunque per completare ad una base di $\R^4$ basta aggiungere un vettore che ha un pivot ad altezza $4$, come $\begin{psmallmatrix} 0\\0\\0\\1 \end{psmallmatrix}$.

    Dunque abbiamo completato la base di $A$ alla seguente base di $\R^4$: \[
        \ang{
            \begin{pmatrix}1\\0\\-1\\\frac13 \end{pmatrix}, 
            \begin{pmatrix}0\\1\\-2\\\frac13 \end{pmatrix}, 
            \begin{pmatrix}0\\0\\-3\\-\frac13 \end{pmatrix},
            \begin{pmatrix} 0\\0\\0\\1 \end{pmatrix}
        }.      
    \]
\end{solution}

\section{Sottospazi somma e intersezione}

\begin{definition}
    Sia $V$ uno spazio vettoriale e siano $A, B \subseteq V$ due sottospazi di $V$. Allora sono sottospazi vettoriali di $V$:
    \begin{subequations}
        \begin{equation}
            A \cap B = \left\{ \bm{v} \in V \mid \bm v \in A, \bm v \in B\right\}
        \end{equation}
        \begin{equation}
            A + B = \left\{ (\bm{v} + \bm{w}) \in V \mid \bm v \in A, \bm w \in B\right\}
        \end{equation}
    \end{subequations}
\end{definition}

\begin{remark}
    Possiamo verificare molto semplicemente che i due spazi sopra sono effettivamente sottospazi di $V$. Inoltre $A \cup B$ non e' un sottospazio vettoriale, ma possiamo notare che $(A \cup B) \subset (A + B)$ in quanto $A \subset A + B$ e $B \subset A + B$.
\end{remark}

\begin{proposition}
    Sia $V$ uno spazio vettoriale e siano $A, B \subseteq V$ due sottospazi di $V$ tali che $A = \Span{\bm{v_1}, \dots, \bm{v_n}}$ e $B = \Span{\bm{w_1}, \dots, \bm{w_m}}$. Allora \begin{equation}
        A + B = \Span{\bm{v_1}, \dots, \bm{v_n}, \bm{w_1}, \dots, \bm{w_m}}.
    \end{equation}
\end{proposition}
\begin{proof}
    Consideriamo un generico $\bm u \in A + B$. Allora per definizione di $A + B$ segue che $\bm u = \bm v + \bm w$ per qualche $\bm v \in A, \bm w \in B$.
    Dato che $A = \Span{\bm{v_1}, \dots, \bm{v_n}}$ e $B = \Span{\bm{w_1}, \dots, \bm{w_m}}$, allora possiamo scrivere 
    \begin{alignat*}{1}
        &\bm v = a_1\bm{v_1} + \dots + a_n\bm{v_n},  \quad \bm w = b_1\bm{w_1} + \dots + b_n\bm{w_m} \\
        \intertext{per qualche $a_1, \dots, a_n, b_1, \dots, b_m \in \R$. Quindi $\bm u = \bm v + \bm w$ diventa}
        \implies &\bm v + \bm w = a_1\bm{v_1} + \dots + a_n\bm{v_n} + b_1\bm{w_1} + \dots + b_n\bm{w_m}
    \end{alignat*}
    dunque ogni vettore in $A + B$ puo' essere scritto come combinazione lineare di $\bm{v_1}, \dots, \bm{v_n}, \bm{w_1}, \dots, \bm{w_m}$, dunque questi vettori generano $A + B$. 
\end{proof}

\begin{remark}
    I vettori $\bm{v_1}, \dots, \bm{v_n}, \bm{w_1}, \dots, \bm{w_m}$ \textbf{generano} $A+B$ ma non sono una base: dobbiamo prima assicurarci che siano linearmente indipendenti.
\end{remark}

\begin{definition}
    Sia $V$ uno spazio vettoriale e siano $A, B \subseteq V$ due sottospazi di $V$. Allora il sottospazio somma $A + B$ si dice in somma diretta se per ogni $\bm v \in A$, $\bm w \in B$ allora $\bm v, \bm w$ sono indipendenti. Se la somma e' diretta scrivo $A \oplus B$.
\end{definition}

\begin{proposition}
    Sia $V$ uno spazio vettoriale e siano $A, B \subseteq V$ due sottospazi di $V$. Allora il sottospazio somma $A + B$ e' in somma diretta se e solo se $A \cap B = \{\bm 0\}$.
\end{proposition}
\begin{proof}
    Innanzitutto notiamo che $\bm{0} \in A$ e $\bm 0 \in B$, dunque $\{\bm 0\} \subseteq A \cap B$.
    \begin{itemize}
        \item[($\implies$)] Supponiamo $A \oplus B$. 
        
        Allora supponiamo per assurdo che esista $\bm u \in A \cap B$ non nullo. Per definizione di intersezione segue che $\bm u \in A$ e $\bm u \in B$, ma questo significa che in $A$ e in $B$ ci sono almeno due vettori $\bm v \in A$ e $\bm w \in B$ non indipendenti tra loro: basta scegliere $\bm v = \bm u$ e $\bm w = \bm u$. 
        
        Tuttavia questo e' assurdo poiche' abbiamo assunto che $A$ e $B$ siano in somma diretta, dunque non puo' esserci un $\bm u \in A \cap B$ non nullo, dunque $A \cap B = \{\bm 0\}$.
        \item[($\impliedby$)] Supponiamo che $A \cap B = \{\bm 0\}$. 
        
        Siano $\bm v \in A$, $\bm{w} \in B$ entrambi non nulli. Per dimostrare che $A$ e $B$ sono in somma diretta e' sufficiente dimostrare che sono necessariamente indipendenti, cioe' che l'unica combinazione lineare $a\bm v + b\bm w$ che e' uguale a $\bm 0$ e' quella con $a = b = 0$. 
        
        Notiamo che $a\bm v + b\bm w = \bm 0$ se e solo se $a\bm v = -b\bm w$; ma dato che i due vettori (che fanno parte di sottospazi diversi) sono uguali segue che devono entrambi far parte del sottospazio intersezione, cioe' $a\bm v, -b\bm w \in A \cap B$.
        
        Per ipotesi $A \cap B = \{\bm 0\}$, dunque $a\bm v = -b\bm w = \bm 0$. Inoltre abbiamo assunto che i vettori $\bm v$ e $\bm{w}$ siano non nulli, dunque segue che $a = b = 0$, come volevasi dimostrare. \qedhere
    \end{itemize}
\end{proof}

\begin{theorem}
    [di Grassman] \label{th_grassman}
    Sia $V$ uno spazio vettoriale e $A, B \subseteq V$ due sottospazi. Allora \begin{equation}
        \dim(A + B) = \dim A + \dim B - \dim(A \cap B).
    \end{equation}
\end{theorem}
\begin{proof}
    Consideriamo una base $\gamma = \ang{\bm{u_1}, \dots, \bm{u_k}}$ di $A \cap B$.

    Dato che $A \cap B$ e' un sottospazio sia di $A$ che di $B$, allora per il teorema del completamento ad una base (\ref{th_completamento}) possiamo completarla ad una base $\alpha = \ang{\bm{u_1}, \dots, \bm{u_k}, \bm{v_1}, \dots, \bm{v_n}}$ di $A$ e ad una base $\beta = \ang{\bm{u_1}, \dots, \bm{u_k}, \bm{w_1}, \dots, \bm{w_m}}$ di $B$.

    Dimostriamo che $\ang{\bm{u_1}, \dots, \bm{u_k}, \bm{v_1}, \dots, \bm{v_n}, \bm{w_1}, \dots, \bm{w_m}}$ e' una base di $A + B$.

    \begin{itemize}
        \item Mostriamo che $\{\bm{u_1}, \dots, \bm{u_k}, \bm{v_1}, \dots, \bm{v_n}, \bm{w_1}, \dots, \bm{w_m}\}$ generano $A + B$.
        
        Sia $\bm u \in A + B$ generico. Allora esistono $\bm v \in A, \bm w \in B$ tali che $\bm{u} = \bm{v} + \bm{w}$. Dato che $\alpha$ e' una base di $A$ e $\beta$ e' una base di $B$ allora possiamo scrivere $\bm v$ e $\bm w$ come \begin{alignat*}{2}
            \bm{v} &=\ && a_1\bm{u_1} + \dots + a_k\bm{u_k} + a_{k+1}\bm{v_1} + \dots + a_{k+n}\bm{v_n} \\
            \bm{w} &=\ && b_1\bm{u_1} + \dots + b_k\bm{u_k} + b_{k+1}\bm{w_1} + \dots + b_{k+m}\bm{w_m} \\
            \intertext{dunque}
            \bm{u} &=\ && \bm{v} + \bm{w}\\
                &=\ && a_1\bm{u_1} + \dots + a_k\bm{u_k} + a_{k+1}\bm{v_1} + \dots + a_{k+n}\bm{v_n} + \\
                & && + b_1\bm{u_1} + \dots + b_k\bm{u_k} + b_{k+1}\bm{w_1} + \dots + b_{k+m}\bm{w_m} \\
                &=\ && (a_1 + b_1)\bm{u_1} + \dots + (a_k + b_k)\bm{u_k} + \\
                & && + a_{k+1}\bm{v_1} + \dots + a_{k+n}\bm{v_n} + \\
                & && + b_{k+1}\bm{w_1} + \dots + b_{k+m}\bm{w_m}.
        \end{alignat*}
        Dunque ogni elemento di $A + B$ puo' essere scritto come combinazione lineare di $\bm{u_1}, \dots, \bm{u_k}, \bm{v_1}, \dots, \bm{v_n}, \bm{w_1}, \dots, \bm{w_m}$, cioe' essi sono generatori di $A + B$.

        \item Mostriamo che $\{\bm{u_1}, \dots, \bm{u_k}, \bm{v_1}, \dots, \bm{v_n}, \bm{w_1}, \dots, \bm{w_m}\}$ e' un insieme di vettori linearmente indipendenti.
        
        Consideriamo una combinazione lineare dei vettori $\bm{u_1}, \dots, \bm{u_k}$, $\bm{v_1}, \dots, \bm{v_n}$, $\bm{w_1}, \dots, \bm{w_m}$ e verifichiamo che essa e' uguale al vettore $\bm{0}$ se e solo se tutti i coefficienti sono uguali a $0$.

        \begin{alignat*}{1}
            &\bm{0} = x_1\bm{v_1} + \dots + x_n\bm{v_n} + y_1\bm{u_1} + \dots + y_k\bm{u_k} + z_1\bm{w_1} + \dots +z_m\bm{w_m} \\
            \iff &x_1\bm{v_1} + \dots + x_n\bm{v_n} = -(y_1\bm{u_1} + \dots + y_k\bm{u_k} + z_1\bm{w_1} + \dots +z_m\bm{w_m}).
        \end{alignat*}
        Notiamo che il primo membro e' un vettore del sottospazio $A$, mentre il secondo membro e' un vettore del sottospazio $B$: dato che i due vettori sono uguali allora devono trovarsi in entrambi i sottospazi e dunque anche nel sottospazio $A \cap B$. Dato che $\gamma$ e' una base di $A \cap B$ possiamo scrivere \begin{alignat*}{1}
            &x_1\bm{v_1} + \dots + x_n\bm{v_n} = a_1\bm{u_1} + \dots + a_k\bm{u_k} \\
            \iff &x_1\bm{v_1} + \dots + x_n\bm{v_n} - a_1\bm{u_1} - \dots - a_k\bm{u_k} = \bm{0}
            \intertext{ma $\bm{u_1}, \dots, \bm{u_k}, \bm{v_1}, \dots, \bm{v_n}$ formano una base di $A$, dunque devono essere indipendenti, quindi per definizione segue che}
            \iff &x_1 = \dots = x_n = 0
        \end{alignat*}
        Dunque nella combinazione lineare i termini con $\bm{v_i}$ scompaiono, e rimangono solo
        \begin{alignat*}{1}
            &\bm{0} = y_1\bm{u_1} + \dots + y_k\bm{u_k} + z_1\bm{w_1} + \dots +z_m\bm{w_m}
            \intertext{ma questi vettori formano la base $\beta$ di $B$, dunque devono essere indipendenti, cioe' per definizione}
            \iff &y_1 = \dots = y_k = z_1 = \dots = z_m = 0.
        \end{alignat*}

        Segue quindi che $\{\bm{u_1}, \dots, \bm{u_k}, \bm{v_1}, \dots, \bm{v_n}, \bm{w_1}, \dots, \bm{w_m}\}$ e' un insieme di vettori linearmente indipendenti.
    \end{itemize}
    Dato che l'insieme $\{\bm{u_1}, \dots, \bm{u_k}, \bm{v_1}, \dots, \bm{v_n}, \bm{w_1}, \dots, \bm{w_m}\}$ e' un insieme di vettori linearmente indipendenti e genera $A+B$, allora esso e' una base di $A + B$.

    Dunque \begin{alignat*}
        {1}
        \dim(A + B) &= n + k + m\\ 
                &= (n + k) + (m + k) - k \\
                &= \dim A + \dim B - dim(A \cap B)
    \end{alignat*}
    come volevasi dimostrare.
\end{proof}



\chapter{Applicazioni lineari}

\section{Applicazioni lineari}

\begin{definition}
    Siano $V, W$ spazi vettoriali. Allora un'applicazione $f : V \to W$ si dice lineare
    se
    \begin{align}
        &f(\bm{0_V}) = \bm{0_W} \\
        &f(\bm{v} + \bm{w}) = f(\bm{v}) + f(\bm{w}) &&\forall v, w \in V \\
        &f(k\bm{v}) = kf(\bm{v})                    &&\forall v\in V, k \in \R 
    \end{align}
    $V$ si dice dominio dell'applicazione lineare, $W$ si dice codominio.
\end{definition}

\begin{definition}
    Siano $V, W$ spazi vettoriali e sia $f : V \to W$ lineare. Allora si dice immagine di $f$ l'insieme \begin{equation}
        \Imm{f} = \left\{ f(\bm{v}) \mid \bm v \in V\right\}.
    \end{equation}
\end{definition}

\begin{remark}
    Se $f : V \to W$ allora $\Imm{f} \subseteq W$. In particolare si puo' dimostrare che $\Imm{f}$ e' un sottospazio di $W$, e dunque che $0 \leq \dim\Imm{f} \leq \dim W$.
\end{remark}

\begin{definition}
    Siano $V, W$ spazi vettoriali e sia $f : V \to W$ lineare. Allora si dice kernel (o nucleo) di $f$ l'insieme \begin{equation}
        \ker{f} = \left\{ \bm{v} \in V \mid f(\bm v) = \bm{0_W}\right\}.
    \end{equation}
\end{definition}

\begin{proposition}\label{base_mappata_generatori_immagine}
    Sia $f : V \to W$. Allora se $\ang{\bm{v_1}, \dots, \bm{v_n}}$ e' una base di $V$ segue che $\left\{ f(\bm{v_1}), \dots, f(\bm{v_n})\right\}$ e' un insieme di generatori di $\Imm{f}$.
\end{proposition}
\begin{proof}
    Sia $\bm{w} \in \Imm{f}$ generico; allora questo equivale a dire che esiste $\bm{v} \in V$ tale che $f(\bm{v}) = \bm{w}$.
    Dato che $\ang{\bm{v_1}, \dots, \bm{v_n}}$ e' una base di $V$, allora possiamo scrivere $\bm{v}$ come $a_1\bm{v_1} + \dots a_n\bm{v_n}$, dunque 
    \begin{equation*}
        \bm{w} = f(a_1\bm{v_1} + \dots + a_n\bm{v_n}) = a_1f(\bm{v_1}) + \dots + a_nf(\bm{v_n}).
    \end{equation*}
    Dunque per la generalita' di $w$ segue che ogni elemento di $\Imm{f}$ appartiene allo span di $f(\bm{v_1}), \dots, f(\bm{v_n})$, cioe' $\left\{ f(\bm{v_1}), \dots, f(\bm{v_n})\right\}$ e' un insieme di generatori di $\Imm{f}$.
\end{proof}

\begin{theorem} 
    [delle dimensioni] \label{th_dimensioni}
    Siano $V, W$ spazi vettoriali e sia $f : V \to W$ lineare. Allora vale il seguente fatto:
    \begin{equation}
        \dim V = \dim \Imm f + \dim \ker f.
    \end{equation}
\end{theorem}
\begin{proof}
    Sia $k$ la dimensione di $\ker f$ e $n$ la dimensione di $V$.
    Sia $\alpha = \ang{\bm{v_1}, \dots, \bm{v_k}}$ una base di $\ker f$. Dato che $\ker f$ e' un sottospazio di $V$, per il teorema del completamento ad una base (\ref{th_completamento}) possiamo completare $\alpha$ ad una base $\beta = \ang{\bm{v_1}, \dots, \bm{v_k}, \bm{u_1}, \dots, \bm{u_{n-k}}}$ di $V$.

    Per la proposizione \ref{base_mappata_generatori_immagine} segue che l'immagine della base $\beta$, cioe' $f(\beta) = \ang{f(\bm{v_1}), \dots, f(\bm{v_k}), f(\bm{u_1}), \dots, f(\bm{u_{n-k}})}$, e' una insieme di generatori di $\Imm{f}$. Dato che $\bm{v_1}, \dots, \bm{v_k} \in \ker f$, allora \begin{alignat*}
        {1}
        \Imm{f} &= \Span{f(\bm{v_1}), \dots, f(\bm{v_k}), f(\bm{u_1}), \dots, f(\bm{u_{n-k}})}\\
        &= \Span{0, \dots, 0, f(\bm{u_1}), \dots, f(\bm{u_{n-k}})}\\
        &= \Span{f(\bm{u_1}), \dots, f(\bm{u_{n-k}})}.
    \end{alignat*}

    Se $f(\bm{u_1}), \dots, f(\bm{u_{n-k}})$ sono indipendenti allora segue che essi formano una base di $\Imm{f}$, cioe' che $\dim \Imm{f} = n - k = \dim V - \dim \ker f$.

    Consideriamo quindi una generica combinazione lineare \[
        x_1f(\bm{u_1}) + \dots + x_{n-k}f(\bm{u_{n-k}})
    \] e dimostriamo che imponendola uguale a $\bm 0$ segue che i coefficienti della combinazione devono essere uguali a 0.
    \begin{alignat*}
        {1}
        &x_1f(\bm{u_1}) + \dots + x_{n-k}f(\bm{u_{n-k}}) = \bm 0 \\
        \iff &f(x_1\bm{u_1} + \dots + x_{n-k}\bm{u_{n-k}}) = \bm 0 \\
        \intertext{che per definizione di kernel significa}
        \iff &x_1\bm{u_1} + \dots + x_{n-k}\bm{u_{n-k}} \in \ker f.
    \end{alignat*}
    Dato che $\alpha$ e' una base di $\ker f$ allora segue che
    \begin{alignat*}{1}
        &x_1\bm{u_1} + \dots + x_{n-k}\bm{u_{n-k}} = a_1\bm{v_1} + \dots + a_k\bm{v_k} \\
        \iff &x_1\bm{u_1} + \dots + x_{n-k}\bm{u_{n-k}} - a_1\bm{v_1} - \dots - a_k\bm{v_k} = \bm 0.
    \end{alignat*}
    Ma $\beta = \ang{\bm{v_1}, \dots, \bm{v_k}, \bm{u_1}, \dots, \bm{u_{n-k}}}$ e' una base di $V$, dunque i vettori che la compongono devono essere indipendenti, da cui segue \[
        x_1 = \dots = x_{n-k} = 0.   
    \]
    Dunque $\ang{f(\bm{u_1}), \dots, f(\bm{u_{n-k}})}$ e' una base di $\Imm{f}$ e dunque segue che $\dim V = \dim \ker f + \dim \Imm{f}$, come volevasi dimostrare.
\end{proof}

Una conseguenza diretta del teorema delle dimensioni e' che data una matrice $A$ e riducendola a scalini tramite mosse di riga non cambia la dimensione dello spazio delle colonne.
\begin{proposition}\label{invarianza_dim_colonne_per_mosse_riga}
    Sia $A \in \M_{n\times m}$ e siano $C_1, \dots, C_m \in \R^n$ le colonne della matrice. Sia $S$ la matrice ottenuta riducendo a scalini per riga la matrice $A$, e siano $C'_1, \dots, C'_m$ le colonne di $S$. Allora \begin{equation}
        \dim \Span{C_1, \dots, C_m} = \dim \Span{C'_1, \dots, C'_m}.
    \end{equation}
\end{proposition}
\begin{proof}
    Dato che $S$ e' ottenuta riducendo $A$ a scalini, allora le soluzioni dei sistemi $A\bm{x} = \bm 0$ e $S\bm{x} = \bm 0$ devono essere le stesse. 
    
    Siano $L_A : \R^m \to \R^n$ e $L_S : \R^m \to \R^n$ le applicazioni lineari associate ad $A$ e $S$; trascrivendo le equazioni di prima in termini delle applicazioni otteniamo che $L_A(\bm x) = \bm 0$ se e solo se $L_S(\bm x) = \bm 0$. 
    
    Allora segue che $\ker L_A = \ker L_S$, dunque $\dim \ker L_A = \dim \ker L_S$. Per la proposizione \ref{span_colonne=immagine_applicazione_associata} e per il teorema delle dimensioni (\ref{th_dimensioni}) segue quindi che
    \begin{alignat*}
        {1}
        \dim \Span{C_1, \dots, C_m} &= \dim \Imm{L_A}\\
        &= \dim \R^m - \dim \ker L_A\\
        &= \dim \R^m - \dim \ker L_S\\
        &= \dim \Imm{L_S} \\
        &= \dim \Span{C'_1, \dots, C'_m}
    \end{alignat*}
    che e' la tesi.
\end{proof}

Ovviamente lo stesso ragionamento ci dice che ridurre una matrice a scalini tramite mosse di colonna non cambia la dimensione dello spazio delle righe.

\begin{corollary}\label{invarianza_dim_righe_per_mosse_colonna}
    Sia $A \in \M_{n\times m}$ e siano $R_1, \dots, R_m \in \R^n$ le righe della matrice. Sia $S$ la matrice ottenuta riducendo a scalini per colonna la matrice $A$, e siano $R'_1, \dots, R'_m$ le righe di $S$. Allora \begin{equation}
        \dim \Span{R_1, \dots, R_m} = \dim \Span{R'_1, \dots, R'_m}.
    \end{equation}
\end{corollary}
\begin{proof}
    Consideriamo la matrice $A^T$ e riduciamola per righe, ottenendo la matrice $S$. Per la proposizione \ref{invarianza_dim_colonne_per_mosse_riga} la dimensione dello spazio delle righe di $S$ e' uguale alla dimensione dello spazio delle righe di $A^T$. Notiamo inoltre che $S$ e' la trasposta della matrice ottenuta riducendo $A$ per colonne, dunque la dimensione dello spazio delle colonne di $S^T$ e' la dimensione dello spazio delle colonne di $A$, cioe' la tesi.
\end{proof}

\section{Applicazioni iniettive e surgettive}

Le applicazioni lineari sono funzioni, dunque possono essere iniettive e surgettive, ma essendo lineari hanno delle proprieta' particolari.

\begin{definition}
    Siano $V, W$ spazi vettoriali e sia $f : V \to W$ lineare. Allora $f$ si dice iniettiva se per ogni $\bm{v}, \bm{u} \in V$ vale che $f(\bm{v}) = f(\bm{u})$ se e solo se $\bm{v} = \bm{u}$.
\end{definition}

\begin{proposition}\label{ker_funzione_iniettiva}
    Siano $V, W$ spazi vettoriali e sia $f : V \to W$ lineare. Allora $f$ e' iniettiva se e solo se $\ker f = \{\bm{0_V}\}$. 
\end{proposition}
\begin{proof}
    Notiamo che dato che $f$ e' lineare allora per definizione $f(\bm{0_V}) = \bm{0_W}$, dunque $\bm{0_V} \in \ker f$.
    \begin{description}
        \item [($\implies$).] Supponiamo che $f$ sia iniettiva e supponiamo che per qualche $\bm{v} \in V$ valga $\bm{v} \in \ker f$. Allora per definizione di kernel $f(\bm{v}) = \bm{0_W} = f(\bm{0_V})$, dunque per iniettivita' di $f$ da $f(\bm{v}) = f(\bm{0_V})$ segue che $\bm v = \bm{0_V}$. Dunque $\ker f = \{\bm{0_V}\}$.
        \item [($\impliedby$).] Supponiamo che $\ker f = \left\{ \bm{0_V}\right\}$. Per dimostrare che $f$ e' iniettiva e' sufficiente dimostrare che per ogni $\bm{v}, \bm{w} \in V$ segue che $f(\bm{v}) = f(\bm{w}) \implies \bm{v} = \bm{w}$.
        \begin{alignat*}{1}
            &f(\bm{v}) = f(\bm{w}) \\
            \iff &f(\bm{v}) - f(\bm{w}) = \bm{0_W} \\
            \iff &f(\bm{v} - \bm{w}) = \bm{0_W} \\
            \iff &\bm{v} - \bm{w} \in \ker f \\
            \intertext{ma l'unico elemento di $\ker f$ e' $\bm{0_V}$, dunque}
            \implies &\bm{v} - \bm{w} = \bm{0_V}\\
            \iff &\bm{v} = \bm{w}
        \end{alignat*}
        cioe' $f$ e' iniettiva. \qedhere
    \end{description}
\end{proof}

\begin{corollary}\label{iniettiva_allora_dimIm_uguale_dimV}
    Se $f$ e' iniettiva allora $\dim \Imm{f} = \dim V$.
\end{corollary}
\begin{proof}
    Infatti per la proposizione \ref{ker_funzione_iniettiva} $\dim \ker f = 0$, dunque per il teorema delle dimensioni (\ref{th_dimensioni}) segue che $\dim V = \dim \Imm{f} + \dim \ker f = \dim \Imm{f}$.
\end{proof}

\begin{corollary}
    Siano $V, W$ spazi vettoriali tali che $\dim V > \dim W$. Allora non puo' esistere $f : V \to W$ iniettiva.
\end{corollary}
\begin{proof}
    Infatti per il corollario \ref{iniettiva_allora_dimIm_uguale_dimV} segue che $\dim \Imm{f} = \dim V$, ma $\dim \Imm{f} < \dim W$ dunque non puo' essere che $\dim V > \dim W$.
\end{proof}

\begin{proposition}\label{indipendenti_mappati_indipendenti}
    Siano $V, W$ spazi vettoriali, $\bm{v_1}, \dots, \bm{v_n} \in V$ linearmente indipendenti e sia $f : V \to W$ lineare. Se $f$ e' iniettiva allora segue che $f(\bm{v_1}), \dots, f(\bm{v_n})$ sono linearmente indipendenti. 
\end{proposition}
\begin{proof}
    Consideriamo una combinazione lineare di $f(\bm{v_1}), \dots, f(\bm{v_n})$ e dimostriamo che imponendola uguale al vettore nullo segue che i coefficienti devono essere tutti nulli.
    \begin{alignat*}
        {1}
        &x_1f(\bm{v_1}) + \dots + x_nf(\bm{v_n}) = \bm{0_W}\\
        \iff &f(x_1\bm{v_1} + \dots + x_n\bm{v_n}) = \bm{0_W}\\
        \intertext{Per la proposizione \ref{ker_funzione_iniettiva} segue che}
        \iff &x_1\bm{v_1} + \dots + x_n\bm{v_n} = \bm{0_V}\\
        \intertext{Ma i vettori $\bm{v_1}, \dots, \bm{v_n}$ sono linearmente indipendenti, dunque l'unica combinazione lineare che li annulla e' quella a coefficienti nulli, cioe'}
        \iff &x_1 = \dots = x_n = 0
    \end{alignat*}
    Quindi una combinazione lineare di $f(\bm{v_1}), \dots, f(\bm{v_n})$ e' uguale al vettore nullo se e solo se tutti i coefficienti sono nulli, dunque $f(\bm{v_1}), \dots, f(\bm{v_n})$ sono linearmente indipendenti.
\end{proof}

\begin{definition}
    Siano $V, W$ spazi vettoriali e sia $f : V \to W$ lineare. Allora $f$ si dice surgettiva se per ogni $\bm{w} \in W$ esiste $\bm{v} \in V$ tale che $f(\bm{v}) = \bm{w}$.
\end{definition}

\begin{remark}
    Una funzione $f : V \to W$ e' surgettiva se e solo se $\Imm{f} = W$.
\end{remark}

\begin{corollary}\label{base_mappata_generatori_codominio}
    Sia $f : V \to W$ surgettiva. Allora se $\ang{\bm{v_1}, \dots, \bm{v_n}}$ e' una base di $V$ segue che $\left\{ f(\bm{v_1}), \dots, f(\bm{v_n})\right\}$ e' un insieme di generatori di $W$.
\end{corollary}
\begin{proof}
    Segue direttamente dalla proposizione \ref{base_mappata_generatori_immagine}: infatti se una funzione e' surgettiva allora $\Imm{f} = W$, dunque se $\left\{ f(\bm{v_1}), \dots, f(\bm{v_n})\right\}$ e' un insieme di generatori di $\Imm f$ segue che e' anche un insieme di generatori di $W$.
\end{proof}

\section{Isomorfismi}

\begin{definition}
    Siano $V, W$ spazi vettoriali e sia $f : V \to W$ lineare. Allora $f$ si dice bigettiva se $f$ e' sia iniettiva che surgettiva.
\end{definition}

\begin{definition}
    Una funzione $f : V \to W$ si dice invertibile se esiste $f^{-1} : W \to V$ tale che \begin{equation}
        f(\bm{v}) = \bm{w} \iff f^{-1}(\bm{w}) = \bm{v}
    \end{equation}
    Se $f$ e' invertibile allora $f^{-1}$ e' unica e si chiama inversa di $f$.
\end{definition}

\begin{remark}
    Un'applicazione lineare e' invertibile se e solo se e' bigettiva. 
\end{remark}

\begin{definition}
    Siano $V, W$ spazi vettoriali e sia $f : V \to W$ lineare. Allora se $f$ e' bigettiva si dice che $f$ e' un isomorfismo.
    
    Se esiste un isomorfismo tra gli spazi $V$ e $W$ allora si dice che $V$ e' isomorfo a $W$, e si indica con $V \cong W$.
\end{definition}

\begin{remark}
    Le seguenti affermazioni sono equivalenti:
    \begin{itemize}
        \item $f$ e' bigettiva;
        \item $f$ e' invertibile;
        \item $f$ e' un isomorfismo.
    \end{itemize}
\end{remark}

Gli isomorfismi preservano la linearita' dello spazio vettoriale e tutte le sue proprieta', come ci dicono le seguenti proposizioni.

\begin{proposition}
    Sia $V$ uno spazio vettoriale, $\alpha = \ang{\bm{v_1}, \dots, \bm{v_n}}$ una base di $V$. Allora se $f : V \to W$ e' un isomorfismo segue che $\beta = \ang{f(\bm{v_1}), \dots, f(\bm{v_n})}$ e' una base di $W$ (cioe' gli isomorfismi mappano basi in basi).
\end{proposition}
\begin{proof}
    Dato che $f$ e' un isomorfismo allora $f$ e' bigettiva.

    Dunque dato che $f$ e' iniettiva essa mappa un insieme di vettori indipendenti (come la base $\alpha$ di $V$) in un insieme di vettori linearmente indipendenti per la proposizione \ref{indipendenti_mappati_indipendenti}, dunque $\beta$ e' un insieme di vettori linearmente indipendenti. 

    Inoltre, dato che $f$ e' surgettiva, per la proposizione \ref{base_mappata_generatori_codominio} essa mappa una base di $V$ in un insieme di generatori del codominio $W$, dunque i vettori di $\beta$ generano $W$.

    Dunque $\beta$ e' un insieme di generatori linearmente indipendenti, e quindi e' una base di $W$.
\end{proof}

\begin{proposition}
    Se $V$ e' uno spazio vettoriale di dimensione $n = \dim V$, allora $V$ e' isomorfo a tutti e soli gli spazi vettoriali di dimensione $n$.
\end{proposition}
\begin{proof}
    Deriva direttamente dalla proposizione precedente: infatti ogni isomorfismo che ha come dominio $V$ deve portare una base di $V$ in una base di $W$, dunque la dimensione di $V$ deve essere uguale a quella di $W$.
\end{proof}

Quindi per calcolare una base di un sottospazio $W$ di uno spazio $V$ spesso conviene passare allo spazio dei vettori colonna $\R^n$ isomorfo allo spazio $V$, calcolare la base del sottospazio $\tilde{W}$ isomorfo a $W$ e infine tornare allo spazio di partenza.

\begin{example}
    Sia $V = \R[x]^{\leq 2}$ e sia $W \subseteq V$ il sottospazio di $V$ tale che $p \in W \iff p(2) = 0$. Dimostrare che $W$ e' un sottospazio e trovarne una base.
\end{example}
\begin{solution}
    Svolgiamo i due punti separatamente.
    \begin{enumerate}
        \item Dimostriamo che $W$ e' un sottospazio di $V$.
        \begin{itemize}
            \item Sia $\bm{0_V} \in V$ tale che $\bm{0_V}(x) = 0 + 0x + 0x^2$. Allora $\bm{0_V}(2) = 0 + 0\cdot 2 + 0 \cdot 4 = 0$, dunque $\bm{0_V} \in W$.
            \item Supponiamo $p, q \in W$ e mostriamo che $p+q \in W$. Dunque \[
                (p+q)(2) = p(2) + q(2) = 0 + 0 = 0    
            \] dunque $p + q \in W$.
            \item Supponiamo $p \in W$ e mostriamo che $kp \in W$ per un generico $k \in \R$. Dunque \[
                (kp)(2) = kp(2) = k \cdot 0 = 0    
            \] cioe' $kp \in W$ per ogni $k \in \R$.
        \end{itemize} 
        Dunque abbiamo dimostrato che $W$ e' un sottospazio di $V$.
        \item Cerchiamo ora una base per $W$.
         
        Consideriamo un generico $p \in V$, cioe' $p(x) = a + bx + cx^2$. La condizione che definisce $W$ e' $p(2) = a + 2b + 4c = 0$. 
    
        Passiamo ora allo spazio isomorfo $\R^3$. Il vettore corrispondente a $p$ in $\R^3$ e' $\bm{\tilde{p}} = \begin{psmallmatrix} a \\ b \\ c \end{psmallmatrix}$, mentre la condizione di appartenenza allo spazio $\widetilde{W} \subseteq \R^3$ isomorfo a $W$ e' sempre $a+2b+4c = 0$. Cerchiamo una base di $\widetilde{W}$ passando alla forma parametrica, cioe' cercando di esplicitare la condizione di appartenenza allo spazio e inserendola nella definizione stessa del vettore.
        Dato che la condizione e' data dal sistema $a+2b+4c = 0$ che ha due variabili libere, scelgo $b, c$ libere ottenendo $a = -2b - 4c$. Sostituendo in $\bm{\tilde{p}}$:\[
            \bm{\tilde p} = \begin{pmatrix} a \\ b \\ c \end{pmatrix} = \begin{pmatrix}
                -2b-4c\\b\\c
            \end{pmatrix} = b\begin{pmatrix} -2 \\ 1 \\ 0 \end{pmatrix} + c\begin{pmatrix} -4 \\ 0 \\ 1 \end{pmatrix}
        \]
        Dato che ogni vettore generico di $\widetilde{W}$ puo' essere scritto come combinazione lineare dei due vettori $\bm{\tilde{w}_1} = \begin{psmallmatrix} -2 \\ 1 \\ 0 \end{psmallmatrix}$ e $\bm{\tilde{w}_2} = \begin{psmallmatrix} -4 \\ 0 \\ 1 \end{psmallmatrix}$, allora segue che essi sono generatori di $W$. 
            
        Controlliamo ora che siano linearmente indipendenti riducendo a scalini per riga (secondo la proposizione \ref{estrarre_una_base}) la matrice che ha come colonne $\bm{\tilde{w}_1}$ e $\bm{\tilde{w}_2}$.
        \begin{equation*}
            \begin{pmatrix}[c|c]
                -2 & -4 \\ 1 & 0 \\ 0 & 1
            \end{pmatrix} \xrightarrow[]{R_2 + \frac12R_1}
            \begin{pmatrix}[c|c]
                -2 & -4 \\ 0 & -2 \\ 0 & 1
            \end{pmatrix} \xrightarrow[]{R_3 + \frac12R_2}
            \begin{pmatrix}[c|c]
                -2 & -4 \\ 0 & -2 \\ 0 & 0
            \end{pmatrix}
        \end{equation*}
        Dato che ci sono tanti pivot quante colonne segue che tutti i vettori originali sono indipendenti.
        I vettori $\bm{\tilde{w}_1}$ e $\bm{\tilde{w}_2}$ sono quindi indipendenti e generano $\widetilde{W}$: segue che $\ang{\bm{\tilde{w}_1}, \bm{\tilde{w}_1}}$ e' una base di $\widetilde{W}$, quindi $\dim \widetilde{W} = 2$.

        Tornando allo spazio originale, i vettori corrispondenti alla base sono quindi $w_1(x) = (-2 + x)$ e $w_2(x) = (-4 + x^2)$. L'insieme ordinato $\ang{(-2+x), (-4+x^2)}$ forma dunque una base di $W$ e dunque $\dim W = 2$.
    \end{enumerate}
\end{solution}

\section{Matrice associata ad una funzione}

Come avevamo visto nel primo capitolo, le matrici sono associate ad applicazioni lineari da vettori colonna in vettori colonna. Possiamo generalizzare questo concetto e definire una matrice associata ad ogni applicazione lineare.

\begin{definition}\label{matrice_associata}
    Siano $V, W$ spazi vettoriali, $f : V \to W$ lineare, $\alpha$ base di $V$ e $\beta$ base di $W$. Allora si dice chiama \textbf{matrice associata all'applicazione lineare} $f$ la matrice $[f]^{\alpha}_{\beta}$ tale che
    \begin{equation}
        \forall \bm{v} \in V. \quad  [f]^{\alpha}_{\beta} \cdot (\bm{v})_{\alpha}= \left(f(\bm{v}) \right)_{\beta}.
    \end{equation}
    Ovvero se $f$ mappa $\bm{v}$ a $\bm{w}$ allora $[f]^{\alpha}_{\beta}$ e' una matrice che porta (tramite il prodotto) il vettore colonna delle coordinate di $\bm{v}$ rispetto ad una base $\alpha$ nel vettore colonna delle coordinate di $\bm{w}$ rispetto ad una base $\beta$.
\end{definition}

Per trovare la matrice associata ad $f$ rispetto alle basi $\alpha = \ang{\bm{v_1}, \dots, \bm{v_n}}$ e $\beta = \ang{\bm{w_1}, \dots, \bm{w_m}}$ possiamo seguire questo procedimento:
\begin{itemize}
    \item calcoliamo $f(\bm{v_1}) = \bm{u_1}, \dots, f(\bm{v_n}) = \bm{u_n}$;
    \item scriviamo $\bm{u_i}$ in termini della base $\beta$, cioe' \begin{equation*}
        \bm{u_i} = a_{1i}\bm{w_1} + \dots + a_{mi}\bm{w_m} \iff [\bm{u_i}]_{\beta} = \begin{pmatrix}
            a_{1i} \\ \vdots \\ a_{mi}
        \end{pmatrix};
    \end{equation*}
    \item notiamo che dato che $\bm{v_i}$ e' l'$i$-esimo vettore della base $\alpha$, allora la sua rappresentazione in termini della base sara' un vettore colonna con tutti $0$ tranne un $1$ in posizione $i$;
    \item per la proposizione \ref{j-esima_colonna} il risultato del prodotto $[f]^{\alpha}_{\beta} \cdot (\bm{v_i})_{\alpha}$ sara' l'$i$-esima colonna della matrice $[f]^{\alpha}_{\beta}$, ma $[f]^{\alpha}_{\beta} \cdot (\bm{v_i})_{\alpha}$ deve essere uguale a $[f(v_i)]_{\beta} = [\bm{u_i}]_{\beta}$, dunque l'$i$-esima colonna di $[f]^{\alpha}_{\beta}$ sara' data dal vettore colonna $[\bm{u_i}]_{\beta}$;
    \item dunque la matrice avra' per colonne i vettori $[\bm{u_1}]_{\beta}, \dots, [\bm{u_n}]_{\beta}$, cioe'
    \begin{equation}
        [f]^{\alpha}_{\beta} = \begin{pmatrix}
            a_{11} & a_{12} & \dots & a_{1n} \\
            a_{21} & a_{22} & \dots & a_{2n} \\
            \vdots & \vdots & \vdots& \vdots \\
            a_{m1} & a_{m2} & \dots & a_{mn} \\
        \end{pmatrix}.
    \end{equation}
\end{itemize}


\begin{example}
    Sia $V = \M_{2\times 2}(\R)$ e sia $\alpha = \ang{\begin{psmallmatrix}1&0\\0&0\end{psmallmatrix}, \begin{psmallmatrix}0&1\\0&0\end{psmallmatrix}, \begin{psmallmatrix}0&0\\1&0\end{psmallmatrix}, \begin{psmallmatrix}0&0\\0&1\end{psmallmatrix}}$ una sua base.    
    Sia $A = \begin{psmallmatrix}0&1\\1&0\end{psmallmatrix} \in V$ e $f : V \to V$ tale che $f(B) = AB - BA$.
    \begin{enumerate}
        \item Dimostrare che $f$ e' lineare.
        \item Calcolare $[f]^{\alpha}_{\alpha}$.
        \item Dare una base di $\Imm{f}$.
        \item Dare una base di $\ker f$.
    \end{enumerate}
\end{example}
\begin{solution}
    Verifichiamo i quattro punti.
    \begin{enumerate}
        \item Dimostriamo che $f$ e' lineare.
            \begin{alignat*}{2}
                &\text{(a) } f(\bm{0}) = f\left(\begin{psmallmatrix}0&0\\0&0\end{psmallmatrix}\right) = A\begin{psmallmatrix}0&0\\0&0\end{psmallmatrix} - \begin{psmallmatrix}0&0\\0&0\end{psmallmatrix}A = \begin{psmallmatrix}0&0\\0&0\end{psmallmatrix} - \begin{psmallmatrix}0&0\\0&0\end{psmallmatrix} = \bm 0 \\
                &\begin{alignedat}{1}
                    \text{(b) } f(B + C) &= A(B + C) - (B + C)A \\
                    &= AB + AC - BC - CA \\
                    &= (AB - BA) + (AC - CA) \\
                    &= f(B) + f(C)
                \end{alignedat}\\
                &\begin{alignedat}{1}
                    \text{(c) } f(kB) &= A(kB) - (kB)A \\
                    &= k(AB) - k(BA) \\
                    &= k(AB - BA) \\
                    &= kf(B)
                \end{alignedat}
            \end{alignat*}
            dunque $f$ e' lineare.
        \item Seguo il procedimento per ottenere $[f]^{\alpha}_{\alpha}$. Innanzitutto calcolo il risultato di $f$ sui vettori della base $\alpha$:
        \begin{alignat*}{1}
            &f\left(\begin{psmallmatrix}1&0\\0&0\end{psmallmatrix}\right) = 
            \begin{psmallmatrix}0&1\\1&0\end{psmallmatrix}\begin{psmallmatrix}1&0\\0&0\end{psmallmatrix} - \begin{psmallmatrix}1&0\\0&0\end{psmallmatrix}\begin{psmallmatrix}0&1\\1&0\end{psmallmatrix} = \begin{psmallmatrix}0&0\\1&0\end{psmallmatrix} - \begin{psmallmatrix}0&1\\0&0\end{psmallmatrix} = \begin{psmallmatrix}0&-1\\1&0\end{psmallmatrix} \\
            &f\left(\begin{psmallmatrix}0&1\\0&0\end{psmallmatrix}\right) = 
            \begin{psmallmatrix}0&1\\1&0\end{psmallmatrix}\begin{psmallmatrix}0&1\\0&0\end{psmallmatrix} - \begin{psmallmatrix}0&1\\0&0\end{psmallmatrix}\begin{psmallmatrix}0&1\\1&0\end{psmallmatrix} = \begin{psmallmatrix}0&0\\0&1\end{psmallmatrix} - \begin{psmallmatrix}1&0\\0&0\end{psmallmatrix} = \begin{psmallmatrix}-1&0\\0&1\end{psmallmatrix} \\
            &f\left(\begin{psmallmatrix}0&0\\1&0\end{psmallmatrix}\right) = 
            \begin{psmallmatrix}0&1\\1&0\end{psmallmatrix}\begin{psmallmatrix}0&0\\1&0\end{psmallmatrix} - \begin{psmallmatrix}0&0\\1&0\end{psmallmatrix}\begin{psmallmatrix}0&1\\1&0\end{psmallmatrix} = \begin{psmallmatrix}1&0\\0&0\end{psmallmatrix} - \begin{psmallmatrix}0&0\\0&1\end{psmallmatrix} = \begin{psmallmatrix}1&0\\0&-1\end{psmallmatrix} \\
            &f\left(\begin{psmallmatrix}0&0\\0&1\end{psmallmatrix}\right) = 
            \begin{psmallmatrix}0&1\\1&0\end{psmallmatrix}\begin{psmallmatrix}0&0\\0&1\end{psmallmatrix} - \begin{psmallmatrix}0&0\\0&1\end{psmallmatrix}\begin{psmallmatrix}0&1\\1&0\end{psmallmatrix} = \begin{psmallmatrix}0&1\\0&0\end{psmallmatrix} - \begin{psmallmatrix}0&0\\1&0\end{psmallmatrix} = \begin{psmallmatrix}0&1\\-1&0\end{psmallmatrix}
        \end{alignat*}
        dunque le loro coordinate rispetto alla base $\alpha$ sono
        \begin{align*}
            &[f(\bm{v_1})]_{\alpha} = \begin{pmatrix}
                0 \\ -1 \\ 1 \\ 0
            \end{pmatrix} &[f(\bm{v_2})]_{\alpha} = \begin{pmatrix}
                -1 \\ 0 \\ 0 \\ 1
            \end{pmatrix}
            \\&[f(\bm{v_3})]_{\alpha} = \begin{pmatrix}
                1 \\ 0 \\ 0 \\ -1
            \end{pmatrix} &[f(\bm{v_4})]_{\alpha} = \begin{pmatrix}
                0 \\ 1 \\ -1 \\ 0
            \end{pmatrix}
        \end{align*}
        cioe' \begin{equation*}
            [f]^{\alpha}_{\alpha} = \begin{pmatrix}
                0 & -1 & 1 & 0 \\ -1 & 0 & 0 & 1 \\
                1 & 0 & 0 & -1 \\ 0 & 1 & -1 & 0
            \end{pmatrix}.
        \end{equation*}
        \item Per la proposizione \ref{base_mappata_generatori_immagine} sappiamo che l'insieme
        $\{f(\bm{v_1}), f(\bm{v_2}), f(\bm{v_3}), f(\bm{v_4})\}$ e' un insieme di generatori dell'immagine della funzione. 
        Per eliminare i vettori indipendenti passiamo all'isomorfismo con $\R^4$ tramite la base di partenza $\alpha$. Chiamiamo $\widetilde{W}$ lo spazio isomorfo a $\Imm{f}$, allora notiamo che il ruolo di $f$ nel nuovo spazio e' dato dalla matrice $[f]^{\alpha}_{\alpha}$, dunque il corrispondente insieme di generatori di $\widetilde{W}$ sara' \begin{gather*}
            \left\{ [f]^{\alpha}_{\alpha}[v_1]_{\alpha}, [f]^{\alpha}_{\alpha}[v_2]_{\alpha}, [f]^{\alpha}_{\alpha}[v_3]_{\alpha}, [f]^{\alpha}_{\alpha}[v_4]_{\alpha}\right\} \\
            \intertext{che e' uguale a }
            \left\{\begin{psmallmatrix} 0 \\ -1 \\ 1 \\ 0 \end{psmallmatrix}, \begin{psmallmatrix} -1 \\ 0 \\ 0 \\ 1 \end{psmallmatrix}, \begin{psmallmatrix} 1 \\ 0 \\ 0 \\ -1 \end{psmallmatrix}, \begin{psmallmatrix} 0 \\ 1 \\ -1 \\ 0 \end{psmallmatrix}\right\}  
        \end{gather*}
        Semplifichiamolo tramite mosse di colonna: \begin{gather*}
            \begin{pmatrix}[c|c|c|c]
                0 & -1 & 1 & 0 \\ -1 & 0 & 0 & 1 \\
                1 & 0 & 0 & -1 \\ 0 & 1 & -1 & 0
            \end{pmatrix} \xrightarrow[R_2 + R_3]{R_1 + R_4}
            \begin{pmatrix}[c|c|c|c]
                0 & 0 & 1 & 0 \\ 0 & 0 & 0 & 1 \\
                0 & 0 & 0 & -1 \\ 0 & 0 & -1 & 0
            \end{pmatrix} \xrightarrow[]{}
            \begin{pmatrix}[c|c|c|c]
                1 & 0 & 0 & 0\\0 & 1 & 0 & 0 \\
                0 & -1 & 0 & 0\\ -1 & 0 & 0 & 0
            \end{pmatrix}
        \end{gather*}
        dunque i vettori $\begin{psmallmatrix} 1 \\ 0 \\ 0 \\ -1 \end{psmallmatrix}, \begin{psmallmatrix} 0 \\ 1 \\ -1 \\ 0 \end{psmallmatrix}$ sono indipendenti e generano $\widetilde{W}$, dunque sono una base di $\widetilde{W}$.

        Tornando allo spazio originale otteniamo che una base di $\Imm{f}$ e' data da \[
            \beta = \ang{\begin{pmatrix} 1 & 0 \\ 0 & -1 \end{pmatrix}, \begin{pmatrix} 0 & 1 \\ -1 & 0 \end{pmatrix}}.
        \] e dunque $\dim \Imm{f} = 2$.
        \item Per definizione di kernel \[
            \ker f = \left\{ \begin{psmallmatrix}x&y\\z&t \end{psmallmatrix}\in \M_{2\times 2}(\R) \mid f\left(\begin{psmallmatrix}x&y\\z&t \end{psmallmatrix}\right) = \begin{psmallmatrix}0&0\\0&0 \end{psmallmatrix}\right\}.
        \] Sia $\widetilde{V} \subseteq \R^4$ lo spazio isomorfo a $\ker f$ tramite l'isomorfismo dato dalle coordinate dei vettori rispetto alla base $\alpha$. Allora \begin{alignat*}{1}
            \widetilde{V} &= \left\{ \begin{pmatrix}x\\y\\z\\t \end{pmatrix}\in \R^4 \mid [f]^{\alpha}_{\alpha}\begin{pmatrix}x\\y\\z\\t \end{pmatrix} = \begin{pmatrix}0\\0\\0\\0 \end{pmatrix}\right\} \\
            &= \left\{ \begin{pmatrix}x\\y\\z\\t \end{pmatrix}\in \R^4 \mid \begin{pmatrix}
                0 & -1 & 1 & 0 \\ -1 & 0 & 0 & 1 \\
                1 & 0 & 0 & -1 \\ 0 & 1 & -1 & 0
            \end{pmatrix}\begin{pmatrix}x\\y\\z\\t \end{pmatrix} = \begin{pmatrix}0\\0\\0\\0 \end{pmatrix}\right\}
        \end{alignat*}
        Dunque $\widetilde{V}$ e' formato da tutti e solo i vettori $\bm{x} \in \R^4$ che sono soluzione del sistema lineare $[f]^{\alpha}_{\alpha}\bm{x} = \bm{0}$. Risolviamolo tramite eliminazione gaussiana:
        \begin{gather*}
            \begin{pmatrix}
                0  & -1 & 1  & 0  \\ 
                -1 & 0  & 0  & 1  \\
                1  & 0  & 0  & -1 \\ 
                0  & 1  & -1 & 0
            \end{pmatrix} \xrightarrow[]{scambio}
            \begin{pmatrix}
                1  & 0  & 0  & -1 \\
                0  & 1  & -1 & 0  \\
                0  & -1 & 1  & 0  \\ 
                -1 & 0  & 0  & 1
            \end{pmatrix} \xrightarrow[R_4 + R_1]{R_3 + R_2} \\
            \xrightarrow[R_4 + R_1]{R_3 + R_2} \begin{pmatrix}
                1  & 0  & 0  & -1 \\
                0  & 1  & -1 & 0  \\
                0  & 0  & 0  & 0  \\ 
                0  & 0  & 0  & 0
            \end{pmatrix} \iff \left\{
                \begin{array}{@{}roror }
                    x & - & t & = & 0 \\
                    y & - & z & = & 0
                \end{array}
            \right. \iff \left\{
                \begin{array}{@{}ror }
                    x & = & t\\
                    y & = & z
                \end{array}
            \right.\\
            \intertext{dunque scegliendo $z, t \in \R$ libere otteniamo}
            \iff \begin{pmatrix} x \\ y \\ z \\t \end{pmatrix} = \begin{pmatrix} t \\ z \\ z \\ t \end{pmatrix} = z\begin{pmatrix} 0 \\ 1 \\ 1 \\ 0 \end{pmatrix} + t\begin{pmatrix} 1 \\ 0 \\ 0 \\ 1 \end{pmatrix}
        \end{gather*}
        Dunque $\tilde{\gamma} = \ang{\begin{psmallmatrix} 0 \\ 1 \\ 1 \\ 0 \end{psmallmatrix}, \begin{psmallmatrix} 1 \\ 0 \\ 0 \\ 1 \end{psmallmatrix}}$ e' un insieme di generatori di $\widetilde{V}$. Inoltre sono anche indipendenti (poiche' hanno pivot ad altezze diverse), dunque $\tilde{\gamma}$ e' una base di $\widetilde{V}$. 
            
        Tornando tramite la base $\alpha$ allo spazio iniziale otteniamo che \[
            \gamma = \ang{\begin{pmatrix} 0 & 1 \\ 1 & 0 \end{pmatrix}, \begin{pmatrix} 1 & 0 \\ 0 & 1 \end{pmatrix}}  
        \] e' una base di $\ker f$, dunque $\dim \ker f = 2$.
    \end{enumerate}
\end{solution}

\subsection{Inversa di una matrice}

\begin{definition}
    Sia $A \in \M_{n \times n}(\R)$ una matrice quadrata. Allora se $A$ e' invertibile si dice che la sua inversa e' la matrice $A^{-1}$ tale che \[
        AA^{-1} = A^{-1}A = I_n.
    \]
\end{definition}

Una matrice e' invertibile se e solo se la sua applicazione lineare associata e' invertibile, ovvero se e solo se e' bigettiva.

\begin{proposition}
    Siano $V, W$ spazi vettoriali e sia $f : V \to W$. Allora vale che \[
        [f]^{-1} = [f^{-1}]    
    \] ovvero l'inversa della matrice associata ad $f$ e' la matrice associata all'inversa di $f$.
\end{proposition}

Possiamo sfruttare le inverse per risolvere sistemi lineari: \[
    A\bm x = \bm b \iff A^{-1}A\bm x = A^{-1}\bm b \iff \bm x = A^{-1}\bm b.
\]

\subsubsection{Metodo di Gauss per l'inversa}
Possiamo trovare la matrice inversa di una matrice $A \in  \M_{n \times n}(\R)$ in questo modo. 

Consideriamo la matrice $n \times 2n$ formata affiancando alla matrice originale la matrice identita' $n \times n$, e indichiamola con $[A | I_n]$. Tramite mosse di riga riduciamo questa matrice alla forma $[I_n | B]$: allora $B = A^{-1}$.

Questa strategia funziona perche' risolvere il sistema $[A | I_n]$ tramite mosse di Gauss-Jordan e' equivalente a cercare una soluzione della seguente equazione matriciale: \[
    A\begin{pmatrix}
        b_{11} & \dots & b_{1n} \\
        \vdots & \vdots & \vdots \\
        b_{n1} & \dots & b_{nn} \\
    \end{pmatrix} = I_n.
\] Dato che la matrice inversa e' unica, allora la matrice $B$ ottenuta in questo modo dovra' essere l'inversa di $A$.

\begin{proposition}
    \label{inversa_prodotto}
    Siano $A, B$ tali che $AB$ e' invertibile. Allora vale che $(AB)^{-1} = B^{-1}A^{-1}$.
\end{proposition}
\begin{proof}
    Per definizione di inversa deve valere che \begin{alignat*}
        {1}
        (AB)^{-1}(AB) = I_n \\
        \iff (AB)^{-1}AB = I_n \\
        \iff (AB)^{-1}ABB^{-1} = B^{-1}\\
        \iff (AB)^{-1}A = B^{-1}\\
        \iff (AB)^{-1}AA^{-1} = B^{-1}A^{-1}\\
        \iff (AB)^{-1} = B^{-1}A^{-1}
    \end{alignat*} 
    che e' la tesi.
\end{proof}

\begin{example}
    Trovare l'inversa di \[
        A = \begin{pmatrix}
            2  & -1 & 0 \\
            -1 & 2  & -1 \\
            0  & -1 & 2 \\
        \end{pmatrix}. 
    \]
\end{example}
\begin{solution}
    Usiamo il metodo di Gauss.
    \begin{gather*}
        \begin{pmatrix}[ccc|ccc]
            2  & -1 & 0  & 1 & 0 & 0\\
            -1 & 2  & -1 & 0 & 1 & 0\\
            0  & -1 & 2  & 0 & 0 & 1\\
        \end{pmatrix} \xrightarrow[]{R_2 + \frac12 R_1}
        \begin{pmatrix}[ccc|ccc]
            2 & -1      & 0  & 1        & 0 & 0\\
            0 & \nicefrac32 & -1 & \nicefrac12  & 1 & 0\\
            0 & -1      & 2  & 0        & 0 & 1\\
        \end{pmatrix} \xrightarrow[]{R_3 + \frac23 R_2} \\
        \begin{pmatrix}[ccc|ccc]
            2 & -1      & 0         & 1        & 0       & 0\\
            0 & \nicefrac32 & -1        & \nicefrac12  & 1       & 0\\
            0 & 0       & \nicefrac43   & \nicefrac13  & \nicefrac23 & 1\\
        \end{pmatrix} \xrightarrow[]{R_2 + \frac34 R_3}
        \begin{pmatrix}[ccc|ccc]
            2 & -1      & 0         & 1        & 0       & 0\\
            0 & \nicefrac32 & 0         & \nicefrac34  & \nicefrac32 & \nicefrac34\\
            0 & 0       & \nicefrac43   & \nicefrac13  & \nicefrac23 & 1\\
        \end{pmatrix} \xrightarrow[]{R_1 + \frac23 R_2}\\
        \begin{pmatrix}[ccc|ccc]
            2 & 0       & 0         & \nicefrac32  & 1       & \nicefrac12\\
            0 & \nicefrac32 & 0         & \nicefrac34  & \nicefrac32 & \nicefrac34\\
            0 & 0       & \nicefrac43   & \nicefrac13  & \nicefrac23 & 1\\
        \end{pmatrix} \xrightarrow[\frac34 \times R_3]{\frac12 \times R_1, \nicefrac23 \times R_2}
        \begin{pmatrix}[ccc|ccc]
            1 & 0 & 0 & \nicefrac34 & \nicefrac12 & \nicefrac14\\
            0 & 1 & 0 & \nicefrac12 & 1       & \nicefrac12\\
            0 & 0 & 1 & \nicefrac14 & \nicefrac12 & \nicefrac34 \\
        \end{pmatrix} \\
        \implies A^{-1} = \begin{pmatrix}
            \nicefrac34 & \nicefrac12 & \nicefrac14\\
            \nicefrac12 & 1       & \nicefrac12\\
            \nicefrac14 & \nicefrac12 & \nicefrac34 \\
        \end{pmatrix}
    \end{gather*}
\end{solution}

\subsubsection{Inversa di una matrice $2 \times 2$}

Per trovare l'inversa di una matrice $2 \times 2$ possiamo usare questa formula:
\[
    A = \begin{pmatrix}
        a & b \\
        c & d \\
    \end{pmatrix} \implies 
    A^{-1} = \frac{1}{ad-bc}\begin{pmatrix}
        d  & -b \\
        -c & a \\
    \end{pmatrix}.
\] Il numero $ad - bc$ e' il determinante della matrice $2 \times 2$ e analizzeremo il suo significato in un capitolo successivo.

Ad esempio \[
    A = A = \begin{pmatrix}
        3 & 2 \\
        4 & 7 \\
    \end{pmatrix} \implies 
    A^{-1} = \frac{1}{3 \cdot 7 - 2 \cdot 4}\begin{pmatrix}
        7  & -2 \\
        -4 & 3 \\
    \end{pmatrix} = \frac{1}{13}\begin{pmatrix}
        7  & -2 \\
        -4 & 3 \\
    \end{pmatrix}.    
\]

\subsection{Composizione di funzioni come moltiplicazione tra matrici}

Consideriamo tre spazi vettoriali $U, V, W$ e due funzioni $f, g$ tali che \[
    U \xmapsto{f} V \xmapsto{g} W  \quad  \iff  \quad  U \xmapsto{g \circ f} W
\] dunque per ogni $u \in U$ segue che $(g \circ f)(u) = g(f(u))$. 

Consideriamo ora $\alpha$ base di $U$, $\beta$ base di $V$ e $\gamma$ base di $W$ e cerchiamo di rappresentare la relazione tra gli insiemi tramite le matrici associate alle funzioni: \[
    U_{\alpha} \xmapsto{[f]^{\alpha}_{\beta}} V_{\beta} \xmapsto{[g]^{\beta}_{\gamma}} W_{\gamma}  \quad  \iff  \quad  U_{\alpha} \xmapsto{[g \circ f]^{\alpha}_{\gamma}} W_{\gamma}.
\] Dunque per lo stesso ragionamento di prima dovra' valere che \[
    [g \circ f]^{\alpha}_{\gamma}\cdot (u)_{\alpha} = [g]^{\beta}_{\gamma} \cdot \left( [f]^{\alpha}_{\beta} \cdot (u)_{\alpha} \right).
\]

\begin{theorem}
    Siano $U,\ V,\ W$ spazi vettoriali e siano $\alpha$, $\beta$ e $\gamma$ delle basi rispettivamente di $U,\ V$ e $W$. Siano $f : U \to V$, $g : V \to W$ e siano $[f]{\alpha}_{\beta}$ e $[g]^{\beta}_{\gamma}$ le matrici associate a $f$ e $g$.

    Allora vale che \begin{equation}
        [g \circ f]^{\alpha}_{\gamma} = [g]^{\beta}_{\gamma} \cdot [f]^{\alpha}_{\beta}.
    \end{equation}
\end{theorem}
\begin{proof}
    Sia $u \in U$ generico e dimostriamo che $[g \circ f]^{\alpha}_{\gamma} \cdot (u)_{\alpha} = [g]^{\beta}_{\gamma} \cdot [f]^{\alpha}_{\beta} \cdot (u)_{\alpha}$, ricordando la definizione di matrice associata ad una funzione (\ref{matrice_associata}).
    \begin{align*}
        [g]^{\beta}_{\gamma} \cdot [f]^{\alpha}_{\beta} \cdot (u)_{\alpha} &= [g]^{\beta}_{\gamma} \cdot ([f]^{\alpha}_{\beta} \cdot (u)_{\alpha}) \\
        &= [g]^{\beta}_{\gamma} \cdot (f(u))_{\beta} &&\text{per def. di matrice associata ad $f$}\\
        &= (g(f(u)))_{\gamma} &&\text{per def. di matrice associata a $g$}\\
        &= ((g \circ f)(u))_{\gamma}\\
        &= [g \circ f]^{\alpha}_{\gamma} \cdot (u)_{\alpha} &&\text{per def. di matrice associata a $g \circ f$.}
    \end{align*}
    Dunque dato che la relazione vale per qualunque $u \in U$, allora segue che \[
        [g \circ f]^{\alpha}_{\gamma} = [g]^{\beta}_{\gamma} \cdot [f]^{\alpha}_{\beta}    
    \] che e' la tesi.
\end{proof}

\subsection{Matrice del cambiamento di base}

\begin{definition}
    Sia $V$ uno spazio vettoriale. Allora si dice funzione identita' la funzione $\id_V : V \to V$ tale che \[
        \forall v \in V. \quad \id_V(v) = v.
    \]
\end{definition}

Scriveremo sempre $\id$ sottointendendo lo spazio di riferimento.

Possiamo costruire una matrice che trasforma le coordinate di un vettore rispetto ad una base nelle coordinate di un vettore rispetto ad un'altra base sfruttando la funzione identita'.

\begin{definition}
    Sia $V$ uno spazio vettoriale, $\alpha$ e $\beta$ basi di $V$. 
    
    Allora dato un vettore $v \in V$ la matrice $[\id]^{\alpha}_{\beta}$ porta le coordinate di $v$ rispetto ad $\alpha$ nelle coordinate di $v$ rispetto a $\beta$, ovvero \begin{equation}
        [\id]^{\alpha}_{\beta} \cdot (v)_{\alpha} = (v)_{\beta}.
    \end{equation}
    $[\id]^{\alpha}_{\beta}$ si dice \textbf{matrice del cambiamento di base}.
\end{definition}

\begin{example}
    Sia $\id : \R^2 \to \R^2$, sia $\mathcal{C}$ la base canonica di $\R^2$ e sia \[
        \alpha = \ang{\begin{pmatrix} 1 \\ 3 \end{pmatrix}, \begin{pmatrix} 2 \\ 1 \end{pmatrix}} = \ang{\bm{\alpha_1}, \bm{\alpha_2}}
    \] un'altra base di $\R^2$. 
    
    Calcolare $[\id]^{\alpha}_{\mathcal{C}}$ e $[\id]^{\mathcal{C}}_{\alpha}$.
\end{example}
\begin{solution} Scriviamo \[
        [\id]^{\alpha}_{\mathcal{C}} = \begin{pmatrix} a_{11} & a_{12} \\ a_{21} & a_{22} \end{pmatrix}.
    \] Allora:
    \begin{gather*}
        \begin{pmatrix} a_{11} \\ a_{21} \end{pmatrix} = [\id]^{\alpha}_{\mathcal{C}} [\bm{\alpha_1}]_{\alpha} = [\id(\bm{\alpha_1})]_{\mathcal{C}}= [\bm{\alpha_1}]_{\mathcal{C}} = \bm{\alpha_1}  \\
        \begin{pmatrix} a_{12} \\ a_{22} \end{pmatrix} = [\id]^{\alpha}_{\mathcal{C}} [\bm{\alpha_2}]_{\alpha} = [\id(\bm{\alpha_2})]_{\mathcal{C}}= [\bm{\alpha_2}]_{\mathcal{C}} = \bm{\alpha_2}
    \end{gather*}
    dunque \[
        [\id]^{\alpha}_{\mathcal{C}} = \begin{pmatrix} 1 & 2 \\ 3 & 1 \end{pmatrix}.
    \] Per calcolare la matrice che porta i vettori dalla base $\mathcal{C}$ alla base $\alpha$ basta notare che essa deve invertire il comportamento di $[\id]^{\alpha}_{\mathcal{C}}$, ovvero \[
        [\id]^{\mathcal{C}}_{\alpha} = ([\id]^{\alpha}_{\mathcal{C}})^{-1} = \frac{1}{1\cdot 1 - 2 \cdot 3} \begin{pmatrix} 1 & -2 \\ -3 & 1 \end{pmatrix} = -\frac{1}{5} \begin{pmatrix} 1 & -2 \\ -3 & 1 \end{pmatrix}.
    \]
\end{solution}

Le prossime proposizioni aiutano a scrivere piu' semplicemente le matrici di cambiamento di base e le matrici associate ad applicazioni lineari.

\begin{proposition}
    Sia $\alpha = \ang{\bm{\alpha_1}, \dots, \bm{\alpha_n}}$ una base di $\R^n$ e $\mathcal{C}$ la base canonica di $\R^n$. Allora la matrice di cambiamento di base $[\id]^{\alpha}_{\mathcal{C}}$ e' la matrice che ha come colonne $\bm{\alpha_1}, \dots, \bm{\alpha_n}$.
\end{proposition}
\begin{proof}
    Consideriamo $[\id]^{\alpha}_{\mathcal{C}} [\bm{\alpha_i}]_{\alpha}$. Allora \begin{alignat*}
        {1}
        [\id]^{\alpha}_{\mathcal{C}} [\bm{\alpha_i}]_{\alpha} &= [\id(\bm{\alpha_i})]_{\mathcal{C}} \\
        &= [\bm{\alpha_i}]_{\mathcal{C}}\\
        &= \bm{\alpha_i}
    \end{alignat*} dato che rispetto alle basi canoniche le coordinate del vettore $\bm{\alpha_i}$ e' proprio il vettore stesso.
    
    Cerchiamo ora di ricavare $[\bm{\alpha_i}]_{\alpha}$. Il vettore $\bm{\alpha_i}$ rispetto alla base $\alpha$ si puo' scrivere come \begin{alignat*}
        {1}
        &\bm{\alpha_i} = 0\bm{\alpha_1} + \dots + 1\bm{\alpha_i} + \dots + 0\bm{\alpha_n} \\
        \implies &[\bm{\alpha_i}]_{\alpha} = (a_{j1}) = \begin{cases}
            1 & \text{ se } j = i\\
            0 & \text{ altrimenti}\\
        \end{cases}
    \end{alignat*}
    cioe' il vettore $[\bm{\alpha_i}]_{\alpha}$ e' un vettore colonna con tutti zeri tranne un uno in posizione $i$.

    Dunque per la proposizione \ref{j-esima_colonna} il prodotto $[\id]^{\alpha}_{\mathcal{C}} [\bm{\alpha_i}]_{\alpha}$ deve dare la $i$-esima colonna della matrice $[\id]^{\alpha}_{\mathcal{C}}$, che deve essere quindi $\bm{\alpha_i}$.

    Dato che questo ragionamento vale per ogni $i$, la matrice di cambiamento di base dalla base $\alpha$ alla canonica ha come colonne i vettori $\bm{\alpha_1}, \dots, \bm{\alpha_n}$.
\end{proof}


\chapter{Determinanti}

\section{Definizione e significato del determinante}

\begin{definition}
    Sia $A$ una matrice quadrata $n \times n$ e siano $\vec{C_1}, \dots, \vec{C_n} \in \R^n$ le sue colonne. Allora si dice determinante una funzione \[
        \det : \M_{n \times n}(\R) \to \R 
    \] che rispetta le seguenti proprieta':

    \begin{enumerate}[(i)]
        \item $\det I_n = 1$, cioe' il determinante della matrice identita' $n \times n$ deve essere 1;
        \item se per qualche $i, j$ compresi tra $1$ e $n$, con $i \neq j$, vale che $\vec{C_i} = \vec{C_j}$, allora $\det A = 0$, cioe' se due colonne della matrice sono uguali il determinante deve essere 0;
        \item se $A'$ e' la matrice ottenuta moltiplicando una colonna della matrice $A$ per $\lambda \in \R$, cioe' $A' = \begin{pmatrix} \vec{C_1} & \dots & \lambda \vec{C_i} & \dots & \vec{C_n} \end{pmatrix}$ allora \[\det A' = \lambda \det A;\]
        \item se la colonna $\vec{C_i}$ e' esprimibile come $\vec{v_1} + \vec{v_2}$ con $\vec{v_1}, \vec{v_2} \in \R^n$, cioe' $A = \begin{pmatrix} \vec{C_1} & \dots & \vec{v_1} + \vec{v_2} & \dots & \vec{C_n} \end{pmatrix}$
        allora \[
            \det A = \det \begin{pmatrix} \vec{C_1} & \dots & \vec{v_1} & \dots & \vec{C_n} \end{pmatrix} + \det \begin{pmatrix} \vec{C_1} & \dots & \vec{v_2} & \dots & \vec{C_n} \end{pmatrix} 
        \] cioe' il determinante e' lineare nelle colonne della matrice.
    \end{enumerate}
\end{definition}

Il determinante di una matrice si indica anche con \[
    \det A = \abs{A}.    
\]

Possiamo anche definire il determinante come una funzione che prende esattamente $n$ vettori di $\R^n$ e restituisce un numero reale, cioe' $\det : (\R^n)^n \to \R$ e che rispetta le seguenti proprieta':
\begin{enumerate}
    [(i)]
    \item se $\vec{c_1}, \dots, \vec{c_n}$ sono i vettori della base standard di $\R^n$, allora \[\det(\vec{c_1}, \dots, \vec{c_n}) = 1;\]
    \item $\det(\vec{v_1}, \dots, \vec{v_i}, \dots, \vec{v_i}, \dots, \vec{v_n}) = 0$, cioe' se due dei vettori sono uguali allora il determinante e' nullo;
    \item $\det(\vec{v_1}, \dots, \lambda\vec{v_i}, \dots, \vec{v_n}) = \lambda \det (\vec{v_1}, \dots, \vec{v_i}, \dots, \vec{v_n})$;
    \item le somme escono fuori dal determinante, cioe' \begin{alignat*}{1}
        &\det(\vec{v_1}, \dots, \vec{v_i} + \vec{w}, \dots, \vec{v_n})\\
        = &\det (\vec{v_1}, \dots, \vec{v_i}, \dots, \vec{v_n}) + \det (\vec{v_1}, \dots, \vec{w}, \dots, \vec{v_n}).
    \end{alignat*}
\end{enumerate}

Dalle quattro proprieta' base ne discendono altre, che elenchiamo in questa proposizione:
\begin{proposition}
    Il determinante ha le seguenti proprieta':
    \begin{enumerate}
        [(i)]
        \item se scambio due colonne della matrice tra loro il determinante cambia segno;
        \item le combinazioni lineari escono fuori dal determinante;
        \item se una delle $n$ colonne e' combinazione lineare delle restanti, cioe' se gli $n$ vettori formano un insieme di vettori linearmente dipendenti, allora il determinante e' uguale a $0$;
        \item sommando ad una colonna un multiplo di un'altra colonna il determinante non cambia;
        \item il determinante di una matrice e' uguale al determinante della trasposta.
    \end{enumerate}
\end{proposition}

Notiamo che la mossa principale di Gauss-Jordan, cioe' sommare ad una colonna un multiplo di un'altra colonna, non modifica il determinante di una matrice: possiamo calcolare i determinanti quindi tramite mosse di Gauss-Jordan, facendo attenzione a cambiare il segno se scambiamo due colonne o a portare fuori i fattori per cui moltiplichiamo le colonne.

Dall'ultima proprieta' segue che ogni proprieta' che si basa sulle colonne puo' anche essere riformulata in termini delle righe della matrice (che corrispondono alle colonne della trasposta).

Dalle proprieta' precedenti segue che il determinante di una matrice e' $0$ se e solo se ci sono due colonne linearmente dipendenti: dunque il determinante e' una funzione che indica la dipendenza lineare tra i vettori a cui viene applicato.

\subsection{Determinante di matrici particolari}

\subsubsection{Determinante di matrici diagonali}

Consideriamo la matrice diagonale \[
    D = \begin{pmatrix}
        \lambda_1   &0          &\dots  &0 \\
        0           &\lambda_2  &\dots  &0 \\
        \vdots      &\vdots     &\ddots &\vdots \\
        0           &0          &\dots  &\lambda_n \\
    \end{pmatrix}.    
\] Applicando il terzo assioma possiamo estrarre i coefficienti di ogni colonna, ottenendo
\begin{alignat*}{1}
    \det \begin{pmatrix}
        \lambda_1   &0          &\dots  &0 \\
        0           &\lambda_2  &\dots  &0 \\
        \vdots      &\vdots     &\ddots &\vdots \\
        0           &0          &\dots  &\lambda_n \\
    \end{pmatrix} = &\lambda_1 \det \begin{pmatrix}
        1           &0          &\dots  &0 \\
        0           &\lambda_2  &\dots  &0 \\
        \vdots      &\vdots     &\ddots &\vdots \\
        0           &0          &\dots  &\lambda_n \\
    \end{pmatrix} \\
    = &\lambda_1 \lambda_2 \det \begin{pmatrix}
        1           &0          &\dots  &0 \\
        0           &1          &\dots  &0 \\
        \vdots      &\vdots     &\ddots &\vdots \\
        0           &0          &\dots  &\lambda_n \\
    \end{pmatrix} \\
    \intertext{Ripetendo il procedimento per ogni colonna arriviamo a}
    = &\lambda_1 \lambda_2 \dots \lambda_n \det I_n \\
    = &\lambda_1 \lambda_2 \dots \lambda_n.
\end{alignat*}
Dunque il determinante di una matrice diagonale e' il prodotto degli elementi sulla diagonale principale.

\subsubsection{Determinante di matrici triangolari superiori o inferiori}

Consideriamo una matrice triangolare superiore (o inferiore), cioe' una matrice che ha tutti zeri sotto (o sopra) la diagonale principale. Tramite mosse di Gauss-Jordan possiamo trasformare questa matrice in una matrice diagonale senza dover scambiare colonne tra di loro, dunque il determinante della matrice triangolare e' uguale al determinante della matrice diagonale, cioe' e' il prodotto degli elementi sulla sua diagonale principale.

\begin{equation*}
    \det \begin{pmatrix}
        \lambda_1   &\star      &\dots  &\star \\
        0           &\lambda_2  &\dots  &\star \\
        \vdots      &\vdots     &\ddots &\vdots \\
        0           &0          &\dots  &\lambda_n \\
    \end{pmatrix} = \det \begin{pmatrix}
        \lambda_1   &0          &\dots  &0 \\
        \star       &\lambda_2  &\dots  &0 \\
        \vdots      &\vdots     &\ddots &\vdots \\
        \star       &\star      &\dots  &\lambda_n \\
    \end{pmatrix} = \lambda_1 \lambda_2 \dots \lambda_n
\end{equation*}
dove $\star$ indica un qualsiasi numero reale.

\subsubsection{Determinante di matrici $2 \times 2$}

Consideriamo una matrice $A \in \M_{2 \times 2}(\R)$ generica e calcoliamone il determinante. Se $a \neq 0$ allora \begin{equation*}
    \det \begin{pmatrix} a & b \\ c & d \end{pmatrix} = \det \begin{pmatrix} a & b \\ 0 & d - \frac{c}{a}b \end{pmatrix} = ad - bc.
\end{equation*}
Se $a = 0$ allora \begin{equation*}
\det \begin{pmatrix} 0 & b \\ c & d \end{pmatrix} = -\det \begin{pmatrix} b & 0 \\ d & c \end{pmatrix} = -bc = 0d - bc = ad - bc.
\end{equation*}
Dunque il determinante di $\begin{psmallmatrix} a & b \\ c & d \end{psmallmatrix}$ e' $ad - bc$.

Il determinante di una matrice $2 \times 2$ e' l'area del parallelogramma che ha come lati i vettori che formano le colonne della matrice. Notiamo infatti che se i due vettori sono sulla stessa retta, cioe' se sono dipendenti, allora l'area del parallelogramma e' 0, esattamente come il determinante.

\subsubsection{Determinante di matrici $3 \times 3$}

Per calcolare il determinante di una matrice $A \in \M_{3 \times 3}(\R)$ generica possiamo usare la regola di Sarrus: creiamo una matrice $3 \times 5$ dove le ultime due colonne sono le prime due ripetute. Il determinante sara' allora la somma dei prodotti delle prime tre diagonali da sinistra verso destra meno il prodotto delle tre diagonali da destra verso sinistra.
Dunque se \[
    A = \begin{pmatrix}
        a & b & c \\
        d & e & f \\
        g & h & i \\
    \end{pmatrix}
\] consideriamo la matrice \[
    A = \begin{pmatrix}[ccc|cc]
        a & b & c & a & b \\
        d & e & f & d & e\\
        g & h & i & g & h\\
    \end{pmatrix}
\] e otteniamo che \[
    \det A = aei + bfg + cdh - bdi - afh - ceg.    
\]

Il determinante di una matrice $3 \times 3$ e' il volume del "parallelepipedo" che ha come lati i vettori che formano le colonne della matrice: infatti se un vettore e' nello span degli altri due allora il volume viene 0.

\section{Sviluppi di Laplace}

\begin{definition}
    Sia $A \in \M_{n\times m}(\R)$. Allora diciamo che $B$ e' una sottomatrice di $A$ se $B \in \M_{(n-k) \times (m-h)}(\R)$ e $B$ si ottiene eliminando $k$ righe e $h$ colonne di $A$.
\end{definition}

\begin{definition}
    Sia $A \in \M_{n\times m}(\R)$. Allora diciamo che $B$ e' un minore di $A$ se e' una sottomatrice quadrata di $A$.
\end{definition}

Possiamo quindi enunciare il metodo degli sviluppi di Laplace per calcolare il determinante di una matrice.

\begin{theorem}
    [Sviluppi di Laplace]
    Sia $A \in \M_{n \times n}(\R)$ una matrice quadrata. 
    
    Sia $C_j = \begin{pmatrix}
        a_{1j} & \dots & a_{nj}
    \end{pmatrix}^T$ una colonna qualsiasi di $A$. 
    
    Chiamo $M_{ij}$ il minore di $A$ ottenuto eliminando la riga $i$-esima e la colonna $j$-esima. Inoltre per ogni $i$ compreso tra 1 e $n$ chiamo cofattore $c_{ij}$ la quantita' \[
        c_{ij} = (-1)^{i+j} \det M_{ij}.
    \]
    Allora vale che \begin{equation}
        \det A = a_{1j}c_{1j} + \dots + a_{nj}c_{nj} = \sum_{i = 1}^n a_{ij}c_{ij}.
    \end{equation}
\end{theorem}

Il metodo funziona anche scegliendo una colonna invece di una riga.

\begin{example}
    Trovare il determinante di \[
    A = \begin{pmatrix}
        2 & 1 & 0 \\ 3 & 2 & 1 \\ 3 & 3 & 1
    \end{pmatrix}    
    \] tramite sviluppi di Laplace.
\end{example}
\begin{solution}
    Scegliamo come colonna la prima colonna.
    Allora \begin{alignat*}
        {1}
        \det A &= a_{13}(-1)^{1+3} \begin{vmatrix}  3 & 2 \\ 3 & 3 \end{vmatrix} + a_{23}(-1)^{2+3} \begin{vmatrix}  2 & 1 \\ 3 & 3 \end{vmatrix} + a_{33}(-1)^{3+3} \begin{vmatrix}  3 & 2 \\ 2 & 1 \end{vmatrix}\\
            &= -\begin{vmatrix}  2 & 1 \\ 3 & 3 \end{vmatrix} + \begin{vmatrix} 3 & 2 \\ 2 & 1 \end{vmatrix} \\
            &= -(2 \cdot 3 - 1 \cdot 3) + (2 \cdot 2 - 1 \cdot 3)\\
            &= -3 + 6 + 4 - 3\\
            &= 4.
    \end{alignat*}
\end{solution}

\section{Rango e determinanti}

Diamo ora la definizione esatta di rango di una matrice.

\begin{definition}
    Sia $A \in \M_{n \times m}(\R)$ e sia $L_A : \R^m \to \R^n$ l'applicazione lineare associata alla matrice $A$. Allora si dice rango della matrice $A$ la dimensione dell'immagine dell'applicazione lineare associata, cioe'
    \begin{equation}
        \rk{A} = \dim \Imm{L_A}
    \end{equation}
\end{definition}

\begin{proposition}
    Sia $A \in \M_{n \times m}(\R)$. Siano $R_1, \dots, R_n \in \R^m$ le righe di $A$ e $C_1, \dots, C_m \in \R^n$ le colonne di $A$. Sia inoltre $L_A : \R^m \to \R^n$ l'applicazione lineare associata alla matrice $A$.
    
    Allora i seguenti fatti sono equivalenti:
    \begin{itemize}
        \item $k = \rk{A}$
        \item $k = \dim \Span{C_1, \dots, C_m}$;
        \item $k = \dim \Span{R_1, \dots, R_n}$;
        \item $k$ e' il numero di pivot della matrice a scalini $A'$ ottenuta tramite mosse di Gauss a partire dalla matrice $A$;
    \end{itemize}
\end{proposition}
\begin{proof}
    Se $k$ e' il rango della matrice, allora per definizione di rango $k = \dim \Imm{L_A}$. Per la proposizione \ref{span_colonne=immagine_applicazione_associata} lo span delle colonne della matrice e' uguale all'immagine dell'applicazione lineare associata, dunque $k = \dim \Span{C_1, \dots, C_m}$.

    Supponiamo che la matrice sia a scalini per colonne. Allora le colonne indipendenti sono tutte e solo le colonne con i pivot, mentre le altre colonne sono nulle. Dato che le colonne con i pivot formano una base dell'immagine, segue che devono esserci esattamente $k = \rk{A}$ colonne indipendenti, e quindi $k$ pivot. 
    Inoltre le righe indipendenti sono quelle con i pivot, dunque devono esserci anche $k$ righe indipendenti, cioe' $\dim \Span{R_1, \dots, R_n} = k$.

    Supponiamo che la matrice non sia a scalini per colonne. Allora riduciamola a scalini per colonne tramite mosse di Gauss, ottenendo la matrice $A'$ che ha per colonne i vettori $C'_1, \dots, C'_m$ e per righe i vettori $R'_1, \dots, R'_n$.
    Per la proposizione \ref{span_colonne_indipendenti} segue che $\Span{C'_1, \dots, C'_m} = \Span{C_1, \dots, C_m}$, dunque anche le loro dimensioni saranno uguali. Dato che la dimensione di $\Span{C'_1, \dots, C'_m}$ e' data dal numero di colonne indipendenti, cioe' dal numero di pivot per colonna, abbiamo dimostrato che il numero di pivot e' uguale alla dimensione di $\Span{C_1, \dots, C_m}$, cioe' al rango di $A$. Infine per la proposizione \ref{invarianza_dim_righe_per_mosse_colonna} ridurre una matrice a scalini per colonne non cambia la dimensione delle righe, dunque dato che la dimensione di $\Span{R'_1, \dots, R'_n}$ e' uguale al numero di pivot (cioe' $k$), allora anche la dimensione di $\Span{R_1, \dots, R_n}$ dovra' essere uguale a $k$.
\end{proof}

\begin{proposition}
    Sia $A \in \M_{n \times n}(\R)$ e sia $L_A : \R^n \to \R^n$ l'applicazione lineare associata ad $A$. Allora $L_A$ e' bigettiva se e solo se $\rk{A} = n$.
\end{proposition}
\begin{proof}
    Dire che $L_A$ e' bigettiva equivale a dire che per ogni $\vec b \in \R^n$ esiste uno e un solo vettore $\vec x$ tale che $L_A(\vec x) = \vec b$, cioe' che il sistema $A\vec x = \vec b$ ha una e una sola soluzione.

    Per la proposizione \ref{sistema_quadrato_n_pivot_unica_soluzione} questo avviene se e solo se il sistema ridotto a scalini ha $n$ pivot, cioe' $\rk{A} = n$. 
\end{proof}

Ora iniziamo ad analizzare la relazione tra determinanti e rango di una matrice.

\begin{theorem}\label{det_nonnullo_sse_rango_n}
    Sia $A \in \M_{n \times n}(\R)$. 
    
    Allora $\det A \neq 0$ se e solo se $\rk{A} = n$.
\end{theorem}
\begin{proof}
    Riduciamo $A$ a scalini tramite mosse di Gauss, senza moltiplicare le righe per coefficienti. La matrice ottenuta tramite questo processo, chiamiamola $S$, dovra' avere lo stesso determinante di $A$ a meno del segno: \[
        \abs{\det A} = \abs{\det S}.
    \] Dunque $\det A \neq 0$ se e solo se $\det S \neq 0$. 
    
    Dato che $S$ e' quadrata e a scalini segue che $S$ dovra' essere triangolare superiore, dunque il determinante di $S$ e' il prodotto degli elementi sulla diagonale principale, cioe' dei pivot di $S$.
    
    Dunque il determinante di $S$ e' diverso da $0$ se e solo se $S$ ha $n$ pivot, cioe' $\rk{S} = n$. Dato che le mosse di Gauss non cambiano il rango di una matrice, segue che $\det A \neq 0$ se e solo se $\rk{A} = n$, che e' la tesi.
\end{proof}

\begin{proposition}\label{rango_geq_k_sse_det_minore_k_neq_0}
    Sia $A \in \M_{n \times n}(\R)$.
    
    Allora $\rk{A} \geq k$ se e solo se esiste un minore $M_k$ di dimensione $k \times k$ tale che $\det M_k \neq 0$.
\end{proposition}
\begin{proof} Dimostriamo l'implicazione nei due versi.
    \begin{description}
        \item[($\implies$).] Dato che $\rk{A} \geq k$ dovranno esserci almeno $k$ righe e $k$ colonne indipendenti. 
        
        Scelgo $k$ righe indipendenti e elimino le altre, ottenendo una sottomatrice $S$ di dimensione $k \times n$ tale che $\rk{S} = k$. Dato che il rango e' anche il numero di colonne linearmente indipendenti, allora questa sottomatrice avra' anche almeno $k$ colonne linearmente indipendenti, dunque ne scelgo $k$ e elimino le altre.

        A questo punto ho ottenuto una sottomatrice $k \times k$ di $A$ con rango uguale a $k$, che e' la tesi.
        \item[($\impliedby$).] Per la proposizione \ref{det_nonnullo_sse_rango_n} segue che $\rk{B} = k$, dunque $B$ ha $k$ righe indipendenti. Consideriamo le relative $k$ righe di $A$: allora anche esse devono essere indipendenti, in quanto per annullare le prime $k$ componenti e' necessario che i coefficienti della combinazione lineare siano tutti uguali a $0$. Dunque $A$ ha almeno $k$ righe indipendenti, dunque $\rk{A} \geq k$. \qedhere
    \end{description}
\end{proof}

\begin{proposition}\label{rango_tramite_minori}
    Sia $A \in \M_{n \times n}(\R)$.
    Allora $\rk{A} = k$ se e solo se \begin{enumerate}[(i)]
        \item esiste almeno un minore $M_k$ di dimensione $k \times k$ tale che $\det M_k \neq 0$;
        \item per ogni minore $M_{k+1}$ di dimensione $(k + 1) \times (k + 1)$ vale che $\det M_{k+1} = 0$.
    \end{enumerate} 
\end{proposition}
\begin{proof}
    Dimostriamo l'implicazione nei due versi.
    \begin{description}
        \item[($\implies$).] Dato che $\rk{A} = k$ segue che $\rk{A} \geq k$, dunque per la proposizione precedente (\ref{rango_geq_k_sse_det_minore_k_neq_0}) esiste un minore di dimensione $k \times k$ con $\det M_k \neq 0$. 
        
        Dato che $\rk{A} = k$ segue che $\rk{A} < k+1$, dunque per la stessa proposizione non puo' esistere un minore di dimensione $(k+1) \times (k+1)$ con determinante diverso da $0$, cioe' $\det M_{k+1} = 0$ per ogni minore $M_{k+1}$ di dimensione $(k+1) \times (k+1)$.
        \item[($\impliedby$).] Per la proposizione precedente (\ref{rango_geq_k_sse_det_minore_k_neq_0}), dato che esiste almeno un minore con determinante non nullo di dimensione $k \times k$, allora $\rk{A} \geq k$. 
        
        Per la stessa proposizione, dato che non esistono minori di dimensioni $(k+1) \times (k+1)$ con determinante non nullo, segue che $\rk{A} < k+1$, che e' equivalente a $\rk{A} \leq k$.
        
        Dunque $\rk{A} = k$. \qedhere
    \end{description}
\end{proof}

Il seguente teorema riassume le varie definizioni di rango.

\begin{theorem}\label{equivalenza_definizioni_di_rango}
    Sia $A \in \M_{n \times m}(\R)$ e sia $L_A : \R^m \to \R^n$ l'applicazione lineare associata alla matrice $A$. Siano inoltre $R_1, \dots, R_n \in \R^m$ le righe di $A$ e $C_1, \dots, C_m \in \R^n$ le colonne di $A$. 
    
    Allora i seguenti fatti sono equivalenti:
    \begin{enumerate}
        [(i)]
        \item $k = \rk{A} = \dim \Imm L_A$;
        \item $k = \dim \Span{R_1, \dots, R_n}$, cioe' $k$ e' il numero di righe indipendenti di $A$;
        \item $k = \dim \Span{C_1, \dots, C_m}$, cioe' $k$ e' il numero di colonne indipendenti di $A$;
        \item $k$ e' il numero di pivot della matrice $S$ ottenuta riducendo a scalini la matrice $A$;
        \item $k$ e' il massimo numero tale che esiste $M_k \in \M_{k \times k}(\R)$ tale che $M_k$ e' un minore di $A$ con $\det M_k \neq 0$.
    \end{enumerate}
\end{theorem}

Il seguente teorema invece riassume le relazioni tra rango e determinante.

\begin{theorem}\label{relazioni_determinante_rango}
    Sia $A \in \M_{n \times n}(\R)$ e sia $L_A : \R^n \to \R^n$ l'applicazione lineare associata alla matrice $A$. Siano inoltre $R_1, \dots, R_n \in \R^n$ le righe di $A$ e $C_1, \dots, C_n \in \R^n$ le colonne di $A$. 
    
    Allora i seguenti fatti sono equivalenti:
    \begin{enumerate}
        [(i)]
        \item $\det A \neq 0$;
        \item $\rk{A} = n$;
        \item $L_A$ e' bigettiva (ovvero $\Imm L_A = \R^n$ e $\ker L_A = \{\vec 0\}$);
        \item $A$ e' invertibile;
        \item le righe di $A$ sono indipendenti, ovvero $\dim \Span{R_1, \dots, R_n} = n$;
        \item le colonne di $A$ sono indipendenti, ovvero $\dim \Span{C_1, \dots, C_n} = n$.
    \end{enumerate}
\end{theorem}

\end{document}
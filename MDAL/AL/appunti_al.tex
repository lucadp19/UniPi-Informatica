\documentclass[italian,oneside,headinclude,10pt]{scrbook}
    \usepackage[utf8]{inputenc}
    \usepackage[italian]{babel}
    \usepackage[T1]{fontenc}
    \usepackage{textcomp, microtype}
    \usepackage{amsmath, amsthm, amssymb, cases, mathtools, bm, enumerate}
    \usepackage{nicefrac}
    \usepackage{array}
    \usepackage{float}

    \usepackage{letltxmacro}

    \LetLtxMacro\amsproof\proof
    \LetLtxMacro\amsendproof\endproof

    
    \usepackage[pdfspacing]{classicthesis}
    % \usepackage{lmodern}
    
    \usepackage{thmtools}
    \usepackage[framemethod=TikZ]{mdframed}

    \usepackage{cleveref}
    \usepackage{hyperref} % ultimo package da caricare!

\restylefloat{table}

\AtBeginDocument{
    \LetLtxMacro\proof\amsproof
    \LetLtxMacro\endproof\amsendproof
}

\titleformat*{\chapter}{\LARGE\scshape}
\titleformat*{\section}{\Large\scshape}

\renewcommand*{\proofname}{\textsc{Dimostrazione}}
\addto\italian{\renewcommand*{\proofname}{\textsc{Dimostrazione}}}

\declaretheoremstyle[
    spaceabove=6pt, spacebelow=6pt,
    % headindent=6pt,
    headfont=\bfseries,
    notefont=\bfseries, notebraces={ (}{)},
    % headfont=\scshape,
    % notefont=\scshape, notebraces={ (}{)},
    bodyfont=\itshape\normalsize,
    shaded={rulecolor={rgb}{255,255,255}},
    % mdframed={backgroundcolor=halfgray},
    headpunct={\vspace{0.5\topsep}\newline}
]{thmstyle}
\declaretheorem[name=Teorema, numberwithin=section, style=thmstyle]{theorem}
\declaretheorem[name=Corollario, sibling=theorem, style=thmstyle]{corollary}
\declaretheorem[name=Proposizione, sibling=theorem, style=thmstyle]{proposition}
\declaretheorem[name=Lemma, sibling=theorem, style=thmstyle]{lemma}

\declaretheoremstyle[
    spaceabove=0pt, spacebelow=0pt,
    headfont=\bfseries,
    notefont=\normalfont, notebraces={ (}{)},
    % headfont=\scshape,
    % notefont=\scshape, notebraces={(}{)},
    bodyfont=\normalfont\normalsize,
    headpunct={\vspace{0.5\topsep}\newline},
    mdframed={
        linecolor=halfgray,
        linewidth=1pt,
        backgroundcolor=white,
        topline=false,
        bottomline=false,
        rightline=false
    },
]{defstyle}
\declaretheorem[name=Definizione, sibling=theorem, style=defstyle]{definition}

\declaretheoremstyle[
    headfont=\bfseries,
    notefont=\normalfont, notebraces={ (}{)},
    bodyfont=\normalfont,
    postheadspace=1em
]{exmplstyle}
\declaretheorem[name=Esempio, sibling=theorem, style=exmplstyle]{example}
\declaretheorem[name=Esercizio, sibling=theorem, style=exmplstyle]{exercise}


\declaretheoremstyle[
    headfont=\scshape,
    notefont=\normalfont, notebraces={(}{)},
    bodyfont=\normalfont,
    numbered=no,
    postheadspace=1em
]{remarkstyle}
\declaretheorem[name=Osservazione, style=remarkstyle]{remark}
\declaretheorem[name=Soluzione, style=remarkstyle]{solution}
\declaretheorem[name=Intuizione, style=remarkstyle]{intuition}

\DeclareMathOperator{\tc}{\ tale che\ }

\newcolumntype{z}{r<{{}}}
\newcolumntype{o}{@{}>{{}}c<{{}}@{}}

% matrix with lines
\makeatletter
\renewcommand*\env@matrix[1][*\c@MaxMatrixCols c]{%
  \hskip -\arraycolsep
  \let\@ifnextchar\new@ifnextchar
  \array{#1}}
\makeatother

\renewcommand{\vec}[1]{\bm{#1}}

\newcommand{\conj}[1]{\overline{#1}}
\newcommand{\divides}{\mid}
\newcommand{\Mod}[1]{\ \left(#1\right)}
\newcommand{\abs}[1]{\left|#1\right|}
\newcommand{\mcm}[2]{\operatorname{mcm}\left(#1, #2\right)}
\newcommand{\mcd}[2]{\operatorname{mcd}\left(#1, #2\right)}
\newcommand{\Span}[1]{\operatorname{span}\left\{#1\right\}}
\newcommand{\basis}[1]{\left\langle #1 \right\rangle}
\newcommand{\innerprod}[2]{\langle #1{,}#2\rangle}
\newcommand{\norm}[1]{\left\lVert#1\right\rVert}
\newcommand{\ortog}[1]{#1^{\perp}}
\newcommand{\proj}[2]{\operatorname{proj}_{#2}\!\left( #1 \right)}
\newcommand{\rk}[1]{\operatorname{rango}\left( #1 \right)}
\newcommand{\Imm}[1]{\operatorname{Im}#1}
\newcommand{\id}{\operatorname{id}}
\newcommand{\N}{\mathbb{N}}
\newcommand{\Z}{\mathbb{Z}}
\newcommand{\Q}{\mathbb{Q}}
\newcommand{\R}{\mathbb{R}}
\newcommand{\C}{\mathbb{C}}
\newcommand{\K}{\mathbb{K}}
\newcommand{\M}{\mathcal{M}}

\begin{document}

% \author{Luca De Paulis}
\title{Algebra Lineare}
\maketitle

\tableofcontents

\chapter{Insiemi numerici}

\section{Strutture algebriche fondamentali}

\begin{definition}[Gruppo]
    Si dice \textbf{gruppo} una tripla ($G$, $\cdot$, $e$) formata da \begin{itemize}
        \item un insieme di elementi $G$;
        \item un operazione $\cdot : A \times A \to A$ detta prodotto;
        \item un elemento $e \in G$
    \end{itemize} per cui valgono i seguenti assiomi: 
    \begin{description}
        \item[(Assiomi di gruppo)] Per ogni $a, b, c \in G$ vale che
        \begin{align*}
            &\text{(P1)}      &&(ab) \in G            &\text{(chiusura rispetto a $\cdot$)}\\
            &\text{(P2)}      &&(ab)c = a(bc)         &\text{(associatività di $\cdot$)}\\
            &\text{(P3)}      &&a \cdot e=e \cdot a=a &\text{($e$ el. neutro di $\cdot$)}\\
            &\text{(P4)}     &&\exists a^{-1} \in G. \quad aa^{-1} = e &\text{(inverso per $\cdot$)}
            \intertext{Si dice \textbf{gruppo commutativo} un gruppo per cui vale inoltre il seguente assioma:}
            &\text{(P5)}     &&ab = ba               &\text{(commutatività di $\cdot$)}
        \end{align*}
    \end{description}
\end{definition}

\begin{definition}[Anello]
    Si dice \textbf{anello} una quintupla ($A$, $+$, $\cdot$, $0$, $1$) formata da
    \begin{itemize}
        \item un insieme di elementi $A$;
        \item un operazione $+ : A \times A \to A$ detta somma;
        \item un operazione $\cdot : A \times A \to A$ detta prodotto;
        \item un elemento $0 \in A$;
        \item un elemento $1 \in A$
    \end{itemize} per cui valgono i seguenti assiomi: 
    \begin{description}
        \item[(Assiomi di anello)] Per ogni $a, b, c \in A$ vale che
        \begin{align*}
            &\text{(S1)}      &&(a+b) \in A           &\text{(chiusura rispetto a $+$)}\\
            &\text{(S2)}      &&a+b = b+a             &\text{(commutatività di $+$)}\\
            &\text{(S3)}      &&(a+b)+c = a+(b+c)     &\text{(associatività di $+$)}\\
            &\text{(S4)}      &&a+0=0+a=a             &\text{(0 el. neutro di $+$)}\\
            &\text{(S5)}      &&\exists (-a) \in A. \quad a+(-a) = 0 &\text{(opposto per $+$)}\\
            &\text{(P1)}      &&(ab) \in A            &\text{(chiusura rispetto a $\cdot$)}\\
            &\text{(P2)}      &&(ab)c = a(bc)         &\text{(associatività di $\cdot$)}\\
            &\text{(P3)}      &&a \cdot 1=1 \cdot a=a &\text{(1 el. neutro di $\cdot$)}\\
            &\text{(P4)}      &&(a+b)c = ac + bc      &\text{(distributività 1)} \\
            &\text{(P5)}     &&a(b+c) = ab + ac      &\text{(distributività 2)}
            \intertext{Si dice \textbf{anello commutativo} un anello per cui vale inoltre il seguente assioma:}
            &\text{(P6)}     &&ab = ba               &\text{(commutatività di $\cdot$)}
        \end{align*}
    \end{description} 
\end{definition}

Un tipico esempio di anello commutativo è $\Z$: infatti gli anelli generalizzano le operazioni che possiamo fare sui numeri interi e le loro proprietà fondamentali per estenderle ad altri insiemi con la stessa struttura algebrica.

\begin{definition}[Campo]
    Si dice \textbf{campo} una quintupla ($F$, $+$, $\cdot$, $0$, $1$) formata da
    \begin{itemize}
        \item un insieme di elementi $F$;
        \item un operazione $+ : F \times F \to F$ detta somma;
        \item un operazione $\cdot : F \times F \to F$ detta prodotto;
        \item un elemento $0 \in F$;
        \item un elemento $1 \in F$
    \end{itemize}  per cui valgono i seguenti assiomi: 
    \begin{description}
        \item[(Assiomi di campo)] Per ogni $a, b, c \in F$ vale che
        \begin{align*}
            &\text{(S1)}      &&(a+b) \in F           &\text{(chiusura rispetto a $+$)}\\
            &\text{(S2)}      &&a+b = b+a             &\text{(commutatività di $+$)}\\
            &\text{(S3)}      &&(a+b)+c = a+(b+c)     &\text{(associatività di $+$)}\\
            &\text{(S4)}      &&a+0=0+a=a             &\text{(0 el. neutro di $+$)}\\
            &\text{(S5)}      &&\exists (-a) \in F. \quad a+(-a) = 0 &\text{(opposto per $+$)}\\
            &\text{(P1)}      &&(ab) \in F            &\text{(chiusura rispetto a $\cdot$)}\\        
            &\text{(P2)}      &&ab = ba               &\text{(commutatività di $\cdot$)}\\
            &\text{(P3)}      &&(ab)c = a(bc)         &\text{(associatività di $\cdot$)}\\
            &\text{(P4)}      &&a \cdot 1=1 \cdot a=a &\text{(1 el. neutro di $\cdot$)}\\
            &\text{(P5)}     &&(a+b)c = ac + bc      &\text{(distributività)} \\
            &\text{(P6)}     &&a \neq 0 \implies \exists a^{-1} \in F. \quad aa^{-1} = 1 &\text{(inverso per $\cdot$)}
        \end{align*}
    \end{description} 

    La definizione sopra è equivalente a dire che $F$ è un anello commutativo per cui ogni elemento non nullo ha un inverso moltiplicativo.
\end{definition}

Tra gli insiemi numerici classici, gli insiemi $\Q, \R$ e $\C$ sono tutti esempi di campi: infatti le operazioni di addizione e moltiplicazione sono chiuse rispetto all'insieme, rispettano le proprietà commutativa, associativa e distributiva ed esistono gli inversi per la somma e per il prodotto (per ogni numero diverso da $0$). Il concetto di campo serve quindi a generalizzare la struttura algebrica dei numeri razionali/reali/complessi per altri insiemi numerici.

Nei campi vale la seguente proposizione.
\begin{proposition}
    [Regola di annullamento del prodotto] \label{annullamento_prodotto}
    Sia $\K$ un campo e siano $a, b \in \K$. Allora \[
        ab = 0 \implies a = 0 \lor b = 0.    
    \]
\end{proposition}
\begin{proof}
    Sappiamo che $a = 0 \lor b = 0$ è equivalente a $a \neq 0 \implies b = 0$, dunque supponiamo che $a$ sia diverso da $0$ e dimostriamo che $b$ è zero.

    Dato che $a \neq 0$ allora ammette un inverso. Chiamiamolo $a^{-1}$ e moltiplichiamo entrambi i membri per esso:
    \begin{alignat*}
        {1}
        &a^{-1}(ab) = a^{-1} \cdot 0\\
        \iff &(a^{-1}a)b = 0 \\
        \iff &b = 0
    \end{alignat*}
    che è la tesi.
\end{proof}

\section{Numeri complessi}

\begin{definition}[Unità immaginaria]
    Si dice unità immaginaria il numero $i$ tale che \[
        i^2 = -1.    
    \]
\end{definition}

\begin{definition}[Numeri complessi]
    L'insieme dei numeri complessi $\C$ è l'insieme dei numeri della forma $a+ib$ per qualche $a, b \in \R$, ovvero \[
        \C = \{ a + ib \mid a, b \in \R, i^2 = -1\}.  
    \]
\end{definition}

\begin{definition}[Parte reale e immaginaria]
    Sia $z \in \C$ tale che $z = a + ib$. Allora si dicono rispettivamente \begin{itemize}
        \item parte reale di $z$ il numero $\Re z = a$;
        \item parte immaginaria di $z$ il numero $\Im z = b$.
    \end{itemize}
\end{definition}

\begin{definition}[Somma e prodotto sui complessi]
    Definiamo le seguenti due operazioni su $\C$:
    \begin{itemize}
        \item $+ : \C \times \C \to \C$ tale che $(a + ib) + (c + id) = (a + c) + i(b + d)$;
        \item $\cdot : \C \times \C \to \C$ tale che $(a + ib) \cdot (c + id) = (ac - bd) + i(ad + bc)$.
    \end{itemize}
\end{definition}

\begin{remark}
    Le due operazioni vengono naturalmente dalla somma e dal prodotto tra monomi. Infatti \begin{gather*}
        (a + ib) + (c + id) = a + c + ib + id = (a + c) + i(b + d);\\
        \begin{alignedat}{1}
            (a + ib) \cdot (c + id) &= ac + iad + ibc + i^2bd \\
            &= ac + i(ad + bc) -bd \\
            &= (ac - bd) + i(ad + bc).
        \end{alignedat}
    \end{gather*}
\end{remark}

Notiamo che i numeri complessi della forma $a + i0$ sono numeri reali, dunque $\R \subset \C$. Inoltre possiamo rappresentare i numeri complessi come punti in uno spazio bidimensionale dove la parte reale rappresenta l'ascissa e la parte immaginaria rappresenta l'ordinata: la retta corrispondente all'asse x è il sottoinsieme dei numeri reali.

\begin{definition}[Coniugato complesso]
    Sia $z = a+ib \in \C$. Allora si dice coniugato complesso (o semplicemente coniugato) di $z$ il numero \[
        \conj{z} = a - ib.    
    \]
\end{definition}

\begin{definition}[Norma di un numero complesso]
    Sia $z = a + ib \in \C$. Allora si dice norma di $z$ il numero reale \[
        \abs{z} = \sqrt{a^2 + b^2}.    
    \]
\end{definition}

Notiamo che $\abs{z} = 0$ se e solo se $a = b = 0$, ovvero se $z = 0$.

\begin{proposition}\label{somma_prodotto_tra_coniugati}
    Siano $z, w \in \C$ tali che $z = a+ib$, $w = c + id$. Allora \begin{enumerate}[(i)]
        \item $\conj{z} + \conj{w} = \conj{z + w}$;
        \item $\conj{z}\cdot\conj{w} = \conj{zw}$;
        \item $(\conj{z})^n = \conj{z^n}$.
    \end{enumerate}
\end{proposition}
\begin{proof}
    Dimostriamo i tre fatti.
    \begin{enumerate}[(i)]
        \item Per definizione di somma \begin{alignat*}
            {1}
            \conj{z} + \conj{w} &= (a-ib) + (c - id)\\
            &= (a+c) - i(b+d)\\
            &= \conj{z + w}.
        \end{alignat*}
        \item Per definizione di prodotto \begin{alignat*}
            {1}
            \conj{z}\cdot\conj{w} &= (a-ib)(c - id)\\
            &= (ac - bd) + i(-ad-bc)\\
            &= (ac - bd) - i(ad+bc)\\
            &= \conj{zw}.
        \end{alignat*}
        \item Dimostriamolo per induzione su $n$.
        \begin{description}
            \item[Caso base.] Se $n = 1$ allora banalmente $(\conj{z})^1 = \conj{z} = \conj{z^1}$.
            \item[Passo induttivo.] Supponiamo che la tesi valga per $n$ e dimostriamola per $n+1$. Allora \[
                (\conj{z})^{n+1} = (\conj{z})^{n} \cdot \conj{z} = \conj{z^n} \cdot \conj{z} = \conj{z^{n+1}}
            \] dove l'ultimo passaggio è giustificato dal punto precedente della dimostrazione. \qedhere
        \end{description}
    \end{enumerate}
\end{proof}

\begin{proposition}\label{somma_prodotto_col_coniugato}
    Sia $z = a+ib \in \C$. Allora valgono i seguenti fatti:
    \begin{enumerate}[(i)]
        \item $z + \conj{z} = 2\Re{z}$;
        \item $z\conj{z} = \abs{z}^2$.
    \end{enumerate}
\end{proposition}
\begin{proof}
    Dimostriamo i due fatti.
    \begin{enumerate}[(i)]
        \item Per definizione di somma $z + \conj{z} = (a + ib) + (a - ib) = 2a = 2\Re z$.
        \item Per definizione di prodotto \[
            z\conj{z} = (a + ib)(a - ib) = a^2 - iab + iab - i^2b^2 = a^2 + b^2 = \abs{z}^2.\qedhere
        \] 
    \end{enumerate}
\end{proof}

La proposizione precedente ci consente di trovare l'inverso di qualunque numero non nullo in $\C$.

\begin{proposition}[Inverso tra i complessi]
    Sia $z \in \C, z \neq 0$. Allora \[\frac{1}{z} = \frac{\conj{z}}{\abs{z}^2}.\]
\end{proposition}
\begin{proof}
    Per la proposizione \ref{somma_prodotto_col_coniugato} segue che \[
        z\conj{z} = \abs{z}^2 \iff \frac{1}{z} = \frac{\conj{z}}{\abs{z}^2}. \qedhere   
    \]
\end{proof}

\begin{proposition}[I numeri complessi formano un campo]
    L'insieme $\C$ insieme alle operazioni di somma e prodotto con i rispettivi elementi neutri $0, 1 \in \C$ forma un campo.
\end{proposition}

\subsection{Rappresentazione polare dei numeri complessi}

Dato che possiamo considerare i numeri complessi come punti di un piano bidimensionale possiamo rappresentarli in forma polare, cioè considerando il vettore che congiunge l'origine degli assi con il punto $(a, b)$ che rappresenta il numero complesso $a + ib$. La forma polare di un numero complesso è data dalla coppia $(r, \theta)$, dove $r$ è il raggio del vettore e $\theta$ è l'angolo tra l'asse x e il vettore.

Dunque se $z = a+ib$ è un numero complesso in forma cartesiana, possiamo esprimerlo come $r(\cos\theta + i\sin\theta)$, dove $r = \sqrt{a^2 + b^2} = \abs{z}$ e $\theta = \arctan \frac{a}{b}$.

\begin{definition}[Esponenziale complesso]
    $e^{i\theta} = \cos\theta + i\sin\theta.$
\end{definition}

Sfruttando la definizione precedente possiamo scrivere ogni numero complesso nella forma $re^{i\theta}$ che è la forma polare del numero.

\begin{proposition}
    Siano $e^{i\alpha}, e^{i\beta} \in \C$. Allora vale \[
        e^{i\alpha} e^{i\beta} = e^{i(\alpha + \beta)}.
    \]
\end{proposition}
\begin{proof}
    Per definizione di esponenziale complesso:
    \begin{alignat*}{1}
        e^{i\alpha} e^{i\beta} &= (\cos\alpha + i\sin\alpha)(\cos\beta + i\sin\beta)\\
        &= (\cos\alpha \cos\beta - \sin\alpha \sin\beta) + i(\sin\alpha \cos\beta + \cos\alpha \sin\beta)\\
        &= \cos(\alpha + \beta) + i\sin(\alpha + \beta)\\
        &= e^{i(\alpha + \beta)}. \tag*{\qedhere}
    \end{alignat*}
\end{proof}

\section{Successioni per ricorrenza}

\begin{definition}[Successione]
    Si dice successione a valori in un insieme $A$ una funzione $(a_n) : \N \to A$.
\end{definition}

Solitamente analizzeremo successioni a valori reali, ovvero $(a_n) : \N \to \R$. Inoltre usiamo equivalentemente le notazioni $(a_n)_k$ o $a_k$ per riferirci alla funzione valutata nel punto $k \in \N$.

\begin{definition}[Somma di successioni e prodotto per una costante]
    Sia $S_{\R}$ l'insieme delle successioni a valori reali. Allora definisco una somma tra successioni $+ : S_{\R} \times S_{\R} \to S_{\R}$ tale che \[
        (a_n) + (b_n) = (a_n + b_n)  
    \] e un prodotto per una costante $\cdot : \R \times S_{\R} \to S_{\R}$ tale che \[
        k(a_n) = (ka_n).    
    \]
\end{definition}

\begin{example}
    Sia $a_n = 3^n$ e $b_n = 2n + 1$. Allora $(c_n) = (a_n) + (b_n)$ è la successione definita dalla legge $c_n = 3^n + 2n + 1$, mentre $(d_n) = 3(b_n)$ è la successione definita da $d_n = 6n + 3$.
\end{example}

Queste operazioni rispettano le solite proprietà (associativa, commutativa, distributiva). In particolare vale quindi la seguente proposizione.

\begin{proposition}[L'insieme delle successioni è uno spazio vettoriale]
    L'insieme delle successioni a valori reali $S_{\R}$ insieme alle operazioni di somma e prodotto per costanti e alla successione identicamente nulla $(0_n)$ è uno spazio vettoriale su $\R$.
\end{proposition}

\begin{definition}[Ricorrenza lineare omogenea]
    Si dice ricorrenza lineare omogenea di ordine $k$ un'equazione della forma \begin{equation} \label{ricorrenza}
        a_{n+k} = r_{k-1}a_{n+k-1} + r_{k-2}a_{n+k-2} + \dots + r_{1}a_{n+1} + r_0a_n. 
    \end{equation}
    Una soluzione della ricorrenza lineare \ref{ricorrenza} è una successione $(s_n)$ tale che per ogni $n \in \N$ vale che $s_n, s_{n+1}, \dots, s_{n+k}$ soddisfano la ricorrenza.
\end{definition}

\begin{proposition}
    Sia $A$ l'insieme delle successioni che soddisfano la ricorrenza lineare omogenea \[
        s_{n+k} = r_{k-1}s_{n+k-1} + r_{k-2}s_{n+k-2} + \dots + r_{1}s_{n+1} + r_0s_n.
    \] Allora $A$ è un sottospazio vettoriale di $S_{\R}$.
\end{proposition}
\begin{proof}
    Dobbiamo dimostrare tre fatti:
    \begin{enumerate}[(i)]
        \item $(0_n) \in A$;
        \item se $(a_n), (b_n) \in A$ allora $(c_n) = (a_n) + (b_n) \in A$;
        \item se $h \in \R$, $(a_n) \in A$ allora $(d_n) = h(a_n) \in A$.
    \end{enumerate}

    Sia $n \in \N$ qualsiasi.
    \begin{enumerate}[(i)]
        \item Verifichiamo che $(0_n)$ sia soluzione. La ricorrenza da verificare è \[
            0_{n+k} = r_{k-1}0_{n+k-1} + \dots + r_{1}0_{n+1} + r_00_n.
        \] Ma dato che $(0_n)$ è la successione identicamente nulla, allora questo equivale a dire $0 = 0r_{k-1} + \dots +  + 0r_0 = 0$, che è verificata e quindi $(0_n) \in A$.
        \item Verifichiamo che $(c_n)$ sia soluzione. \begin{align*}
            c_{n+k} &= a_{n+k} + b_{n+k}\\
            &= (r_{k-1}a_{n+k-1} + \dots + r_0a_n) + (r_{k-1}b_{n+k-1} + \dots + r_0b_n) \\
            &= r_{k-1}(a_{n+k-1} + b_{n+k-1}) + \dots + r_0(a_n + b_n) \\
            &= r_{k-1}c_{n+k-1} + \dots + r_0c_0
        \end{align*}
        dunque $(c_n) \in A$.
        \item Verifichiamo che $(d_n)$ sia soluzione. \begin{align*}
            d_{n+k} &= ha_{n+k}\\
            &= h(r_{k-1}a_{n+k-1} + \dots + r_0a_n)\\
            &= r_{k-1}(ha_{n+k-1}) + \dots + r_0(ha_n) \\
            &= r_{k-1}d_{n+k-1} + \dots + r_0d_0
        \end{align*}
        dunque $(d_n) \in A$. \qedhere
    \end{enumerate}
\end{proof}

La proposizione precedente ci permette di trovare una soluzione generale ad una ricorrenza lineare omogenea.

\begin{example}
    Siano $a_n = 3^n$ e $b_n = (-1)^n$ due soluzioni di una ricorrenza lineare omogenea. Allora per la proposizione precedente anche $k_1a_n = k_13^n$ e $k_2b_n = k_2(-1)^n$ saranno soluzioni (per ogni $k_1, k_2 \in \R$), e di conseguenza anche $k_1a_n + k_2b_n = k_13^n + k_2(-1)^n$.
\end{example}

Cerchiamo di risolvere una ricorrenza lineare omogenea.
\begin{example}
    Sia $a_{n+2} = 2a_{n+1} + 3a_n$ una ricorrenza lineare omogenea di ordine $2$. Trovare la soluzione generale. Inoltre trovare una soluzione particolare che soddisfi le condizioni iniziali $a_0 = 0$ e $a_1 = 1$.
\end{example}
\begin{solution}
    Proviamo a risolvere la ricorrenza con una soluzione esponenziale della forma $(\lambda^n)$ al variare di $n \in \N$. Sostituendo otteniamo \begin{alignat*}{1}
        &\lambda^{n+2} = 2\lambda^{n+1} + 3\lambda^{n} \\
        \iff &\lambda^2 = 2\lambda + 3 \\
        \iff &\lambda^2 - 2\lambda - 3.
    \end{alignat*}
    Dunque se $(\lambda^n)$ è una soluzione allora $\lambda$ deve essere radice di quel polinomio di secondo grado, detto polinomio caratteristico della ricorrenza.
    Risolvendolo segue che $\lambda_1 = 3$ e $\lambda_2 = -1$ sono soluzioni, dunque le successioni $(3^n)$ e $((-1)^n)$ sono soluzioni della ricorrenza.

    La soluzione generale della ricorrenza è dunque una successione della forma $(a_n) = k_1(3^n) + k_2((-1)^n)$ al variare di $k_1, k_2 \in \R$.

    Imponiamo ora che $a_0 = 0$ e $a_1 = 1$.
    \begin{equation*}
        \left\{
        \begin{array}{@{}roror }
        3^0k_1 & + & (-1)^0k_2 & = & 0 \\
        3^1k_1 & + & (-1)^1k_2 & = & 1 \\
        \end{array}
        \right. \iff \left\{
        \begin{array}{@{}ror }
        k_1 + k_2 & = & 0 \\
        3k_1 -k_2 & = & 1 \\
        \end{array}
        \right. 
    \end{equation*}
    da cui segue $k_1 = \frac14$, $k_2 = -\frac14$. La successione che soddisfa le condizioni iniziali è dunque $a_n = \frac14(3)^n - \frac14(-1)^n$.
\end{solution}

\begin{definition}
    [Polinomio caratteristico di una ricorrenza]
    Sia $a_{n+k} = r_{k-1}a_{n+k-1} +  \dots +  r_0a_n$ una ricorrenza lineare omogenea di ordine $k$. Allora si dice polinomio caratteristico associato alla ricorrenza il polinomio \[
        p(\lambda) = \lambda^k - r_{k-1}\lambda^{k-1} - \dots - r_0.    
    \]
\end{definition}

Il polinomio caratteristico si ottiene sostituendo alla ricorrenza lineare la successione $(\lambda^n)$, esattamente come abbiamo fatto nell'esempio precedente.

\begin{example}
    Consideriamo la successione di Fibonacci $f_{n+2} = f_{n+1} + f_n$ con $f_0 = 0$, $f_1 = 1$. Trovare una successione che risolva la ricorrenza e soddisfi i casi base.
\end{example}
\begin{solution}
    Il polinomio caratteristico di questa ricorrenza è \[
        p(\lambda) = \lambda^2 - \lambda - 1    
    \] che ha come radici i numeri $\varphi = \frac12(1 + \sqrt5)$ e $\bar{\varphi} = \frac12(1 - \sqrt5)$.

    La soluzione generale della ricorrenza è dunque una successione della forma $(f_n) = k_1(\varphi^n) + k_2(\bar{\varphi}^n)$ al variare di $k_1, k_2 \in \R$.

    Imponiamo ora che $f_0 = 0$ e $f_1 = 1$.
    \begin{equation*}
        \arraycolsep=1.2pt\def\arraystretch{1.3}
        \left\{
        \begin{array}{@{}roror }
        \varphi^0k_1 & + & \bar{\varphi}^0k_2 & = & 0\\
        \varphi^1k_1 & + & \bar{\varphi}^1k_2 & = & 1 \\
        \end{array}
        \right. \iff \left\{
        \begin{array}{@{}roror }
        k_1 & + & k_2 & = & 0\\
        \varphi k_1 & + & \bar{\varphi}k_2 & = & 1 \\
        \end{array}
        \right. 
    \end{equation*}
    da cui segue $k_1 = \frac{1}{\sqrt5}$, $k_2 = -\frac{1}{\sqrt5}$. La successione che soddisfa le condizioni iniziali è dunque \[
        f_n = \frac{1}{\sqrt5}\left(\frac{1 + \sqrt5}{2}\right)^n - \frac{1}{\sqrt5}\left(\frac{1 - \sqrt5}{2}\right)^n.
    \]
\end{solution}

Nel caso che una radice del polinomio caratteristico abbia una molteplicità maggiore di $1$ essa darà luogo a più di una soluzione della ricorrenza, come ci dice la seguente proposizione.
\begin{proposition}
    Sia $p(\lambda)$ il polinomio caratteristico di una ricorrenza lineare omogenea e sia $\lambda_0$ una radice di molteplicità $h$ (ovvero $h$ è il massimo intero per cui $(x - \lambda_0)^h$ compare nella fattorizzazione di $p(\lambda)$) con $h \leq 2$. 
    
    Allora $(\lambda_0^n), (n\lambda_0^n), \dots, (n^{h-1}\lambda_0^n)$ sono tutte soluzioni della ricorrenza lineare omogenea.
\end{proposition}

\begin{example}
    Sia $p(\lambda) = (\lambda - 3)^3(\lambda + 1)^2(\lambda - \sqrt2)^4$. Allora le seguenti sono tutte soluzioni indipendenti della ricorrenza lineare omogenea associata a $p(\lambda)$:
    \begin{multicols}{3}
        \begin{enumerate}[(i)]
        \item $(3^n)$;
        \item $(n3^n)$;
        \item $(n^23^n)$;
        \item $((-1)^n)$;
        \item $(n(-1)^n)$;
        \item $(\sqrt{2}^n)$;
        \item $(n\sqrt{2}^n)$;
        \item $(n^2\sqrt{2}^n)$;
        \item $(n^3\sqrt{2}^n)$.
    \end{enumerate}
    \end{multicols}
    
    La soluzione generale sarà dunque della forma \begin{align*}
        (a_n) = 
            &\ k_1(3^n) + k_2(n3^n) + k_3(n^23^n) + k_4(n(-1)^n) + k_5(n(-1)^n) + \\
            + &\ k_6(\sqrt{2}^n) + k_7(n\sqrt{2}^n) + k_8(n^2\sqrt{2}^n) + k_9(n^3\sqrt{2}^n)
    \end{align*}
    al variare di $k_1, \dots, k_9 \in \R$.
\end{example}
\chapter{Cinematica del punto materiale}

\section{Definizioni fondamentali}
\begin{definition}
    Si dice raggio vettore o \textbf{vettore posizione} il vettore $\bvv{r}$ che descrive la posizione del punto materiale rispetto ai tre assi al variare del tempo.
    \[\bvv{r}(t) = x(t)\bh{i} + y(t)\bh{j} + z(t)\bh{k}\]
\end{definition}

\begin{definition}
    Si dice \textbf{vettore spostamento} il vettore $\bvv{s}$ che descrive lo spostamento del punto materiale tra due istanti di tempo.
    \[\bvv{s} = \Delta\bvv{r} = \bvv{r_f} - \bvv{r_i} = \bvv{r}(t_f) - \bvv{r}(t_i)\]
\end{definition}

\begin{definition}
    Si dice \textbf{velocita' media} il vettore $\ang{\bvv{v}}$ che descrive la velocita' media del punto materiale tra due istanti di tempo.
        \begin{equation} \label{def_vel_media}
            \ang{\bvv{v}} = \bvv{v_m}(t_1, t_2) = \frac{\Delta\bvv{r}}{\Delta t} = \frac{\bvv{r}(t_2) - \bvv{r}(t_1)}{t_2-t_1}
        \end{equation}
    Si dice invece \textbf{velocita' istantanea} il vettore $\bvv{v}$ che descrive la velocita' del punto materiale in ogni istante di tempo.
    La velocita' istantanea e' definita come il limite della velocita' media quando $t_2 \to t_1$, o equivalentemente se supponiamo
    $t_2 = t_1 + \Delta t$ la velocita' istantanea e' il limite della velocita' media quando $\Delta t \to 0$.
    \[\bvv{v} = \lim_{t_2 \to t_1} \frac{\bvv{r}(t_2) - \bvv{r}(t_1)}{t_2-t_1} = \lim_{\Delta t \to 0} \frac{\bvv{r}(t_1 + \Delta t) - \bvv{r}(t_1)}{\Delta t} = \dot{\bvv{r}}(t)\]
\end{definition}

Notiamo che la velocita' istantanea e' un vettore parallelo allo spostamento infinitesimo, e quindi e' in particolare tangente alla traiettoria
del punto materiale. Scrivendola come derivata delle componenti otteniamo:
\[\bvv{v} = \dot{x}(t)\bh{i} + \dot{y}(t)\bh{j} + \dot{z}(t)\bh{k} = v_x\bh{i} + v_y\bh{j} + v_z\bh{k}\]

\begin{definition}
    Si dice \textbf{accelerazione media} il vettore $\ang{\bvv{a}}$ che descrive il cambiamento medio della velocita' del punto materiale tra due istanti di tempo.
    \[\ang{\bvv{a}} = \bvv{a_m}(t_1, t_2) = \frac{\Delta\bvv{v}}{\Delta t} = \frac{\bvv{v}(t_2) - \bvv{v}(t_1)}{t_2-t_1}\] 
    Si dice invece \textbf{accelerazione istantanea} il vettore $\bvv{a}$ che descrive l'accelerazione del punto materiale in ogni istante di tempo.
    \[\bvv{a} = \lim_{t_2 \to t_1} \frac{\bvv{v}(t_2) - \bvv{v}(t_1)}{t_2-t_1} = \lim_{\Delta t \to 0} \frac{\bvv{v}(t_1 + \Delta t) - \bvv{v}(t_1)}{\Delta t} = \dot{\bvv{v}}(t) = \ddot{\bvv{r}}(t)\]
\end{definition}

\theoremstyle{plain}
\begin{remark}
    L'accelerazione e' diversa da 0 se la velocita' cambia in modulo, ma anche se cambia in direzione!
\end{remark}

\section{Moto ad una dimensione}
\theoremstyle{definition}
\begin{definition}
    Si dice \textbf{legge oraria del moto} la legge che associa ad ogni istante di tempo la posizione del corpo
    sull'asse di riferimento:
    \[x(t) = f(t)\]
\end{definition}

Dalla legge oraria possiamo ricavare la velocita' e l'accelerazione del corpo tramite la derivata:
\[x(t) = f(t) \implies v(t) = \dot{x}(t) \implies a(t) = \dot{v}(t) = \ddot{x}(t)\]

\subsection{Stato di quiete}
Si dice che un punto materiale e' in stato di quiete se vale che $x(t) = x_0$ costante $\forall t > 0$.
Da questa relazione si ricavano la velocita' e l'accelerazione:
\begin{subequations}
\begin{align}
    &x(t) = x_0 \\
    &v(t) = \dot{x}(t) = 0 \\
    &a(t) = \dot{v}(t) = 0 
\end{align}
\end{subequations}

\subsection{Moto a velocita' costante}
Si dice che un punto materiale si muove di moto rettilineo uniforme se vale che $v(t) = v_0$ costante $\forall t > 0$.
Da questa relazione si ricavano la posizione e l'accelerazione:
\begin{subequations}
\begin{align}
    &x(t) = \int_{0}^t v(t) dt = x_0 + v_0t \\
    &v(t) = v_0 \\
    &a(t) = \dot{v}(t) = 0 
\end{align}    
\end{subequations}
dove $x_0 = x(0)$.
Se consideriamo il vettore posizione e il vettore velocita', otteniamo che
\[\bvv{v}(t) = \ang{\bvv{v}} = \frac{\Delta \bvv{r}}{\Delta t} = \frac{\bvv{r}(t) - \bvv{r_0}}{\Delta t}\]
da cui segue
\[\bvv{r}(t) = \bvv{r_0} + \ang{\bvv{v}}(t - t_0)\]
dove $\ang{\bvv{v}}$ e' il vettore velocita' media (che e' sempre uguale alla velocita' 
istantanea nel caso di moto a velocita' costante). 
Da questo segue il fatto che il moto sia lungo una traiettoria rettilinea.

\subsection{Moto ad accelerazione costante}
Si dice che un punto materiale si muove di moto rettilineo uniformemente accelerato
se vale che $a(t) = a_0$ costante $\forall t > 0$.
Da questa relazione si ricavano la posizione e la velocita':
\begin{subequations}
\begin{align}
&a(t) = a_0 \\
&v(t) = \int_{0}^t a(t) dt = v_0 + a_0t \\
&x(t) = \int_{0}^t v(t) dt = x_0 + v_0t + \frac{a_0}{2}t^2
\end{align}
\end{subequations}

dove $x_0 = x(0)$ e $v_0 = v(0)$.
Dalla seconda possiamo ricavare
\begin{numcases}{v(t) = v_0 + a_0t \implies}
    t = \frac{v(t) - v_0}{a_0} \label{time} \\
    a_0 = \frac{v(t) - v_0}{t} \label{accel}
\end{numcases}
Sostituendo la \ref{time} nell'espressione per $x(t)$ e riordinando otteniamo
\begin{equation}
    v^2(t) = v_0^2 + 2a_0(x-x_0) \label{MUA_senza_t}
\end{equation}
Sostituendo invece la \ref{accel} nell'espressione per $x(t)$ e riordinando otteniamo
\begin{equation}
    x(t) = x_0 + \frac{1}{2}(v(t)+v_0)t \label{MUA_senza_a}
\end{equation}

Se consideriamo il vettore velocita' e il vettore accelerazione, otteniamo che
\[\bvv{a}(t) = \ang{\bvv{a}} = \frac{\Delta \bvv{v}}{\Delta t} = \frac{\bvv{v}(t) - \bvv{v_0}}{\Delta t}\]
da cui segue
\[\bvv{v}(t) = \bvv{v_0} + \ang{\bvv{a}}(t - t_0)\]
dove $\ang{\bvv{a}}$ e' il vettore accelerazione media (che e' sempre uguale all'accelerazione 
istantanea nel caso di moto ad accelerazione costante).
Da questo segue il fatto che il moto sia lungo una traiettoria rettilinea.


\subsection{Moto a caduta libera}
E' un caso particolare di un moto uniformemente accelerato. Consideriamo un sistema ortogonale $XY$ e un corpo
che si trova inizialmente nel punto $(x_0, y_0) = (0, y_0)$ e che si muove verso il basso con una velocita' di modulo iniziale $v_0$. 
I vettori che rappresentano lo stato del corpo saranno quindi:
\begin{subequations}
\begin{align}
    &\bvv{r}(0) = y_0\bh{j}\\
    &\bvv{v}(0) = v_0\bh{j}\\
    &\bvv{a}(0) = -g\bh{j}
\end{align}    
\end{subequations}
dove $g$ e' l'accelerazione di gravita' terrestre.
Notiamo quindi che il moto si svolge unicamente nella direzione dell'asse $Y$.

\subsubsection{Caduta da un'altezza}
Supponiamo che il corpo cada da un'altezza $h$ da fermo (cioe' $v_0 = 0$).
Avremo:
\begin{subequations}
\begin{align}
    &y(t) = h - \frac{1}{2}gt^2 \\
    &v(t) = -gt \label{v_caduta_libera}\\
    &a(t) = -g
\end{align}    
\end{subequations}

Da queste equazioni possiamo ricavare il tempo di caduta e la velocita' di impatto del corpo con il suolo.
Infatti quando il corpo tocca il suolo all'istante $t_f$, avremo che
\begin{alignat}{2}
          y(t_f) &= h - \frac{1}{2}gt_f^2 = 0    \nonumber\\
    \implies t_f &= \sqrt{\frac{2h}{g}}             &\rlap{\text{(Tempo di caduta)}}\\
    \intertext{dunque sostituendo $t_f$ nell'equazione della velocita' \ref{v_caduta_libera}:}
          v(t_f) &= -\sqrt{2gh}                  &\rlap{\text{(Velocita' finale)}}
\end{alignat}    


\subsubsection{Lancio verso l'alto}
Supponiamo ora che il corpo venga lanciato verso l'alto con una velocita' iniziale $v_0 \neq 0$.
Avremo:
\begin{subequations}
\begin{align}
    &y(t) = y_0 + v_0t - \frac{1}{2}gt^2 \label{y_lancio}\\
    &v(t) = v_0 - gt \\
    &a(t) = -g
\end{align}    
\end{subequations}

Possiamo calcolare il punto di altezza massima $y_M$ e il tempo necessario per raggiungerlo $t_M$ da queste equazioni.
Infatti al punto di altezza massima la velocita' sara' nulla, dunque avremo che    
\begin{alignat}
    {2}
          v(t_M) &= v_0 - gt_M = 0       \nonumber \\
    \implies t_M &= \frac{v_0}{g}               \\
    \intertext{dunque sostituendo $t_M$ nell'equazione della posizione \ref{y_lancio}}
    y_M     &= y_0 + v_0t_M - \frac{1}{2}gt_M^2 \nonumber \\
            &= y_0 + \frac{v_0^2}{2g}               
\end{alignat}

\section{Moto in due dimensioni}
Consideriamo ora moti che avvengono in due dimensioni.
Siano $\bvv{r}$, $\bvv{v}$ e $\bvv{a}$ i vettori che descrivono la posizione, la velocita' e l'accelerazione del corpo nel tempo.
Allora il moto e' rettilineo se $\bvv{a} \parallel \bvv{v}$ oppure se $\bvv{a} = \bvv{0}$, altrimenti il moto e' bidimensionale.
Le leggi del moto sono le stesse del caso unidimensionale
\begin{subequations}
\begin{align}
    &\bvv{a}(t) = \bvv{a_0} \\
    &\bvv{v}(t) = \bvv{v_0} + \bvv{a_0}t \\
    &\bvv{r}(t) = \bvv{r_0} + \bvv{v_0}t + \frac{\bvv{a_0}}{2}t^2
\end{align}
\end{subequations}
ma possono essere separate in due equazioni che si riferiscono al moto sui due assi
\begin{subequations}
    \begin{align}
        &\bvv{r}\text{:}
        \begin{cases}{}
            x(t) = x_0 + (v_0\cos{\theta})t + \frac{1}{2}(a_0\cos{\psi})t^2 \\
            y(t) = y_0 + (v_0\sin{\theta})t + \frac{1}{2}(a_0\sin{\psi})t^2
        \end{cases} \\
        &\bvv{v}\text{:}
        \begin{cases}{}
            v_x(t) = v_0\cos{\theta} + (a_0\cos{\psi})t \\
            v_y(t) = v_0\sin{\theta} + (a_0\sin{\psi})t
        \end{cases} \\
        &\bvv{a}\text{:}
        \begin{cases}{}
            a_x(t) = a_0\cos{\psi} \\
            a_y(t) = a_0\sin{\psi}
        \end{cases}
    \end{align}
\end{subequations}
dove $v_0$ e' il modulo del vettore $\bvv{v_0}$, $\theta$ e' l'angolo formato da $\bvv{v_0}$ con l'asse $X$, 
$a_0$ e' il modulo del vettore $\bvv{a_0}$ e $\psi$ e' l'angolo formato da $\bvv{a_0}$ con l'asse $X$.

\subsection{Moto del proiettile}
Consideriamo un caso particolare del moto accelerato bidimensionale in cui $\bvv{a} = -g\bh{j}$.
Se sostituiamo nelle equazioni precedenti otteniamo
\begin{subequations}
    \begin{align}
        &\bvv{r}\text{:}
        \begin{cases}{}
            x(t) = x_0 + (v_0\cos{\theta})t\\
            y(t) = y_0 + (v_0\sin{\theta})t - \frac{1}{2}gt^2
        \end{cases} \label{proj_pos}\\
        &\bvv{v}\text{:}
        \begin{cases}{}
            v_x(t) = v_0\cos{\theta} \\
            v_y(t) = v_0\sin{\theta} - gt
        \end{cases} \\
        &\bvv{a}\text{:}
        \begin{cases}{}
            a_x(t) = 0 \\
            a_y(t) = -g
        \end{cases}
    \end{align}
\end{subequations}
Possiamo notare che, considerando i moti sui due assi separatamente, il moto del punto sull'asse $X$ e' rettilineo uniforme, 
mentre quello sull'asse $Y$ e' uniformemente accelerato.

\subsubsection{Traiettoria del proiettile}
Se ricaviamo $t$ dalla formula di $x(t)$ da \ref{proj_pos} (ottenendo $t = \frac{x-x_0}{v_0\cos{\theta}}$)
e lo sostituiamo nella formula di $y(t)$ otteniamo la traiettoria tracciata dal proiettile, 
cioe' una curva di secondo grado del tipo
\begin{equation}
    y(x) = y_0 + \tan{\theta}(x-x_0) - \frac{g}{2(v\cos{\theta})^2}(x-x_0)^2 \label{proj_traj}
\end{equation}
che rappresenta la traiettoria del proiettile al variare della $x$.

\subsubsection{Punto di altezza massima}
Come nel caso del corpo lanciato verticalmente, il punto di altezza massima viene raggiunto nell'istante
di tempo $t_h$ tale che $v_y(t_h) = 0$.
Sostituendo nelle equazioni otteniamo:   
\begin{alignat}
    {2}
        v_y(t_h) &= v_0\sin{\theta} - gt_h = 0             \nonumber \\       
    \implies t_h &= \frac{v_0\sin{\theta}}{g}                   \\
    \intertext{dunque sostituendo $t_h$ nell'equazione della posizione \ref{proj_pos}}
    y(t_h)   &= y_0 + (v_0\sin{\theta})t_h - \frac{1}{2}gt_h^2  \nonumber \\
             &= y_0 + \frac{(v_0\sin{\theta})^2}{g} - \frac{(v_0\sin{\theta})^2}{2g} \nonumber \\
             &= y_0 + \frac{(v_0\sin{\theta})^2}{2g}            \label{h_max_proj_theta}
\end{alignat}
che e' massimo quando $\theta = \frac{\pi}{2}$, ed e' dunque uguale a
\begin{equation}
    y_M = \frac{v_0^2}{2g} \label{h_map_proj}
\end{equation}

\subsubsection{Gittata}
Per calcolare la gittata del proiettile ci bastera' capire in che punto esso raggiunge 
l'altezza che aveva all'inizio del lancio;
bastera' cioe' trovare $\Delta x = x(t_g)-x_0$, dove $t_g > 0$ e' tale che $y(t_g) = y_0$.
\begin{alignat}
    {2}
             y(t_g) &= y_0 + (v_0\sin{\theta})t_g - \frac{1}{2}gt_g^2 = y_0 \nonumber \\
    \implies t_g &= \frac{2v_0\sin{\theta}}{g}                   \\
    \intertext{dunque sostituendo $t_g$ nell'equazione della posizione \ref{proj_pos}}
    \Delta x &= (v_0\sin{\theta})t_g  \nonumber \\
             &= \frac{v_0^2\sin{2\theta}}{g}            \label{gittata_proj_theta}
\end{alignat}
che e' massimo quando $\theta = \frac{\pi}{4}$ ed e' dunque uguale a
\begin{equation}
    x_{g_M} = \frac{x_0^2}{g}   \label{gittata_proj}
\end{equation}

\subsubsection{Impatto col suolo}
Invece per calcolare la distanza percorsa per impattare il suolo e' sufficiente trovare l'intersezione con l'asse $X$;
cioe' bastera' trovare $\Delta x = x(t_s)-x_0$, dove $t_s > 0$ e' tale che $y(t_s) = 0$.
\begin{alignat}
    {2}
             y(t_s) &= y_0 + (v_0\sin{\theta})t_s - \frac{1}{2}gt_s^2 = 0               \nonumber \\
    \implies t_s &= \frac{1}{g}\left(v_0\sin{\theta} + \sqrt{(v_0\sin{\theta})^2 + 2gy_0}\right)   \\
    \intertext{dunque sostituendo $t_s$ nell'equazione della posizione \ref{proj_pos}}
    \Delta x &= (v_0\sin{\theta})t_s   \label{impatto_proj_theta}
\end{alignat}
\chapter{Applicazioni lineari}

\section{Applicazioni lineari}

\begin{definition}
    Siano $V, W$ spazi vettoriali. Allora un'applicazione $f : V \to W$ si dice lineare
    se
    \begin{align}
        &f(\bm{0_V}) = \bm{0_W} \\
        &f(\bm{v} + \bm{w}) = f(\bm{v}) + f(\bm{w}) &&\forall v, w \in V \\
        &f(k\bm{v}) = kf(\bm{v})                    &&\forall v\in V, k \in \R 
    \end{align}
    $V$ si dice dominio dell'applicazione lineare, $W$ si dice codominio.
\end{definition}

\begin{definition}
    Siano $V, W$ spazi vettoriali e sia $f : V \to W$ lineare. Allora si dice immagine di $f$ l'insieme \begin{equation}
        \Imm{f} = \left\{ f(\bm{v}) \mid \bm v \in V\right\}.
    \end{equation}
\end{definition}

\begin{remark}
    Se $f : V \to W$ allora $\Imm{f} \subseteq W$. In particolare si puo' dimostrare che $\Imm{f}$ e' un sottospazio di $W$, e dunque che $0 \leq \dim\Imm{f} \leq \dim W$.
\end{remark}

\begin{definition}
    Siano $V, W$ spazi vettoriali e sia $f : V \to W$ lineare. Allora si dice kernel (o nucleo) di $f$ l'insieme \begin{equation}
        \ker{f} = \left\{ \bm{v} \in V \mid f(\bm v) = \bm{0_W}\right\}.
    \end{equation}
\end{definition}

\begin{theorem} 
    [delle dimensioni] \label{th_dimensioni}
    Siano $V, W$ spazi vettoriali e sia $f : V \to W$ lineare. Allora vale il seguente fatto:
    \begin{equation}
        \dim V = \dim \Imm f + \dim \ker f.
    \end{equation}
\end{theorem}
\begin{proof}
    Sia $k$ la dimensione di $\ker f$ e $n$ la dimensione di $V$.
    Sia $\alpha = \ang{\bm{v_1}, \dots, \bm{v_k}}$ una base di $\ker f$. Dato che $\ker f$ e' un sottospazio di $V$, per il teorema del completamento ad una base (\ref{th_completamento}) possiamo completare $\alpha$ ad una base $\beta = \ang{\bm{v_1}, \dots, \bm{v_k}, \bm{u_1}, \dots, \bm{u_{n-k}}}$ di $V$.

    Per la proposizione \ref{base_mappata_generatori_immagine} segue che l'immagine della base $\beta$, cioe' $f(\beta) = \ang{f(\bm{v_1}), \dots, f(\bm{v_k}), f(\bm{u_1}), \dots, f(\bm{u_{n-k}})}$, e' una insieme di generatori di $\Imm{f}$. Dato che $\bm{v_1}, \dots, \bm{v_k} \in \ker f$, allora \begin{alignat*}
        {1}
        \Imm{f} &= \Span{f(\bm{v_1}), \dots, f(\bm{v_k}), f(\bm{u_1}), \dots, f(\bm{u_{n-k}})}\\
        &= \Span{0, \dots, 0, f(\bm{u_1}), \dots, f(\bm{u_{n-k}})}\\
        &= \Span{f(\bm{u_1}), \dots, f(\bm{u_{n-k}})}.
    \end{alignat*}

    Se $f(\bm{u_1}), \dots, f(\bm{u_{n-k}})$ sono indipendenti allora segue che essi formano una base di $\Imm{f}$, cioe' che $\dim \Imm{f} = n - k = \dim V - \dim \ker f$.

    Consideriamo quindi una generica combinazione lineare \[
        x_1f(\bm{u_1}) + \dots + x_{n-k}f(\bm{u_{n-k}})
    \] e dimostriamo che imponendola uguale a $\bm 0$ segue che i coefficienti della combinazione devono essere uguali a 0.
    \begin{alignat*}
        {1}
        &x_1f(\bm{u_1}) + \dots + x_{n-k}f(\bm{u_{n-k}}) = \bm 0 \\
        \iff &f(x_1\bm{u_1} + \dots + x_{n-k}\bm{u_{n-k}}) = \bm 0 \\
        \intertext{che per definizione di kernel significa}
        \iff &x_1\bm{u_1} + \dots + x_{n-k}\bm{u_{n-k}} \in \ker f.
    \end{alignat*}
    Dato che $\alpha$ e' una base di $\ker f$ allora segue che
    \begin{alignat*}{1}
        &x_1\bm{u_1} + \dots + x_{n-k}\bm{u_{n-k}} = a_1\bm{v_1} + \dots + a_k\bm{v_k} \\
        \iff &x_1\bm{u_1} + \dots + x_{n-k}\bm{u_{n-k}} - a_1\bm{v_1} - \dots - a_k\bm{v_k} = \bm 0.
    \end{alignat*}
    Ma $\beta = \ang{\bm{v_1}, \dots, \bm{v_k}, \bm{u_1}, \dots, \bm{u_{n-k}}}$ e' una base di $V$, dunque i vettori che la compongono devono essere indipendenti, da cui segue \[
        x_1 = \dots = x_{n-k} = 0.   
    \]
    Dunque $\ang{f(\bm{u_1}), \dots, f(\bm{u_{n-k}})}$ e' una base di $\Imm{f}$ e dunque segue che $\dim V = \dim \ker f + \dim \Imm{f}$, come volevasi dimostrare.
\end{proof}

Una conseguenza diretta del teorema delle dimensioni e' che data una matrice $A$ e riducendola a scalini tramite mosse di riga non cambia la dimensione dello spazio delle colonne.
\begin{proposition}\label{invarianza_dim_colonne_per_mosse_riga}
    Sia $A \in \M_{n\times m}$ e siano $C_1, \dots, C_m \in \R^n$ le colonne della matrice. Sia $S$ la matrice ottenuta riducendo a scalini per riga la matrice $A$, e siano $C'_1, \dots, C'_m$ le colonne di $S$. Allora \begin{equation}
        \dim \Span{C_1, \dots, C_m} = \dim \Span{C'_1, \dots, C'_m}.
    \end{equation}
\end{proposition}
\begin{proof}
    Dato che $S$ e' ottenuta riducendo $A$ a scalini, allora le soluzioni dei sistemi $A\bm{x} = \bm 0$ e $S\bm{x} = \bm 0$ devono essere le stesse. 
    
    Siano $L_A : \R^m \to \R^n$ e $L_S : \R^m \to \R^n$ le applicazioni lineari associate ad $A$ e $S$; trascrivendo le equazioni di prima in termini delle applicazioni otteniamo che $L_A(\bm x) = \bm 0$ se e solo se $L_S(\bm x) = \bm 0$. 
    
    Allora segue che $\ker L_A = \ker L_S$, dunque $\dim \ker L_A = \dim \ker L_S$. Per la proposizione \ref{span_colonne=immagine_applicazione_associata} e per il teorema delle dimensioni (\ref{th_dimensioni}) segue quindi che
    \begin{alignat*}
        {1}
        \dim \Span{C_1, \dots, C_m} &= \dim \Imm{L_A}\\
        &= \dim \R^m - \dim \ker L_A\\
        &= \dim \R^m - \dim \ker L_S\\
        &= \dim \Imm{L_S} \\
        &= \dim \Span{C'_1, \dots, C'_m}
    \end{alignat*}
    che e' la tesi.
\end{proof}

Ovviamente lo stesso ragionamento ci dice che ridurre una matrice a scalini tramite mosse di colonna non cambia la dimensione dello spazio delle righe.

\begin{corollary}\label{invarianza_dim_righe_per_mosse_colonna}
    Sia $A \in \M_{n\times m}$ e siano $R_1, \dots, R_m \in \R^n$ le righe della matrice. Sia $S$ la matrice ottenuta riducendo a scalini per colonna la matrice $A$, e siano $R'_1, \dots, R'_m$ le righe di $S$. Allora \begin{equation}
        \dim \Span{R_1, \dots, R_m} = \dim \Span{R'_1, \dots, R'_m}.
    \end{equation}
\end{corollary}
\begin{proof}
    Consideriamo la matrice $A^T$ e riduciamola per righe, ottenendo la matrice $S$. Per la proposizione \ref{invarianza_dim_colonne_per_mosse_riga} la dimensione dello spazio delle righe di $S$ e' uguale alla dimensione dello spazio delle righe di $A^T$. Notiamo inoltre che $S$ e' la trasposta della matrice ottenuta riducendo $A$ per colonne, dunque la dimensione dello spazio delle colonne di $S^T$ e' la dimensione dello spazio delle colonne di $A$, cioe' la tesi.
\end{proof}

\section{Applicazioni iniettive e surgettive}

Le applicazioni lineari sono funzioni, dunque possono essere iniettive e surgettive, ma essendo lineari hanno delle proprieta' particolari.

\begin{definition}
    Siano $V, W$ spazi vettoriali e sia $f : V \to W$ lineare. Allora $f$ si dice iniettiva se per ogni $\bm{v}, \bm{u} \in V$ vale che $f(\bm{v}) = f(\bm{u})$ se e solo se $\bm{v} = \bm{u}$.
\end{definition}

\begin{proposition}\label{ker_funzione_iniettiva}
    Siano $V, W$ spazi vettoriali e sia $f : V \to W$ lineare. Allora $f$ e' iniettiva se e solo se $\ker f = \{\bm{0_V}\}$. 
\end{proposition}
\begin{proof}
    Notiamo che dato che $f$ e' lineare allora per definizione $f(\bm{0_V}) = \bm{0_W}$, dunque $\bm{0_V} \in \ker f$.
    \begin{description}
        \item [($\implies$).] Supponiamo che $f$ sia iniettiva e supponiamo che per qualche $\bm{v} \in V$ valga $\bm{v} \in \ker f$. Allora per definizione di kernel $f(\bm{v}) = \bm{0_W} = f(\bm{0_V})$, dunque per iniettivita' di $f$ da $f(\bm{v}) = f(\bm{0_V})$ segue che $\bm v = \bm{0_V}$. Dunque $\ker f = \{\bm{0_V}\}$.
        \item [($\impliedby$).] Supponiamo che $\ker f = \left\{ \bm{0_V}\right\}$. Per dimostrare che $f$ e' iniettiva e' sufficiente dimostrare che per ogni $\bm{v}, \bm{w} \in V$ segue che $f(\bm{v}) = f(\bm{w}) \implies \bm{v} = \bm{w}$.
        \begin{alignat*}{1}
            &f(\bm{v}) = f(\bm{w}) \\
            \iff &f(\bm{v}) - f(\bm{w}) = \bm{0_W} \\
            \iff &f(\bm{v} - \bm{w}) = \bm{0_W} \\
            \iff &\bm{v} - \bm{w} \in \ker f \\
            \intertext{ma l'unico elemento di $\ker f$ e' $\bm{0_V}$, dunque}
            \implies &\bm{v} - \bm{w} = \bm{0_V}\\
            \iff &\bm{v} = \bm{w}
        \end{alignat*}
        cioe' $f$ e' iniettiva. \qedhere
    \end{description}
\end{proof}

\begin{corollary}\label{iniettiva_allora_dimIm_uguale_dimV}
    Se $f$ e' iniettiva allora $\dim \Imm{f} = \dim V$.
\end{corollary}
\begin{proof}
    Infatti per la proposizione \ref{ker_funzione_iniettiva} $\dim \ker f = 0$, dunque per il teorema delle dimensioni (\ref{th_dimensioni}) segue che $\dim V = \dim \Imm{f} + \dim \ker f = \dim \Imm{f}$.
\end{proof}

\begin{corollary}
    Siano $V, W$ spazi vettoriali tali che $\dim V > \dim W$. Allora non puo' esistere $f : V \to W$ iniettiva.
\end{corollary}
\begin{proof}
    Infatti per il corollario \ref{iniettiva_allora_dimIm_uguale_dimV} segue che $\dim \Imm{f} = \dim V$, ma $\dim \Imm{f} < \dim W$ dunque non puo' essere che $\dim V > \dim W$.
\end{proof}

\begin{proposition}\label{indipendenti_mappati_indipendenti}
    Siano $V, W$ spazi vettoriali, $\bm{v_1}, \dots, \bm{v_n} \in V$ linearmente indipendenti e sia $f : V \to W$ lineare. Se $f$ e' iniettiva allora segue che $f(\bm{v_1}), \dots, f(\bm{v_n})$ sono linearmente indipendenti. 
\end{proposition}
\begin{proof}
    Consideriamo una combinazione lineare di $f(\bm{v_1}), \dots, f(\bm{v_n})$ e dimostriamo che imponendola uguale al vettore nullo segue che i coefficienti devono essere tutti nulli.
    \begin{alignat*}
        {1}
        &x_1f(\bm{v_1}) + \dots + x_nf(\bm{v_n}) = \bm{0_W}\\
        \iff &f(x_1\bm{v_1} + \dots + x_n\bm{v_n}) = \bm{0_W}\\
        \intertext{Per la proposizione \ref{ker_funzione_iniettiva} segue che}
        \iff &x_1\bm{v_1} + \dots + x_n\bm{v_n} = \bm{0_V}\\
        \intertext{Ma i vettori $\bm{v_1}, \dots, \bm{v_n}$ sono linearmente indipendenti, dunque l'unica combinazione lineare che li annulla e' quella a coefficienti nulli, cioe'}
        \iff &x_1 = \dots = x_n = 0
    \end{alignat*}
    Quindi una combinazione lineare di $f(\bm{v_1}), \dots, f(\bm{v_n})$ e' uguale al vettore nullo se e solo se tutti i coefficienti sono nulli, dunque $f(\bm{v_1}), \dots, f(\bm{v_n})$ sono linearmente indipendenti.
\end{proof}

\begin{definition}
    Siano $V, W$ spazi vettoriali e sia $f : V \to W$ lineare. Allora $f$ si dice surgettiva se per ogni $\bm{w} \in W$ esiste $\bm{v} \in V$ tale che $f(\bm{v}) = \bm{w}$.
\end{definition}

\begin{remark}
    Una funzione $f : V \to W$ e' surgettiva se e solo se $\Imm{f} = W$.
\end{remark}

\begin{proposition}\label{base_mappata_generatori_immagine}
    Sia $f : V \to W$. Allora se $\ang{\bm{v_1}, \dots, \bm{v_n}}$ e' una base di $V$ segue che $\left\{ f(\bm{v_1}), \dots, f(\bm{v_n})\right\}$ e' un insieme di generatori di $\Imm{f}$.
\end{proposition}
\begin{proof}
    Sia $\bm{w} \in \Imm{f}$ generico; allora questo equivale a dire che esiste $\bm{v} \in V$ tale che $f(\bm{v}) = \bm{w}$.
    Dato che $\ang{\bm{v_1}, \dots, \bm{v_n}}$ e' una base di $V$, allora possiamo scrivere $\bm{v}$ come $a_1\bm{v_1} + \dots a_n\bm{v_n}$, dunque 
    \begin{equation*}
        \bm{w} = f(a_1\bm{v_1} + \dots + a_n\bm{v_n}) = a_1f(\bm{v_1}) + \dots + a_nf(\bm{v_n}).
    \end{equation*}
    Dunque per la generalita' di $w$ segue che ogni elemento di $\Imm{f}$ appartiene allo span di $f(\bm{v_1}), \dots, f(\bm{v_n})$, cioe' $\left\{ f(\bm{v_1}), \dots, f(\bm{v_n})\right\}$ e' un insieme di generatori di $\Imm{f}$.
\end{proof}

\begin{corollary}\label{base_mappata_generatori_codominio}
    Sia $f : V \to W$ surgettiva. Allora se $\ang{\bm{v_1}, \dots, \bm{v_n}}$ e' una base di $V$ segue che $\left\{ f(\bm{v_1}), \dots, f(\bm{v_n})\right\}$ e' un insieme di generatori di $W$.
\end{corollary}
\begin{proof}
    Segue direttamente dalla proposizione \ref{base_mappata_generatori_immagine}: infatti se una funzione e' surgettiva allora $\Imm{f} = W$, dunque se $\left\{ f(\bm{v_1}), \dots, f(\bm{v_n})\right\}$ e' un insieme di generatori di $\Imm f$ segue che e' anche un insieme di generatori di $W$.
\end{proof}

\section{Isomorfismi}

\begin{definition}
    Siano $V, W$ spazi vettoriali e sia $f : V \to W$ lineare. Allora $f$ si dice bigettiva se $f$ e' sia iniettiva che surgettiva.
\end{definition}

\begin{definition}
    Una funzione $f : V \to W$ si dice invertibile se esiste $f^{-1} : W \to V$ tale che \begin{equation}
        f(\bm{v}) = \bm{w} \iff f^{-1}(\bm{w}) = \bm{v}
    \end{equation}
    Se $f$ e' invertibile allora $f^{-1}$ e' unica e si chiama inversa di $f$.
\end{definition}

\begin{remark}
    Un'applicazione lineare e' invertibile se e solo se e' bigettiva. 
\end{remark}

\begin{definition}
    Siano $V, W$ spazi vettoriali e sia $f : V \to W$ lineare. Allora se $f$ e' bigettiva si dice che $f$ e' un isomorfismo.
    
    Se esiste un isomorfismo tra gli spazi $V$ e $W$ allora si dice che $V$ e' isomorfo a $W$, e si indica con $V \cong W$.
\end{definition}

\begin{remark}
    Le seguenti affermazioni sono equivalenti:
    \begin{itemize}
        \item $f$ e' bigettiva;
        \item $f$ e' invertibile;
        \item $f$ e' un isomorfismo.
    \end{itemize}
\end{remark}

Gli isomorfismi preservano la linearita' dello spazio vettoriale e tutte le sue proprieta', come ci dicono le seguenti proposizioni.

\begin{proposition}
    Sia $V$ uno spazio vettoriale, $\alpha = \ang{\bm{v_1}, \dots, \bm{v_n}}$ una base di $V$. Allora se $f : V \to W$ e' un isomorfismo segue che $\beta = \ang{f(\bm{v_1}), \dots, f(\bm{v_n})}$ e' una base di $W$ (cioe' gli isomorfismi mappano basi in basi).
\end{proposition}
\begin{proof}
    Dato che $f$ e' un isomorfismo allora $f$ e' bigettiva.

    Dunque dato che $f$ e' iniettiva essa mappa un insieme di vettori indipendenti (come la base $\alpha$ di $V$) in un insieme di vettori linearmente indipendenti per la proposizione \ref{indipendenti_mappati_indipendenti}, dunque $\beta$ e' un insieme di vettori linearmente indipendenti. 

    Inoltre, dato che $f$ e' surgettiva, per la proposizione \ref{base_mappata_generatori_codominio} essa mappa una base di $V$ in un insieme di generatori del codominio $W$, dunque i vettori di $\beta$ generano $W$.

    Dunque $\beta$ e' un insieme di generatori linearmente indipendenti, e quindi e' una base di $W$.
\end{proof}

\begin{proposition}
    Se $V$ e' uno spazio vettoriale di dimensione $n = \dim V$, allora $V$ e' isomorfo a tutti e soli gli spazi vettoriali di dimensione $n$.
\end{proposition}
\begin{proof}
    Deriva direttamente dalla proposizione precedente: infatti ogni isomorfismo che ha come dominio $V$ deve portare una base di $V$ in una base di $W$, dunque la dimensione di $V$ deve essere uguale a quella di $W$.
\end{proof}

Quindi per calcolare una base di un sottospazio $W$ di uno spazio $V$ spesso conviene passare allo spazio dei vettori colonna $\R^n$ isomorfo allo spazio $V$, calcolare la base del sottospazio $\tilde{W}$ isomorfo a $W$ e infine tornare allo spazio di partenza.

\begin{example}
    Sia $V = \R[x]^{\leq 2}$ e sia $W \subseteq V$ il sottospazio di $V$ tale che $p \in W \iff p(2) = 0$. Dimostrare che $W$ e' un sottospazio e trovarne una base.
\end{example}
\begin{solution}
    Svolgiamo i due punti separatamente.
    \begin{enumerate}
        \item Dimostriamo che $W$ e' un sottospazio di $V$.
        \begin{itemize}
            \item Sia $\bm{0_V} \in V$ tale che $\bm{0_V}(x) = 0 + 0x + 0x^2$. Allora $\bm{0_V}(2) = 0 + 0\cdot 2 + 0 \cdot 4 = 0$, dunque $\bm{0_V} \in W$.
            \item Supponiamo $p, q \in W$ e mostriamo che $p+q \in W$. Dunque \[
                (p+q)(2) = p(2) + q(2) = 0 + 0 = 0    
            \] dunque $p + q \in W$.
            \item Supponiamo $p \in W$ e mostriamo che $kp \in W$ per un generico $k \in \R$. Dunque \[
                (kp)(2) = kp(2) = k \cdot 0 = 0    
            \] cioe' $kp \in W$ per ogni $k \in \R$.
        \end{itemize} 
        Dunque abbiamo dimostrato che $W$ e' un sottospazio di $V$.
        \item Cerchiamo ora una base per $W$.
         
        Consideriamo un generico $p \in V$, cioe' $p(x) = a + bx + cx^2$. La condizione che definisce $W$ e' $p(2) = a + 2b + 4c = 0$. 
    
        Passiamo ora allo spazio isomorfo $\R^3$. Il vettore corrispondente a $p$ in $\R^3$ e' $\bm{\tilde{p}} = \begin{psmallmatrix} a \\ b \\ c \end{psmallmatrix}$, mentre la condizione di appartenenza allo spazio $\widetilde{W} \subseteq \R^3$ isomorfo a $W$ e' sempre $a+2b+4c = 0$. Cerchiamo una base di $\widetilde{W}$ passando alla forma parametrica, cioe' cercando di esplicitare la condizione di appartenenza allo spazio e inserendola nella definizione stessa del vettore.
        Dato che la condizione e' data dal sistema $a+2b+4c = 0$ che ha due variabili libere, scelgo $b, c$ libere ottenendo $a = -2b - 4c$. Sostituendo in $\bm{\tilde{p}}$:\[
            \bm{\tilde p} = \begin{pmatrix} a \\ b \\ c \end{pmatrix} = \begin{pmatrix}
                -2b-4c\\b\\c
            \end{pmatrix} = b\begin{pmatrix} -2 \\ 1 \\ 0 \end{pmatrix} + c\begin{pmatrix} -4 \\ 0 \\ 1 \end{pmatrix}
        \]
        Dato che ogni vettore generico di $\widetilde{W}$ puo' essere scritto come combinazione lineare dei due vettori $\bm{\tilde{w}_1} = \begin{psmallmatrix} -2 \\ 1 \\ 0 \end{psmallmatrix}$ e $\bm{\tilde{w}_2} = \begin{psmallmatrix} -4 \\ 0 \\ 1 \end{psmallmatrix}$, allora segue che essi sono generatori di $W$. 
            
        Controlliamo ora che siano linearmente indipendenti riducendo a scalini per riga (secondo la proposizione \ref{estrarre_una_base}) la matrice che ha come colonne $\bm{\tilde{w}_1}$ e $\bm{\tilde{w}_2}$.
        \begin{equation*}
            \begin{pmatrix}[c|c]
                -2 & -4 \\ 1 & 0 \\ 0 & 1
            \end{pmatrix} \xrightarrow[]{R_2 + \frac12R_1}
            \begin{pmatrix}[c|c]
                -2 & -4 \\ 0 & -2 \\ 0 & 1
            \end{pmatrix} \xrightarrow[]{R_3 + \frac12R_2}
            \begin{pmatrix}[c|c]
                -2 & -4 \\ 0 & -2 \\ 0 & 0
            \end{pmatrix}
        \end{equation*}
        Dato che ci sono tanti pivot quante colonne segue che tutti i vettori originali sono indipendenti.
        I vettori $\bm{\tilde{w}_1}$ e $\bm{\tilde{w}_2}$ sono quindi indipendenti e generano $\widetilde{W}$: segue che $\ang{\bm{\tilde{w}_1}, \bm{\tilde{w}_1}}$ e' una base di $\widetilde{W}$, quindi $\dim \widetilde{W} = 2$.

        Tornando allo spazio originale, i vettori corrispondenti alla base sono quindi $w_1(x) = (-2 + x)$ e $w_2(x) = (-4 + x^2)$. L'insieme ordinato $\ang{(-2+x), (-4+x^2)}$ forma dunque una base di $W$ e dunque $\dim W = 2$.
    \end{enumerate}
\end{solution}

\section{Matrice associata ad una funzione}

Come avevamo visto nel primo capitolo, le matrici sono associate ad applicazioni lineari da vettori colonna in vettori colonna. Possiamo generalizzare questo concetto e definire una matrice associata ad ogni applicazione lineare.

\begin{definition}
    Siano $V, W$ spazi vettoriali, $f : V \to W$ lineare, $\alpha$ base di $V$ e $\beta$ base di $W$. Allora si dice chiama \textbf{matrice associata all'applicazione lineare} $f$ la matrice $[f]^{\alpha}_{\beta}$ tale che
    \begin{equation}
        \forall \bm{v} \in V. \quad \left( f(\bm{v}) \right)_{\beta} = [f]^{\alpha}_{\beta} \cdot (\bm{v})_{\alpha}.
    \end{equation}
    Cioe' se $f$ mappa $\bm{v} \mapsto \bm{w}$ allora $[f]^{\alpha}_{\beta}$ e' una matrice che porta (tramite il prodotto) il vettore colonna delle coordinate di $\bm{v}$ rispetto ad una base $\alpha$ nel vettore colonna delle coordinate di $\bm{w}$ rispetto ad una base $\beta$.
\end{definition}

Per trovare la matrice associata ad $f$ rispetto alle basi $\alpha = \ang{\bm{v_1}, \dots, \bm{v_n}}$ e $\beta = \ang{\bm{w_1}, \dots, \bm{w_m}}$ possiamo seguire questo procedimento:
\begin{itemize}
    \item calcoliamo $f(\bm{v_1}) = \bm{u_1}, \dots, f(\bm{v_n}) = \bm{u_n}$;
    \item scriviamo $\bm{u_i}$ in termini della base $\beta$, cioe' \begin{equation*}
        \bm{u_i} = a_{1i}\bm{w_1} + \dots + a_{mi}\bm{w_m} \iff [\bm{u_i}]_{\beta} = \begin{pmatrix}
            a_{1i} \\ \vdots \\ a_{mi}
        \end{pmatrix};
    \end{equation*}
    \item notiamo che dato che $\bm{v_i}$ e' l'$i$-esimo vettore della base $\alpha$, allora la sua rappresentazione in termini della base sara' un vettore colonna con tutti $0$ tranne un $1$ in posizione $i$;
    \item per la proposizione \ref{j-esima_colonna} il risultato del prodotto $[f]^{\alpha}_{\beta} \cdot (\bm{v_i})_{\alpha}$ sara' l'$i$-esima colonna della matrice $[f]^{\alpha}_{\beta}$, ma $[f]^{\alpha}_{\beta} \cdot (\bm{v_i})_{\alpha}$ deve essere uguale a $[f(v_i)]_{\beta} = [\bm{u_i}]_{\beta}$, dunque l'$i$-esima colonna di $[f]^{\alpha}_{\beta}$ sara' data dal vettore colonna $[\bm{u_i}]_{\beta}$;
    \item dunque la matrice avra' per colonne i vettori $[\bm{u_1}]_{\beta}, \dots, [\bm{u_n}]_{\beta}$, cioe'
    \begin{equation}
        [f]^{\alpha}_{\beta} = \begin{pmatrix}
            a_{11} & a_{12} & \dots & a_{1n} \\
            a_{21} & a_{22} & \dots & a_{2n} \\
            \vdots & \vdots & \vdots& \vdots \\
            a_{m1} & a_{m2} & \dots & a_{mn} \\
        \end{pmatrix}.
    \end{equation}
\end{itemize}

\begin{example}
    Sia $V = \M_{2\times 2}(\R)$ e sia $\alpha = \ang{\begin{psmallmatrix}1&0\\0&0\end{psmallmatrix}, \begin{psmallmatrix}0&1\\0&0\end{psmallmatrix}, \begin{psmallmatrix}0&0\\1&0\end{psmallmatrix}, \begin{psmallmatrix}0&0\\0&1\end{psmallmatrix}}$ una sua base.    
    Sia $A = \begin{psmallmatrix}0&1\\1&0\end{psmallmatrix} \in V$ e $f : V \to V$ tale che $f(B) = AB - BA$.
    \begin{enumerate}
        \item Dimostrare che $f$ e' lineare.
        \item Calcolare $[f]^{\alpha}_{\alpha}$.
        \item Dare una base di $\Imm{f}$.
        \item Dare una base di $\ker f$.
    \end{enumerate}
\end{example}
\begin{solution}
    Verifichiamo i quattro punti.
    \begin{enumerate}
        \item Dimostriamo che $f$ e' lineare.
            \begin{alignat*}{2}
                &\text{(a) } f(\bm{0}) = f\left(\begin{psmallmatrix}0&0\\0&0\end{psmallmatrix}\right) = A\begin{psmallmatrix}0&0\\0&0\end{psmallmatrix} - \begin{psmallmatrix}0&0\\0&0\end{psmallmatrix}A = \begin{psmallmatrix}0&0\\0&0\end{psmallmatrix} - \begin{psmallmatrix}0&0\\0&0\end{psmallmatrix} = \bm 0 \\
                &\begin{alignedat}{1}
                    \text{(b) } f(B + C) &= A(B + C) - (B + C)A \\
                    &= AB + AC - BC - CA \\
                    &= (AB - BA) + (AC - CA) \\
                    &= f(B) + f(C)
                \end{alignedat}\\
                &\begin{alignedat}{1}
                    \text{(c) } f(kB) &= A(kB) - (kB)A \\
                    &= k(AB) - k(BA) \\
                    &= k(AB - BA) \\
                    &= kf(B)
                \end{alignedat}
            \end{alignat*}
            dunque $f$ e' lineare.
        \item Seguo il procedimento per ottenere $[f]^{\alpha}_{\alpha}$. Innanzitutto calcolo il risultato di $f$ sui vettori della base $\alpha$:
        \begin{alignat*}{1}
            &f\left(\begin{psmallmatrix}1&0\\0&0\end{psmallmatrix}\right) = 
            \begin{psmallmatrix}0&1\\1&0\end{psmallmatrix}\begin{psmallmatrix}1&0\\0&0\end{psmallmatrix} - \begin{psmallmatrix}1&0\\0&0\end{psmallmatrix}\begin{psmallmatrix}0&1\\1&0\end{psmallmatrix} = \begin{psmallmatrix}0&0\\1&0\end{psmallmatrix} - \begin{psmallmatrix}0&1\\0&0\end{psmallmatrix} = \begin{psmallmatrix}0&-1\\1&0\end{psmallmatrix} \\
            &f\left(\begin{psmallmatrix}0&1\\0&0\end{psmallmatrix}\right) = 
            \begin{psmallmatrix}0&1\\1&0\end{psmallmatrix}\begin{psmallmatrix}0&1\\0&0\end{psmallmatrix} - \begin{psmallmatrix}0&1\\0&0\end{psmallmatrix}\begin{psmallmatrix}0&1\\1&0\end{psmallmatrix} = \begin{psmallmatrix}0&0\\0&1\end{psmallmatrix} - \begin{psmallmatrix}1&0\\0&0\end{psmallmatrix} = \begin{psmallmatrix}-1&0\\0&1\end{psmallmatrix} \\
            &f\left(\begin{psmallmatrix}0&0\\1&0\end{psmallmatrix}\right) = 
            \begin{psmallmatrix}0&1\\1&0\end{psmallmatrix}\begin{psmallmatrix}0&0\\1&0\end{psmallmatrix} - \begin{psmallmatrix}0&0\\1&0\end{psmallmatrix}\begin{psmallmatrix}0&1\\1&0\end{psmallmatrix} = \begin{psmallmatrix}1&0\\0&0\end{psmallmatrix} - \begin{psmallmatrix}0&0\\0&1\end{psmallmatrix} = \begin{psmallmatrix}1&0\\0&-1\end{psmallmatrix} \\
            &f\left(\begin{psmallmatrix}0&0\\0&1\end{psmallmatrix}\right) = 
            \begin{psmallmatrix}0&1\\1&0\end{psmallmatrix}\begin{psmallmatrix}0&0\\0&1\end{psmallmatrix} - \begin{psmallmatrix}0&0\\0&1\end{psmallmatrix}\begin{psmallmatrix}0&1\\1&0\end{psmallmatrix} = \begin{psmallmatrix}0&1\\0&0\end{psmallmatrix} - \begin{psmallmatrix}0&0\\1&0\end{psmallmatrix} = \begin{psmallmatrix}0&1\\-1&0\end{psmallmatrix}
        \end{alignat*}
        dunque le loro coordinate rispetto alla base $\alpha$ sono
        \begin{align*}
            &[f(\bm{v_1})]_{\alpha} = \begin{pmatrix}
                0 \\ -1 \\ 1 \\ 0
            \end{pmatrix} &[f(\bm{v_2})]_{\alpha} = \begin{pmatrix}
                -1 \\ 0 \\ 0 \\ 1
            \end{pmatrix}
            \\&[f(\bm{v_3})]_{\alpha} = \begin{pmatrix}
                1 \\ 0 \\ 0 \\ -1
            \end{pmatrix} &[f(\bm{v_4})]_{\alpha} = \begin{pmatrix}
                0 \\ 1 \\ -1 \\ 0
            \end{pmatrix}
        \end{align*}
        cioe' \begin{equation*}
            [f]^{\alpha}_{\alpha} = \begin{pmatrix}
                0 & -1 & 1 & 0 \\ -1 & 0 & 0 & 1 \\
                1 & 0 & 0 & -1 \\ 0 & 1 & -1 & 0
            \end{pmatrix}.
        \end{equation*}
        \item Per la proposizione \ref{base_mappata_generatori_immagine} sappiamo che l'insieme
        $\{f(\bm{v_1}), f(\bm{v_2}), f(\bm{v_3}), f(\bm{v_4})\}$ e' un insieme di generatori dell'immagine della funzione. 
        Per eliminare i vettori indipendenti passiamo all'isomorfismo con $\R^4$ tramite la base di partenza $\alpha$. Chiamiamo $\widetilde{W}$ lo spazio isomorfo a $\Imm{f}$, allora notiamo che il ruolo di $f$ nel nuovo spazio e' dato dalla matrice $[f]^{\alpha}_{\alpha}$, dunque il corrispondente insieme di generatori di $\widetilde{W}$ sara' \begin{gather*}
            \left\{ [f]^{\alpha}_{\alpha}[v_1]_{\alpha}, [f]^{\alpha}_{\alpha}[v_2]_{\alpha}, [f]^{\alpha}_{\alpha}[v_3]_{\alpha}, [f]^{\alpha}_{\alpha}[v_4]_{\alpha}\right\} \\
            \intertext{che e' uguale a }
            \left\{\begin{psmallmatrix} 0 \\ -1 \\ 1 \\ 0 \end{psmallmatrix}, \begin{psmallmatrix} -1 \\ 0 \\ 0 \\ 1 \end{psmallmatrix}, \begin{psmallmatrix} 1 \\ 0 \\ 0 \\ -1 \end{psmallmatrix}, \begin{psmallmatrix} 0 \\ 1 \\ -1 \\ 0 \end{psmallmatrix}\right\}  
        \end{gather*}
        Semplifichiamolo tramite mosse di colonna: \begin{gather*}
            \begin{pmatrix}[c|c|c|c]
                0 & -1 & 1 & 0 \\ -1 & 0 & 0 & 1 \\
                1 & 0 & 0 & -1 \\ 0 & 1 & -1 & 0
            \end{pmatrix} \xrightarrow[R_2 + R_3]{R_1 + R_4}
            \begin{pmatrix}[c|c|c|c]
                0 & 0 & 1 & 0 \\ 0 & 0 & 0 & 1 \\
                0 & 0 & 0 & -1 \\ 0 & 0 & -1 & 0
            \end{pmatrix} \xrightarrow[]{}
            \begin{pmatrix}[c|c|c|c]
                1 & 0 & 0 & 0\\0 & 1 & 0 & 0 \\
                0 & -1 & 0 & 0\\ -1 & 0 & 0 & 0
            \end{pmatrix}
        \end{gather*}
        dunque i vettori $\begin{psmallmatrix} 1 \\ 0 \\ 0 \\ -1 \end{psmallmatrix}, \begin{psmallmatrix} 0 \\ 1 \\ -1 \\ 0 \end{psmallmatrix}$ sono indipendenti e generano $\widetilde{W}$, dunque sono una base di $\widetilde{W}$.

        Tornando allo spazio originale otteniamo che una base di $\Imm{f}$ e' data da \[
            \beta = \ang{\begin{pmatrix} 1 & 0 \\ 0 & -1 \end{pmatrix}, \begin{pmatrix} 0 & 1 \\ -1 & 0 \end{pmatrix}}.
        \] e dunque $\dim \Imm{f} = 2$.
        \item Per definizione di kernel \[
            \ker f = \left\{ \begin{psmallmatrix}x&y\\z&t \end{psmallmatrix}\in \M_{2\times 2}(\R) \mid f\left(\begin{psmallmatrix}x&y\\z&t \end{psmallmatrix}\right) = \begin{psmallmatrix}0&0\\0&0 \end{psmallmatrix}\right\}.
        \] Sia $\widetilde{V} \subseteq \R^4$ lo spazio isomorfo a $\ker f$ tramite l'isomorfismo dato dalle coordinate dei vettori rispetto alla base $\alpha$. Allora \begin{alignat*}{1}
            \widetilde{V} &= \left\{ \begin{pmatrix}x\\y\\z\\t \end{pmatrix}\in \R^4 \mid [f]^{\alpha}_{\alpha}\begin{pmatrix}x\\y\\z\\t \end{pmatrix} = \begin{pmatrix}0\\0\\0\\0 \end{pmatrix}\right\} \\
            &= \left\{ \begin{pmatrix}x\\y\\z\\t \end{pmatrix}\in \R^4 \mid \begin{pmatrix}
                0 & -1 & 1 & 0 \\ -1 & 0 & 0 & 1 \\
                1 & 0 & 0 & -1 \\ 0 & 1 & -1 & 0
            \end{pmatrix}\begin{pmatrix}x\\y\\z\\t \end{pmatrix} = \begin{pmatrix}0\\0\\0\\0 \end{pmatrix}\right\}
        \end{alignat*}
        Dunque $\widetilde{V}$ e' formato da tutti e solo i vettori $\bm{x} \in \R^4$ che sono soluzione del sistema lineare $[f]^{\alpha}_{\alpha}\bm{x} = \bm{0}$. Risolviamolo tramite eliminazione gaussiana:
        \begin{gather*}
            \begin{pmatrix}
                0  & -1 & 1  & 0  \\ 
                -1 & 0  & 0  & 1  \\
                1  & 0  & 0  & -1 \\ 
                0  & 1  & -1 & 0
            \end{pmatrix} \xrightarrow[]{scambio}
            \begin{pmatrix}
                1  & 0  & 0  & -1 \\
                0  & 1  & -1 & 0  \\
                0  & -1 & 1  & 0  \\ 
                -1 & 0  & 0  & 1
            \end{pmatrix} \xrightarrow[R_4 + R_1]{R_3 + R_2} \\
            \xrightarrow[R_4 + R_1]{R_3 + R_2} \begin{pmatrix}
                1  & 0  & 0  & -1 \\
                0  & 1  & -1 & 0  \\
                0  & 0  & 0  & 0  \\ 
                0  & 0  & 0  & 0
            \end{pmatrix} \iff \left\{
                \begin{array}{@{}roror }
                    x & - & t & = & 0 \\
                    y & - & z & = & 0
                \end{array}
            \right. \iff \left\{
                \begin{array}{@{}ror }
                    x & = & t\\
                    y & = & z
                \end{array}
            \right.\\
            \intertext{dunque scegliendo $z, t \in \R$ libere otteniamo}
            \iff \begin{pmatrix} x \\ y \\ z \\t \end{pmatrix} = \begin{pmatrix} t \\ z \\ z \\ t \end{pmatrix} = z\begin{pmatrix} 0 \\ 1 \\ 1 \\ 0 \end{pmatrix} + t\begin{pmatrix} 1 \\ 0 \\ 0 \\ 1 \end{pmatrix}
        \end{gather*}
        Dunque $\tilde{\gamma} = \ang{\begin{psmallmatrix} 0 \\ 1 \\ 1 \\ 0 \end{psmallmatrix}, \begin{psmallmatrix} 1 \\ 0 \\ 0 \\ 1 \end{psmallmatrix}}$ e' un insieme di generatori di $\widetilde{V}$. Inoltre sono anche indipendenti (poiche' hanno pivot ad altezze diverse), dunque $\tilde{\gamma}$ e' una base di $\widetilde{V}$. 
            
        Tornando tramite la base $\alpha$ allo spazio iniziale otteniamo che \[
            \gamma = \ang{\begin{pmatrix} 0 & 1 \\ 1 & 0 \end{pmatrix}, \begin{pmatrix} 1 & 0 \\ 0 & 1 \end{pmatrix}}  
        \] e' una base di $\ker f$, dunque $\dim \ker f = 2$.
    \end{enumerate}
\end{solution}

\chapter{Determinanti}

\section{Definizione e significato del determinante}

\begin{definition}
    Sia $A$ una matrice quadrata $n \times n$ e siano $\bm{C_1}, \dots, \bm{C_n} \in \R^n$ le sue colonne. Allora si dice determinante una funzione \[
        \det : \M_{n \times n}(\R) \to \R 
    \] che rispetta le seguenti proprieta':

    \begin{enumerate}[(i)]
        \item $\det I_n = 1$, cioe' il determinante della matrice identita' $n \times n$ deve essere 1;
        \item se per qualche $i, j$ compresi tra $1$ e $n$, con $i \neq j$, vale che $\bm{C_i} = \bm{C_j}$, allora $\det A = 0$, cioe' se due colonne della matrice sono uguali il determinante deve essere 0;
        \item se $A'$ e' la matrice ottenuta moltiplicando una colonna della matrice $A$ per $\lambda \in \R$, cioe' $A' = \begin{pmatrix} \bm{C_1} & \dots & \lambda \bm{C_i} & \dots & \bm{C_n} \end{pmatrix}$ allora \[\det A' = \lambda \det A;\]
        \item se la colonna $\bm{C_i}$ e' esprimibile come $\bm{v_1} + \bm{v_2}$ con $\bm{v_1}, \bm{v_2} \in \R^n$, cioe' $A = \begin{pmatrix} \bm{C_1} & \dots & \bm{v_1} + \bm{v_2} & \dots & \bm{C_n} \end{pmatrix}$
        allora \[
            \det A = \det \begin{pmatrix} \bm{C_1} & \dots & \bm{v_1} & \dots & \bm{C_n} \end{pmatrix} + \det \begin{pmatrix} \bm{C_1} & \dots & \bm{v_2} & \dots & \bm{C_n} \end{pmatrix} 
        \] cioe' il determinante e' lineare nelle colonne della matrice.
    \end{enumerate}
\end{definition}

Possiamo anche definire il determinante come una funzione che prende esattamente $n$ vettori di $\R^n$ e restituisce un numero reale, cioe' $\det : (\R^n)^n \to \R$ e che rispetta le seguenti proprieta':
\begin{enumerate}
    [(i)]
    \item se $\bm{c_1}, \dots, \bm{c_n}$ sono i vettori della base standard di $\R^n$, allora \[\det(\bm{c_1}, \dots, \bm{c_n}) = 1;\]
    \item $\det(\bm{v_1}, \dots, \bm{v_i}, \dots, \bm{v_i}, \dots, \bm{v_n}) = 0$, cioe' se due dei vettori sono uguali allora il determinante e' nullo;
    \item $\det(\bm{v_1}, \dots, \lambda\bm{v_i}, \dots, \bm{v_n}) = \lambda \det (\bm{v_1}, \dots, \bm{v_i}, \dots, \bm{v_n})$;
    \item le somme escono fuori dal determinante, cioe' \begin{alignat*}{1}
        &\det(\bm{v_1}, \dots, \bm{v_i} + \bm{w}, \dots, \bm{v_n})\\
        = &\det (\bm{v_1}, \dots, \bm{v_i}, \dots, \bm{v_n}) + \det (\bm{v_1}, \dots, \bm{w}, \dots, \bm{v_n}).
    \end{alignat*}
\end{enumerate}

Dalle quattro proprieta' base ne discendono altre, che elenchiamo in questa proposizione:
\begin{proposition}
    Il determinante ha le seguenti proprieta':
    \begin{enumerate}
        [(i)]
        \item se scambio due colonne della matrice tra loro il determinante cambia segno;
        \item le combinazioni lineari escono fuori dal determinante;
        \item se una delle $n$ colonne e' combinazione lineare delle restanti, cioe' se gli $n$ vettori formano un insieme di vettori linearmente dipendenti, allora il determinante e' uguale a $0$;
        \item sommando ad una colonna un multiplo di un'altra colonna il determinante non cambia;
        \item il determinante di una matrice e' uguale al determinante della trasposta.
    \end{enumerate}
\end{proposition}

Notiamo che la mossa principale di Gauss-Jordan, cioe' sommare ad una colonna un multiplo di un'altra colonna, non modifica il determinante di una matrice: possiamo calcolare i determinanti quindi tramite mosse di Gauss-Jordan, facendo attenzione a cambiare il segno se scambiamo due colonne o a portare fuori i fattori per cui moltiplichiamo le colonne.

Dall'ultima proprieta' segue che ogni proprieta' che si basa sulle colonne puo' anche essere riformulata in termini delle righe della matrice (che corrispondono alle colonne della trasposta).

Dalle proprieta' precedenti segue che il determinante di una matrice e' $0$ se e solo se ci sono due colonne linearmente dipendenti: dunque il determinante e' una funzione che indica la dipendenza lineare tra i vettori a cui viene applicato.

\subsection{Determinante di matrici particolari}

\subsection{Determinante di matrici diagonali}

Consideriamo la matrice diagonale \[
    D = \begin{pmatrix}
        \lambda_1   &0          &\dots  &0 \\
        0           &\lambda_2  &\dots  &0 \\
        \vdots      &\vdots     &\ddots &\vdots \\
        0           &0          &\dots  &\lambda_n \\
    \end{pmatrix}.    
\] Applicando il terzo assioma possiamo estrarre i coefficienti di ogni colonna, ottenendo
\begin{alignat*}{1}
    \det \begin{pmatrix}
        \lambda_1   &0          &\dots  &0 \\
        0           &\lambda_2  &\dots  &0 \\
        \vdots      &\vdots     &\ddots &\vdots \\
        0           &0          &\dots  &\lambda_n \\
    \end{pmatrix} = &\lambda_1 \det \begin{pmatrix}
        1           &0          &\dots  &0 \\
        0           &\lambda_2  &\dots  &0 \\
        \vdots      &\vdots     &\ddots &\vdots \\
        0           &0          &\dots  &\lambda_n \\
    \end{pmatrix} \\
    = &\lambda_1 \lambda_2 \det \begin{pmatrix}
        1           &0          &\dots  &0 \\
        0           &1          &\dots  &0 \\
        \vdots      &\vdots     &\ddots &\vdots \\
        0           &0          &\dots  &\lambda_n \\
    \end{pmatrix} \\
    \intertext{Ripetendo il procedimento per ogni colonna arriviamo a}
    = &\lambda_1 \lambda_2 \dots \lambda_n \det I_n \\
    = &\lambda_1 \lambda_2 \dots \lambda_n.
\end{alignat*}
Dunque il determinante di una matrice diagonale e' il prodotto degli elementi sulla diagonale principale.

\subsubsection{Determinante di matrici triangolari superiori o inferiori}

Consideriamo una matrice triangolare superiore (o inferiore), cioe' una matrice che ha tutti zeri sotto (o sopra) la diagonale principale. Tramite mosse di Gauss-Jordan possiamo trasformare questa matrice in una matrice diagonale senza dover scambiare colonne tra di loro, dunque il determinante della matrice triangolare e' uguale al determinante della matrice diagonale, cioe' e' il prodotto degli elementi sulla sua diagonale principale.

\begin{equation*}
    \det \begin{pmatrix}
        \lambda_1   &\star      &\dots  &\star \\
        0           &\lambda_2  &\dots  &\star \\
        \vdots      &\vdots     &\ddots &\vdots \\
        0           &0          &\dots  &\lambda_n \\
    \end{pmatrix} = \det \begin{pmatrix}
        \lambda_1   &0          &\dots  &0 \\
        \star       &\lambda_2  &\dots  &0 \\
        \vdots      &\vdots     &\ddots &\vdots \\
        \star       &\star      &\dots  &\lambda_n \\
    \end{pmatrix} = \lambda_1 \lambda_2 \dots \lambda_n
\end{equation*}
dove $\star$ indica un qualsiasi numero reale.

\subsubsection{Determinante di matrici $2 \times 2$}

Consideriamo una matrice $A \in \M_{2 \times 2}(\R)$ generica e calcoliamone il determinante. Se $a \neq 0$ allora \begin{equation*}
    \det \begin{pmatrix} a & b \\ c & d \end{pmatrix} = \det \begin{pmatrix} a & b \\ 0 & d - \frac{c}{a}b \end{pmatrix} = ad - bc.
\end{equation*}
Se $a = 0$ allora \begin{equation*}
\det \begin{pmatrix} 0 & b \\ c & d \end{pmatrix} = -\det \begin{pmatrix} b & 0 \\ d & c \end{pmatrix} = -bc = 0d - bc = ad - bc.
\end{equation*}
Dunque il determinante di $\begin{psmallmatrix} a & b \\ c & d \end{psmallmatrix}$ e' $ad - bc$.

Il determinante di una matrice $2 \times 2$ e' l'area del parallelogramma che ha come lati i vettori che formano le colonne della matrice. Notiamo infatti che se i due vettori sono sulla stessa retta, cioe' se sono dipendenti, allora l'area del parallelogramma e' 0, esattamente come il determinante.

\subsubsection{Determinante di matrici $3 \times 3$}

Per calcolare il determinante di una matrice $A \in \M_{3 \times 3}(\R)$ generica possiamo usare la regola di Sarrus: creiamo una matrice $3 \times 5$ dove le ultime due colonne sono le prime due ripetute. Il determinante sara' allora la somma dei prodotti delle prime tre diagonali da sinistra verso destra meno il prodotto delle tre diagonali da destra verso sinistra.
Dunque se \[
    A = \begin{pmatrix}
        a & b & c \\
        d & e & f \\
        g & h & i \\
    \end{pmatrix}
\] consideriamo la matrice \[
    A = \begin{pmatrix}[ccc|cc]
        a & b & c & a & b \\
        d & e & f & d & e\\
        g & h & i & g & h\\
    \end{pmatrix}
\] e otteniamo che \[
    \det A = aei + bfg + cdh - bdi - afh - ceg.    
\]

Il determinante di una matrice $3 \times 3$ e' il volume del "parallelepipedo" che ha come lati i vettori che formano le colonne della matrice: infatti se un vettore e' nello span degli altri due allora il volume viene 0.

\section{Sviluppi di Laplace}

\begin{definition}
    Sia $A \in \M_{n\times m}(\R)$. Allora diciamo che $B$ e' una sottomatrice di $A$ se $B \in \M_{(n-k) \times (m-h)}(\R)$ e $B$ si ottiene eliminando $k$ righe e $h$ colonne di $A$.
\end{definition}

\begin{definition}
    Sia $A \in \M_{n\times m}(\R)$. Allora diciamo che $B$ e' un minore di $A$ se e' una sottomatrice quadrata di $A$.
\end{definition}

Possiamo quindi enunciare il metodo degli sviluppi di Laplace per calcolare il determinante di una matrice.

\begin{theorem}
    [Sviluppi di Laplace]
    Sia $A \in \M_{n \times n}(\R)$ una matrice quadrata. 
    
    Sia $C_j = \begin{pmatrix}
        a_{1j} & \dots & a_{nj}
    \end{pmatrix}^T$ una colonna qualsiasi di $A$. 
    
    Chiamo $M_{ij}$ il minore di $A$ ottenuto eliminando la riga $i$-esima e la colonna $j$-esima. Inoltre per ogni $i$ compreso tra 1 e $n$ chiamo cofattore $c_{ij}$ la quantita' \[
        c_{ij} = (-1)^{i+j} \det M_{ij}.
    \]
    Allora vale che \begin{equation}
        \det A = a_{1j}c_{1j} + \dots + a_{nj}c_{nj} = \sum_{i = 1}^n a_{ij}c_{ij}.
    \end{equation}
\end{theorem}

\section{Rango e determinanti}

Diamo ora la definizione esatta di rango di una matrice.

\begin{definition}
    Sia $A \in \M_{n \times m}(\R)$ e sia $L_A : \R^m \to \R^n$ l'applicazione lineare associata alla matrice $A$. Allora si dice rango della matrice $A$ la dimensione dell'immagine dell'applicazione lineare associata, cioe'
    \begin{equation}
        \rk{A} = \dim \Imm{L_A}
    \end{equation}
\end{definition}

\begin{proposition}
    Sia $A \in \M_{n \times m}(\R)$. Siano $R_1, \dots, R_n \in \R^m$ le righe di $A$ e $C_1, \dots, C_m \in \R^n$ le colonne di $A$. Sia inoltre $L_A : \R^m \to \R^n$ l'applicazione lineare associata alla matrice $A$.
    
    Allora i seguenti fatti sono equivalenti:
    \begin{itemize}
        \item $k = \rk{A}$
        \item $k = \dim \Span{C_1, \dots, C_m}$;
        \item $k = \dim \Span{R_1, \dots, R_n}$;
        \item $k$ e' il numero di pivot della matrice a scalini $A'$ ottenuta tramite mosse di Gauss a partire dalla matrice $A$;
    \end{itemize}
\end{proposition}
\begin{proof}
    Se $k$ e' il rango della matrice, allora per definizione di rango $k = \dim \Imm{L_A}$. Per la proposizione \ref{span_colonne=immagine_applicazione_associata} lo span delle colonne della matrice e' uguale all'immagine dell'applicazione lineare associata, dunque $k = \dim \Span{C_1, \dots, C_m}$.

    Supponiamo che la matrice sia a scalini per colonne. Allora le colonne indipendenti sono tutte e solo le colonne con i pivot, mentre le altre colonne sono nulle. Dato che le colonne con i pivot formano una base dell'immagine, segue che devono esserci esattamente $k = \rk{A}$ colonne indipendenti, e quindi $k$ pivot. 
    Inoltre le righe indipendenti sono quelle con i pivot, dunque devono esserci anche $k$ righe indipendenti, cioe' $\dim \Span{R_1, \dots, R_n} = k$.

    Supponiamo che la matrice non sia a scalini per colonne. Allora riduciamola a scalini per colonne tramite mosse di Gauss, ottenendo la matrice $A'$ che ha per colonne i vettori $C'_1, \dots, C'_m$ e per righe i vettori $R'_1, \dots, R'_n$.
    Per la proposizione \ref{span_colonne_indipendenti} segue che $\Span{C'_1, \dots, C'_m} = \Span{C_1, \dots, C_m}$, dunque anche le loro dimensioni saranno uguali. Dato che la dimensione di $\Span{C'_1, \dots, C'_m}$ e' data dal numero di colonne indipendenti, cioe' dal numero di pivot per colonna, abbiamo dimostrato che il numero di pivot e' uguale alla dimensione di $\Span{C_1, \dots, C_m}$, cioe' al rango di $A$. Infine per la proposizione \ref{invarianza_dim_righe_per_mosse_colonna} ridurre una matrice a scalini per colonne non cambia la dimensione delle righe, dunque dato che la dimensione di $\Span{R'_1, \dots, R'_n}$ e' uguale al numero di pivot (cioe' $k$), allora anche la dimensione di $\Span{R_1, \dots, R_n}$ dovra' essere uguale a $k$.
\end{proof}

\chapter{Autovettori e diagonalizzabilita'}

\section{Autovettori e matrici simili}

\subsection{Matrici simili}

\begin{definition}
    Siano $A, B \in \M_{n \times n}(\R)$. Allora $A$ e' simile a $B$ se esiste $M \in \M_{n \times n}(\R)$ invertibile tale che \begin{equation}
        A = M^{-1}BM.
    \end{equation}
\end{definition}

\begin{remark}
    Posso anche scrivere la relazione di similitudine come $A = PBP^{-1}$: basta sostituire $P = M^{-1}$ da cui otteniamo $P^{-1} = (M^{-1})^{-1} = M$.
\end{remark}
\begin{proposition}
    La relazione di similitudine e' una relazione di equivalenza, ovvero: siano $A, B, C \in \M_{n \times n}(\R)$; allora valgono le seguenti:
    \begin{align*}
        &i.     &&A \text{ e' simile ad } A  &&\text{(riflessivita')}\\
        &ii.    &&A \text{ e' simile a } B \implies B \text{ e' simile ad } A  &&\text{(simmetria)}\\
        &iii.   &&A \text{ e' simile a } B,\ B \text{ e' simile a } C \implies A \text{ e' simile a } C.   &&\text{(transitivita')}
    \end{align*}
\end{proposition}
\begin{proof}
    Dimostriamo i tre fatti.
    \begin{enumerate}[(i)]
        \item Basta scegliere come matrice $M$ l'identita' $I_n$. Dato che $(I_n)^{-1} = I_n$ allora vale che $A = I_nAI_n = (I_n)^{-1}AI_n$, cioe' $A$ e' simile a se stessa.
        \item Dato che $A$ e' simile a $B$ possiamo scrivere $A = M^{-1}BM$. Allora vale che \begin{alignat*}{1}
            &A = M^{-1}BM\\
            \iff &MA = MM^{-1}BM \\
            \iff &MA = BM \\
            \iff &MAM^{-1} = BMM^{-1} \\
            \iff &MAM^{-1} = B
        \end{alignat*}
        cioe' $B$ e' simile ad $A$.
        \item Scriviamo $A = M^{-1}BM$ e $B = P^{-1}CP$. Allora segue che \begin{alignat*}
            {1}
            A &= M^{-1}BM\\
            &= M^{-1}(P^{-1}CP)M \\
            &= (M^{-1}P^{-1})C(PM) \\
            &= (PM)^{-1}C(PM)
        \end{alignat*}
        dove l'ultimo passaggio e' giustificato dalla proposizione \ref{inversa_prodotto}. \qedhere
    \end{enumerate}
\end{proof}

\begin{proposition}
    Sia $f : V \to V$ un endomorfismo e siano $\alpha, \beta$ due basi di $V$. Allora $[f]_{\alpha}^{\alpha}$ e' simile a $[f]_{\beta}^{\beta}$.
\end{proposition}
\begin{proof}
    Basta moltiplicare la matrice $[f]_{\alpha}^{\alpha}$ a destra per la matrice del cambio di base da $\beta$ ad $\alpha$ e a sinistra per la sua inversa. Infatti \[
        ([\id]^{\beta}_{\alpha})^{-1}[f]_{\alpha}^{\alpha}[\id]^{\beta}_{\alpha} = [\id]^{\alpha}_{\beta}[f]_{\alpha}^{\alpha}[\id]^{\beta}_{\alpha} = [\id]^{\alpha}_{\beta}[f]_{\beta}^{\alpha} = [f]_{\beta}^{\beta}
    \] come volevasi dimostrare.
\end{proof}

\begin{definition}
    Sia $A \in \M_{n \times n}(\R)$. Allora $A$ si dice diagonalizzabile se esiste una matrice diagonale $D$ simile ad $A$.
\end{definition}

\subsection{Autovettori, autovalori e autospazio}

\begin{definition}
    Sia $V$ uno spazio vettoriale. Allora ogni applicazione lineare $f : V \to V$ si dice \textbf{endomorfismo}.
\end{definition}

\begin{definition}
    Sia $V$ uno spazio vettoriale, $f : V \to V$ un endomorfismo e sia $\lambda \in \R$. Allora si dice \textbf{autospazio di $\bm{V}$ relativo a $\bm{\lambda}$} l'insieme \begin{equation}
        V_{\lambda} = \left\{ v \in V \mid f(v) = \lambda v \right\}.
    \end{equation}
\end{definition}

\begin{proposition}
    $V_{\lambda}$ e' un sottospazio vettoriale di $V$.
\end{proposition}

\begin{definition}
    Sia $f : V \to V$ un endomorfismo. Allora $\bm{v} \in V$, $\bm{v} \neq \bm{0}$ si dice \textbf{autovettore di $\bm{V}$ relativo a $\bm{\lambda}$} se $\bm{v} \in V_{\lambda}$.
\end{definition}

\begin{definition}
    Sia $f : V \to V$ un endomorfismo. Allora $\lambda \in \R$ si dice \textbf{autovalore di $\bm{V}$} se $V_{\lambda} \neq \left\{ \bm{0}\right\}$.
\end{definition}

Possiamo dare le stesse definizioni di autovalore, autovettore e autospazio considerando una matrice quadrata $A \in \M_{n \times n}(\R)$.

\begin{theorem}
    Sia $f : V \to V$ un endomorfismo e siano $\lambda_1, \dots, \lambda_k$ i suoi autovalori.

    Allora per qualunque $\bm{v_1} \in V_{\lambda_{1}}, \dots, \bm{v_n} \in V_{\lambda_n}$ (con $\bm{v_1}, \dots, \bm{v_n} \neq \bm 0$) segue che $\{\bm{v_1}, \dots, \bm{v_n}\}$ e' un insieme di vettori linearmente indipendenti, ovvero gli autospazi $V_{\lambda_1}, \dots, V_{\lambda_{n}}$ sono in somma diretta.
\end{theorem}
\begin{proof}
    Per induzione su $k$.
    \begin{description}
        \item[Caso base.] Se $k = 1$ allora c'e' un solo autovettore nell'insieme. Inoltre dato che $\bm{v_1} \neq 0$ sicuramente esso forma un insieme di vettori linearmente indipendenti.
        \item[Passo induttivo.] Supponiamo che l'insieme con $k-1$ vettori sia linearmente indipendente e dimostriamo che anche l'insieme formato da $k$ vettori lo e'.
        
        Consideriamo una combinazione lineare di questi vettori e poniamola uguale al vettore nullo: \begin{equation} \label{comb_lin_th_autospazi_somma_diretta}
            c_1\bm{v_1} + \dots + c_{k-1}\bm{v_{k-1}} + c_k\bm{v_k} = \bm 0.
        \end{equation}
        Applichiamo ad entrambi i membri l'applicazione lineare $(f - \lambda_k\id) : V \to V$, ottenendo:
        \begin{alignat*}
            {1}
            &(f - \lambda_k\id)(c_1\bm{v_1} + \dots + c_{k-1}\bm{v_{k-1}} + c_k\bm{v_k}) = (f - \lambda_k\id)(\bm 0)\\
            \iff &c_1(f - \lambda_k\id)(\bm{v_1}) + \dots + c_{k-1}(f - \lambda_k\id)(\bm{v_{k-1}}) +\\
                 &+ c_k(f - \lambda_k\id)(\bm{v_k}) = \bm 0 \\
            \iff &c_1(f(\bm{v_1}) - \lambda_k\id(\bm{v_1})) + \dots + c_{k-1}(f(\bm{v_{k-1}}) - \lambda_k\id(\bm{v_{k-1}})) +\\
                 &+ c_k(f(\bm{v_k}) - \lambda_k\id(\bm{v_k})) = \bm 0 \\
            \iff &c_1(\lambda_1\bm{v_1} - \lambda_k\bm{v_1}) + \dots + c_{k-1}(\lambda_{k-1}\bm{v_{k-1}} - \lambda_k\bm{v_{k-1}}) + \\
                 &+ c_k(\lambda_k\bm{v_k} - \lambda_k\bm{v_k}) = \bm 0\\
            \iff &c_1(\lambda_1 - \lambda_k)\bm{v_1} + \dots + c_{k-1}(\lambda_{k-1} - \lambda_k)\bm{v_{k-1}} = \bm 0.
        \end{alignat*}
        Per ipotesi induttiva sappiamo che i vettori $\bm{v_1}, \dots, \bm{v_{k-1}}$ sono indipendenti, dunque segue che \[
            c_1(\lambda_1 - \lambda_k) = \dots = c_{k-1}(\lambda_{k-1} - \lambda_k) = 0.
        \]
        Dato che gli autovalori sono distinti segue che $\lambda_i - \lambda_k \neq 0$ per $i < k$, dunque \[
            c_1 = \dots = c_{k-1} = 0.    
        \]
        Sostituendo cio' nell'equazione \ref{comb_lin_th_autospazi_somma_diretta} otteniamo $c_k\bm{v_k} = \bm 0$, ma dato che $\bm{v_k} \neq \bm{0}$ segue che $c_k = 0$, cioe' i vettori sono indipendenti. \qedhere
    \end{description}
\end{proof}

Possiamo sfruttare gli autovettori per diagonalizzare una matrice, come ci conferma la prossima proposizione.

\begin{proposition}
    Sia $V$ uno spazio vettoriale, $f: V \to V$ un endomorfismo e siano $\bm{v_1}, \dots, \bm{v_n}$ gli autovettori di $f$ relativi agli autovalori $\lambda_1, \dots, \lambda_n$. 
    
    Allora se $\beta = \ang{\bm{v_1}, \dots, \bm{v_n}}$ e' una base di $V$ la matrice $[f]_{\beta}^{\beta}$ e' diagonale ed ha sulla diagonale $\lambda_1, \dots, \lambda_n$.
\end{proposition}
\begin{proof}
    Costruiamo la matrice $[f]_{\beta}^{\beta}$. La prima colonna di questa matrice sara' \[
        [f]_{\beta}^{\beta}[\bm{v_1}]_{\beta} = [f(\bm{v_1})]_{\beta} = [\lambda_1 \bm{v_1}]_{\beta} = \lambda_1 [\bm{v_1}]_{\beta}.
    \]
    Ma dato che $\bm{v_1}$ puo' essere espresso nella base $\beta$ dalla combinazione lineare \[
        \bm{v_1} = 1\cdot \bm{v_1} + 0 \cdot \bm{v_2} + \dots + 0 \cdot \bm{v_n}    
    \]
    segue che la prima colonna della matrice $[f]_{\beta}^{\beta}$ e' \[
        [f]_{\beta}^{\beta}[\bm{v_1}]_{\beta} = \lambda_1 \begin{pmatrix} 1 \\ 0 \\ \vdots \\ 0 \end{pmatrix} = \begin{pmatrix} \lambda_1 \\ 0 \\ \vdots \\ 0 \end{pmatrix}.
    \]
    Ripetendo il procedimento per gli altri vettori, otteniamo che la colonna $i$-esima della matrice e' formata da un vettore con tutti $0$ tranne che in posizione $i$, dove compare l'autovalore $\lambda_i$. Di conseguenza $[f]_{\beta}^{\beta}$ e' la matrice diagonale\[
        [f]_{\beta}^{\beta} = \begin{pmatrix}
            \lambda_1   & 0         & \dots  & 0             & 0 \\
            0           & \lambda_2 & \dots  & 0             & 0 \\
            \vdots      & \vdots    & \ddots & \vdots        & \vdots \\
            0           & 0         & \dots  & \lambda_{n-1} & 0 \\
            0           & 0         & \dots  & 0             & \lambda_n \\
        \end{pmatrix}. \qedhere
    \]
\end{proof}

\begin{proposition}
    Sia $f : V \to V$ un endomorfismo e sia $\lambda \in \R$. 
    
    Allora segue che \[
        V_{\lambda} = \ker (f - \lambda\id)    
    \] ovvero che l'autospazio $V_{\lambda}$ relativo a $\lambda$ e' dato dal kernel della funzione $(f - \lambda\id)$.
\end{proposition}
\begin{proof}
    Sia $\bm v \in V$. Allora \begin{alignat*}
        {1}
        &f(\bm v) = \lambda\bm v \\
        \iff &f(\bm v) = \lambda\id(\bm v)\\
        \intertext{Sia $(\lambda\id) : V \to V$ tale che $(\lambda\id)(\bm v) = \lambda \bm v$: }
        \iff &f(\bm v) = (\lambda\id)(\bm v)\\
        \iff &f(\bm v) - (\lambda\id)(\bm v) = \bm 0\\
        \intertext{Sia $(f - \lambda\id) : V \to V$ tale che $(f - \lambda\id)(\bm v) = f(\bm v) - \lambda \bm v$: }
        \iff &(f - \lambda\id)(\bm v) = \bm 0\\
        \iff &\bm v \in \ker (f - \lambda\id). \tag*{\qedhere}
    \end{alignat*}
\end{proof}

\begin{corollary}\label{autospazio_kernel_matrice}
    Se $A$ e' una matrice $n \times n$ allora l'autospazio $V_{\lambda}$ relativo a $\lambda$ e' dato dal kernel della matrice $A - \lambda I_n$, dove $I_n$ e' la matrice identita' $n \times n$.
\end{corollary}

\begin{proposition}\label{autovalore_sse_kernel_nonnullo}
    Sia $A \in \M_{n \times n}(\R)$. Allora $\lambda$ e' un autovalore di $A$ se e solo se $\ker (A - \lambda I_n) \neq \{\bm 0\}$. 
\end{proposition}
\begin{proof}
    Infatti per la proposizione \ref{autospazio_kernel_matrice} sappiamo che $V_{\lambda} = \ker (A - \lambda I_n)$. Inoltre per definizione di autovalore segue che $\lambda$ e' un autovalore se e solo se $V_{\lambda} \neq \{\bm 0\}$, ovvero se e solo se $\ker (A - \lambda I_n) \neq \{\bm 0\}$.
\end{proof}

\begin{definition}
    Sia $A \in \M_{n \times n}(\R)$. Allora si dice \textbf{polinomio caratteristico} di $A$ il polinomio \[
        p_A(\lambda) = \det (A - \lambda I_n). 
    \]
\end{definition}

\begin{theorem}
    Sia $A \in \M_{n \times n}(\R)$. Allora $\lambda_0$ e' un autovalore di $A$ se e solo se $p_A(\lambda_0) = 0$, ovvero se e solo se $\lambda_0$ e' radice del polinomio caratteristico.
\end{theorem}
\begin{proof}
    Per la proposizione \ref{autovalore_sse_kernel_nonnullo} sappiamo che $\lambda_0$ e' un autovalore se e solo se $\ker (A - \lambda_0 I_n) \neq \{\bm 0\}$.  
    
    Per i teoremi sulla relazione tra determinante e rango di una matrice (\ref{relazioni_determinante_rango}) sappiamo che $\ker (A - \lambda_0 I_n) \neq \{\bm 0\}$ se e solo se $\det (A - \lambda_0 I_n) = 0$, ma dato che $\det (A - \lambda_0 I_n) = p_A(\lambda_0)$ questo e' equivalente a dire che $p_A(\lambda_0) = 0$, ovvero che $\lambda_0$ e' una radice di $p_A$, che e' la tesi.
\end{proof}

\begin{example}
    Sia $f : \R^2 \to \R^2$ tale che $f\begin{psmallmatrix} x \\ y \end{psmallmatrix} = \begin{psmallmatrix} x + y \\ x \end{psmallmatrix}$. Diagonalizzare $[f]$.
\end{example}
\begin{proof}
    Innanzitutto troviamo la matrice $[f]$ rispetto alle basi standard.
    \begin{equation*}
        A\begin{pmatrix} 1 \\ 0 \end{pmatrix} = \begin{pmatrix} 1 \\ 1 \end{pmatrix},
            \,A\begin{pmatrix} 0 \\ 1 \end{pmatrix} = \begin{pmatrix} 1 \\ 0 \end{pmatrix}
        \implies A = \begin{pmatrix}
            1 & 1 \\ 1 & 0
        \end{pmatrix}
    \end{equation*}
\end{proof}
\chapter{Ortogonalita'}

\section{Prodotto scalare e ortogonalita'}

\begin{definition}[Prodotto scalare]
    Siano $\vec{v}, \vec{w} \in \R^n$ tali che $\vec v = (a_1, \dots, a_n)$ e $\vec w = (b_1, \dots, b_n)$.
    
    Allora si dice prodotto scalare canonico tra $\vec{v}$ e $\vec w$ la funzione \[
        \innerprod{\cdot}{\cdot} : \R^n \times \R^n \to \R
    \] tale che \[
        \innerprod{\vec v}{\vec w} = \vec v^T \cdot \vec w = \sum_{k = 1}^n a_kb_k.
    \]
\end{definition}

Se i vettori sono in $\C^n$ si usa un prodotto scalare diverso, detto prodotto hermitiano.

\begin{definition}[Prodotto hermitiano]
    Siano $\vec{v}, \vec{w} \in \C^n$ tali che $\vec v = (a_1, \dots, a_n)$ e $\vec w = (b_1, \dots, b_n)$.
    
    Allora si dice prodotto hermitiano tra $\vec{v}$ e $\vec w$ la funzione \[
        \innerprod{\cdot}{\cdot} : \C^n \times \C^n \to \C
    \] tale che \[
        \innerprod{\vec v}{\vec w} = \vec v^T \cdot \conj{\vec w} = \sum_{k = 1}^n a_k\conj{b_k}.
    \]
\end{definition}

\begin{proposition}
    Siano $\vec u, \vec v, \vec w \in \R^n$. 
    Il prodotto scalare gode delle seguenti proprieta':
    \begin{enumerate}[(i)]
        \item se $k \in \R$ allora $\innerprod{k\vec v}{\vec w} = k\innerprod{\vec v}{\vec w} = \innerprod{\vec v}{k\vec w}$;
        \item $\innerprod{\vec v + \vec u}{\vec w} = \innerprod{\vec v}{\vec w} + \innerprod{\vec u}{\vec w}$;
        \item $\innerprod{\vec v}{\vec 0} = \innerprod{\vec 0}{\vec v} = 0$;
        \item $\innerprod{\vec v}{\vec w} = \innerprod{\vec w}{\vec v}$.
    \end{enumerate}
\end{proposition}

\begin{definition}[Norma di un vettore]
    Sia $\vec{v} \in \R^n$. Allora si dice norma (o lunghezza) di $\vec{v}$ il numero reale \[
        \norm{\vec v} = \sqrt{\innerprod{\vec v}{\vec v}} = \sqrt{\sum_{k = 1}^n a_k^2}.     
    \]
\end{definition}

\begin{proposition}[La norma e' nulla se e solo se il vettore e' nullo]
    \label{norma_nulla_sse_vettore_nullo}
    Sia $\vec v \in \R^n$. Allora $\norm{\vec v} = 0$ se e solo se $\vec v = \vec 0$.
\end{proposition}
\begin{proof}
    Sia $\vec v = (a_1, \dots, a_n)$. Allora
    \begin{align*}
        &\norm{\vec{v}} = 0\\
        \iff &\sqrt{a_1^2 + \dots a_n^2} = 0 \\
        \iff &a_1^2 + \dots + a_n^2 = 0.
    \end{align*}
    Ma $a_1^2 + \dots + a_n^2$ e' una somma di termini maggiori o uguali a 0, dunque $a_1^2 + \dots + a_n^2 = 0$ se e solo se $a_1 = \dots = a_n = 0$, ovvero $\vec v = \vec 0$. 
\end{proof}

\begin{definition}
    Siano $\vec v, \vec w \in \R^n$, entrambi non nulli. Allora $\vec v$ e' ortogonale a $\vec w$ (e si indica con $\vec v \perp \vec w$) se $\innerprod{\vec v}{\vec w} = 0$.
\end{definition}

\begin{proposition}[Ortogonali implica indipendenti]
    \label{ortogonali=>indip_2}
    Siano $\vec v, \vec w \in \R^n$ entrambi non nulli. Allora se $\vec v \perp \vec w$ segue che l'insieme $\{\vec v, \vec w\}$ e' un insieme di vettori linearmente indipendenti.
\end{proposition}
\begin{proof}
    Consideriamo una generica combinazione lineare di $\vec v, \vec w$ (come $a\vec v + b \vec w$ al variare di $a, b \in \R$) e poniamola uguale a $\vec 0$. Dobbiamo dimostrare che segue che $a = b = 0$.

    Consideriamo il prodotto scalare $\innerprod{a\vec v + b \vec w}{\vec v}$. Allora \begin{align*}
        \innerprod{a\vec v + b \vec w}{\vec v} &= \innerprod{a\vec v}{\vec v} +  \innerprod{b\vec w}{\vec v} \\
        &= a\innerprod{\vec v}{\vec v} + b\innerprod{\vec w}{\vec v} &&\text{(per ipotesi $\innerprod{\vec w}{\vec v} = 0$)} \\
        &= a\innerprod{\vec{v}}{\vec v} \\
        &= a\norm{\vec v}^2.
    \end{align*}
    Dato che abbiamo imposto che la combinazione lineare sia uguale a $\vec 0$, avremo \[
        \innerprod{a\vec v + b \vec w}{\vec v} = \innerprod{\vec 0}{\vec v} = 0.    
    \]
    Dunque segue che $a\norm{\vec v}^2 = 0$, ma per ipotesi $\vec v \neq \vec 0$, dunque per la proposizione \ref{norma_nulla_sse_vettore_nullo} segue che $a = 0$.

    Quindi $a\vec v + b\vec w = \vec 0$ se e solo se $b \vec w = \vec 0$, ma dato che $\vec w$ non e' nullo segue che anche $b = 0$, ovvero l'insieme $\{\vec v, \vec w\}$ e' un insieme di vettori linearmente indipendenti.
\end{proof}

\begin{corollary}[Ortogonali implica indipendenti]
    \label{ortogonali=>indip_n}
    Siano $\vec{v_1}, \dots, \vec{v_k} \in \R^n$ non nulli e ortogonali a due a due (ovvero per ogni $i, j \leq k$, $i \neq j$ segue che $\vec{v_i} \perp \vec{v_j}$). Allora l'insieme $\{\vec{v_1}, \dots, \vec{v_k}\}$ e' un insieme di vettori linearmente indipendenti.
\end{corollary}
\begin{proof}
    Consideriamo una generica combinazione lineare dei vettori e poniamola uguale a $0$: \[
        c_1\vec{v_1} + \dots + c_k\vec{v_k} = \vec 0. 
    \] Dimostriamo che segue che $c_1 = \dots = c_k = 0$.

    Consideriamo il prodotto scalare $\innerprod{c_1\vec{v_1} + \dots + c_k\vec{v_k}}{\vec{v_i}}$. Dato che la combinazione lineare e' uguale al vettore nullo, allora questo prodotto scalare sara' uguale a 0, ovvero:
    \begin{align*}
        0 &= \innerprod{c_1\vec{v_1} + \dots + c_i\vec{v_i} + \dots c_k\vec{v_k}}{\vec{v_i}} \\
        &= \innerprod{c_1\vec{v_1}}{\vec{v_i}} + \dots + \innerprod{c_i\vec{v_i}}{\vec{v_i}} + \dots + \innerprod{c_k\vec{v_k}}{\vec{v_i}} \\
        &= c_1\innerprod{\vec{v_1}}{\vec{v_i}} + \dots + c_i\innerprod{\vec{v_i}}{\vec{v_i}} + \dots + c_k\innerprod{\vec{v_k}}{\vec{v_i}} \\
        \intertext{Per ipotesi $\innerprod{\vec{v_j}}{\vec{v_i}} = 0$ per ogni $j \neq i$:} 
        &= c_i\innerprod{\vec{v_i}}{\vec{v_i}} \\
        &= a\norm{\vec{v_i}}^2.
    \end{align*}
    Dunque $c_i\norm{\vec{v_i}}^2 = 0$, ma dato che $\vec{v_i} \neq \vec 0$ segue che $c_i = 0$.

    Con un analogo ragionamento si dimostra che tutti i coefficienti devono essere $0$, dunque i vettori sono indipendenti.
\end{proof}

\begin{definition}
    Sia $\mathcal{B}$ una base di $\R^n$. Allora si dice che $\mathcal{B}$ e' una base ortogonale di $\R^n$ se per ogni $\vec v, \vec w \in \mathcal{B}$ con $\vec v \neq \vec w$ vale che $\vec v \perp \vec w$ (ovvero i vettori sono a due a due ortogonali).
\end{definition}

\begin{definition}
    Sia $\mathcal{B}$ una base ortogonale di $\R^n$. Allora si dice che $\mathcal{B}$ e' una base ortonormale di $\R^n$ se per ogni $\vec v \in \mathcal{B}$ vale che $\norm{\vec{v}} = 1$.
\end{definition}

\begin{proposition}
    Siano $\vec{v_1}, \dots, \vec{v_n} \in \R^n$ a due a due ortogonali. Allora $\mathcal{B} = \basis{\vec{v_1}, \dots, \vec{v_n}}$ e' una base ortogonale di $\R^n$.
\end{proposition}
\begin{proof}
    Per la proposizione \ref{ortogonali=>indip_n} gli $n$ vettori sono indipendenti. Inoltre $\dim \R^n = n$, dunque per la proposizione \ref{base=dim_gener_indip} segue che $\mathcal{B}$ e' una base di $\R^n$. In particolare, dato che i vettori sono ortogonali a due a due, $\mathcal{B}$ e' una base ortogonale di $\R^n$.
\end{proof}

\begin{restatable}[Teorema di Gram-Schmidt]{theorem}{gramschmidt}
    Sia $V \subseteq \R^n$ un sottospazio vettoriale di $\R^n$. Allora esiste una base ortogonale di $V$.
\end{restatable}

Le basi ortogonali (e ortonormali) sono utili in quanto possiamo sfruttare la seguente proposizione per trovare le coordinate di un vettore rispetto alla suddetta base.

\begin{proposition}[Coordinate rispetto ad una base ortogonale]
    \label{coordinate_base_ortogonale}
    Sia $V \subseteq \R^n$ un sottospazio di $\R^n$ e sia $\mathcal{B} = \basis{\vec{v_1}, \dots, \vec{v_k}}$ una base ortogonale di $V$. Sia inoltre $\vec v \in V$.

    Allora il vettore delle coordinate di $\vec v$ rispetto alla base $\mathcal{B}$ e' il vettore $(c_1, \dots, c_k)$ tale che per ogni $i \leq k$:
    \[
        c_i = \frac{\innerprod{\vec{v}}{\vec{v_i}}}{\innerprod{\vec{v_i}}{\vec{v_i}}} = \frac{\innerprod{\vec{v}}{\vec{v_i}}}{\norm{\vec{v_i}}^2}.    
    \]
\end{proposition}
\begin{proof}
    Dato che $(c_1, \dots, c_k)$ e' il vettore delle coordinate di $\vec v$ rispetto a $\mathcal{B}$, allora segue che \[
        \vec v = c_1\vec{v_1} + \dots + c_k\vec{v_k}.   
    \] Sia $i \leq k$ generico. Allora
    \begin{align*}
        &\innerprod{\vec v}{\vec{v_i}} \\
        = &\innerprod{c_1\vec{v_1} + \dots + c_i\vec{v_i} + \dots + c_k\vec{v_k}}{\vec{v_i}}\\
        = &c_1\innerprod{\vec{v_1}}{\vec{v_i}} + \dots + c_i\innerprod{\vec{v_i}}{\vec{v_i}} + \dots + c_k\innerprod{\vec{v_k}}{\vec{v_i}} &&\text{(per ortogonalita' della base)}\\
        = &c_i\innerprod{\vec{v_i}}{\vec{v_i}}\\
        = &c_i\norm{\vec{v_1}}^2.
    \end{align*}
    Dunque \[
        c_i = \frac{\innerprod{\vec{v}}{\vec{v_i}}}{\norm{\vec{v_i}}^2}. \qedhere
    \]
\end{proof}
\begin{corollary}[Coordinate rispetto ad una base ortonormale]
    \label{coordinate_base_ortonormale}
    Sia $V \subseteq \R^n$ un sottospazio di $\R^n$ e sia $\mathcal{B} = \basis{\vec{v_1}, \dots, \vec{v_k}}$ una base ortonormale di $V$. Sia inoltre $\vec v \in V$.

    Allora il vettore delle coordinate di $\vec v$ rispetto alla base $\mathcal{B}$ e' il vettore $(c_1, \dots, c_k)$ tale che per ogni $i \leq k$:
    \[
        c_i = \innerprod{\vec{v}}{\vec{v_i}}.    
    \]
\end{corollary}
\begin{proof}
    Dato che una base ortonormale e' anche ortogonale, dalla proposizione \ref{coordinate_base_ortogonale} segue che per ogni $i \leq k$ vale che \[
        c_i = \frac{\innerprod{\vec{v}}{\vec{v_i}}}{\norm{\vec{v_i}}^2}.
    \] Per definizione di base ortonormale sappiamo che $\norm{\vec{v_i}} = 1$, da cui segue la tesi.
\end{proof}

\section{Complemento ortogonale}

\begin{definition}[Sottospazi ortogonali]
    Siano $V, W \subseteq \R^n$ due sottospazi di $\R^n$. Allora dico che $V$ e' ortogonale a $W$ (e scrivo $V \perp W$) se per ogni $\vec v \in V, \vec w \in W$ vale che $\vec v \perp \vec w$.
\end{definition}

\begin{proposition}[Sottospazi ortogonali sono in somma diretta]
    Siano $V, W \subseteq \R^n$ due sottospazi di $\R^n$ tali che $V \perp W$. Allora $V$ e $W$ sono in somma diretta.
\end{proposition}
\begin{proof}
    Dire che $V$ e $W$ sono in somma diretta significa che per ogni coppia di vettori $\vec v \in V, \vec w \in W$ segue che $\vec{v}, \vec w$ sono indipendenti.

    Siano $\vec v \in V, \vec w \in W$ due vettori generici. Dato che $V$ e $W$ sono ortogonali, $\vec v \perp \vec w$, ma per la proposizione \ref{ortogonali=>indip_2} questo implica che $\vec v, \vec w$ siano indipendenti, cioe' la tesi.
\end{proof}

\begin{definition}[Ortogonale di un sottospazio]
    Sia $V \subseteq \R^n$ un sottospazio di $\R^n$. Allora si dice complemento ortogonale di $V$ (o semplicemente ortogonale di $V$) il sottospazio \[
        \ortog{V} = \{\vec w \in \R^n \mid \vec w \perp \vec v \quad \forall \vec v \in V\}.  
    \]
\end{definition}

\begin{proposition}
    Sia $V \subseteq \R^n$ un sottospazio di $\R^n$ e sia $\ortog{V}$ il suo ortogonale. Allora valgono le seguenti:
    \begin{enumerate}[(i)]
        \item $V \perp \ortog{V}$;
        \item $V \oplus \ortog{V} = \R^n$, ovvero $\ortog{V}$ e' un complemento di $V$;
        \item $\ortog{(\ortog{V})} = V$.
    \end{enumerate}
\end{proposition}

\begin{proposition}[Un vettore e' in $\ortog{V}$ se e solo se e' ortogonale ad una base di $V$] \label{w_in_ortog_sse_ortogonale_base}
    Sia $\vec w \in \R^n$ e sia $V \subseteq \R^n$ un sottospazio di $\R^n$. $\mathcal{B} = \basis{\vec{v_1}, \dots, \vec{v_k}}$ una base ortogonale di $V$.
    
    Allora $\vec w \in \ortog{V}$ se e solo se $\vec w \perp \vec{v_1}, \dots, \vec w \perp \vec{v_k}$.
\end{proposition}
\begin{proof}
    Dimostriamo le due implicazioni.
    \begin{itemize}
        \item[($\implies$)] Se $\vec w$ e' un vettore di $\ortog{V}$ allora per definizione di $\ortog{V}$ segue che $\vec w \perp \vec v$ per ogni $\vec v \in V$; di conseguenza, $\vec w$ sara' ortogonale anche ai vettori della base $\mathcal{B}$.
        \item[($\impliedby$)] Sia $\vec v \in V$ generico. 
        Dato che $\mathcal{B}$ e' una base di $V$, allora esisteranno $c_1, \dots, c_k \in \R$ tali che \[
            \vec v = c_1\vec{v_1} + \dots + c_k\vec{v_k}.    
        \]
        E' quindi sufficiente dimostrare che $\vec w \perp \vec v$, ovvero che $\innerprod{\vec v}{\vec w} = 0$.
        \begin{align*}
            \innerprod{\vec v}{\vec w} &= \innerprod{c_1\vec{v_1} + \dots + c_k\vec{v_k}}{\vec w} \\
            &= c_1\innerprod{\vec{v_1}}{\vec w} + \dots + c_k\innerprod{\vec{v_k}}{\vec w} \\
            \intertext{Ma per ipotesi $\vec w \perp \vec{v_1}, \dots, \vec w \perp \vec{v_k}$, dunque}
            &= 0. \qedhere
        \end{align*} 
    \end{itemize}
\end{proof}

\section{Proiezioni ortogonali}

\begin{definition}
    Siano $\vec u, \vec v \in \R^n$. Allora si chiama proiezione ortogonale di $\vec v$ su $\vec u$ il vettore \[
        \proj{\vec v}{\vec u} = \frac{\innerprod{\vec v}{\vec u}}{\innerprod{\vec u}{\vec u}}\vec u.
    \]
\end{definition}

\begin{proposition}
    Siano $\vec u, \vec v \in \R^n$ e sia $\vec w \in \R^n$ tale che $\vec w = \vec v - \proj{\vec v}{\vec u}$. Allora $\vec w \perp \vec u$.
\end{proposition}
\begin{proof}
    Basta dimostrare che $\innerprod{\vec w}{\vec u} = 0$.
    \begin{align*}
        \innerprod{\vec w}{\vec u} &= \innerprod{\vec v - \proj{\vec v}{\vec u}}{\vec u} \\
        &= \innerprod{\vec v}{\vec u} - \innerprod{\proj{\vec v}{\vec u}}{\vec u} \\
        &= \innerprod{\vec v}{\vec u} - \innerprod{\frac{\innerprod{\vec v}{\vec u}}{\innerprod{\vec u}{\vec u}}\vec u}{\vec u} \\
        &= \innerprod{\vec v}{\vec u} - \frac{\innerprod{\vec v}{\vec u}}{\innerprod{\vec u}{\vec u}}\innerprod{\vec u}{\vec u} \\
        &= \innerprod{\vec v}{\vec u} - \innerprod{\vec v}{\vec u} \\
        &= 0. \qedhere
    \end{align*}
\end{proof}

\begin{definition}
    Sia $\vec v \in \R^n$ e sia $V \subseteq \R^n$ un sottospazio di $\R^n$. Sia inoltre $\mathcal{B} = \basis{\vec{v_1}, \dots, \vec{v_k}}$ una base ortogonale di $V$.
    
    Allora si chiama proiezione ortogonale di $\vec v$ sul sottospazio $V$ il vettore \[
        \proj{\vec v}{V} = \proj{\vec v}{\vec{v_1}} + \dots + \proj{\vec v}{\vec{v_k}}.
    \]
\end{definition}

\begin{proposition}\label{(v-proj)_in_ortogonale}
    Siano $\vec v \in \R^n$, sia $V \subseteq \R^n$ un sottospazio di $\R^n$ e sia $\vec w \in \R^n$ tale che $\vec w = \vec v - \proj{\vec v}{V}$. Allora $\vec w \perp \vec u$ per ogni $\vec u \in V$, ovvero $\vec w \in \ortog{V}$.
\end{proposition}
\begin{proof}
    Sia $\mathcal{B} = \basis{\vec{v_1}, \dots, \vec{v_k}}$ una base di $V$. Per la proposizione \ref{w_in_ortog_sse_ortogonale_base} e' sufficiente dimostrare che $\vec{w} \perp \vec{v_i}$ per ogni vettore $\vec{v_i} \in \mathcal{B}$.

    Dimostriamo che $\vec w \perp \vec{v_1}$.
    \begin{align*}
        \innerprod{\vec w}{\vec{v_1}} &= \innerprod{\vec v - \proj{\vec v}{V}}{\vec{v_1}} \\
        &= \innerprod{\vec v}{\vec{v_1}} - \innerprod{\proj{\vec v}{V}}{\vec{v_1}} \\
        &= \innerprod{\vec v}{\vec{v_1}} - \innerprod{\proj{\vec v}{\vec{v_1}} + \dots + \proj{\vec v}{\vec{v_k}}}{\vec{v_1}} \\
        &= \innerprod{\vec v}{\vec{v_1}} - (\innerprod{\proj{\vec v}{\vec{v_1}}}{\vec{v_1}} + \dots + \innerprod{\proj{\vec v}{\vec{v_k}}}{\vec{v_1}}) \\
        \intertext{Sia $c_i = \frac{\innerprod{\vec v}{\vec{v_i}}}{\innerprod{\vec{v_i}}{\vec{v_i}}}$ per ogni $i \leq k$:}
        &= \innerprod{\vec v}{\vec{v_1}} - (\innerprod{c_1\vec{v_1}}{\vec{v_1}} + \dots + \innerprod{c_k\vec{v_k}}{\vec{v_1}}) \\
        &= \innerprod{\vec v}{\vec{v_1}} - (c_1\innerprod{\vec{v_1}}{\vec{v_1}} + \dots + c_k\innerprod{\vec{v_k}}{\vec{v_1}}) \\
        \intertext{Per ortogonalita' della base $\mathcal{B}$:}
        &= \innerprod{\vec v}{\vec{v_1}} - c_1\innerprod{\vec{v_1}}{\vec{v_1}}\\
        &= \innerprod{\vec v}{\vec{v_1}} - \frac{\innerprod{\vec v}{\vec{v_1}}}{\innerprod{\vec{v_1}}{\vec{v_1}}}\innerprod{\vec{v_1}}{\vec{v_1}} \\
        &= \innerprod{\vec v}{\vec{v_1}} - \innerprod{\vec v}{\vec{v_1}}\\
        &= 0.
    \end{align*}
    Tramite un analogo ragionamento si dimostra che $\vec w$ e' perpendicolare a tutti i vettori di $\mathcal{B}$, dunque appartiene a $\ortog{V}$.
\end{proof}

Mostriamo ora una dimostrazione del Teorema di Gram-Schmidt.

\gramschmidt*
\begin{proof}
    Sia $\mathcal{B} = \basis{\vec{v_1}, \dots, \vec{v_m}}$ una base di $V$. Applichiamo il seguente algoritmo per trovare una base ortogonale di $V$.

    Chiamiamo $\mathcal{B}_k$ una base in cui i primi $k$ vettori sono ortogonali a due a due. Mostriamo che questa base esiste per $k = 1$ e mostriamo come ottenerla per un $k$ qualsiasi induttivamente.

    \begin{description}
        \item[Caso base] Sia $k = 1$. Allora banalmente $\mathcal{B}_1 = \mathcal{B}$, poiche' il primo vettore non deve essere ortogonale a nessun altro vettore.
        \item[Passo induttivo] Sia $k \geq 1$ e supponiamo (per ipotesi induttiva) che esista $\mathcal{B}_{k}$. Dimostriamo che esiste $\mathcal{B}_{k+1}$.
        
        Sia $\mathcal{B}_k = \basis{\vec{w_1}, \dots, \vec{w_k}, \vec{v_{k+1}}, \dots, \vec{v_m}}$ tale che i primi $k$ vettori sono ortogonali a due a due. Sia inoltre $W_k = \Span{\vec{w_1}, \dots, \vec{w_k}}$.
        
        Sia $\mathcal{B}_{k+1} = \basis{\vec{w_1}, \dots, \vec{w_k}, \vec{w_{k+1}}, \vec{v_{k+2}}, \dots, \vec{v_m}}$ dove $\vec{w_{k+1}} = \vec{v_{k+1}} - \proj{v_{k+1}}{W_k}$.
        Dobbiamo dimostrare che \begin{enumerate}[(i)]
            \item $\mathcal{B}_{k+1}$ genera ancora tutto $V$ (ovvero lo span dei vettori non e' cambiato);
            \item i primi $k+1$ vettori sono ortogonali tra loro a due a due.
        \end{enumerate}

        Dimostriamo questi due punti:
        \begin{enumerate}[(i)]
            \item Per definizione di $\vec{w_{k+1}}$:
            \begin{align*}
                \vec{w_{k+1}} &= \vec{v_{k+1}} - \proj{v_{k+1}}{W_k} \\
                &= \vec{v_{k+1}} - \proj{v_{k+1}}{\vec{w_1}} - \dots - \proj{v_{k+1}}{\vec{w_k}}\\
                &= \vec{v_{k+1}} - c_1{\vec{w_1}} - \dots - c_k{\vec{w_k}}
            \end{align*} dove $c_1 = \frac{\innerprod{\vec{w_{k+1}}}{\vec{w_i}}}{\innerprod{\vec{w_i}}{\vec{w_i}}}$.

            Dunque stiamo sottraendo ad un vettore della base multipli di altri vettori della base, dunque (per la proposizione \ref{span_Gauss}) lo span non cambia e $\mathcal{B}_{k+1}$ genera ancora tutto $V$ (e in particolare dovra' essere ancora una base di $V$ in quanto ha $m$ elementi).
            \item Dato che per ipotesi induttiva i primi $k$ vettori sono ortogonali tra loro, ci basta mostrare che $\vec{w_{k+1}}$ e' ortogonale con tutti i vettori $\vec{w_1}, \dots, \vec{w_k}$.
            
            Dato che $\vec{w_{k+1}} = \vec{v_{k+1}} - \proj{v_{k+1}}{W_k}$ per la proposizione \ref{(v-proj)_in_ortogonale} segue che $\vec{w_{k+1}} \in \ortog{W_k}$, dunque per definizione di $\ortog{W_k}$ segue che $\vec{w_{k+1}}$ dovra' essere ortogonale ad ogni vettore di $W_k$, ovvero $\vec{w_{k+1}} \perp \vec{w_i}$ per ogni $i \leq k$.
        \end{enumerate}
        
        Dunque $\mathcal{B}_{k+1}$ e' una base di $V$ tale che i primi $k+1$ vettori sono ortogonali tra loro a due a due.
    \end{description}

    Per induzione dunque esistera' $\mathcal{B}_m$, cioe' una base di $V$ in cui tutti i vettori sono ortogonali a due a due, ovvero $\mathcal{B}_m$ e' una base ortogonale di $V$.
\end{proof}

\section{Matrici simmetriche e ortogonalita'}

\begin{proposition}
    Sia $A \in \M_{n \times k}(\R)$, sia $\vec{v} \in \R^k$ e sia $\vec{w} \in \R^n$. Allora \[
        \innerprod{A\vec v}{\vec w} = \innerprod{\vec v}{A^T\vec w}.    
    \]
\end{proposition}
\begin{proof}
    $\innerprod{A\vec v}{\vec w} = (A\vec v)^T \vec w = {\vec v}^T A^T \vec w = \innerprod{\vec v}{A^T \vec w}$.
\end{proof}

\begin{corollary}
    Sia $A \in \M_{n \times n}(\R)$ simmetrica e siano $\vec v, \vec w \in \R^n$. Allora \[
        \innerprod{A\vec v}{\vec w} = \innerprod{\vec v}{A\vec w}.    
    \]
\end{corollary}
\begin{proof}
    Deriva direttamente dalla proposizione precedente: dato che $A$ e' simmetrica segue che $A^T = A$, dunque $\innerprod{A\vec v}{\vec w} = \innerprod{\vec v}{A^T\vec w} = \innerprod{\vec v}{A\vec w}$.
\end{proof}

\begin{proposition}
    Sia $A \in \M_{n \times n}(\R)$ una matrice simmetrica e sia $\vec u$ un suo autovettore. Allora per ogni $\vec v \in \R^n$ segue che\[
        \vec v \perp \vec u \implies A \vec v \perp \vec u.    
    \]
\end{proposition}
\begin{proof}
    Per la proposizione precedente $\innerprod{A\vec v}{\vec u} = \innerprod{\vec v}{A\vec u}$. Dato che $\vec u$ e' un autovettore di $A$ segue che $A\vec u = \lambda\vec u$ per qualche $\lambda \in \R$. Allora \begin{align*}
        \innerprod{\vec v}{A\vec u} &= \innerprod{\vec v}{\lambda\vec u}\\
            &= \lambda\innerprod{\vec v}{\vec u}\\
            &= 0 \cdot \lambda\\
            &= 0. \qedhere
    \end{align*}
\end{proof}

\begin{theorem}
    [Teorema Spettrale]
    Sia $A \in \M_{n \times n}(\R)$ una matrice simmetrica. Allora \begin{enumerate}[(i)]
        \item $A$ e' diagonalizzabile su $\R$;
        \item autospazi relativi ad autovalori diversi sono ortogonali tra loro.
    \end{enumerate}
\end{theorem}

Il teorema spettrale vale solo per matrici a valori in $\R$ o in $\C$.

\section{Matrici ortogonali}

\begin{definition}
    Una matrice $A \in \M_{n \times n}(\R)$ si dice ortogonale se le sue colonne sono ortonormali a due a due.
\end{definition}

\begin{proposition}
    Sia $A \in \M_{n \times n}(\R)$ ortogonale. Allora l'inversa di $A$ e' $A^T$.
\end{proposition}
\begin{proof}
    Siano $\vec{c_1}, \dots, \vec{c_n}$ le colonne di $A$; allora $\vec{c_1}^T, \dots, \vec{c_n}^T$ saranno le righe di $A^T$. Consideriamo il valore in posizione $ij$ della matrice prodotto $A^T \cdot A$:
    \[
        [A^TA]_{ij} = \vec{c_i}^T \cdot \vec{c_j} = \innerprod{\vec{c_i}}{\vec{c_j}} = \begin{cases}
            0, &\text{se } i \neq j \\
            1, &\text{se } i = j.
        \end{cases}
    \]
    Dunque $A^TA = I_n$, ovvero $A^T = A^{-1}$.
\end{proof}

\begin{definition}
    Si dice distanza tra i vettori $\vec v, \vec w \in \R^n$ il numero reale $\norm{\vec v - \vec w}$.
\end{definition}

\begin{proposition}
    Sia $A \in \M_{n \times n}(\R)$ ortogonale. Allora valgono le seguenti:
    \begin{enumerate}[(i)]
        \item $\norm{A\vec v} = \norm{\vec v}$;
        \item $\norm{A\vec v - A\vec w} = \norm{\vec v - \vec w}$
    \end{enumerate}
    ovvero la matrice $A$ trasforma lo spazio $\R^n$ tramite una trasformazione rigida.
\end{proposition}
\begin{proof}
    \begin{enumerate}
        \item $\norm{A\vec v} = \sqrt{\innerprod{A\vec v}{A\vec v}} = \sqrt{\innerprod{\vec v}{A^TA\vec v}} = \sqrt{\innerprod{\vec v}{\vec v}} = \norm{\vec v}$.
        \item $\norm{A\vec v - A\vec w} = \norm{A(\vec v - \vec w)} = \norm{\vec v - \vec w}$ per il punto precedente. \qedhere
    \end{enumerate}
\end{proof}

\end{document}
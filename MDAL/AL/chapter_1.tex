\chapter{Matrici e sistemi lineari}

\section{Matrici}

\begin{definition}
    Si dice matrice $m \times n$ una tabella di $m$ righe e $n$ colonne i cui elementi appartengono ad un campo $\K$ fissato, della forma
    \begin{equation}
        A = \begin{pmatrix}
            a_{11}  &   a_{12}  & \dots     & a_{1n} \\
            a_{21}  &   a_{22}  & \dots     & a_{2n} \\
            \vdots  &   \vdots  & \vdots    & \vdots \\
            a_{m1}  &   a_{m2}  & \dots     & a_{mn} \\
        \end{pmatrix}
        = [A_{ij}]_{i\leq m, j \leq n}
    \end{equation}
\end{definition}

\begin{definition}
    Si dice vettore colonna una matrice $n \times 1$ del tipo
    \begin{equation}
        \bm{v} = \begin{pmatrix}
            a_{1} \\
            a_{2} \\
            \vdots \\
            a_n
        \end{pmatrix}
    \end{equation}
    Si dice vettore riga una matrice $1 \times n$ del tipo
    \begin{equation}
        \bm{w} = \begin{pmatrix}
            a_{1} & a_2 & \dots & a_n
        \end{pmatrix}
    \end{equation}
    L'insieme dei vettori colonna di $n$ elementi appartenenti ad un campo $\K$ si indica con $\K^n$, mentre l'insieme dei vettori riga di $n$ elementiappartenenti ad un campo $\K$ si indica con $\K^{\times n}$.
\end{definition}

E' evidente che se i due vettori $\bm{v}$ e $\bm{w}$ hanno la stessa dimensione e contengono gli stessi elementi allora rappresentano la stessa informazione, ma sotto forme diverse. Verificheremo piu' avanti infatti che $\K^n$ e $\K^{\times n}$ sono isomorfi, cioe' contengono gli stessi elementi in due forme diverse.

\subsection{Operazioni sulle matrici}

Consideriamo le operazioni fondamentali che coinvolgono matrici.

\subsubsection{Somma di matrici}

Siano $A, B$ due matrici $m \times n$ a coefficienti reali. Allora possiamo definire un'operazione di somma $+ : \M_{m \times n}(\R) \times \M_{m \times n}(\R) \to \M_{m \times n}(\R)$ tale che \begin{equation}
    A + B = [A_{ij} + B_{ij}]_{ij}.
\end{equation}
Cioe' se \begin{gather*}
    A = \begin{pmatrix}
        a_{11}  & \dots     & a_{1n} \\
        a_{21}  & \dots     & a_{2n} \\
        \vdots  & \vdots    & \vdots \\
        a_{m1}  & \dots     & a_{mn} \\
    \end{pmatrix},\quad
    B = \begin{pmatrix}
        b_{11}  & \dots     & b_{1n} \\
        b_{21}  & \dots     & b_{2n} \\
        \vdots  & \vdots    & \vdots \\
        b_{m1}  & \dots     & b_{mn} \\
    \end{pmatrix} \\
    \\
    \implies
    A + B = \begin{pmatrix}
        a_{11} + b_{11}  & \dots     & a_{1n} + b_{1n} \\
        a_{21} + b_{21}  & \dots     & a_{2n} + b_{2n} \\
        \vdots           & \vdots    & \vdots          \\
        a_{m1} + b_{m1}  & \dots     & a_{mn} + b_{mn} \\
    \end{pmatrix}   
\end{gather*}
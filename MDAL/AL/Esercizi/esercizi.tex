\documentclass[italian,oneside,headinclude,11pt]{scrreprt}
    \usepackage[utf8]{inputenc}
    \usepackage[italian]{babel}
    \usepackage[T1]{fontenc}
    \usepackage{textcomp, microtype}
    \usepackage{amsmath, amsthm, amssymb, cases, mathtools, bm, enumerate}
    \usepackage{nicefrac}
    \usepackage{array}
    \usepackage{float}

    \usepackage{letltxmacro}

    \LetLtxMacro\amsproof\proof
    \LetLtxMacro\amsendproof\endproof

    
    \usepackage[pdfspacing, eulermath]{classicthesis}
    % \usepackage{lmodern}
    
    \usepackage{thmtools}
    \usepackage[framemethod=TikZ]{mdframed}

    \usepackage{cleveref}
    \usepackage{hyperref} % ultimo package da caricare!

\restylefloat{table}

\AtBeginDocument{
    \LetLtxMacro\proof\amsproof
    \LetLtxMacro\endproof\amsendproof
}

% \titleformat*{\chapter}{\LARGE\scshape}
% \titleformat*{\section}{\Large\scshape}

\renewcommand*{\proofname}{\textsc{Dimostrazione}}
\addto\italian{\renewcommand*{\proofname}{\textsc{Dimostrazione}}}

\declaretheoremstyle[
    spaceabove=10pt, spacebelow=10pt,
    headindent=-20pt,
    headfont=\scshape,
    notefont=\scshape, notebraces={\newline(}{)},
    headformat={\NUMBER. \NAME.\NOTE},
    % headfont=\scshape,
    % notefont=\scshape, notebraces={ (}{)},
    bodyfont=\itshape\normalsize,
    % shaded={rulecolor={rgb}{255,255,255}},
    % mdframed={backgroundcolor=halfgray},
    headpunct={\vspace{0.5\topsep}\newline}
]{thmstyle}
\declaretheorem[name=Teorema, numberwithin=section, style=thmstyle]{theorem}
\declaretheorem[name=Corollario, sibling=theorem, style=thmstyle]{corollary}
\declaretheorem[name=Proposizione, sibling=theorem, style=thmstyle]{proposition}
\declaretheorem[name=Lemma, sibling=theorem, style=thmstyle]{lemma}

\declaretheoremstyle[
    spaceabove=0pt, spacebelow=0pt,
    headfont=\scshape,
    notefont=\normalfont, notebraces={ (}{)},
    % headfont=\scshape,
    % notefont=\scshape, notebraces={(}{)},
    bodyfont=\normalfont\normalsize,
    headpunct={\vspace{0.5\topsep}\newline},
    mdframed={
        linecolor=halfgray,
        linewidth=1pt,
        backgroundcolor=white,
        topline=false,
        bottomline=false,
        rightline=false
    },
]{defstyle}
\declaretheorem[name=Definizione, sibling=theorem, style=defstyle]{definition}

\declaretheoremstyle[
    spaceabove=10pt, spacebelow=10pt,
    headfont=\scshape,
    notefont=\normalfont, notebraces={ (}{)},
    bodyfont=\normalfont,
    postheadspace=1em
]{exmplstyle}
\declaretheorem[name=Esempio, sibling=theorem, style=exmplstyle]{example}
\declaretheorem[name=Esercizio, sibling=theorem, style=exmplstyle]{exercise}


\declaretheoremstyle[
    headfont=\scshape,
    notefont=\normalfont, notebraces={(}{)},
    bodyfont=\normalfont,
    numbered=no,
    postheadspace=1em
]{remarkstyle}
\declaretheorem[name=Osservazione, style=remarkstyle]{remark}
\declaretheorem[name=Soluzione, style=remarkstyle]{solution}
\declaretheorem[name=Intuizione, style=remarkstyle]{intuition}

\newcolumntype{z}{r<{{}}}
\newcolumntype{o}{@{}>{{}}c<{{}}@{}}

% matrix with lines
\makeatletter
\renewcommand*\env@matrix[1][*\c@MaxMatrixCols c]{%
  \hskip -\arraycolsep
  \let\@ifnextchar\new@ifnextchar
  \array{#1}}
\makeatother

% Set related symbols
\newcommand{\set}[1]{ \left\{\,#1\,\right\} }
\newcommand{\union}{\cup}
\newcommand{\inters}{\cap}
\newcommand{\suchthat}{\,:\,} % oppure con {:}
\DeclareMathOperator{\tc}{\text{ tale che }}

\renewcommand{\epsilon}{\varepsilon}
\renewcommand{\theta}{\vartheta}
\renewcommand{\rho}{\varrho}
\renewcommand{\phi}{\varphi}

\let\oldre\Re
\let\oldim\Im
\renewcommand{\Re}[1]{\operatorname{Re}(#1)}
\renewcommand{\Im}[1]{\operatorname{Im}(#1)}
\newcommand{\conj}[1]{\overline{#1}}

\newcommand{\deq}{:=}
\newcommand{\iseq}{\overset{?}{=}}
\newcommand{\seteq}{\overset{!}{=}}
\newcommand{\divides}{\mid} % divide esattamente
\newcommand{\ndivides}{\not\mid} % non divide esattamente
\newcommand{\congr}{\equiv} % congruo 
\newcommand{\ncongr}{\not\congr} % non congruo

\renewcommand{\vec}[1]{\bm{#1}}

\newcommand{\Mod}[1]{\ \left(#1\right)}
\newcommand{\abs}[1]{\left|#1\right|}
\newcommand{\mcm}[2]{\operatorname{mcm}\left(#1, #2\right)}
\newcommand{\mcd}[2]{\operatorname{mcd}\left(#1, #2\right)}
\newcommand{\Span}[1]{\operatorname{span}\left\{#1\right\}}
\newcommand{\basis}[1]{\left\langle #1 \right\rangle}
\newcommand{\innerprod}[2]{\langle #1{,}#2\rangle}
\newcommand{\norm}[1]{\left\lVert#1\right\rVert}
\newcommand{\ortog}[1]{#1^{\perp}}
\newcommand{\proj}[2]{\operatorname{proj}_{#2}\!\left( #1 \right)}
\newcommand{\rk}[1]{\operatorname{rango}\left( #1 \right)}
\newcommand{\Imm}[1]{\operatorname{Im}#1}
\newcommand{\id}{\operatorname{id}}
\newcommand{\N}{\mathbb{N}}
\newcommand{\Z}{\mathbb{Z}}
\newcommand{\Q}{\mathbb{Q}}
\newcommand{\R}{\mathbb{R}}
\newcommand{\C}{\mathbb{C}}
\newcommand{\K}{\mathbb{K}}
\newcommand{\M}{\mathcal{M}}
\newcommand{\BB}{\mathcal{B}}

\begin{document}

% \author{Luca De Paulis}
\title{Esercizi di Algebra Lineare}
\maketitle

\tableofcontents

% \chapter{Indipendenza lineare}

\section{Teoremi utili}

\begin{definition}[Indipendenza lineare]
    Sia $V$ uno spazio vettoriale, $\vec{v_1}, \dots, \vec{v_n} \in V$. Allora l'insieme $\left\{ \vec{v_1}, \dots, \vec{v_n} \right\}$ si dice insieme di vettori linearmente indipendenti se
    \begin{equation}
        a_1\vec{v_1} + \dots + a_n\vec{v_n} = \vec{0_V} \iff a_1 = \dots = a_n = 0
    \end{equation}
    cioe' se l'unica combinazione lineare di $\vec{v_1}, \dots, \vec{v_n}$ che da' come risultato il vettore nullo e' quella con $a_1 = \dots = a_n = 0$.
\end{definition}

\begin{proposition}[Mosse di Gauss per colonna non modificano lo span] \label{span_Gauss}

    Sia $V$ uno spazio vettoriale e $\vec{v_1}, \dots, \vec{v_n} \in V$. Allora per ogni $k \in \R$ e per ogni $i, j \leq n$.
    \begin{equation}
        \Span{\vec{v_1}, \dots, \vec{v_i}, \vec{v_j}, \dots, \vec{v_n}} = \Span{\vec{v_1}, \dots, \vec{v_i} + k\vec{v_j}, \vec{v_j}, \dots, \vec{v_n}}.
    \end{equation}
    
    Inoltre scambiare due vettori o sostituire ad un vettore un suo multiplo non cambia lo span, dunque le mosse di Gauss non modificano lo span dei vettori.
\end{proposition}

\begin{proposition}[Mosse di colonna per ottenere uno span di vettori indipendenti] \label{span_colonne_indipendenti}

    Siano $\vec{v_1}, \dots, \vec{v_n} \in \R^m$ dei vettori colonna. Allora per stabilire quali di questi vettori sono indipendenti consideriamo la matrice $A$ che contiene come colonna $i$-esima il vettore colonna $v_i$ e riduciamola a scalini per colonna. Lo span delle colonne non nulle della matrice ridotta a scalini e' uguale allo span di $\vec{v_1}, \dots, \vec{v_n}$.
\end{proposition}
\chapter{Sottospazi vettoriali}

In questo capitolo vogliamo scoprire come verificare se un dato sottoinsieme di uno spazio vettoriale e' un sottospazio vettoriale.

\section{Teoremi e definizioni utili}

\begin{definition}[Sottospazio vettoriale]
    Sia $V$ uno spazio vettoriale, $A \subseteq V$. Allora si dice che $A$ e' un sottospazio vettoriale di $V$ (o semplicemente sottospazio) se
    \begin{align}
        &\vec{0_V} \in A \label{0_in_A}\\
        &(\vec{v} + \vec{w}) \in A    &&\forall \vec{v}, \vec{w} \in A \label{somma_in_A}\\
        &(k\vec{v}) \in A            &&\forall k \in \R, \vec{v} \in A \label{prodotto_in_A}
    \end{align}
\end{definition}

\section{Verifica}

\begin{example}
    Sia $S \subseteq \R^3$ tale che \[
        S = \set{ \begin{pmatrix}
            x \\ y \\ z
        \end{pmatrix} \in \R^3 \suchthat x -2y + 3z = 0}.
    \]

    Per verificare se $S$ e' un sottospazio di $\R^3$ e' sufficiente verificare che $S$ rispetti le tre condizioni di sopra:
    \begin{description}
        \item[($0 \in S$)] Verifichiamo che il vettore $\vec{0_{\R^3}} = (0, 0, 0)$ appartenga ad $S$, ovvero soddisfi la condizione $x -2y + 3z = 0$: \[
            0 -2\cdot 0 + 3 \cdot 0 = 0 + 0 + 0 = 0
        \] La condizione quindi e' verificata e $0_V \in S$.
        \item[($\vec v + \vec w \in S$)] Verifichiamo che se \[
            \vec v = \begin{pmatrix}
                v_1 \\ v_2 \\ v_3
            \end{pmatrix}, \quad \vec w = \begin{pmatrix}
                w_1 \\ w_2 \\ w_3
            \end{pmatrix} 
        \] appartengono ad $S$, cioe' \[
            v_1 - 2v_2 + 3v_3 = 0, \quad w_1 - 2w_2 + 3w_3 = 0
        \] allora il vettore $\vec{v} + \vec{w} = (v_1+w_1, v_2+w_2, v_3+w_3)$ appartiene ad $S$, ovvero soddisfa la condizione $x -2y + 3z = 0$. \begin{align*}
            (v_1+w_1) -2(v_2+w_2) + 3(v_3+w_3) \\
            =\ &v_1+w_1 -2v_2-2w_2 + 3v_3+3w_3 \\
            =\ &(v_1 -2v_2 +3v_3) + (w_1 -2w_2 +3w_3)\\
            \intertext{dunque per l'ipotesi che $\vec v \in S$ e $\vec w \in S$}
            =\ &0 + 0\\
            =\ &0.
        \end{align*}
        Dunque $\vec v + \vec w \in S$.
        \item[($k\vec v  \in S$)] Verifichiamo che se \[
            \vec v = \begin{pmatrix}
                v_1 \\ v_2 \\ v_3
            \end{pmatrix}
        \] appartiene ad $S$, cioe' \[
            v_1 - 2v_2 + 3v_3 = 0
        \] allora per ogni $k \in \R$ il vettore $k\vec{v} = (kv_1, kv_2, kv_3)$ appartiene ad $S$, ovvero soddisfa la condizione $x -2y + 3z = 0$. \begin{align*}
            (kv_1) -2(kv_2) + 3(kv_3) &= kv_1 -2kv_2 + 3kv_3 \\
            &= k(v_1 -2v_2 +3v_3)\\
            \intertext{dunque per l'ipotesi che $\vec v \in S$}
            &= k \cdot 0\\
            &= 0.
        \end{align*}
        Dunque $k\vec v \in S$.
    \end{description}
    Concludiamo che $S$ e' un sottospazio di $\R^3$.
\end{example}

\begin{example}
    Sia $S \subseteq \R^3$ tale che \[
        S = \set{ \begin{pmatrix}
            x \\ y \\ z
        \end{pmatrix} \in \R^3 \suchthat x -2y + 3z = 4}.
    \]

    Per verificare se $S$ e' un sottospazio di $\R^3$ e' sufficiente verificare che $S$ rispetti le tre condizioni di sopra:
    \begin{description}
        \item[($0 \in S$)] Verifichiamo che il vettore $\vec{0_\R^3} = (0, 0, 0)$ appartenga ad $S$, ovvero soddisfi la condizione $x -2y + 3z = 4$: \[
            0 -2\cdot 0 + 3 \cdot 0 = 0 + 0 + 0 = 0 \neq 4
        \] La condizione quindi non e' verificata.
    \end{description}
    Possiamo subito concludere che $S$ non e' un sottospazio di $\R^3$.
\end{example}

\begin{example}
    Sia $S \subseteq \R^3$ tale che \[
        S = \set{ \begin{pmatrix}
            x \\ y \\ z
        \end{pmatrix} \in \R^3 \suchthat x^2 - y^2 = 0}.
    \]

    Per verificare se $S$ e' un sottospazio di $\R^3$ e' sufficiente verificare che $S$ rispetti le tre condizioni di sopra:
    \begin{description}
        \item[($0 \in S$)] Verifichiamo che il vettore $\vec{0_\R^3} = (0, 0, 0)$ appartenga ad $S$, ovvero soddisfi la condizione $x^2 - y^2 = 0$: \[
            0^2 - 0y^2 = 0 + 0 = 0
        \] La condizione quindi e' verificata e $0_V \in S$.
        \item[($\vec v + \vec w \in S$)] Verifichiamo che se \[
            \vec v = \begin{pmatrix}
                v_1 \\ v_2 \\ v_3
            \end{pmatrix}, \quad \vec w = \begin{pmatrix}
                w_1 \\ w_2 \\ w_3
            \end{pmatrix} 
        \] appartengono ad $S$, cioe' \[
            v_1^2 - v_2^2 = 0, \quad w_1^2 - w_2^2 = 0
        \] allora il vettore $\vec{v} + \vec{w} = (v_1+w_1, v_2+w_2, v_3+w_3)$ appartiene ad $S$, ovvero soddisfa la condizione $x^2 - y^2 = 0$. \begin{align*}
            &(v_1+w_1)^2 - (v_2+w_2)^2 \\
            =\ &v_1^2+w_1^2 + 2v_1w_1 -v_2^2-w_2^2-2v_2w_2 \\
            =\ &(v_1^2 - v_2^2) + (w_1^2 - w_2^2) + 2v_1w_1 -2v_2w_2\\
            \intertext{dunque per l'ipotesi che $\vec v \in S$ e $\vec w \in S$}
            =\ &0 + 0  + 2v_1w_1 -2v_2w_2\\
            =\ &2v_1w_1 -2v_2w_2.
        \end{align*}
        Ma nessuno ci assicura che questa somma sia uguale a $0$ (ad esempio basta scegliere $\vec v = (1, -1, 0)$ e $\vec w = (1, 1, 0)$), dunque la condizione non e' sempre rispettata.
    \end{description}
    Concludiamo che $S$ non e' un sottospazio di $\R^3$.
\end{example}

\begin{example}
    Sia $V = \M_{2 \times 2}(\R)$ e sia $A \subseteq V$ tale che \[
        A = \set{M \in \M_{2 \times 2}(\R) \suchthat M = M^T}.
    \]

    Vedere se questo e' un sottospazio sembra piu' difficile degli esercizi precedenti. Tuttavia, possiamo cercare di rendere la definizione di $A$ piu' esplicita in modo da capire meglio quale sia la condizione di appartenenza al sottospazio.

    Notiamo che tutta la definizione di $A$ si basa su una matrice generica $M \in \M_{2 \times 2}(\R)$. Rendiamo piu' esplicita questa definizione, scrivendo \[
        M = \begin{pmatrix}
            a & b \\ c & d
        \end{pmatrix}    
    \] con $a, b, c, d \in \R$ generici.

    A questo punto ricordando la definizione di matrice trasposta (ovvero una matrice ottenuta trasformando le righe in colonne) possiamo scrivere la condizione di appartenenza al sottospazio $A$ come \[
        M = M^T \iff \begin{pmatrix}
            a & b \\ c & d
        \end{pmatrix} = \begin{pmatrix}
            a & c \\ b & d
        \end{pmatrix}
    \].
    
    Dunque la matrice $M$ appartiene ad $A$ se e solo se $b = c$ (ovvero la seconda e la terza coordinata sono uguali), cioe' \[
        A = \set{\begin{pmatrix}
            a & b \\ c & d
        \end{pmatrix} \in \M_{2 \times 2}(\R) \suchthat b = c}.
    \]

    A questo punto possiamo verificare se $A$ e' effettivamente un sottospazio di $\M_{2 \times 2}(\R)$.

    \begin{description}
        \item[($0 \in A$)] Verifichiamo che il vettore $\vec{0_V} = \begin{psmallmatrix}0 & 0 \\ 0 & 0 \end{psmallmatrix}$ appartenga ad $A$, ovvero soddisfi la condizione $b = c$. La condizione e' ovviamente verificata e dunque $0_V \in A$.
        \item[($\vec v + \vec w \in A$)] Verifichiamo che se \[
            M_1 = \begin{pmatrix}
                p & q \\ r & s
            \end{pmatrix}, \quad M_2 = \begin{pmatrix}
                x & y \\ z & t
            \end{pmatrix} 
        \] appartengono ad $A$, cioe' \[
            q = r, \quad y = z
        \] allora la matrice \[
            M_1 + M_2 = \begin{pmatrix}
                p+x & q+y \\ r+z & s+t
            \end{pmatrix}
        \] appartiene ad $A$, ovvero soddisfa la condizione $b = c$. \begin{align*}
            &(q+y) \iseq (r+z) \\
            \intertext{Per l'ipotesi che $M_1 \in A$ e $M_2 \in A$ sappiamo che $q = r$ e $y = z$:}
            \iff &r + z = r+z
        \end{align*}
        che e' ovvia. Dunque $M_1 + M_2 \in A$.
        \item[($k\vec v  \in A$)] Verifichiamo che se \[
            M = \begin{pmatrix}
                x & y \\ z & t
            \end{pmatrix} 
        \] appartiene ad $A$, cioe' \[
            y = z
        \] allora per ogni $k \in \R$ la matrice \[
            kM = \begin{pmatrix}
                kx & ky \\ kz & kt
            \end{pmatrix} 
        \]
        appartiene ad $A$, ovvero soddisfi la condizione $ky = kz$. 
        
        Ma per ipotesi $y = z$, dunque moltiplicando entrambi i membri per $k$ otteniamo $ky = kz$, che e' quello che stavamo cercando di dimostrare.

        Dunque $k\M \in A$.
    \end{description}

    Segue quindi che $A$ e' un sottospazio di $\M_{2 \times 2}(\R)$. Tale sottospazio si chiama \emph{spazio delle matrici simmetriche}.
\end{example}

\begin{exercise}
    Dire se i seguenti sottoinsiemi sono sottospazi oppure no.

    \begin{enumerate}
        \item $V \subseteq \R^4$ tale che \[
            V = \set{ \begin{pmatrix}
                x \\ y \\ z
            \end{pmatrix} \in R^4 \suchthat 
                3x - 2y + z + t = 0 }
        \]
        \item $V \subseteq \R^3$ tale che \[
            V = \set{ \begin{pmatrix}
                x \\ y \\ z
            \end{pmatrix} \in R^3 \suchthat \left\{
                \begin{array}{@{}rororor }
                3x & - & 2y & + & z & = & 0 \\
                -x & + & y & + & 4z & = & 2
                \end{array}
            \right.}
        \]
        \item $V \subseteq \R^3$ tale che \[
            V = \Span{\begin{pmatrix}
                1 \\ 2 \\ 0
            \end{pmatrix}, \begin{pmatrix}
                -1 \\ 0 \\ 1
            \end{pmatrix}}   
        \]
        \item $V \subseteq \R[x]^{\leq 3}$ tale che \[
            V = \set{p(x) \in \R[x]^{\leq 3} \suchthat p(2) = 0}
        \]
        \item $V \subseteq \R[x]^{\leq 3}$ tale che \[
            V = \set{p(x) \in \R[x]^{\leq 3} \suchthat p(2) = -1}
        \]
        \item $V \subseteq \M_{2\times 2}(\R)$ tale che \[
            V = \set{M \in \M_{2\times 2}(\R) \suchthat AM - MA = O_2}
        \] dove $A$ e $O_2$ sono due matrici $2 \times 2$ tali che \begin{align*}
            A = \begin{pmatrix}
                0 & -1 \\ -1 & 0
            \end{pmatrix}, &&O_2 = \begin{pmatrix}
                0 & 0 \\ 0 & 0
            \end{pmatrix}.
        \end{align*}
    \end{enumerate}
\end{exercise}
\chapter{Forma parametrica e cartesiana}

\section{Definizioni}

\begin{definition}
    [Forma parametrica e cartesiana]
    Sia $V$ uno spazio vettoriale e sia $A$ un sottospazio di $V$. Allora si dice che $A$ e' espresso in forma parametrica se e' scritto come \[
        A = \Span{\vec{v_1}, \vec{v_2}, \dots, \vec{v_n}}    
    \] con $\vec{v_1}, \dots, \vec{v_n} \in A$.

    Invece si dice che $A$ e' espresso in forma cartesiana se e' scritto come \[
        A = \{\; \vec{v} \in V : \vec{v} \text{ rispetta qualche condizione} \;\}.
    \]
\end{definition}

Ad esempio se $A$ e' un sottospazio di $\R^3$ allora
 \[
    A = \Span{
        \begin{pmatrix} 1 \\ 2 \\ 3 \end{pmatrix}; \begin{pmatrix} 1 \\ 0 \\ -10 \end{pmatrix}
    }    
\] e' espresso in forma parametrica, mentre
\[
    A = \{\; \begin{pmatrix}
        x \\ y \\ z
    \end{pmatrix} \in R^3 : x-2y+3 = 0 \;\}    
\] e' espresso in forma cartesiana.

\section{Passare dalla forma cartesiana alla forma parametrica}

\begin{example}
    Sia $A \subseteq R^3$ tale che \[
        A = \left\{\; \begin{pmatrix}
            x \\ y \\ z
        \end{pmatrix} \in R^3 : \left\{
            \begin{array}{@{}rororor }
            3x & - & 2y & + & z & = & 0 \\
            -x & + & y & + & 4z & = & 0
            \end{array}
        \right. \;\right\}.
    \]

    Per scriverlo in forma parametrica dobbiamo risolvere il sistema e sostituire le informazioni ricavate nell'espressione per il vettore.

    Risolviamo il sistema:
    \begin{align*}
        \begin{pmatrix}
            3 & -2 & 1 \\
            -1 & 1 & 4
        \end{pmatrix} \xrightarrow[R_1 \times -1]{\text{scambio}} 
        \begin{pmatrix}
            1 & -1 & -4 \\
            3 & -2 & 1
        \end{pmatrix} \\ \xrightarrow[]{R_2 - 3R_1} 
        \begin{pmatrix}
            1 & -1 & -4 \\
            0 & 1 & 13
        \end{pmatrix} \xrightarrow[]{R_1 + R_2} 
        \begin{pmatrix}
            1 & 0 & 9 \\
            0 & 1 & 13
        \end{pmatrix}
    \end{align*}

    Dunque la soluzione al sistema e' $x = -9z$, $y = -13z$ con $z \in \R$ libera. Sostituiamolo nell'espressione per $(x, y, z)$:

    \begin{align*}
        A &= \left\{\; \begin{pmatrix}
            -9z \\ -13z \\ z
        \end{pmatrix} \in R^3 :z \in R \;\right\} \\
        &= \left\{\; z\begin{pmatrix}
            -9 \\ -13 \\ 1
        \end{pmatrix} \in R^3 :z \in R \;\right\} \\
        &= \Span{\begin{pmatrix}
            -9 \\ -13 \\ 1
        \end{pmatrix}}
    \end{align*}
    che e' la forma parametrica del sottospazio $A$.
\end{example}

\begin{exercise}
    Dato $A$ in forma cartesiana, scriverlo in forma parametrica.
    \begin{enumerate}[(1)]
        \item $A$ sottospazio di $\R^4$ tale che \[
            A = \set{ \begin{pmatrix}
                x \\ y \\ z \\ t
            \end{pmatrix} \in \R^4 \suchthat \left\{
                \begin{array}{@{}rorororor }
                4x & - & 2y & + & 2z & - & 6t &= & 0 \\
                -x & + & 3y & + & 4z & + & 2t &= & 0
                \end{array}
            \right.}    
        \]
        \item $A$ sottospazio di $\R^3$ tale che \[
            A = \set{ \begin{pmatrix}
                x \\ y \\ z
            \end{pmatrix} \in \R^3 \suchthat \left\{
                \begin{array}{@{}rorororor }
                x & + & y & - & 2z & = & 0 \\
                -x &  & & + & 9z & = & 0
                \end{array}
            \right.}    
        \]
        \item $A$ sottospazio di $\R[x]^{\leq 2}$ tale che \[
            A = \set{ p(x) \in \R[x]^{\leq 2} \suchthat p(1) = 0}    
        \]
        \item $A$ sottospazio di $\M_{2 \times 2}(\R)$ tale che \[
            A = \set{ M \in \M_{2 \times 2}(\R) \suchthat M = M^T}    
        \]
    \end{enumerate}
\end{exercise}

\textsc{Hint:} se la condizione non e' totalmente esplicita (accade spesso quando si hanno spazi diversi da $\R^n$) basta esplicitarla. 

Ad esempio, se lo spazio e' $\R[x]^{\leq 2}$, invece di scrivere la condizione in termini di un polinomio generico $p(x)$ basta esplicitare il polinomio scrivendolo per esteso (in questo caso scriviamo $p(x) = a + bx + cx^2$ lasciando libere $a, b, c \in \R$) e poi riscrivere la condizione in termini delle nuove variabili $a, b, c$. 

A questo punto e' anche facile fare l'isomorfismo con $\R^{\text{quello che ti pare}}$ per risolvere l'esercizio come se fosse con i vettori colonna.

\section{Passare dalla forma parametrica alla forma cartesiana}

\begin{example}
    Sia $A$ sottospazio di $\R^3$ tale che \[
        A = \Span{\begin{pmatrix}
            1 \\ 2 \\ 3
        \end{pmatrix}, \begin{pmatrix}
            -1 \\ 0 \\ 1
        \end{pmatrix}}.   
    \]

    Per definizione di span, sappiamo che \[
        A = \set{a\begin{pmatrix}
            1 \\ 2 \\ 3
        \end{pmatrix} + b\begin{pmatrix}
            -1 \\ 0 \\ 1
        \end{pmatrix} \suchthat a, b \in \R}.
    \] Dunque un vettore generico $(x, y, z)$ e' in $A$ se e solo se \[
        \exists a, b \in \R \text{ tali che } \begin{pmatrix}
            x \\ y \\ z
        \end{pmatrix} = a\begin{pmatrix}
            1 \\ 2 \\ 3
        \end{pmatrix} + b\begin{pmatrix}
            -1 \\ 0 \\ 1
        \end{pmatrix} = \begin{pmatrix}
            a-b \\ 2a \\ 3a+b
        \end{pmatrix} = \begin{pmatrix}
            1 & -1 \\ 2 & 0 \\ 3 & 1
        \end{pmatrix}\begin{pmatrix}
            a \\ b
        \end{pmatrix}.
    \]

    Dunque la condizione per cui $(x, y, z) \in A$ dipende dalla \emph{risolubilita'} del sistema \[
        \begin{pmatrix}
            x \\ y \\ z
        \end{pmatrix} = \begin{pmatrix}
            1 & -1 \\ 2 & 0 \\ 3 & 1
        \end{pmatrix}\begin{pmatrix}
            a \\ b
        \end{pmatrix}.
    \]

    Proviamo a risolverlo e imponiamo che non vi siano equazioni impossibili.

    \begin{align*}
        \begin{pmatrix}[cc|c]
            1 & -1 & x\\ 2 & 0 & y\\ 3 & 1 & z
        \end{pmatrix} \xrightarrow[R_3 - 3R_1]{R_2 - 2R_1}
        \begin{pmatrix}[cc|c]
            1 & -1 & x\\ 0 & 2 & y - 2x\\ 0 & 4 & z - 3x
        \end{pmatrix} \xrightarrow[]{R_3 - 2R_2}
        \begin{pmatrix}[cc|c]
            1 & -1 & x\\ 0 & 2 & y - 2x\\ 0 & 0 & z - 3x - 2(y - 2x)
        \end{pmatrix}.
    \end{align*}

    Dunque il sistema ha soluzione se e solo se \[
        z - 3x - 2(y - 2x) = 0
    \] ovvero se e solo se \[
        x -2y + z = 0
    \] che e' la condizione che cercavamo.

    Di conseguenza, il sottospazio $A$ in forma cartesiana e' dato da \[
        A = \set{ \begin{pmatrix}
            x \\ y \\ z
        \end{pmatrix} \in \R^3 \suchthat x - 2y + z = 0}.    
    \]
\end{example}

\begin{exercise}
    Dato $A$ in forma parametrica, scriverlo in forma cartesiana.

    \begin{enumerate}[(1)]
        \item $A$ sottospazio di $\R^3$ tale che \[
            A = \Span{\begin{pmatrix} 1 \\ 0 \\ 2 \end{pmatrix}; \begin{pmatrix} 0 \\ 1 \\ -3 \end{pmatrix}}
        \]
        \item $A$ sottospazio di $\R^4$ tale che \[
            A = \Span{\begin{pmatrix} 2 \\ 1 \\ 1 \\ 2\end{pmatrix}; \begin{pmatrix} 0 \\ 1 \\ 1 \\ 0 \end{pmatrix}}
        \]
        \item $A$ sottospazio di $\R^2$ tale che \[
            A = \Span{\begin{pmatrix} 1 \\ 2 \end{pmatrix}; \begin{pmatrix} 0 \\ -1 \end{pmatrix}}
        \]
        \item $A$ sottospazio di $\R^4$ tale che \[
            A = \Span{\begin{pmatrix} 1 \\ 0 \\ 2 \\ -3 \end{pmatrix}; \begin{pmatrix} 0 \\ 1 \\ -3 \\ 2 \end{pmatrix}; \begin{pmatrix} 3 \\ -1 \\ 2 \\ 1 \end{pmatrix}}
        \]
    \end{enumerate}
\end{exercise}
\chapter{Determinare basi di sottospazi}

Dati sottospazi di uno spazio vettoriale $V$, scritti in forma parametrica o cartesiana, vorremmo riuscire a ricavare una base del sottospazio.

\section{Teoremi e definizioni utili}

\begin{definition}[Base di uno spazio vettoriale]
    Sia $V$ uno spazio vettoriale, $\vec{v_1}, \dots, \vec{v_n} \in V$. Allora si dice che $\mathcal{B} = \basis{\vec{v_1}, \dots, \vec{v_n}}$ e' una base di $V$ se
    \begin{itemize}
        \item i vettori $\vec{v_1}, \dots, \vec{v_n}$ generano $V$;
        \item i vettori $\vec{v_1}, \dots, \vec{v_n}$ sono linearmente indipendenti.
    \end{itemize}
\end{definition}

Le basi canoniche degli spazi vettoriali piu' comuni sono:
\begin{description}
    \item[base canonica di $R^n$] \hfill \\La base canonica di $\R^n$ e' \[
        \basis{\begin{pmatrix}
            1 \\ 0 \\ \vdots \\ 0
        \end{pmatrix}, \begin{pmatrix}
            0 \\ 1 \\ \vdots \\ 0
        \end{pmatrix}, \dots, \begin{pmatrix}
            0 \\ 0 \\ \vdots \\ 1
        \end{pmatrix}}.  
    \] 
    \item[base canonica di $\M_{n \times m}(\R)$] \hfill \\La base canonica di $\M_{2 \times 2}(\R)$ e' \[
        \basis{\begin{pmatrix}
            1 & 0 \\ 0 & 0
        \end{pmatrix}, \begin{pmatrix}
            0 & 1 \\ 0 & 0
        \end{pmatrix}, \begin{pmatrix}
            0 & 0 \\ 1 & 0
        \end{pmatrix}, \begin{pmatrix}
            0 & 0 \\ 0 & 1
        \end{pmatrix}}.  
    \] Ragionamento analogo per le $n \times m$. Lo spazio delle matrici $n \times m$ e' isomorfo a $\R^{nm}$.
    \item[base canonica dello spazio dei polinomi] \hfill \\
    La base canonica di $\R[x]^{\leq n}$ e' \[
        \basis{1, x, x^2, \dots, x^{n-1}, x^n}.  
    \] 
    Lo spazio dei polinomi di grado minore o uguale a $n$ e' isomorfo a $\R^{n+1}$.
\end{description}

\begin{proposition}[Mosse di colonna per ottenere uno span di vettori indipendenti] \label{span_colonne_indipendenti}

    Sia $V$ un sottospazio di $\R^n$ tale che $\vec{v_1}, \dots, \vec{v_m} \in \R^n$ siano suoi generatori, ovvero \[
        V = \Span{\vec{v_1}, \dots, \vec{v_m}}.
    \] 
    
    Consideriamo la matrice $A$ formata dai vettori $\vec{v_i}$ messi in colonna e riduciamola a scalini per colonna.
    Siano $\vec{c_1}, \dots, \vec{c_k}$ le colonne non nulle della matrice $A$ ridotta a scalini. Allora \begin{enumerate}[(i)]
        \item $\vec{c_1}, \dots, \vec{c_k}$ sono indipendenti;
        \item lo span di $\vec{c_1}, \dots, \vec{c_k}$ e' uguale allo span di $\vec{v_1}, \dots, \vec{v_n}$
    \end{enumerate}
    ovvero $\basis{\vec{c_1}, \dots, \vec{c_k}}$ e' una base di $V$.
\end{proposition}

\begin{proposition}[Estrazione di una base tramite mosse di riga]\label{estrarre_una_base}
    Sia $V$ un sottospazio di $\R^n$ tale che $\vec{v_1}, \dots, \vec{v_m} \in \R^n$ siano suoi generatori, ovvero \[
        V = \Span{\vec{v_1}, \dots, \vec{v_m}}.  
    \] 
    Allora possiamo porre i vettori come colonne di una matrice e ridurla a scalini per riga. Alla fine del procedimento i vettori che originariamente erano nelle colonne con i pivot sono indipendenti e generano $V$, dunque formano una base di $V$.
\end{proposition}

\section{Trovare una base tramite mosse di colonna}

Cerchiamo di sfruttare la proposizione \ref{span_colonne_indipendenti} per trovare una base di sottospazi vettoriali.

\begin{example}
    Sia $A \subseteq \R^3$ tale che \[
        A = \Span{\begin{pmatrix}
            1 \\ -2 \\ 0
        \end{pmatrix}; \begin{pmatrix}
            3 \\ -1 \\ 4
        \end{pmatrix}; \begin{pmatrix}
            -1 \\ 2 \\ -1
        \end{pmatrix}; \begin{pmatrix}
            4 \\ -3 \\ 0
        \end{pmatrix}}    
    \]

    Per trovare una base di $A$ tramite mosse di colonna mettiamo i vettori come colonne di una matrice e riduciamola a scalini per colonna.

    \begin{align*}
        \begin{pmatrix}
            1 & 3 & -1 & 4 \\ -2 & -1 & 2 & -3 \\ 0 & 4 & -1 & 0
        \end{pmatrix} \xrightarrow[C_4 - 4C_1]{C_2 - 3C_1, C_3 + C_1}
        \begin{pmatrix}
            1 & 0 & 0 & 0 \\ -2 & 5 & 0 & 5 \\ 0 & 4 & -1 & 0
        \end{pmatrix} \xrightarrow[]{C_4 - C_2} \\
        \xrightarrow[]{C_4 - C_2}\begin{pmatrix}
            1 & 0 & 0 & 0 \\ -2 & 5 & 0 & 0 \\ 0 & 4 & -1 & -4
        \end{pmatrix} \xrightarrow[]{C_4 + 4C_3}
        \begin{pmatrix}
            1 & 0 & 0 & 0 \\ -2 & 5 & 0 & 0 \\ 0 & 4 & -1 & 0
        \end{pmatrix}
    \end{align*}

    Dunque per la proposizione \ref{span_colonne_indipendenti} i vettori $(1, -2, 0), (0, 5, 4), (0, 0, -1)$ sono indipendenti e generano $V$, ovvero \[
        \Span{\begin{pmatrix}
            1 \\ -2 \\ 0
        \end{pmatrix}; \begin{pmatrix}
            3 \\ -1 \\ 4
        \end{pmatrix}; \begin{pmatrix}
            -1 \\ 2 \\ -1
        \end{pmatrix}; \begin{pmatrix}
            4 \\ -3 \\ 0
        \end{pmatrix}} = \Span{\begin{pmatrix} 1 \\ -2 \\ 0 \end{pmatrix}, \begin{pmatrix} 0 \\ 5 \\ 4 \end{pmatrix}, \begin{pmatrix} 0 \\ 0 \\ -1 \end{pmatrix}}    
    \]
    dunque $\BB = \basis{(1, -2, 0);\ (0, 5, 4);\ (0, 0, -1)}$ e' una base di $V$.
\end{example}

\begin{exercise}
    Dati uno spazio vettoriale $V$ e un sottospazio $A$, trovare una base di $A$.
    \begin{enumerate}
        \item Sia $V = \R^4$ e $A$ sottospazio di $V$ dato da \[
            A = \Span{\begin{pmatrix}
                1 \\ 2 \\ 1 \\ 3
            \end{pmatrix}, \begin{pmatrix}
                -1 \\ -6 \\ 5 \\ 1
            \end{pmatrix}, \begin{pmatrix}
                4 \\ 2 \\ 1 \\ 3
            \end{pmatrix}, \begin{pmatrix}
                2 \\ 2 \\ 3 \\ 1
            \end{pmatrix}}.  
        \]
        \item Sia $V = \R^3$ e $A$ sottospazio di $V$ dato da \[
            A = \Span{\begin{pmatrix}
                1 \\ -1 \\ 1
            \end{pmatrix}, \begin{pmatrix}
                -1 \\ 2 \\ 4
            \end{pmatrix}, \begin{pmatrix}
                3 \\ -1 \\ 7
            \end{pmatrix}, \begin{pmatrix}
                2 \\ 5 \\ 1
            \end{pmatrix}}.
        \]
        \item Sia $V = \R^3$ e $A$ sottospazio di $V$ dato da \[
            A = \set{\begin{pmatrix}
                x \\ y \\ z
            \end{pmatrix} \in \R^3 \suchthat x - 3y + 2z = 0}.
        \]
        \item Sia $V = \R[x]^{\leq 2}$ e $A$ sottospazio di $V$ dato da \[
            A = \set{p(x) \in V \suchthat p(3) = 0}.
        \]
        \item Sia $V = \M_{2 \times 2}(\R)$ e $A$ sottospazio di $V$ dato da \[
            A = \set{M \in V \suchthat M + M^T = O_2}
        \] dove $O_2$ e' la matrice $2 \times 2$ con zero in tutte le posizioni, mentre $M^T$ e' la matrice trasposta di $M$ (quella ottenuta trasformando le righe in colonne).
    \end{enumerate}
\end{exercise}


\textsc{Hint:} se il sottospazio e' in forma cartesiana, va prima portato in forma parametrica per fare i calcoli con gli span.

\textsc{Hint:} come nel capitolo precedente, se la condizione non e' totalmente esplicita (accade spesso quando si hanno spazi diversi da $\R^n$) basta esplicitarla. 

Ad esempio, se lo spazio e' $\R[x]^{\leq 2}$, invece di scrivere la condizione in termini di un polinomio generico $p(x)$ basta esplicitare il polinomio scrivendolo per esteso (in questo caso scriviamo $p(x) = a + bx + cx^2$ lasciando libere $a, b, c \in \R$) e poi riscrivere la condizione in termini delle nuove variabili $a, b, c$. 

A questo punto e' anche facile fare l'isomorfismo con $\R^{\text{quello che ti pare}}$ per risolvere l'esercizio come se fosse con i vettori colonna.

\section{Trovare una base tramite estrazione e mosse di riga}

Cerchiamo di sfruttare la proposizione \ref{estrarre_una_base} per trovare una base di sottospazi vettoriali.

\begin{example}
    Sia $A \subseteq \R^3$ tale che \[
        A = \Span{\begin{pmatrix}
            1 \\ -2 \\ 0
        \end{pmatrix}; \begin{pmatrix}
            3 \\ -1 \\ 4
        \end{pmatrix}; \begin{pmatrix}
            -1 \\ 2 \\ -1
        \end{pmatrix}; \begin{pmatrix}
            4 \\ -3 \\ 0
        \end{pmatrix}}    
    \]

    Per trovare una base di $A$ tramite mosse di riga mettiamo i vettori come colonne di una matrice e riduciamola a scalini per riga.

    \begin{align*}
        \begin{pmatrix}
            1 & 3 & -1 & 4 \\ -2 & -1 & 2 & -3 \\ 0 & 4 & -1 & 0
        \end{pmatrix} \xrightarrow[]{R_2 + 2R_1}
        \begin{pmatrix}
            1 & 3 & -1 & 4 \\ 0 & 5 & 0 & 5 \\ 0 & 4 & -1 & 0
        \end{pmatrix} \xrightarrow[]{R_2 \times \dfrac{1}{5}} \\
        \begin{pmatrix}
            1 & 3 & -1 & 4 \\ 0 & 1 & 0 & 1 \\ 0 & 4 & -1 & 0
        \end{pmatrix} \xrightarrow[]{R_3 -4R_2}
        \begin{pmatrix}
            1 & 3 & -1 & 4 \\ 0 & 1 & 0 & 1 \\ 0 & 0 & -1 & -4
        \end{pmatrix}
    \end{align*}

    I pivot di questa matrice sono nelle colonne $1$, $2$ e $3$, dunque per la proposizione \ref{estrarre_una_base} i vettori $(1, -2, 0), (3, -1, 4), (-1, 2, -1)$ sono indipendenti e generano $V$, ovvero \[
        \Span{\begin{pmatrix}
            1 \\ -2 \\ 0
        \end{pmatrix}; \begin{pmatrix}
            3 \\ -1 \\ 4
        \end{pmatrix}; \begin{pmatrix}
            -1 \\ 2 \\ -1
        \end{pmatrix}; \begin{pmatrix}
            4 \\ -3 \\ 0
        \end{pmatrix}} = \Span{\begin{pmatrix} 1 \\ -2 \\ 0 \end{pmatrix}, \begin{pmatrix} 3 \\ -1 \\ 4 \end{pmatrix}, \begin{pmatrix} -1 \\ 2 \\ -1 \end{pmatrix}}    
    \]
    dunque $\BB = \basis{(1, -2, 0);\ (3, -1, 4);\ (-1, 2, -1)}$ e' una base di $V$.
\end{example}

\textsc{Nota bene:} i due procedimenti (per colonna e per riga) danno quasi sempre due basi diverse, ma ugualmente valide.

\begin{exercise}
    Dati uno spazio vettoriale $V$ e un sottospazio $A$, estrarre una base di $A$.
    \begin{enumerate}
        \item Sia $V = \R^4$ e $A$ sottospazio di $V$ dato da \[
            A = \Span{\begin{pmatrix}
                1 \\ 2 \\ 1 \\ 3
            \end{pmatrix}, \begin{pmatrix}
                -1 \\ -6 \\ 5 \\ 1
            \end{pmatrix}, \begin{pmatrix}
                4 \\ 2 \\ 1 \\ 3
            \end{pmatrix}, \begin{pmatrix}
                2 \\ 2 \\ 3 \\ 1
            \end{pmatrix}}.  
        \]
        \item Sia $V = \R^3$ e $A$ sottospazio di $V$ dato da \[
            A = \Span{\begin{pmatrix}
                1 \\ -1 \\ 1
            \end{pmatrix}, \begin{pmatrix}
                -1 \\ 2 \\ 4
            \end{pmatrix}, \begin{pmatrix}
                3 \\ -1 \\ 7
            \end{pmatrix}, \begin{pmatrix}
                2 \\ 5 \\ 1
            \end{pmatrix}}.
        \]
        \item Sia $V = \R^3$ e $A$ sottospazio di $V$ dato da \[
            A = \set{\begin{pmatrix}
                x \\ y \\ z
            \end{pmatrix} \in \R^3 \suchthat x - 3y + 2z = 0}.
        \]
        \item Sia $V = \R[x]^{\leq 2}$ e $A$ sottospazio di $V$ dato da \[
            A = \set{p(x) \in V \suchthat p(3) = 0}.
        \]
        \item Sia $V = \M_{2 \times 2}(\R)$ e $A$ sottospazio di $V$ dato da \[
            A = \set{M \in V \suchthat M + M^T = O_2}
        \] dove $O_2$ e' la matrice $2 \times 2$ con zero in tutte le posizioni, mentre $M^T$ e' la matrice trasposta di $M$ (quella ottenuta trasformando le righe in colonne).
    \end{enumerate}
\end{exercise}

\textsc{Hint:} valgono gli stessi hint della sezione predecente.


\end{document}
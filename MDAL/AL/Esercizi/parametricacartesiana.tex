\chapter{Forma parametrica e cartesiana}

\section{Definizioni}

\begin{definition}
    [Forma parametrica e cartesiana]
    Sia $V$ uno spazio vettoriale e sia $A$ un sottospazio di $V$. Allora si dice che $A$ e' espresso in forma parametrica se e' scritto come \[
        A = \Span{\vec{v_1}, \vec{v_2}, \dots, \vec{v_n}}    
    \] con $\vec{v_1}, \dots, \vec{v_n} \in A$.

    Invece si dice che $A$ e' espresso in forma cartesiana se e' scritto come \[
        A = \{\; \vec{v} \in V : \vec{v} \text{ rispetta qualche condizione} \;\}.
    \]
\end{definition}

Ad esempio se $A$ e' un sottospazio di $\R^3$ allora
 \[
    A = \Span{
        \begin{pmatrix} 1 \\ 2 \\ 3 \end{pmatrix}; \begin{pmatrix} 1 \\ 0 \\ -10 \end{pmatrix}
    }    
\] e' espresso in forma parametrica, mentre
\[
    A = \{\; \begin{pmatrix}
        x \\ y \\ z
    \end{pmatrix} \in R^3 : x-2y+3 = 0 \;\}    
\] e' espresso in forma cartesiana.

\section{Passare dalla forma cartesiana alla forma parametrica}

\begin{example}
    Sia $A \subseteq R^3$ tale che \[
        A = \left\{\; \begin{pmatrix}
            x \\ y \\ z
        \end{pmatrix} \in R^3 : \left\{
            \begin{array}{@{}rororor }
            3x & - & 2y & + & z & = & 0 \\
            -x & + & y & + & 4z & = & 0
            \end{array}
        \right. \;\right\}.
    \]

    Per scriverlo in forma parametrica dobbiamo risolvere il sistema e sostituire le informazioni ricavate nell'espressione per il vettore.

    Risolviamo il sistema:
    \begin{align*}
        \begin{pmatrix}
            3 & -2 & 1 \\
            -1 & 1 & 4
        \end{pmatrix} \xrightarrow[R_1 \times -1]{\text{scambio}} 
        \begin{pmatrix}
            1 & -1 & -4 \\
            3 & -2 & 1
        \end{pmatrix} \\ \xrightarrow[]{R_2 - 3R_1} 
        \begin{pmatrix}
            1 & -1 & -4 \\
            0 & 1 & 13
        \end{pmatrix} \xrightarrow[]{R_1 + R_2} 
        \begin{pmatrix}
            1 & 0 & 9 \\
            0 & 1 & 13
        \end{pmatrix}
    \end{align*}

    Dunque la soluzione al sistema e' $x = -9z$, $y = -13z$ con $z \in \R$ libera. Sostituiamolo nell'espressione per $(x, y, z)$:

    \begin{align*}
        A &= \left\{\; \begin{pmatrix}
            -9z \\ -13z \\ z
        \end{pmatrix} \in R^3 :z \in R \;\right\} \\
        &= \left\{\; z\begin{pmatrix}
            -9 \\ -13 \\ 1
        \end{pmatrix} \in R^3 :z \in R \;\right\} \\
        &= \Span{\begin{pmatrix}
            -9 \\ -13 \\ 1
        \end{pmatrix}}
    \end{align*}
    che e' la forma parametrica del sottospazio $A$.
\end{example}

\begin{exercise}
    Dato $A$ in forma cartesiana, scriverlo in forma parametrica.
    \begin{enumerate}[(1)]
        \item $A$ sottospazio di $\R^4$ tale che \[
            A = \set{ \begin{pmatrix}
                x \\ y \\ z \\ t
            \end{pmatrix} \in \R^4 \suchthat \left\{
                \begin{array}{@{}rorororor }
                4x & - & 2y & + & 2z & - & 6t &= & 0 \\
                -x & + & 3y & + & 4z & + & 2t &= & 0
                \end{array}
            \right.}    
        \]
        \item $A$ sottospazio di $\R^3$ tale che \[
            A = \set{ \begin{pmatrix}
                x \\ y \\ z
            \end{pmatrix} \in \R^3 \suchthat \left\{
                \begin{array}{@{}rorororor }
                x & + & y & - & 2z & = & 0 \\
                -x &  & & + & 9z & = & 0
                \end{array}
            \right.}    
        \]
        \item $A$ sottospazio di $\R[x]^{\leq 2}$ tale che \[
            A = \set{ p(x) \in \R[x]^{\leq 2} \suchthat p(1) = 0}    
        \]
        \item $A$ sottospazio di $\M_{2 \times 2}(\R)$ tale che \[
            A = \set{ M \in \M_{2 \times 2}(\R) \suchthat M = M^T}    
        \]
    \end{enumerate}
\end{exercise}

\textsc{Hint:} se la condizione non e' totalmente esplicita (accade spesso quando si hanno spazi diversi da $\R^n$) basta esplicitarla. 

Ad esempio, se lo spazio e' $\R[x]^{\leq 2}$, invece di scrivere la condizione in termini di un polinomio generico $p(x)$ basta esplicitare il polinomio scrivendolo per esteso (in questo caso scriviamo $p(x) = a + bx + cx^2$ lasciando libere $a, b, c \in \R$) e poi riscrivere la condizione in termini delle nuove variabili $a, b, c$. 

A questo punto e' anche facile fare l'isomorfismo con $\R^{\text{quello che ti pare}}$ per risolvere l'esercizio come se fosse con i vettori colonna.

\section{Passare dalla forma parametrica alla forma cartesiana}

\begin{example}
    Sia $A$ sottospazio di $\R^3$ tale che \[
        A = \Span{\begin{pmatrix}
            1 \\ 2 \\ 3
        \end{pmatrix}, \begin{pmatrix}
            -1 \\ 0 \\ 1
        \end{pmatrix}}.   
    \]

    Per definizione di span, sappiamo che \[
        A = \set{a\begin{pmatrix}
            1 \\ 2 \\ 3
        \end{pmatrix} + b\begin{pmatrix}
            -1 \\ 0 \\ 1
        \end{pmatrix} \suchthat a, b \in \R}.
    \] Dunque un vettore generico $(x, y, z)$ e' in $A$ se e solo se \[
        \exists a, b \in \R \text{ tali che } \begin{pmatrix}
            x \\ y \\ z
        \end{pmatrix} = a\begin{pmatrix}
            1 \\ 2 \\ 3
        \end{pmatrix} + b\begin{pmatrix}
            -1 \\ 0 \\ 1
        \end{pmatrix} = \begin{pmatrix}
            a-b \\ 2a \\ 3a+b
        \end{pmatrix} = \begin{pmatrix}
            1 & -1 \\ 2 & 0 \\ 3 & 1
        \end{pmatrix}\begin{pmatrix}
            a \\ b
        \end{pmatrix}.
    \]

    Dunque la condizione per cui $(x, y, z) \in A$ dipende dalla \emph{risolubilita'} del sistema \[
        \begin{pmatrix}
            x \\ y \\ z
        \end{pmatrix} = \begin{pmatrix}
            1 & -1 \\ 2 & 0 \\ 3 & 1
        \end{pmatrix}\begin{pmatrix}
            a \\ b
        \end{pmatrix}.
    \]

    Proviamo a risolverlo e imponiamo che non vi siano equazioni impossibili.

    \begin{align*}
        \begin{pmatrix}[cc|c]
            1 & -1 & x\\ 2 & 0 & y\\ 3 & 1 & z
        \end{pmatrix} \xrightarrow[R_3 - 3R_1]{R_2 - 2R_1}
        \begin{pmatrix}[cc|c]
            1 & -1 & x\\ 0 & 2 & y - 2x\\ 0 & 4 & z - 3x
        \end{pmatrix} \xrightarrow[]{R_3 - 2R_2}
        \begin{pmatrix}[cc|c]
            1 & -1 & x\\ 0 & 2 & y - 2x\\ 0 & 0 & z - 3x - 2(y - 2x)
        \end{pmatrix}.
    \end{align*}

    Dunque il sistema ha soluzione se e solo se \[
        z - 3x - 2(y - 2x) = 0
    \] ovvero se e solo se \[
        x -2y + z = 0
    \] che e' la condizione che cercavamo.

    Di conseguenza, il sottospazio $A$ in forma cartesiana e' dato da \[
        A = \set{ \begin{pmatrix}
            x \\ y \\ z
        \end{pmatrix} \in \R^3 \suchthat x - 2y + z = 0}.    
    \]
\end{example}

\begin{exercise}
    Dato $A$ in forma parametrica, scriverlo in forma cartesiana.

    \begin{enumerate}[(1)]
        \item $A$ sottospazio di $\R^3$ tale che \[
            A = \Span{\begin{pmatrix} 1 \\ 0 \\ 2 \end{pmatrix}; \begin{pmatrix} 0 \\ 1 \\ -3 \end{pmatrix}}
        \]
        \item $A$ sottospazio di $\R^4$ tale che \[
            A = \Span{\begin{pmatrix} 2 \\ 1 \\ 1 \\ 2\end{pmatrix}; \begin{pmatrix} 0 \\ 1 \\ 1 \\ 0 \end{pmatrix}}
        \]
        \item $A$ sottospazio di $\R^2$ tale che \[
            A = \Span{\begin{pmatrix} 1 \\ 2 \end{pmatrix}; \begin{pmatrix} 0 \\ -1 \end{pmatrix}}
        \]
        \item $A$ sottospazio di $\R^4$ tale che \[
            A = \Span{\begin{pmatrix} 1 \\ 0 \\ 2 \\ -3 \end{pmatrix}; \begin{pmatrix} 0 \\ 1 \\ -3 \\ 2 \end{pmatrix}; \begin{pmatrix} 3 \\ -1 \\ 2 \\ 1 \end{pmatrix}}
        \]
    \end{enumerate}
\end{exercise}